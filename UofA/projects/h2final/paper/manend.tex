%  start of file manend.tex
%          
%   This file is part of the files in the REVTeX 3.1 distribution.
%   Version 3.1 of REVTeX, September 1996.
%
%   Copyright (c) 1992 American Physical Society, Optical Society of America,
%   American Institute of Physics.
%
%   See the REVTeX 3.1 README file for restrictions and more information.
%
%

%\documentstyle[twocolumn,aps]{revtex}
%
% comment out the documentstyle line above and use one of the other lines
% if you have the AMSFonts installed and would like to see the characters
% and fonts available with these options enabled. There is no guarantee
% that any specific TeX installation will be able to handle these options.
%
% the amsfonts option gives better handling of \bbox output in odd sizes
% and superscripts, along with the \frak and \Bbb font commands:
% \documentstyle[amsfonts,twocolumn,aps]{revtex}
%
% the amssymb option gives all the capabilities of the amsfonts option, but
% also defines names for many extra symbols:
\documentstyle[amsfonts,twocolumn,aps]{revtex}
%
\def\SNG{{\em Physical Review Style and Notation Guide}}
\def\LUG {{\em \LaTeX{} User's Guide \& Reference Manual}}
\def\btt#1{{\tt$\backslash$\string#1}}%
\def\REVTeX{REV\TeX}
\def\AmS{{\protect\the\textfont2
        A\kern-.1667em\lower.5ex\hbox{M}\kern-.125emS}}
\def\AmSLaTeX{\AmS-\LaTeX}
\def\BibTeX{\rm B{\sc ib}\TeX}
\makeatletter
% run page numbers by "chapter"
\def\thepage{A-\@arabic\c@page}
% these page numbers need a bit more width
\def\@pnumwidth{2em}
\makeatother
% next line overrules aps style neutralization of \slantfrac
\def\slantfrac#1#2{\kern1em^{#1}\kern-.3em/\kern-.1em_{#2}}
\begin{document}

\appendix
\twocolumn

\makeatletter
\global\@specialpagefalse
\def\@oddhead{\REVTeX{} 3.1\hfill Released September 1996}
\let\@evenhead\@oddhead
% run page numbers by "chapter", with copyright for first page
\def\@oddfoot{\reset@font\rm\hfill \thepage\hfill
\ifnum\c@page=1
  \llap{\protect\copyright{} 1996~~%
  $\vcenter{\baselineskip10pt
    \hbox{American Physical Society}
    \hbox{Optical Society of America}
    \hbox{American Institute of Physics}
  }$}\fi
} \let\@evenfoot\@oddfoot
\makeatother

\section{Character Set Listing}
\label{sec:chars}

This appendix provides tables which show all of the special characters
and mathematical symbols that are available within \REVTeX{}. Some of these
symbols require the AMSFonts to be available.

\REVTeX{} version 3.1 supports an extensive set of symbols, alphabets, and
special fonts. Their availability does not relieve an author (editor)
of considering whether a chosen notation or symbol will  convey the
intended meaning, and whether there is a more conventional alternative.
As always, for the benefit of the reader, notation should be clear, as simple
as possible, and consistent with standard usage. Nonstandard symbols should
only be used if necessary; their meaning should be explained in the paper
at the first occurrence.

Editorial policy on this issue may vary from journal to journal. Check recent
issues of a given journal and/or query the editor. APS authors
may also consult the journal's
``Information for Contributors'' as well as the \SNG{}.
In preparing an accepted paper for publication, the
editor may suggest (require) the use of alternative notation.

\def\xxx{4pt}

\subsection[\protect\LaTeX{} notations]{LaTeX notations}

\subsubsection[Standard \protect\LaTeX{} symbols]{Standard LaTeX symbols}
The following tables show the standard symbols for \LaTeX{} users.

\begin{table}
\caption{Text accents with letter a.}
\begin{tabular}{c@{\hspace{\xxx}}lc@{\hspace{\xxx}}lc@{\hspace{\xxx}}%
                lc@{\hspace{\xxx}}l}
\`{a} & \verb+\`{a}+&
\'{a} & \verb+\'{a}+&
\^{a} & \verb+\^{a}+&
\"{a} & \verb+\"{a}+\\
\~{a} & \verb+\~{a}+&
\={a} & \verb+\={a}+&
\.{a} & \verb+\.{a}+&
\u{a} & \verb+\u{a}+\\
\v{a} & \verb+\v{a}+&
\H{a} & \verb+\H{a}+&
\t{aa} & \verb+\t{aa}+&
\c{a} & \verb+\c{a}+\\
\d{a} & \verb+\d{a}+&
\b{a} & \verb+\b{a}+
\end{tabular}
\end{table}


\begin{table}
\caption{Math accents with letter a.}
\begin{tabular}%
  {c@{\hspace{\xxx}}lc@{\hspace{\xxx}}lc@{\hspace{\xxx}}lc@{\hspace{\xxx}}l}
$\hat{a}$ & \verb+\hat{a}+ &
$\check{a}$ & \verb+\check{a}+ &
 $\dot{a}$ & \verb+\dot{a}+ &
 $\ddot{a}$ & \verb+\ddot{a}+ \\
$\breve{a}$ & \verb+\breve{a}+ &
 $\tilde{a}$ & \verb+\tilde{a}+ &
 $\grave{a}$ & \verb+\grave{a}+ &
 $\acute{a}$ & \verb+\acute{a}+ \\
  $\bar{a}$ & \verb+\bar{a}+ &
  $\vec{a}$ & \verb+\vec{a}+ &
\end{tabular}
\end{table}


\begin{table}
\caption{Special symbols; any mode.}
\begin{tabular}{c@{\hspace{\xxx}}lc@{\hspace{\xxx}}lc@{\hspace{\xxx}}l}
$\dagger$ & \verb+\dagger+& \S & \verb+\S+& \copyright & \verb+\copyright+\\
$\ddagger$ & \verb+\ddagger+& \P & \verb+\P+& \pounds & \verb+\pounds+\\
\end{tabular}
\end{table}



\begin{table}
\caption{Other special (foreign) symbols; text mode.}
\begin{tabular}%
  {c@{\hspace{\xxx}}lc@{\hspace{\xxx}}lc@{\hspace{\xxx}}lc@{\hspace{\xxx}}l}
 \aa & \verb+\aa+&
 \AA & \verb+\AA+&
\ae & \verb+\ae+&
\AE & \verb+\AE+\\
 \o & \verb+\o+&
 \O & \verb+\O+&
\oe & \verb+\oe+&
\OE & \verb+\OE+\\
 \l & \verb+\l+&
 \L & \verb+\L+&
 ?` & \verb+?`+&
 !` & \verb+!`+\\
 \ss & \verb+\ss+
\end{tabular}
\end{table}

\begin{table}
\caption{Greek letters; used in math mode.}
\begin{tabular}%
  {c@{\hspace{\xxx}}lc@{\hspace{\xxx}}lc@{\hspace{\xxx}}lc@{\hspace{\xxx}}l}
\multicolumn{8}{c}{\it Lowercase}\\
$\alpha$ & \verb+\alpha+ &
$\beta$ & \verb+\beta+ &
$\gamma$ & \verb+\gamma+ &
$\delta$ & \verb+\delta+\\
$\epsilon$ & \verb+\epsilon+ &
$\varepsilon$ & \verb+\varepsilon+ &
$\zeta$ & \verb+\zeta+ &
$\eta$ & \verb+\eta+ \\
 $\theta$ & \verb+\theta+ &
 $\vartheta$ & \verb+\vartheta+ &
 $\iota$ & \verb+\iota+ &
 $\kappa$ & \verb+\kappa+ \\
 $\lambda$ & \verb+\lambda+ &
 $\mu$ & \verb+\mu+ &
 $\nu$ & \verb+\nu+ &
 $\xi$ & \verb+\xi+\\
 $o$ & \verb+o+ &
 $\pi$ & \verb+\pi+ &
 $\varpi$ & \verb+\varpi+ &
 $\rho$ & \verb+\rho+ \\
   $\varrho$ & \verb+\varrho+ &
   $\sigma$ & \verb+\sigma+ &
 $\varsigma$ & \verb+\varsigma+ &
    $\tau$ & \verb+\tau+ \\
   $\upsilon$ & \verb+\upsilon+ &
   $\phi$ & \verb+\phi+ &
   $\varphi$ & \verb+\varphi+ &
 $\chi$ & \verb+\chi+ \\
 $\psi$ & \verb+\psi+ &
  $\omega$ & \verb+\omega+ &
 \\[\baselineskip]
\multicolumn{8}{c}{\it Uppercase}\\
$\Gamma$ & \verb+\Gamma+ &
$\Delta$ & \verb+\Delta+ &
$\Theta$ & \verb+\Theta+ &
 $\Lambda$ & \verb+\Lambda+\\
 $\Xi$ & \verb+\Xi+ &
 $\Pi$ & \verb+\Pi+ &
   $\Sigma$ & \verb+\Sigma+ &
    $\Upsilon$ & \verb+\Upsilon+ \\
 $\Phi$ & \verb+\Phi+ &
 $\Psi$ & \verb+\Psi+ &
 $\Omega$ & \verb+\Omega+ &
\end{tabular}
\end{table}


\begin{table}
\caption{Binary operation symbols; used in math mode.}
\def\xxx{3.2pt}
\begin{tabular}%
  {c@{\hspace{\xxx}}lc@{\hspace{\xxx}}lc@{\hspace{\xxx}}lc@{\hspace{\xxx}}l}
$\pm$ & \verb+\pm+ &
$\mp$ & \verb+\mp+ &
$\times$ & \verb+\times+ &
$\div$ & \verb+\div+ \\
$\ast$ & \verb+\ast+ &
$\star$ & \verb+\star+ &
$\circ$ & \verb+\circ+ &
$\bullet$ & \verb+\bullet+\\
 $\cap$ & \verb+\cap+ &
 $\cup$ & \verb+\cup+ &
 $\uplus$ & \verb+\uplus+ &
$\cdot$ & \verb+\cdot+ \\
 $\sqcap$ & \verb+\sqcap+  &
 $\sqcup$ & \verb+\sqcup+ &
 $\vee$ & \verb+\vee+ &
 $\wedge$ & \verb+\wedge+\\
   $\oplus$ & \verb+\oplus+ &
 $\ominus$ & \verb+\ominus+ &
 $\otimes$ & \verb+\otimes+ &
 $\oslash$ & \verb+\oslash+ \\
$\bigtriangleup$ & \verb+\bigtriangleup+ &
 $\odot$ & \verb+\odot+ &
  $\lhd$ & \verb+\lhd+ &
 $\dagger$ & \verb+\dagger+ \\
$\bigtriangledown$ & \verb+\bigtriangledown+ &
 $\bigcirc$ & \verb+\bigcirc+ &
  $\rhd$ & \verb+\rhd+ &
 $\ddagger$ & \verb+\ddagger+\\
$\triangleleft$ & \verb+\triangleleft+ &
 $\diamond$ & \verb+\diamond+ &
  $\unlhd$ & \verb+\unlhd+ &
 $\setminus$ & \verb+\setminus+\\
$\triangleright$ & \verb+\triangleright+ &
 $\wr$ & \verb+\wr+  &
 $\unrhd$ & \verb+\unrhd+  &
  $\amalg$ & \verb+\amalg+
\end{tabular}
\end{table}



\begin{table}
\caption{Relation symbols; used in math mode. }
\begin{tabular}%
  {c@{\hspace{\xxx}}lc@{\hspace{\xxx}}lc@{\hspace{\xxx}}lc@{\hspace{\xxx}}l}
$\leq$ & \verb+\leq+ &
 $\geq$ & \verb+\geq+ &
$\ll$ & \verb+\ll+ &
 $\gg$ & \verb+\gg+ \\
 $\equiv$ & \verb+\equiv+ &
 $\asymp$ & \verb+\asymp+ &
   $\neq$ & \verb+\neq+ &
   $\doteq$ & \verb+\doteq+  \\
$\subset$ & \verb+\subset+ &
 $\supset$ & \verb+\supset+ &
$\subseteq$ & \verb+\subseteq+ &
 $\supseteq$ & \verb+\supseteq+  \\
$\sqsubset$ & \verb+\sqsubset+ &
 $\sqsupset$ & \verb+\sqsupset+ &
$\sqsubseteq$ & \verb+\sqsubseteq+ &
 $\sqsupseteq$ & \verb+\sqsupseteq+\hidewidth \\
   $\models$ & \verb+\models+ &
   $\perp$ & \verb+\perp+  &
 $\mid$ & \verb+\mid+  &
   $\parallel$ & \verb+\parallel+ \\
$\prec$ & \verb+\prec+ &
 $\succ$ & \verb+\succ+ &
$\preceq$ & \verb+\preceq+ &
 $\succeq$ & \verb+\succeq+ \\
 $\sim$ & \verb+\sim+ &
   $\simeq$ & \verb+\simeq+ &
   $\approx$ & \verb+\approx+ &
   $\cong$ & \verb+\cong+ \\
 $\bowtie$ & \verb+\bowtie+  &
 $\Join$ & \verb+\Join+  &
 $\smile$ & \verb+\smile+ &
 $\frown$ & \verb+\frown+ \\
$\in$ & \verb+\in+ &
 $\ni$ & \verb+\ni+ &
$\vdash$ & \verb+\vdash+ &
 $\dashv$ & \verb+\dashv+  \\
 $\propto$ & \verb+\propto+ &
\end{tabular}
\end{table}

 Negated relations can sometimes
be constructed with \verb+\not+. For example,
\[
\hbox{\verb+If $x \not< y$ then $x \not\leq z$.+}
\]
gives
\[
\hbox{If $x \not< y$ then $x \not\leq z$.}
\]
The AMSFonts have many negated relations already constructed. See
Appendix \ref{AMSFonts,notations}.

%mtp

\begin{table}
\caption{Arrow symbols; used in math mode.}
\begin{tabular}{c@{\hspace{\xxx}}lc@{\hspace{\xxx}}l}
$\leftarrow$ & \verb+\leftarrow+ &
$\rightarrow$ & \verb+\rightarrow+ \\
 $\longleftarrow$ & \verb+\longleftarrow+ &
 $\longrightarrow$ & \verb+\longrightarrow+ \\
$\Leftarrow$ & \verb+\Leftarrow+ &
$\Rightarrow$ & \verb+\Rightarrow+ \\
 $\Longleftarrow$ & \verb+\Longleftarrow+ &
 $\Longrightarrow$ & \verb+\Longrightarrow+ \\
$\hookleftarrow$ & \verb+\hookleftarrow+ & $\hookrightarrow$
  & \verb+\hookrightarrow+ \\
$\leftharpoonup$ & \verb+\leftharpoonup+ & $\rightharpoonup$
  & \verb+\rightharpoonup+\\
$\leftharpoondown$ & \verb+\leftharpoondown+ & $\rightharpoondown$
  & \verb+\rightharpoondown+ \\
$\rightleftharpoons$&\verb+\rightleftharpoons+&$\leadsto$&\verb+\leadsto+ \\
$\leftrightarrow$ & \verb+\leftrightarrow+ & $\longleftrightarrow$
  & \verb+\longleftrightarrow+  \\
$\Leftrightarrow$ & \verb+\Leftrightarrow+ & $\Longleftrightarrow$
  & \verb+\Longleftrightarrow+  \\
$\mapsto$ & \verb+\mapsto+ & $\longmapsto$ & \verb+\longmapsto+ \\
\multicolumn{4}{c}{%
   \begin{tabular}{c@{\hspace{\xxx}}l}
   $\uparrow$ & \btt{uparrow}\\
   $\downarrow$ & \btt{downarrow} \\
   $\Uparrow$ & \btt{Uparrow}\\
   $\Downarrow$ & \btt{Downarrow}\\
   $\updownarrow$ & \btt{updownarrow}\\
   $\Updownarrow$ & \btt{Updownarrow}\\
   $\nearrow$ & \btt{nearrow}\\
   $\searrow$ & \btt{searrow}\\
   $\swarrow$ & \btt{swarrow}\\
   $\nwarrow$ & \btt{nwarrow}
   \end{tabular}%
} % end of multicolumn
%
\end{tabular}
\end{table}

\begin{table}
\caption{Miscellaneous symbols; used in math mode.}
\begin{tabular}%
  {c@{\hspace{\xxx}}lc@{\hspace{\xxx}}lc@{\hspace{\xxx}}lc@{\hspace{\xxx}}l}
 $\flat$ & \verb+\flat+ &
 $\natural$ & \verb+\natural+ &
 $\sharp$ & \verb+\sharp+ &
 $\prime$ & \verb+\prime+ \\
 $\backslash$ & \verb+\backslash+ &
 $\forall$ & \verb+\forall+ &
  $\infty$ & \verb+\infty+ &
 $\exists$ & \verb+\exists+\\
 $\emptyset$ & \verb+\emptyset+ &
  $\Box$ & \verb+\Box+ &
 $\nabla$ & \verb+\nabla+ &
 $\neg$ & \verb+\neg+ \\
  $\Diamond$ & \verb+\Diamond+ &
 $\surd$ & \verb+\surd+ &
  $\triangle$ & \verb+\triangle+&
 $\|$ & \verb+\|+ \\
  $\clubsuit$ & \verb+\clubsuit+ &
$\aleph$ & \verb+\aleph+ &
$\wp$ & \verb+\wp+ &
 $\top$ & \verb+\top+\\
  $\diamondsuit$ & \verb+\diamondsuit+&
$\Re$ & \verb+\Re+ &
$\ell$ & \verb+\ell+ &
 $\bot$ & \verb+\bot+ \\
  $\heartsuit$ & \verb+\heartsuit+ &
$\Im$ & \verb+\Im+ &
$\imath$ & \verb+\imath+ &
 $\partial$ & \verb+\partial+ \\
  $\spadesuit$ & \verb+\spadesuit+ &
$\hbar$ & \verb+\hbar+ &
$\jmath$ & \verb+\jmath+ &
 $\angle$ & \verb+\angle+ \\
$\mho$ & \verb+\mho+
\end{tabular}
\end{table}



\begin{table}
\caption{Log-like functions; used in math mode.}
\begin{tabular}{llllllll}
\verb+\arccos+ &
\verb+\arcsin+ &
\verb+\arctan+ &
\verb+\arg+ &
 \verb+\cos+ \\
 \verb+\cosh+ &
 \verb+\cot+ &
 \verb+\coth+ &
 \verb+\csc+ &
 \verb+\deg+ \\
 \verb+\det+ &
 \verb+\dim+ &
 \verb+\exp+ &
 \verb+\gcd+ &
 \verb+\hom+ \\
 \verb+\inf+ &
 \verb+\ker+ &
 \verb+\lg+ &
 \verb+\lim+ &
 \verb+\liminf+ \\
  \verb+\limsup+ &
  \verb+\ln+ &
  \verb+\log+ &
  \verb+\max+ &
 \verb+\min+ \\
 \verb+\Pr+ &
 \verb+\sec+ &
 \verb+\sin+ &
 \verb+\sinh+ &
 \verb+\sup+ \\
 \verb+\tan+ &
 \verb+\tanh+
\end{tabular}
\end{table}


\begin{table}
\caption{Delimiters; used in math mode.}
\begin{tabular}{c@{\hspace{\xxx}}lc@{\hspace{\xxx}}lc@{\hspace{\xxx}}l}
$($ & \verb+(+ &
 $)$ & \verb+)+ &
$/$ & \verb+/+ \\
$[$ & \verb+[+ &
 $]$ & \verb+]+ &
 $\backslash$ & \verb+\backslash+ \\
$\{$ & \verb+\{+ &
 $\}$ & \verb+\}+ &
$|$ & \verb+|+ \\
$\langle$ & \verb+\langle+ &
 $\rangle$ & \verb+\rangle+ &
 $\|$ & \verb+\|+ \\
 $\uparrow$ & \verb+\uparrow+ &
  $\Uparrow$ & \verb+\Uparrow+&
$\lfloor$ & \verb+\lfloor+ \\
 $\downarrow$ & \verb+\downarrow+ &
  $\Downarrow$ & \verb+\Downarrow+ &
 $\rfloor$ & \verb+\rfloor+ \\
 $\updownarrow$ & \verb+\updownarrow+ &
  $\Updownarrow$ & \verb+\Updownarrow+ &
$\lceil$ & \verb+\lceil+ \\
 & & & & $\rceil$ & \verb+\rceil+
\end{tabular}
\end{table}




\begin{table}
\caption{Miscellaneous symbols; used in math mode.}
\begin{tabular}{c@{\hspace{\xxx}}lc@{\hspace{\xxx}}lc@{\hspace{\xxx}}l}
$\textstyle\sum$ $\displaystyle\sum$ & \verb+\sum+ &
$\textstyle\prod$ $\displaystyle\prod$ & \verb+\prod+ &
$\textstyle\coprod$ $\displaystyle\coprod$ & \verb+\coprod+ \\
$\textstyle\int$ $\displaystyle\int$ & \verb+\int+ &
$\textstyle\oint$ $\displaystyle\oint$ & \verb+\oint+ &
$\textstyle\biguplus$ $\displaystyle\biguplus$ & \verb+\biguplus+ \\
  $\textstyle\bigcap$ $\displaystyle\bigcap$ & \verb+\bigcap+ &
  $\textstyle\bigcup$ $\displaystyle\bigcup$ & \verb+\bigcup+ &
  $\textstyle\bigsqcup$ $\displaystyle\bigsqcup$ & \verb+\bigsqcup+ \\
  $\textstyle\bigodot$ $\displaystyle\bigodot$ & \verb+\bigodot+ &
  $\textstyle\bigotimes$ $\displaystyle\bigotimes$ & \verb+\bigotimes+ &
  $\textstyle\bigoplus$ $\displaystyle\bigoplus$ & \verb+\bigoplus+ \\
  $\textstyle\bigvee$ $\displaystyle\bigvee$ & \verb+\bigvee+ &
  $\textstyle\bigwedge$ $\displaystyle\bigwedge$ & \verb+\bigwedge+
\end{tabular}
\end{table}

\subsubsection[Standard \protect\LaTeX{} typefaces]{Standard LaTeX typefaces}

You can access a pair of special typefaces in \LaTeX.

% only math mode?
You can switch to script (calligraphic) letters by using the \verb+\cal+
command (note the $\cal L$):
\begin{verbatim}
{\cal L}_{\text{int}} = e F^{3}_{\pi} r^{2}
  B^{0}(r,t) \epsilon \sin(\Omega t)
  \exp(\eta t),
\end{verbatim}
gives
\[
{\cal L}_{\text{int}} = eF^{3}_{\pi} r^{2}
 B^{0}(r,t)\epsilon\sin(\Omega t)\exp(\eta t),
\]
Only uppercase letters are available in the \verb+\cal+ font.

\penalty-10000

% only math mode?
You can switch to sans serif letters by using the \verb+\sf+
command (note the $\sf M$):
\begin{verbatim}
R({\cal Q}-{\cal Q}_{0})
=R_{0} \exp\left(-\case1/2\Delta {\cal Q}
 \cdot{\sf M}\cdot\Delta{\cal Q}\right).
  \label{eq:rdef}
\end{verbatim}
gives
\[
  R({\cal Q}-{\cal Q}_{0}) =
  R_{0} \exp\left(-\case1/2\Delta {\cal Q} \cdot {\sf M}
  \cdot \Delta {\cal Q}\right).
\]
Both uppercase and lowercase letters are available with \verb+\sf+.


\subsubsection{Other notations}

% math mode only?
The \verb+\overline+ command puts a horizontal line above its argument
in math mode:
\begin{verbatim}
$\overline{x}+\overline{y}$
\end{verbatim}
gives
\[
\overline{x}+\overline{y}
\]

There is an analogous \verb+\underline+ command that works in text
or math mode:
\begin{verbatim}
The equation \underline{is} $\underline{x+y}$.
\end{verbatim}
gives
\[
\hbox{The equation \underline{is} $\underline{x+y}$.}
\]

% math mode?
Horizontal braces are put above or below an expression with the
\verb+\overbrace+ and \verb+\underbrace+ commands:
\begin{verbatim}
$\underbrace{a_{1} + \overbrace{a_{2}+a_{3}}
  + a_{4}}$
\end{verbatim}
gives
\[
\underbrace{a_{1} + \overbrace{a_{2}+a_{3}} + a_{4}}
\]
and in displayed math, a subscript or a superscript puts a label on
the brace:
\begin{verbatim}
$\underbrace{a_{1} +
  \overbrace{a_{2}+\cdots+a_{n-1}}^{n-2}
   + a_{n}}_{n}$
\end{verbatim}
gives
\[
\underbrace{a_{1} + \overbrace{a_{2}+\cdots+a_{n-1}}^{n-2} + a_{n}}_{n}
\]

Wide versions of the \verb+\hat+ and \verb+\tilde+ commands are available.
They are called \verb+\widehat+ and \verb+\widetilde+, respectively.
Here is an example:
\begin{verbatim}
$\widehat{a} + \widehat{ab}
 + \widehat{abc} + \widehat{abcd}$
\end{verbatim}
gives
\[
\widehat{a} + \widehat{ab} + \widehat{abc} + \widehat{abcd}
\]


\subsection{AMSFonts notations}
\label{AMSFonts,notations}


The AMSFonts are fonts that were developed by the American Mathematical Society
and are now made available free of charge by the AMS. The METAFONT source
files for these
fonts are freely available, as are precompiled .pk files. There are two style
options that can be used to access the AMSFonts:
\verb+amsfonts+ and \verb+amssymb+. These style options are explained in
The appropriate society-specific documentation. Not distributed
with \REVTeX{} are the files amsfonts.sty and amssymb.sty of the
\AmSLaTeX{} distribution.
These files are called in by \REVTeX{}, when the NFSS is in effect,
to give you access
to the AMSFonts; under the OFSS \REVTeX{} itself will do the work necessary
to allow access.


% \subsubsection{Getting the AMSFonts}
% \label{sec:AMSFonts,getting}
%
% This information on getting the AMSFonts was put together from Internet
% posts by Rafal Zbikowski and George D. Greenwade. They are responsible
% for any correct information herein.
%
% The original distributing site for AMSFonts Version 2.1 (released
% in August 1991) is
% \begin{verbatim}
% e-math.ams.com       130.44.1.100      /ams
% \end{verbatim}
% available via ftp. Version 2.1 is incompatible with earlier
% versions, so upgrades are strongly recommended.
% Users having Unix-compatible compress/uncompress and tar/untar
% utilities (such exist for DOS and VMS) can get the following
% (binary) files:
% \begin{verbatim}
%    637421  Oct 28  1991 amsfonts-sources.tar.Z
%    4915200 Sep 27  1991 amsfonts300.tar
%      78823 Jul  2  1991 tfm-files.tar.Z
%    2447360 Sep 27  1991 amsfonts118.tar
%    3235840 Sep 27  1991 amsfonts180.tar
%    3788800 Sep 27  1991 amsfonts240.tar
%    4915200 Sep 27  1991 amsfonts300.tar
%    6512640 Sep 30  1991 amsfonts400.tar
% \end{verbatim}
% from the \verb+/ams+ directory, which covers the whole distribution together
% with documentation printable with plain \TeX. The files
% \verb+amsfonts$$$.tar+ (where \verb+$$$+ is 118 or 180 or 240 or 300 or 400)
% contain .pk files (packed generic font files for AMSFonts),
% the number \verb+$$$+ indicating the required printer/previewer resolution
% in dots per inch (dpi).
% Note that \verb+amsfonts$$$.tar+ are {\em not\/} compressed using Unix's
% compress facility.
%
% Users not having the Unix-compatible utilities will have to pull
% the files from subdirectories
% \begin{verbatim}
%   /ams/amsfonts
%   /ams/amsfonts/doc
%   /ams/amsfonts/pk-files
%   /ams/amsfonts/sources
%   /ams/amsfonts/sources/cyrillic
%   /ams/amsfonts/sources/euler
%   /ams/amsfonts/sources/extracm
%   /ams/amsfonts/sources/symbols
%   /ams/tfm-files
% \end{verbatim}
% Subdirectory \verb+/ams/amsfonts/pk-files+ contains .pk files (compressed
% generic font files for AMSFonts) organized in directories
% according to the required printer/previewer resolution, i.e.
% \begin{verbatim}
%   /ams/amsfonts/pk-files/118dpi
%   /ams/amsfonts/pk-files/180dpi
%   /ams/amsfonts/pk-files/240dpi
%   /ams/amsfonts/pk-files/300dpi
%   /ams/amsfonts/pk-files/400dpi
% \end{verbatim}
% where dpi is dots per inch.
%
% The files of the AMS distribution are rather big, even in the compressed
% form (as seen from the above listings). It is recommended to pull
% only the relevant files (especially the .pk ones). For example,
% files necessary for a 300dpi installation (\verb+amsfonts-sources.tar.Z+,
% \verb+amsfonts300.tar+, tfm-files.tar.Z) occupy around eight megabytes in
% uncompressed form (untarred \verb+amsfonts300.tar+ occupies 4533
% kilobytes).
%
% Note that \verb+amsfonts$$$.tar+ are {\em not\/} compressed using the Unix
% compress facility.
%
%
% The \AmSLaTeX files \verb+amsfonts.sty+ and \verb+amssymb.sty+,
% can also be obtained via anonymous ftp, from
% the directory \verb+/ams/amslatex/inputs+.
%
% The AMSFonts can be obtained by retrieving the font packages from
% Sam Houston State University. Send a mail message ({\em not interactive\/})
% with the lines
% \begin{verbatim}
%   SENDME AMSFONTS
%   SENDME AMSFONTS_DOC
%   SENDME AMSFONTS_EULER
%   SENDME AMSFONTS_EXTRACM
%   SENDME AMSFONTS_SYMBOLS
%   SENDME AMSFONTS_SOURCES
% \end{verbatim}
% to \verb+fileserv@shsu.edu+ (Internet) or \verb+fileserv@shsu+ (Bitnet).
% The server there will queue your request and send you the needed files
% automatically. Be forewarned that you can only retrieve the METAFONT sources
% from the server, and these packages can be large. You will also need
% to have a functioning copy of METAFONT and know how to use it. Here
% is a list of the packages, the number of files in each, and the approximate
% size of the package.
% \begin{quasitable}
% \begin{tabular}{lll}
% Package & No.\ files & Size \\
% \hline
% AMSFONTS & 5 & 37.5 kB \\
% AMSFONTS\_DOC & 4 & 87 kB \\
% AMSFONTS\_EULER & 68 & 1.2 MB \\
% AMSFONTS\_EXTRACM & 15 & 95.5 kB \\
% AMSFONTS\_SYMBOLS & 19 & 285 kB \\
% AMSFONTS\_SOURCES & 3 & 17.5 kB
% \end{tabular}
% \end{quasitable}
% You may wish to split up your requests to simplify things, sending only
% a single request at a time.
%
%
% You can also obtain \verb+amsfonts.sty+ and \verb+amssymb.sty+ by
% sending a mail message containing the lines
% \begin{verbatim}
% SENDME AMSLaTeX_INPUTS.AMSFONTS_STY
% SENDME AMSLaTeX_INPUTS.AMSSYMB_STY
% \end{verbatim}
% to \verb+fileserv@shsu.edu+ (Internet) or \verb+fileserv@shsu+ (Bitnet).
%
%


\subsubsection{Using the {\protect\tt amsfonts} option}
\label{sec:AMSFonts,amsfonts}

The \verb+amsfonts+ style option will give you
access to the \verb+\frak+ and \verb+\Bbb+ fonts and will also
use the extra Computer Modern fonts from the AMS in order to provide
better access to bold math characters at smaller sizes and in
super- and subscripts.

{\em AMSFonts typefaces.}
With the AMSFonts installed and in use through either the \verb+amsfonts+ or
\verb+amssymb+ style option, the \verb+\frak+ and \verb+\Bbb+ commands
are available. \verb+\frak+ switches to the AMS Fraktur font, while
\verb+\Bbb+ switches to the so-called ``Blackboard Bold'' font.
Only uppercase letters are available in Blackboard Bold, and there is
no bold version of the font. Fraktur has both uppercase and lowercase letters
and will become bold in a bbox.

\makeatletter

\def\foobar{\let\foo\iftrue}\let\foo\iffalse
\if@amssymbols\foobar\else\if@amsfonts\foobar\fi\fi

\foo
\makeatother

Here are the letters ``ABCDE'' from \verb+\frak+: $\frak ABCDE$.
And here are the letters ``RIZN'' from \verb+\Bbb+: $\Bbb RIZN$.

Here is some math with superscripts and \verb+\frak+. It demonstrates
the output of \verb+\bbox{#1}+.
\[\text{Normal: } {\frak E}=mc^{2\pi},
  \text{\ \ bbox: } \bbox{{\frak E}=mc^{2\pi}}
\]

\else
\makeatother
\bigskip
\begin{center}
\bf You do not have the amsfonts or amssymb option
selected, therefore the characters in the AMSFonts will not be
printed using \verb+\frak+ and \verb+\Bbb+.
\end{center}
\fi


\subsubsection{Using the {\protect\tt amssymb} option}
\label{sec:AMSFonts,symb}

The \verb+amssymb+ style option gives all the font capabilities of the
\verb+amsfonts+ option. It also defines names for many extra symbols that
are present in the AMSFonts. The names are the same as those the AMS uses.
These symbols and their names are shown below, if you have the AMSFonts
installed and the \verb+amssymb+ option selected.

\makeatletter
\if@amssymbols
\makeatother

% some of these characters are very high
\renewcommand{\arraystretch}{1.2}

Please be aware that no bold versions are available for any of the characters
in this subsection.


% Lowercase Greek
\begin{table}
\caption{Extra lowercase Greek letters available with amssymb option selected.}
\begin{tabular}{c@{\hspace{\xxx}}lc@{\hspace{\xxx}}l}
$\digamma$ & \verb+\digamma+ &
  $\varkappa$ & \verb+\varkappa+
\end{tabular}
\end{table}

% Hebrew letters
\begin{table}
\caption{Extra Hebrew letters available with amssymb option selected.}
\begin{tabular}{c@{\hspace{\xxx}}lc@{\hspace{\xxx}}l}
$\beth$ & \verb+\beth+ &
  $\gimel$ & \verb+\gimel+ \\
$\daleth$ & \verb+\daleth+
\end{tabular}
\end{table}


% Bin Rel
\begin{table}
\caption{Binary relations available with amssymb option selected.}
\begin{tabular}{c@{\hspace{\xxx}}lc@{\hspace{\xxx}}l}
$\leqq$ & \verb+\leqq+ &
  $\geqq$ & \verb+\geqq+ \\
$\leqslant$ & \verb+\leqslant+ &
  $\geqslant$ & \verb+\geqslant+ \\
$\eqslantless$ & \verb+\eqslantless+ &
  $\eqslantgtr$ & \verb+\eqslantgtr+ \\
$\lesssim$ & \verb+\lesssim+ &
  $\gtrsim$ & \verb+\gtrsim+ \\
$\lessapprox$ & \verb+\lessapprox+ &
  $\gtrapprox$ & \verb+\gtrapprox+ \\
$\approxeq$ & \verb+\approxeq+ \\
$\lessdot$ & \verb+\lessdot+ &
  $\gtrdot$ & \verb+\gtrdot+ \\
$\lll$ & \verb+\lll,\llless+ &
  $\ggg$ & \verb+\ggg,\gggtr+ \\
$\lessgtr$ & \verb+\lessgtr+ &
  $\gtrless$ & \verb+\gtrless+ \\
$\lesseqgtr$ & \verb+\lesseqgtr+ &
  $\gtreqless$ & \verb+\gtreqless+ \\[4pt]
$\lesseqqgtr$ & \verb+\lesseqqgtr+ &
  $\gtreqqless$ & \verb+\gtreqqless+ \\
$\preccurlyeq$ & \verb+\preccurlyeq+ &
  $\succcurlyeq$ & \verb+\succcurlyeq+ \\
$\curlyeqprec$ & \verb+\curlyeqprec+ &
  $\curlyeqsucc$ & \verb+\curlyeqsucc+ \\
$\precsim$ & \verb+\precsim+ &
  $\succsim$ & \verb+\succsim+ \\
$\precapprox$ & \verb+\precapprox+ &
  $\succapprox$ & \verb+\succapprox+ \\
$\subseteqq$ & \verb+\subseteqq+ &
  $\supseteqq$ & \verb+\supseteqq+ \\
$\Subset$ & \verb+\Subset+ &
  $\Supset$ & \verb+\Supset+ \\
$\sqsubset$ & \verb+\sqsubset+ &
  $\sqsupset$ & \verb+\sqsupset+ \\
$\backsim$ & \verb+\backsim+ &
  $\thicksim$ & \verb+\thicksim+ \\
$\backsimeq$ & \verb+\backsimeq+ &
  $\thickapprox$ & \verb+\thickapprox+ \\
$\doteqdot$ & \verb+\doteqdot,\Doteq+ &
  $\eqcirc$ & \verb+\eqcirc+ \\
$\risingdotseq$ & \verb+\risingdotseq+ &
  $\circeq$ & \verb+\circeq+ \\
$\fallingdotseq$ & \verb+\fallingdotseq+ &
  $\triangleq$ & \verb+\triangleq+ \\
$\vartriangleleft$ & \verb+\vartriangleleft+ &
  $\vartriangleright$ & \verb+\vartriangleright+ \\
$\trianglelefteq$ & \verb+\trianglelefteq+ &
  $\trianglerighteq$ & \verb+\trianglerighteq+ \\
$\vDash$ & \verb+\vDash+ &
  $\Vdash$ & \verb+\Vdash+ \\
$\Vvdash$ & \verb+\Vvdash+ \\
$\smallsmile$ & \verb+\smallsmile+ &
$\smallfrown$ & \verb+\smallfrown+ \\
  $\shortmid$ & \verb+\shortmid+ &
  $\shortparallel$ & \verb+\shortparallel+ \\
$\bumpeq$ & \verb+\bumpeq+ &
$\Bumpeq$ & \verb+\Bumpeq+ \\
  $\between$ & \verb+\between+ &
  $\pitchfork$ & \verb+\pitchfork+
\end{tabular}
\end{table}

\begin{table}
\caption{Miscellaneous symbols available with amssymb option selected.}
\begin{tabular}{c@{\hspace{\xxx}}lc@{\hspace{\xxx}}l}
$\hbar $ & \verb+\hbar+ &
$\hslash$ & \verb+\hslash+ \\
  $\backprime $ & \verb+\backprime+ &
  $\varnothing$ & \verb+\varnothing+ \\
$\vartriangle$ & \verb+\vartriangle+ &
  $\blacktriangle$ & \verb+\blacktriangle+ \\
$\triangledown$ & \verb+\triangledown+ &
  $\blacktriangledown$ & \verb+\blacktriangledown+ \\
$\square$ & \verb+\square+ &
  $\blacksquare$ & \verb+\blacksquare+ \\
$\lozenge$ & \verb+\lozenge+ &
  $\blacklozenge$ & \verb+\blacklozenge+ \\
$\circledS$ & \verb+\circledS+ &
  $\bigstar$ & \verb+\bigstar+ \\
$\angle $ & \verb+\angle+ &
  $\sphericalangle$ & \verb+\sphericalangle+ \\
$\measuredangle$ & \verb+\measuredangle+ \\
$\nexists$ & \verb+\nexists+ &
  $\complement$ & \verb+\complement+ \\
$\mho$ & \verb+\mho+ &
  $\eth$ & \verb+\eth+ \\
$\Finv$ & \verb+\Finv+ &
$\Game$ & \verb+\Game+ \\
  $\diagup$ & \verb+\diagup+ &
  $\diagdown$ & \verb+\diagdown+ \\
$\Bbbk$ & \verb+\Bbbk+
\end{tabular}
\label{tab:a}
\end{table}


% Bin Op
\begin{table}
\caption{Binary operators available with amssymb option selected.}
\begin{tabular}{c@{\hspace{\xxx}}lc@{\hspace{\xxx}}l}
$\dotplus$ & \verb+\dotplus+ &
  $\ltimes$ & \verb+\ltimes+ \\
$\smallsetminus$ & \verb+\smallsetminus+ &
  $\rtimes$ & \verb+\rtimes+ \\
$\barwedge$ & \verb+\barwedge+ &
  $\curlywedge$ & \verb+\curlywedge+ \\
$\veebar$ & \verb+\veebar+ &
  $\curlyvee$ & \verb+\curlyvee+ \\
$\doublebarwedge$ & \verb+\doublebarwedge+ \\
$\Cap$ &   \verb+\Cap,\doublecap+ &
  $\leftthreetimes$ & \verb+\leftthreetimes+  \\
$\Cup$ &   \verb+\Cup,\doublecup+ &
  $\rightthreetimes$ & \verb+\rightthreetimes+ \\
$\boxtimes$ & \verb+\boxtimes+ &
  $\circledast$ & \verb+\circledast+ \\
$\boxminus$ & \verb+\boxminus+ &
  $\circleddash$ & \verb+\circleddash+ \\
$\boxplus$ & \verb+\boxplus+ &
  $\centerdot$ & \verb+\centerdot+ \\
$\boxdot$ & \verb+\boxdot+ &
  $\circledcirc$ & \verb+\circledcirc+ \\
$\divideontimes$ & \verb+\divideontimes+ &
  $\intercal$ & \verb+\intercal+
\end{tabular}
\end{table}


% other junk
\begin{table}
\caption{Other miscellaneous symbols available with amssymb option selected.}
\begin{tabular}{c@{\hspace{\xxx}}lc@{\hspace{\xxx}}l}
$\varpropto$ & \verb+\varpropto+ &
  $\backepsilon$ & \verb+\backepsilon+ \\
$\blacktriangleleft$ & \verb+\blacktriangleleft+ &
  $\blacktriangleright$ & \verb+\blacktriangleright+ \\
$\therefore$ & \verb+\therefore+ &
  $\because$ & \verb+\because+
\end{tabular}
\end{table}

% negated relations
\begin{table}
\caption{Negated relations available with amssymb option selected.}
\begin{tabular}{c@{\hspace{\xxx}}lc@{\hspace{\xxx}}l}
$\nsim$ & \verb+\nsim+ &
  $\ncong$ & \verb+\ncong+ \\
$\nless$ & \verb+\nless+ &
  $\ngtr$ & \verb+\ngtr+ \\
$\nleq$ & \verb+\nleq+ &
  $\ngeq$ & \verb+\ngeq+ \\
$\nleqslant$ & \verb+\nleqslant+ &
  $\ngeqslant$ & \verb+\ngeqslant+ \\
$\nleqq$ & \verb+\nleqq+ &
  $\ngeqq$ & \verb+\ngeqq+ \\
$\lneq$ & \verb+\lneq+ &
  $\gneq$ & \verb+\gneq+ \\
$\lneqq$ & \verb+\lneqq+ &
  $\gneqq$ & \verb+\gneqq+ \\
$\lvertneqq$ & \verb+\lvertneqq+ &
  $\gvertneqq$ & \verb+\gvertneqq+ \\
$\lnsim$ & \verb+\lnsim+ &
  $\gnsim$ & \verb+\gnsim+ \\
$\lnapprox$ & \verb+\lnapprox+ &
  $\gnapprox$ & \verb+\gnapprox+ \\
$\nprec$ & \verb+\nprec+ &
  $\nsucc$ & \verb+\nsucc+ \\
$\npreceq$ & \verb+\npreceq+ &
  $\nsucceq$ & \verb+\nsucceq+ \\
$\precneqq$ & \verb+\precneqq+ &
  $\succneqq$ & \verb+\succneqq+ \\
$\precnsim$ & \verb+\precnsim+ &
  $\succnsim$ & \verb+\succnsim+ \\
$\precnapprox$ & \verb+\precnapprox+ &
  $\succnapprox$ & \verb+\succnapprox+ \\
$\ntriangleleft$ & \verb+\ntriangleleft+ &
  $\ntriangleright$ & \verb+\ntriangleright+ \\
$\ntrianglelefteq$ & \verb+\ntrianglelefteq+ &
  $\ntrianglerighteq$ & \verb+\ntrianglerighteq+ \\
$\nshortmid$ & \verb+\nshortmid+ &
$\nmid$ & \verb+\nmid+ \\
  $\nshortparallel$ & \verb+\nshortparallel+ &
  $\nparallel$ & \verb+\nparallel+ \\
$\nvdash$ & \verb+\nvdash+ &
  $\nvDash$ & \verb+\nvDash+ \\
$\nVdash$ & \verb+\nVdash+ &
  $\nVDash$ & \verb+\nVDash+ \\
$\nsubseteq$ & \verb+\nsubseteq+ &
  $\nsupseteq$ & \verb+\nsupseteq+ \\
$\nsubseteqq$ & \verb+\nsubseteqq+ &
  $\nsupseteqq$ & \verb+\nsupseteqq+ \\
$\varsubsetneq$ & \verb+\varsubsetneq+ &
  $\varsupsetneq$ & \verb+\varsupsetneq+ \\
$\subsetneq$ & \verb+\subsetneq+ &
  $\supsetneq$ & \verb+\supsetneq+ \\
$\varsubsetneqq$ & \verb+\varsubsetneqq+ &
  $\varsupsetneqq$ & \verb+\varsupsetneqq+ \\
$\subsetneqq$ & \verb+\subsetneqq+ &
  $\supsetneqq$ & \verb+\supsetneqq+
\end{tabular}
\end{table}

\begin{table}
\caption{Yet more
miscellaneous symbols available with amssymb option selected.}
\begin{tabular}{c@{\hspace{\xxx}}lc@{\hspace{\xxx}}l}
$\dashrightarrow$ & \verb+\dashrightarrow+ &
  $\dashleftarrow$ & \verb+\dashleftarrow+ \\
$\dasharrow$ & \verb+\dasharrow+ \\
  $\ulcorner$ & \verb+\ulcorner+ &
$\urcorner$ & \verb+\urcorner+ \\
  $\llcorner$ & \verb+\llcorner+ &
$\lrcorner$ & \verb+\lrcorner+ \\
  $\yen$ & \verb+\yen+ &
$\checkmark$ & \verb+\checkmark+ \\
  $\circledR$ & \verb+\circledR+ &
$\maltese$ & \verb+\maltese+
\end{tabular}
\end{table}

% Negated arrows
\begin{table}
\caption{Extra negated arrows available with amssymb option selected.}
\begin{tabular}{c@{\hspace{\xxx}}lc@{\hspace{\xxx}}l}
$\nleftrightarrow$ & \verb+\nleftrightarrow+ &
  $\nLeftrightarrow$ & \verb+\nLeftrightarrow+\\
$\nleftarrow$ & \verb+\nleftarrow+ &
  $\nrightarrow$ & \verb+\nrightarrow+ \\
$\nLeftarrow$ & \verb+\nLeftarrow+ &
  $\nRightarrow$ & \verb+\nRightarrow+ \\
\end{tabular}
\end{table}

%%  Arrows
\begin{table}
\caption{Extra arrows available with amssymb option selected.}
\begin{tabular}{c@{\hspace{\xxx}}lc@{\hspace{\xxx}}l}
$\leftrightarrows$ & \verb+\leftrightarrows+ &
  $\rightleftarrows$ & \verb+\rightleftarrows+ \\
$\leftleftarrows$ & \verb+\leftleftarrows+ &
  $\rightrightarrows$ & \verb+\rightrightarrows+ \\
$\leftrightharpoons$ & \verb+\leftrightharpoons+ &
  $\rightleftharpoons$ & \verb+\rightleftharpoons+ \\
$\Lleftarrow$ & \verb+\Lleftarrow+ &
  $\Rrightarrow$ & \verb+\Rrightarrow+ \\
$\twoheadleftarrow$ & \verb+\twoheadleftarrow+ &
  $\twoheadrightarrow$ & \verb+\twoheadrightarrow+ \\
$\leftarrowtail$ & \verb+\leftarrowtail+ &
  $\rightarrowtail$ & \verb+\rightarrowtail+ \\
$\looparrowleft$ & \verb+\looparrowleft+ &
  $\looparrowright$ & \verb+\looparrowright+ \\
$\Lsh$ & \verb+\Lsh+ &
  $\Rsh$ & \verb+\Rsh+ \\
$\upuparrows$ & \verb+\upuparrows+ &
  $\downdownarrows$ & \verb+\downdownarrows+ \\
$\upharpoonleft$ & \verb+\upharpoonleft+ &
  $\upharpoonright$ & \verb+\upharpoonright,+ \\
&&&\hskip1pc\verb+\restriction+ \\
$\downharpoonleft$ & \verb+\downharpoonleft+ &
  $\downharpoonright$ & \verb+\downharpoonright+ \\
$\curvearrowleft$ & \verb+\curvearrowleft+ &
  $\curvearrowright$ & \verb+\curvearrowright+ \\
$\circlearrowleft$ & \verb+\circlearrowleft+ &
  $\circlearrowright$ & \verb+\circlearrowright+ \\
$\multimap$ & \verb+\multimap+ &
  $\rightsquigarrow$ & \verb+\rightsquigarrow+ \\
$\leftrightsquigarrow$ & \verb+\leftrightsquigarrow+
\end{tabular}
\end{table}

% other stuff

%\widehat
%\widetilde
%\eqsim

\else
\makeatother
\bigskip
\begin{center}
\bf You do not have the amssymb option
selected, therefore the characters in the AMSFonts will not be
printed.
\end{center}
\fi


\subsection{\REVTeX{} notations}
\label{sec:revtexnotations}


An openface one is available. It does not change size in superscripts.
Here is an example: \verb+$\openone$+ gives $\openone$.
\verb+\openone+ is a fragile command and must be immediately preceded by
\verb+\protect+ when used in section headings and captions.

Bold large bracketing is also available. The normal commands
\verb+\Biggl+,\verb+\Bigl+,$\ldots$, when used with an extra ``b'' on the
end of the command, come out bold:
\begin{verbatim}
\[
\Biggl(\biggl(\Bigl(\bigl(
(x)
\bigr)\Bigr)\biggr)\Biggr)
\]
\end{verbatim}
gives
\[
\Biggl(\biggl(\Bigl(\bigl(
(x)
\bigr)\Bigr)\biggr)\Biggr)
\]
while
\begin{verbatim}
\[
\Bigglb(\bigglb(\Biglb(\biglb(
(x)
\bigrb)\Bigrb)\biggrb)\Biggrb)
\]
\end{verbatim}
gives
\[
\Bigglb(\bigglb(\Biglb(\biglb(
(x)
\bigrb)\Bigrb)\biggrb)\Biggrb)
\]

{\makeatletter\if@amssymbols\penalty-10000\fi}


The commands \verb+\lesssim+,\verb+\gtrsim+ give the output
$\lesssim,\gtrsim$, even without the \verb+amssymb+ style option.
(The commands \verb+\alt+,\verb+\agt+, respectively, may also be used.)
These commands will be fragile if you are not using the \verb+amssymb+ option.


Some extra diacritics have been provided. They scale correctly in
superscripts. Some examples follow.
\verb+$\tensor{x}$+ gives $\tensor{x}$.
\verb+$\overstar{x}$+ gives $\overstar{x}$.
\verb+$\overdots{x}$+ gives $\overdots{x}$.
\verb+$\overcirc{x}$+ gives $\overcirc{x}$.
\verb+$\loarrow{x}$+ gives $\loarrow{x}$.
\verb+$\roarrow{x}$+ gives $\roarrow{x}$.
These commands all work correctly in superscripts.

\verb+\slantfrac{#1}{#2}+ produces a slanted fraction in math mode:
        $\slantfrac{1}{2}$.
 This command should not be used in files destined to be submitted to
 the APS (normal upright fractions will be produced).

 \verb+\corresponds+ produces the symbol $\corresponds$ math mode,
 \verb+\precsim+ produces $\precsim$ in math mode, and
 \verb+\succsim+ produces $\succsim$ in math mode. The AMSFonts will be used
 for these symbols if you have them, but are not necessary.


 \verb+\lambdabar+ produces ``lambda-bar'' in math mode: $\lambdabar$.

\onecolumn


\makeatletter
\c@page1
\def\thepage{B-\@arabic\c@page}
\makeatother


\section{Command List}
\label{sec:commands}

In the following pages are brief descriptions of some
necessary commands.
Those commands that are unique to \REVTeX{} are so noted
with (R). Please consult the \LUG{} if you have further questions
regarding \LaTeX{} commands.

If commands require arguments, they are so noted
with \verb+#1+, \verb+#2+, etc.
The commands are in order of their probable occurrence in a file.

\bigskip\hrule\bigskip
\begin{quasitable}
\begin{tabular}{lp{4.5in}}
\verb+\documentstyle[#1]{revtex}+ &
              Will allow for proper formatting of paper. Selecting
              a society style option (either \verb+osa+, \verb+aps+
              or \verb+seg+) is mandatory.

              Use \verb+[manuscript,osa]+ for \verb+#1+ if
              manuscript style is desired for the OSA macros,
              and use \verb+[osa]+ for
              \verb+#1+ if galley style is desired for the OSA macros.

              Use \verb+[preprint,aps]+ for \verb+#1+ if
              preprint style is desired for the APS macros,
              and use \verb+[aps]+ for
              \verb+#1+ if galley style is desired for the APS macros.
              The preprint and manuscript styles are similar ideas.

              The author will also need to add a journal option.
              Consult the society-specific documentation for the
              relevant options.

              To number
              equations by section, use the \verb+eqsecnum+
               option.\\[4pt]


\verb+\tighten+  &
           Preprint style outputs a double-spaced manuscript.
           When used in preprint style, this command reverts to
           single spacing to save paper. Has no effect in galley
           style.  Use before \verb+\begin{document}+. You can
           also use the \verb+tighten+ style option to get
           single-spaced preprint output for the whole paper. (R)
             \\[4pt]

\verb+\begin{document}+  &
        Begins the body of the \REVTeX{} document.
                          \\[4pt]

\verb+\preprint{#1}+  &
           When used as the first command of a document, places
           \verb+#1+ at the top right corner of the first page in
           preprint style. Used for site-specific preprint
           numbers. (R)
             \\[4pt]

\verb+\draft+  &
                  Omission of this command will cause printing of PACS
                        numbers to be stifled. (R)
                          \\[4pt]

\verb+\title{#1}+  &
              \verb+#1+ is the title of the paper. The title should be
                        broken with the \verb+\\+ command.
                          \\[4pt]

\verb+\author{#1}+  &
             \verb+#1+ represents a list of authors. Use \verb+\\+ to force
                        linebreaks.
                          \\[4pt]

\verb+\address{#1}+  &
            \verb+#1+ represents an author's address (institution).  The
            address should be broken with \verb+\\+ if necessary. (R)
                          \\[4pt]

\verb+\date{#1}+  &
               \verb+#1+ represents the date of receipt at the Editorial
                        Offices. This date will be inserted at the production
                        site.
                          \\[4pt]

\verb+\maketitle+  &
              Prints the material contained in the \verb+\title{#1}+,
              \verb+\author{#1}+, \verb+\address{#1}+ and
              \verb+\date{#1}+ commands.
                \\[4pt]

$\vcenter{\hbox{\verb+\begin{abstract},+}
          \hbox{\verb+\end{abstract},+}}$
&
       Signals the beginning or end of the abstract.
                          \\[4pt]

\verb+\pacs{#1}+  &
   \verb+#1+ represents valid PACS numbers. This command should be
    used after the abstract, even if \verb+#1+ is empty.  Use the
            \verb+\draft+ command to have \verb+#1+ printed. (R)
              \\[4pt]

\verb+\narrowtext+  &
             For galley style, will set all text that follows into
            a 3.4-in.\ column. Does not affect preprint output. (R)
                          \\[4pt]

\verb+\mediumtext+  &
             For galley style, will set figure captions and tables
                        5.5-in.\ wide. Does not affect preprint output. (R)
                          \\[4pt]

\verb+\widetext+  &
               For galley style, will set all text that follows into
                 a 7-in.-wide column. Does not affect preprint output. (R)
                          \\[4pt]

\verb+\section{#1}+  &
            \verb+#1+ represents a primary heading. Fragile commands
                        should be preceded by \verb+\protect+.
                          \\[4pt]

\verb+\subsection{#1}+  &
         \verb+#1+ represents a secondary heading. Fragile commands
                        should be preceded by \verb+\protect+.
                          \\[4pt]

\verb+\subsubsection{#1}+  &
      \verb+#1+ represents a third-level heading.
      Fragile commands should be preceded by \verb+\protect+.
                          \\[4pt]

\verb+\paragraph{#1}+  &
          \verb+#1+ represents a fourth-level heading. Fragile commands
                        should be preceded by \verb+\protect+.
                          \\[4pt]

\verb+\cite{#1}+  &
        Sets a reference or byline footnote citation.  \verb+#1+
        represents a list of reference tags used with
        \verb+\bibitem{#1}+.
        Lists of consecutive numbers will be collapsed; e.g.,
        [1,2,3] will become [1--3]. The style of citation in
        output will depend on the society and/or journal
        option selected.
        Fragile.
                          \\[4pt]

\verb+\onlinecite{#1}+  &
         Sets a reference citation just
           like \verb+\cite{#1}+ does, except that it places the citation
                        on-line in styles where the citations are usually
                        superscripts.  Fragile. (R)
                          \\[4pt]

\verb+\case{#1}{#2}+  &
      Sets textstyle (smaller) fractions in displayed
    equations. \verb+#1+ is the numerator, \verb+#2+ is denominator. An
    optional \verb+/+ may be added between \verb+{#1}+ and \verb+{#2}+. (R)
                     \\[4pt]

\verb+\openone+  &
                Produces an openface one ($\openone$). Fragile. (R) \\[4pt]

\verb+\precsim,\succsim+  &
               Produce the signs $\precsim,\succsim$, respectively,
                        in math mode. \\[4pt]

\verb+\lesssim,\gtrsim+  &
               Produce ``approximately less than'' and ``approximately
               greater than'' signs \hbox{($\lesssim,\gtrsim$)},
               respectively, in math mode. Fragile. \\[4pt]

\verb+\tensor{#1}+ & Gives double-headed overarrow in math mode:
        \verb+$\tensor{x}$+ gives $\tensor{x}$. (R)\\
\verb+\loarrow{#1}+ & Gives left-going overarrow in math mode:
        \verb+$\loarrow{x}$+ gives $\loarrow{x}$. (R)\\
\verb+\roarrow{#1}+ & Gives right-going overarrow in math mode to match
        \verb+\loarrow{#1}+: \verb+$\roarrow{x}$+ gives $\roarrow{x}$. (R)\\
\verb+\overstar{#1}+ & Gives overstar in math mode:
     \verb+$\overstar{x}$+ gives $\overstar{x}$. (R)\\
\verb+\overcirc{#1}+ & Gives overcircle in math mode:
     \verb+$\overcirc{x}$+ gives $\overcirc{x}$. (R)\\[4pt]

\verb+\biglb(+, etc.& Commands to produce large bold bracketing. (R)\\[4pt]

\verb+\corresponds+ & Produces ``corresponds'' sign in math mode:
  $\corresponds$.                       \\[4pt]

\verb+\slantfrac{#1}{#2}+ & Produces a slanted fraction in math mode:
        $\slantfrac{1}{2}$. Should not be used for APS files. (R) \\[4pt]

\verb+\lambdabar+ & Produces ``lambda-bar'' in math mode: $\lambdabar$. (R)
                       \\[4pt]

\verb+\FL+  &
                     Sets the displayed equation that follows flush left
                        with the margin. Only works in galley style. (R)
                          \\[4pt]

\verb+\FR+  &
                     Sets the displayed equation that follows flush right.
                        Only works in galley style. (R)
                          \\[4pt]

\verb+\[,\]+  &
                   Signals beginning or end of unnumbered displayed
                        equation.
                          \\[4pt]

\verb+\begin{equation},\end{equation}+  &
       Signals beginning or end of single-line displayed equation.
                          \\[4pt]

\verb+\begin{eqnarray},\end{eqnarray}+  &
       Signals beginning or end of multiline displayed equation.
                          \\[4pt]

\verb+\nonumber+  &
               Suppresses the numbering of a single line in a
                        eqnarray environment.
                          \\[4pt]

\verb+\eqnum{#1}+ & Uses \verb+#1+ as the number for an equation or for
  a single line of an eqnarray. The number can be cross-referenced with
  \verb+\ref{#1}+ if \verb+\label{#1}+ is used right after \verb+\eqnum{#1}+.
  Numbers set with \verb+\eqnum{#1}+ are completely independent of the
  automatic numbering. (R) \\[4pt]

\verb+\begin{quasitable},\end{quasitable}+  &
     Environment to produce tables in text. See apssamp.tex for an example.
                       (R)   \\[4pt]

\verb+\label{#1}+  &
     \verb+#1+ represents the tag. This command appears in
               displayed equations that need cross-referencing, all
               tables, and all figure captions. Also used following
               section headings that need cross-referencing.
                 \\[4pt]

\verb+\ref{#1}+  &
                \verb+#1+ represents the tag. This command appears in text
                        wherever sections, equations, tables, or figures are
                        cited. Fragile.
                          \\[4pt]

\verb+\acknowledgments+  &
        Sets a section heading for the acknowledgment section.
                          \\[4pt]

\verb+\appendix+  &
               After using this command, all \verb+\section{#1}+ commands
      will set \verb+#1+ as an appendix heading.  \verb+\section*{#1}+ will
      set \verb+#1+ as an appendix heading without a letter
      (A, B, etc.) and should be used when there is only one
      appendix.
        \\[4pt]

\verb+\begin{references},\end{references}+
        & Signals beginning or end of reference section. The
         normal \LaTeX{} \verb+thebibliography+ environment can also
                        be used. (R)
                          \\[4pt]

\verb+\bibitem[#1]{#2}+  &
        Sets a reference in the reference section.  \verb+#1+
                        represents an optional, author-specified reference
                        symbol. This is used for byline endnotes that are
                        not numbered (e.g., those in {\em Physical Review\/}).
                        \verb+#2+ represents the reference tag.  \\[4pt]

\verb+\begin{figure}+  &
          Begins the environment for the figure caption.\\[4pt]
\verb+\caption{#1}+    &
        \verb+#1+ represents the text of the caption.
      Fragile commands must be preceded by \verb+\protect+.
        \\[4pt]
\verb+\label{#2}+      &
        \verb+#2+ represents the figure caption tag.\\[4pt]
\verb+\end{figure}+    &
        Ends the environment for the figure caption.
                          \\[4pt]

\verb+\begin{table}+  &
           Signals the beginning of a table.
                          \\[4pt]

\verb+\squeezetable+  &
           Used immediately after \verb+\begin{table}+, shrinks tables
                        that would not otherwise fit. (R)
                          \\[4pt]

\verb+\caption{#1}+  &
            Sets the table caption. \verb+#1+ represents the
                        text of the caption.
      Fragile commands must be preceded by \verb+\protect+.
        \\[4pt]

\verb+\begin{tabular}{#1}+  &
     Signals the beginning of the tabular material.  \verb+#1+
                        represents formatting commands for the columns.
                          \\[4pt]

\verb+\hline+  &
                  Sets a horizontal rule, separating column headings
                        from data. \verb+\tableline+ may also be used.
                          \\[4pt]

\verb+\end{tabular}+  &
           Signals end of tabular material.
                          \\[4pt]

\verb+\end{table}+  &
             Signals the end of a table.
                          \\[4pt]

\verb+\end{document}+  &
          Ends the body of the \REVTeX{} document.
                          \\[4pt]
\end{tabular}
\end{quasitable}


\smallskip\hrule

\end{document}
% end of file manend.tex
