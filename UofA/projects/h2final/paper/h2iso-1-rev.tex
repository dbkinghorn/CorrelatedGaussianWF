% H2 isotopomer paper from pcgr1 calculations
% Donald B. Kinghorn
% University of Arizona
% Wed Feb 23 16:01:50 MST 2000

% using revtex 3.1 template file

\documentstyle[aps]{revtex}
\linespread{1.6}

\begin{document}

% \draft command makes pacs numbers print
\draft

\title{High Accuracy Non-Born--Oppenheimer Calculations for the
        Isotopomers of the Hydrogen Molecule  }
\author{Donald B. Kinghorn and Ludwik Adamowicz}
\address{ Department of Chemistry\\
        University of Arizona\\
        Tucson AZ 85721}
\date{\today}
\maketitle

\begin{abstract}
  The first rigorous variational results for the non-adiabatic,
  (Non-Born--Oppenheimer), ground state of the six isotopomers of the
  hydrogen molecule are reported.  Ground state energy results are;
  (Using hartree energy units: H$_2$[-1.1640250232],
  D$_2$[-1.16716878], T$_2$[-1.16853565], HD[-1.165471906],
  HT[-1.166002033], DT[-1.167819642]).  Expectation values for kinetic
  and potential energy, inter-nuclear distance and the square of the
  inter-nuclear distance, the virial coefficient, and square of the
  energy gradient norm are also reported.  The calculations were
  performed with a direct non-adiabatic variational approach using a
  new correlated {G}aussian basis set including powers of the the
  inter-nuclear distance.
\end{abstract}

% insert suggested PACS numbers in braces on next line
\pacs{31.15.Pf, 31.25.Nj, 31.30.-i}

%\narrowtext

%\section{Introduction}
We recently published an improved non-adiabatic energy upper-bound for
the ground state of H$_2$ utilizing a new correlated Gaussian basis
set which includes powers of the inter-nuclear
distance\cite{Kinghorn99b}.  This result established a new benchmark
for non-adiabatic molecular calculations. This letter is a
continuation of that work and reports new energy upper-bounds for the
six isotopic species formed from Hydrogen, Deuterium, and Tritium;
H$_2$, D$_2$, T$_2$, HD, HT and DT. These new bounds should provide
accurate reference energies for future non-adiabatic calculations and
for evaluating the quality of adiabatic and non-adiabatic
``corrections'' determined by methods utilizing the Born-Oppenheimer
approximation as the zero order solution.  A detailed description of
this new N-body correlated basis set including analytic formulas for
the Hamiltonian matrix elements and the energy gradient components
together with information on implementation can be found in the
references\cite{Kinghorn99a}.  The calculations in this work utilized
a parallel implementation of these formulas using MPI and a cluster of
Linux workstations.

%\section{Non-adiabatic Hamiltonian}
In the non-adiabatic non-relativistic approach all particles are
treated equally utilizing their given masses and Coulombic
interactions with all other particles.  Without invoking any
approximations, the total Hamiltonian can be separated into an
operator representing the translational motion of the center of mass
and an operator representing the internal energy.  To demonstrate this
for the four body case we perform the separation by making a
transformation to an internal reference frame with origin at particle
one,
\begin{equation} \label{rdef}
\mathbf{r}=\left[
  \begin{array}[c]{c}
    \mathbf{r}_{1}\\
    \mathbf{r}_{2}\\
    \mathbf{r}_{3}
  \end{array}
  \right]  
  = \left[
    \begin{array}[c]{c}
      \mathbf{R}_{2}-\mathbf{R}_{1}\\
      \mathbf{R}_{3}-\mathbf{R}_{1}\\
      \mathbf{R}_{4}-\mathbf{R}_{1}\\
    \end{array}
  \right],
\end{equation}
where the $\mathbf{R}_{i}$ are the original particle coordinates.
This transformation to  internal coordinates together with the 
conjugate momentum transformation yields the non-adiabatic Hamiltonian
for the internal energy of a four particle system,
\begin{equation} \label{intham1}
H=-\frac{1}{2}\left(  \sum_{i}^{3}\frac{1}{\mu_{i}}\nabla_{i}^{2}
  +\sum_{i\neq j}^{3}\frac{1}{M_{1}}\nabla_{i}^{\prime}\nabla_{j} \right)
  +\sum_{i=1}^{3}\frac{q_{0}q_{i}}{r_{i}}  
  +\sum_{i<j}^{3}\frac{q_{i}q_{j}}{r_{ij}},
\end{equation}
here the $\mu_{i}$ are reduced masses, $M_{1}$ is the mass of particle 1,
(the coordinate reference particle), and $\nabla_{i}$ is the gradient with
respect to the $x,y,z$ coordinates $\mathbf{r}_{i}$. The potential energy is
the same as in the total Hamiltonian, but is now written using internal
distance coordinates. The charges are mapped from the original particles as
$\{Q_{1},Q_{2},Q_{3},Q_{4}\}\mapsto\{q_{0},q_{1},q_{2},q_{3}\}$. 
In internal coordinates, distances are denoted,(using the standard 2-norm), 
$r_{ij}=\left\|  \mathbf{r}_{i}-\mathbf{r}_{j}\right\|  =$ 
$\left\|  \mathbf{R}_{i+1}-\mathbf{R}_{j+1}\right\|$ with 
$r_{j}=\left\|  \mathbf{r}_{j}\right\|  =
\left\|  \mathbf{R}_{j+1}-\mathbf{R}_{1}\right\|$.
More general information on the N-body non-adiabatic Hamiltonian and the 
center of mass transformation can be found in the references\cite{Kinghorn99a}.

%\section{Basis set} 
The basis set consists of explicitly correlated Gaussians multiplied
by powers of the inter-nuclear distance.  For diatomic molecules the
general form is, ( $^{\prime }$ represents vector/matrix transposition
and $\otimes $ is the Kronecker product symbol)
\begin{equation} \label{basis}
 \phi _k =
  r_{1}^{m_k} \exp \left[ -\mathbf{r}^{\prime }
      \left( L_kL_k^{\prime }\otimes I_3\right)\mathbf{r}\right],
\end{equation}
where, for H$_2$ $ \mathbf{r}$ is a $9\times 1$ vector of internal
Cartesian coordinates, $L_k$ is an $3\times 3$ rank $3$ lower
triangular matrix of nonlinear variation parameters, and $I_3$ is the
$3\times 3$ identity matrix. The Kronecker product with the identity
insures rotational invariance of the basis functions (the $\phi _k$
are angular momentum eigen-functions with J=0). The exponent
parameters are written in Cholesky factored form, $L_kL_k^{\prime },$
to insure positive definiteness of the quadratic form in the
exponential thereby, insuring $L^2$ integrability of the basis
functions.  For a more complete discussion of this basis and
derivation of the Hamiltonian matrix elements and derivatives in
matrix form see reference\cite{Kinghorn99a}.

Basis functions for the ground state wave function are obtained by
symmetry projecting the $\phi _k$ using a projection operator
$\mathcal{P}$. Thus,
\begin{equation}
{\mathcal{P}} \phi_{k} = \sum_{P} \chi_{P} r_{1}^{m_{k}} 
   \exp \left[-\mathbf{r}^{\prime} \left( \tau_{P}^{\prime} A_{k} \tau_{P}
                                 \otimes I_{3} \right) \mathbf{r} \right],
\end{equation}
where for the homo-nuclear systems, H$_2$, D$_2$ and T$2$ the
$\tau_{P}$ are the permutation matrices transforming the internal
coordinates given by,
\begin{equation}
\tau_{(1)}=
\left( 
\begin{array}{ccc}
1 & 0 & 0 \\ 
0 & 1 & 0 \\
0 & 0 & 1
\end{array} 
\right),
\tau_{(12)}=
\left( 
\begin{array}{ccc}
-1 & 0 & 0 \\ 
-1 & 1 & 0 \\
-1 & 0 & 1
\end{array} 
\right),
\tau_{(34)}=
\left( 
\begin{array}{ccc}
1 & 0 & 0 \\ 
0 & 0 & 1 \\
0 & 1 & 0
\end{array} 
\right),
\tau_{(12)(34)}=
\left( 
\begin{array}{ccc}
-1 & 0 & 0 \\ 
-1 & 0 & 1 \\
-1 & 1 & 0
\end{array} 
\right),
\end{equation}
and for the hetero-nuclear systems, HD, HT and DT, only $\tau_{(1)}$ and
$\tau_{(34)}$ are needed. The coefficients $\chi_{P}$ are from the
matrix elements of the irreducible representation for the desired
state, and for the ground states are all ones.

%\section{Mass values and other constants}
The nuclear masses were computed using the atomic masses published in
``\emph{The 1993 atomic mass evaluation}'' of Audi and
Wapstra\cite{Audi93}, H[$1836.152693$a.u.], D[$3670.483008$a.u.],
T[$5496.921571$a.u.]. These values are derived from the atomic masses
by correcting for the electron mass and the binding energy of the
electron.  We use quantum units in this work except where otherwise
noted. Thus, $\hbar =1,$ $m_e=1,$ energy is in hartree$\left(
  =2R_\infty \right) $, and distance is in bohr.

%\section{Implementation}
The wave function for the ground state is obtained by minimizing
the Rayleigh quotient; 
\begin{equation}
E\left( a;c\right) =\min_{\left\{ a,c\right\} }\frac{c^{\prime }H(a)c}{
c^{\prime }S(a)c}  \label{energy}
\end{equation}
where, $H\left( a\right) $ and $S\left( a\right) $ are the Hamiltonian
and overlap matrices, respectively, which are functions of the
nonlinear parameters contained in the basis set exponent matrices
$L_k$. We write $a$ for the collection of these nonlinear parameters
and $c$ is the vector of linear coefficients in the basis expansion of
the wavefunction.  Our experience indicates that much more thorough
optimization can be achieved by letting the optimizer simultaneously
vary both the linear and nonlinear parameters in the Rayleigh quotient
rather than alternately solving the eigen-problem for the $c$'s and
only letting the optimizer vary the nonlinear parameters, $a$.  The
optimization software employed was the package TN by Stephen Nash\cite
{NashTN} --- available from netlib\cite{netlib}. TN is a truncated
Newton method utilizing a user supplied gradient. The analytic
gradient of the energy functional was derived using matrix
differential calculus\cite{Kinghorn95a,Kinghorn95b} and is given in
the references\cite{Kinghorn99a}.

%\section{Results}
Starting values for the wave function parameters, (exponent matrices,
$L_k$, and powers of the inter-nuclear distance, $m_k$), were first
obtained for small wave functions by random trials. Exponential
parameters for larger wave functions were generated from random normal
distributions with mean and covariance determined from the parameter
sets for the smaller wave functions and then re-optimized.  The values
for the powers of the inter-nuclear distances, $m_k$, were randomly
distributed around the optimal values determined for small wave
functions.  These ``optimal'' values of $m_k$ range from 18 for H$_2$
to 34 for T$_2$. These increasing values of $m_k$ with increasing
mass of the nuclei represent the changing nature of the vibrational
components of the corresponding wave functions. This is consistent
with what one would see in the solutions of the Morse problem.  The
$m_k$ were not optimized during energy minimization.


Included in table~(\ref{evals}) are expectation values for the
Hamiltonian, $\langle H \rangle$, the kinetic, $\langle T \rangle$,
and potential energy, $\langle V \rangle$, the virial coefficient
$\eta=-\langle V \rangle /2\langle T \rangle$, the squared norm of the
energy gradient, $\|g\|_{2}^{2}$, and $r_{1},$ and $r_{1}^{2}$
computed using wave functions with 512 basis functions.  The values of
the virial coefficient and the gradient norm indicate the high level
of optimization obtained for these wave functions.  The energy values
we report for the non-adiabatic ground state of the isotopomers of
hydrogen are new rigorous variational upper-bounds for these systems.
The only other direct non-adiabatic energy results for a molecular
system with more than three particles is for
H$_2$\cite{Wolniewicz95,Bishop77b,Chen95}. The H$_2$ result presented
in this work is same as our previously established upper-bound for
this system\cite{Kinghorn99b}. It should be noted that wavefunctions
for the hetero-nuclear systems, HD, HT and DT, contain the same number of
optimization parameters as the homo-nuclear systems but only half as
many terms in the symmetry projectors, this means that the energy
results for the hetero-nuclear systems are less accurate than the
homo-nuclear systems. We estimate that the energy results for the
hetero-nuclear systems are 10 to 20 nano-hartree higher in absolute
precision than for the homo-nuclear cases.

%\section{Future Work}

The computer code for this work was parallelized using MPI and run on
a four node cluster of dual processor Linux workstations.  The code
scales very well and we see nearly an eight fold speedup when all
eight system processors are used. This parallel system gives us enough
processing power to build wave functions for larger systems.  We have
already begun work on the six particle system LiH and preliminary
results look very promising. With our existing code and a larger
parallel cluster we should realistically be able to obtain high
accuracy results for systems with two heavy particles and six
electrons. The main limitation being the size of the symmetry
projector which scales as the factorial of the number of
indistinguishable particles. We have also begun work on a basis set
suitable for describing the J=0 angular momentum states for tri-atomic
systems of the form,
\begin{equation} 
 \phi _k =
  r_{1}^{m_k}r_{2}^{p_k}r_{12}^{q_k} \exp \left[ -\mathbf{r}^{\prime }
      \left( L_kL_k^{\prime }\otimes I_3\right)\mathbf{r}\right].
\end{equation} 
Our work represents the most ambitious direct non-adiabatic calculations
ever attempted.
 

\acknowledgements
This work was supported by the National Science Foundation.


\begin{references}

\bibitem{Kinghorn99b}
D.~B. Kinghorn and L.~Adamowicz,
\newblock Phys. Rev. Lett. {\bf 83}, 2541 (1999).

\bibitem{Kinghorn99a}
D.~B. Kinghorn and L.~Adamowicz,
\newblock J. Chem. Phys. {\bf 110}, 7166 (1999).

\bibitem{Audi93}
G. Audi and A.~H. Wapstra, Nucl. Phys. A {\bf 565},  1  (1993).

\bibitem{CODATAweb} 
See CODATA database of fundamental constants at http://www.codata.org

\bibitem{NashTN}
S.~G. Nash,
\newblock SIAM J. Numer. Anal. {\bf 21}, 770 (1984).

\bibitem{netlib}
netlib can be accessed by ftp at netlib@ornl.gov and 
netlib@research.att.com or by World Wide Web access at http://www.netlib.org.

\bibitem{Kinghorn95a}
D.~B. Kinghorn,
\newblock Int. J. Quantum Chem. {\bf 57}, 141 (1996).

\bibitem{Kinghorn95b}
D.~B. Kinghorn and R.~D. Poshusta,
\newblock Int. J. Quantum Chem. {\bf 62}, 223 (1997).

\bibitem{Wolniewicz95}
L.~Wolniewicz,
\newblock J. Chem. Phys. {\bf 103}, 1792 (1995).

\bibitem{Bishop77b}
D.~M. Bishop and L.~M. Cheung,
\newblock Phys. Rev. A {\bf 18}, 1846 (1977).

\bibitem{Chen95}
B.~Chen and J.~B. Anderson,
\newblock J. Chem. Phys. {\bf 102}, 2802 (1995).


\end{references}

\newpage
\mediumtext


\begin{table}[!pth]
\caption{Expectation values for the six hydrogen molecule isotopomers 
         in the non-adiabatic ground state
         using 512 term, symmetry projected, 
         correlated {G}aussian wave functions.
         Energy in Hartree, distance in Bohr.  
\label{evals}}

\begin{tabular}{lrrrrrr}
%\hline \hline
     &\multicolumn{1}{c}{H$_2$} &\multicolumn{1}{c}{D$_2$}
     &\multicolumn{1}{c}{T$_2$} &\multicolumn{1}{c}{HD} 
     &\multicolumn{1}{c}{HT}    &\multicolumn{1}{c}{DT}   \\

$\langle H \rangle$ & $-1.1640250232 $ 
                    & $-1.1671688033 $
                    & $-1.1685356688 $ 
                    & $-1.1654718927 $
                    & $-1.1660020061 $
                    & $-1.1678196334 $ \\

$\langle T \rangle$ & $1.1640250041 $ 
                    & $1.1671686971 $
                    & $1.1685356851 $
                    & $1.1654718798 $ 
                    & $1.1660019664 $ 
                    & $1.1678195332 $ \\

$\langle V \rangle$ & $-2.3280500273 $
                    & $-2.3343375004 $ 
                    & $-2.3370713538 $
                    & $-2.3309437724 $
                    & $-2.3320039726 $
                    & $-2.3356391666 $ \\

$\eta$              & $1.0000000081 $
                    & $1.0000000455 $
                    & $0.9999999930 $
                    & $1.0000000055 $
                    & $1.0000000170 $
                    & $1.0000000428 $ \\

$\|g\|_{2}^{2}$     & $4.074 \times 10^{-15} $
                    & $3.619 \times 10^{-15} $
                    & $2.494 \times 10^{-15} $
                    & $1.505 \times 10^{-15} $
                    & $1.047 \times 10^{-15} $
                    & $1.170 \times 10^{-14} $ \\

$\langle r_1 \rangle$ & $1.4487380001 $
                      & $1.4345619772 $
                      & $1.4283586996 $
                      & $1.4422285474 $
                      & $1.4398373583 $
                      & $1.4316113713 $ \\

$\langle r_{1}^{2} \rangle$ & $2.1270459595 $
                            & $2.0776874796 $ 
                            & $2.0562410073 $
                            & $2.1043209366 $
                            & $2.0959987422 $
                            & $2.0674752747 $ \\

%\hline \hline
\end{tabular}
\end{table}

\end{document}
%
% ****** End of file h2iso-1-rev.tex ******

