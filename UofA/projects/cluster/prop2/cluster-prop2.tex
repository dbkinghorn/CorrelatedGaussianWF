
\documentstyle[12pt,overcite]{article}

\setlength{\textwidth}{7in}
\setlength{\textheight}{9.0in}
\setlength{\oddsidemargin}{-.25in}
%
\setlength{\topmargin}{-.10in}
\def\Real{\rm I\hspace{-0.2em}R}
\def\nin{{\in \! \! \! \! \! /} \,}
\def\nequiv{{\equiv \! \! \! \! \! \! /} \,}

\newcommand{\half}{\mbox{$\frac{1}{2}$}}
\newcommand{\mul}{\multicolumn{2}{c}{none}}
\newcommand{\qua}{\mbox{$\frac{1}{4}$}}
\newcommand{\six}{\mbox{$\frac{1}{6}$}}
\newcommand{\tfour}{\mbox{$\frac{1}{24}$}}
\newcommand{\beq}{\begin{equation}}
\newcommand{\eeq}{\end{equation}}
\newcommand{\tripler}{T_{3}\!\left(^{a^{\prime}b^{\prime}
{\rm C}^{\prime}}_{{\rm I}^{\prime}j^{\prime}k^{\prime}}\right)}
\newcommand{\quadrupler}{T_{4}\!\left(^{a^{\prime}b^{\prime}
{\rm C}^{\prime}{\rm D}^{\prime}}_{{\rm I}^{\prime}{\rm J}^{\prime}
k^{\prime}l^{\prime}}\right)}
% \newcommand{\hhline}{\hline \vspace{-4.8mm} \\ \hline}


\begin{document}



\setlength{\baselineskip}{1em}


\noindent
{\bf Building a Parallel Super--Computer for Molecular
Quantum Mechanical Calculations}




%\begin{abstract}
\section{Abstract}
We propose the construction of a prototype high--performance /
low--cost parallel computational platform for a new type of quantum
mechanical (q.m.) calculations on molecules and clusters with
explicitly correlated gaussian functions.  The methods for employing
these novel basis functions to both adiabatic and non--adiabatic q.m.
calculations have been under development in my research group for
several years. This work has been coupled with extended computaional
implementation work which has recently culminated in an integrated
design for parallel computational software and hardware for performing
practical calculations on non--adiabatic and ro--vibrational behavior
of chemical systems.  The computational hardware will consist of a
Linux ``Beowulf'' cluster built from low cost commodity PC components
running the Linux operating system connencted with switched fast
ethernet.  Building of a parallel computing platform represents a new
direction in the research effort of our group at the University of
Arizona.  It also represents a novel and very promissing direction in
high--performance low--cost scientific computing in the area of q.m.
molecular calculations.  Our research group has the necessary
interdisciplinary skills to construct and fully utilize this high
performance computational platform.  We have already built a low cost
prototype system consisting of a cluster server and two diskless
compute nodes providing six processors for parallel program execution.
This system demonstrates the feasibilty of our approach.  In this
pre--proposal we provide background information related to the design
of the parallel cluster consisting of 48 cpu's and to component
specifications and cost estimates.
%\end{abstract}

\newpage


\section{Introduction}

The performance of commodity personal computer (PC) components has
been steadily increasing while at the same time prices for these
components have been decreasing. Today it's possible to construct a
computer with off the shelf components for arround \$1000 that will
out perform a \$40,000 workstation of just a few years ago. Freely
availabile high quality, open source, network operating system
components and tools for parallel software development together with
high performance low cost PC components and inexpensive high speed
networking hardware has made PC clustering, by far, the most cost
effective means of ``super computing''.

The most  promising implementation of this type of PC clustering
for high performance computing
originated with the ``Beowulf Project''(
http://beowulf.gsfc.nasa.gov/beowulf.html ) at the Center of
Excellence in Space Data and Information Sciences (CESDIS) in 1994 (
http://cesdis.gsfc.nasa.gov/ ). Since then the Beowulf project
development has spread to research labs and universities around the
world.  Today the term Beowulf cluster loosely refers to any PC  
clustering  based on the freely available, open source, Linux operating
system where the nodes are dedicated to the cluster (as opposed to a simple 
network of workstations).  The software tools, information and expertise for building
Beowulf class PC clusters is readily available from numerous sites on
the web. A few good starting points are:
http://www.sci.usq.edu.au/staff/jacek/beowulf/ and
http://www.beowulf.org for Beowulf information, 
http://www.tdl.com/~netex/ for hardware information,
  and for parallel
computing in general, the IEEE Computer Society's ParaScope site
http://www.computer.org/parascope/.

\section{Proposed system}

The parallel computing platform we are proposing will utilize high end
PC components running RedHat Linux with Beowulf kernel modifications
and network drivers together with tools for parallel software
development. The compute nodes will communicate over a private,
switched, 100Mb ethernet LAN connected to a cluster master node via
1000Mb ethernet. The master node will server as the primary network
gateway and the NFS and DQS (Distributed Que System) server for the  cluster.
The compute nodes will each contain two cpu's and a small hard drive
for system code, swap and scratch space. The
networking hardware will allow us to convienently scale the cluster in units of 12
or 24 nodes for a maximum of 192 compute nodes ( 284 cpu's ). We are initially
planning on building a 24 node system. The present request is made to
provide funding to purchase the hardware components for the system and
licensing for a High Performance Fortran development package.

The quantum chemistry programs we have developed (and are developing)
readily lend themselves to coarse grained parallelism which scales
well and is ideally suited for implementation on large parallel
clusters. We are currently running large, high accuracy calculations
which require weeks or months of cpu time with well optimized
efficient algorithms on single processor systems.
Our only logical recourse to increase
productivity is to go parallel. We have recently assembled a small two node
test system and are currently working on parallel implementations of
our quantum chemistry codes. Our experiences so far are very
encouraging and have given use great motivation for further effort in
this direction. We are now confident that we have the human resources
necessary to build and fully utilize the system we are proposing.

The specifications for our system will be given in the tables below along with cost
estimates.  It should be noted that PC components are commodities and
prices fluctuate with supply and demand, and new technology. However,
the trend is for prices to move downward while quality and performance
increase. Therefore, the specifications will likely change by the time
you read this, and, this change will most likely be for higher
performance components at the same total system cost. Also note that
components are not necessiarily chosen on the basis of lowest cost but rather
highest reliability and performance. Up time for the cluster
is maximized by paying attention to the details.


\section{Scientific goals to be accomplished}



The low--cost super--computer cluster, 
which we propose to build, will
be empolyed to perform highly accurate quantum mechanical 
calculations of structures and properties of molecues
and clusters using explicitly correlated gaussian
functions and both variational and non--variational
methods of qunatum chemistry. 
The use of explicitly correlated functions, which
depend on the interparticle distances, has the
potential to revolutionize the quantum chemical
calculations by bringing their accuracy to a higher level
which is not
attainable with the methodologies currently is use.
Particularly 
calculations with the
the methods allowing study of molecular systems without
assuming the Born--Oppenheimer (BO) approximation regarding the
separability of the nuclear and electronic motions, which we have
developed in our research group,\cite{1,2,3,4}
will be the focus of this project.
The approach with the explicitly correlated
gaussian functions will also be applied to
a more general problem of stationary bound
quantum states of multi--particle
systems which interact through an isotropic potential.
The non--BO molecular system consisting of electrons and nuclei
with
Coulombic repulsive and attractive potentials
is an example of such a problem.
Another example
is the problem of ro--vibrational structures
of molecules and clusters.
In this case the isotropic interaction potential is provided
in the form of the potential energy hypersurface (PES)
calculated for the system using electronic structure methods.
We will consider PES'es of the
ground or an excited electronic state in such a calculation.
Unlike the Coulombic potential, which includes only two-body
components, the potential given by PES has N--body character and,
although in most cases it is dominated by two-body interactions,
the three- and higher-body contributions can be important.
New functional forms for the multi--body expansion of the potential
energy
surface  
will be developed.
%and for the basis function expansion of the ro--vibrational
%wave function to be used variational calculations of vibrational
%and rotational levels of N-atom systems
%are proposed.
We propose that N--body
explicitly correlated gaussians with pre-multiplying factors consisting
of
products of powers of internal distance coordinates are utilized in a
dual
role to analytically represent the isotropic N--atom potential and the
wave functions expressed in terms of internal cartesian coordinates.
We propose to develop and implement the methodology for the general
N--body case.

The initial non--Born--Oppenheimer
applications of the present methodology
will focus on
small atomic and molecular systems.
Following our recent non--adiabatic calculations
of the electron affinities of hydrogen, deuterium and tritium,
as well as different isotopes of the Li atom,
which produced results matching very well the experimental
values and in some cases providing results which have not 
yet been experimentally measured, we will now
consider small molecular systems.
In the first calculations
we will study
hydrogen molecules and ions and their isotopomers,
H$_2^+$, H$_2$, H$_3$, {\it etc.},
%small dipole--bound anions, {\it e.g.}, LiH$^-$, LiD$^-$,
and other molecular systems, {\it e.g.}, HeH$^+$, HeH$_2^+$, HeH$_2$.
The aim of these calculations will be to determine
electron affinities and electron detachment energies
of these systems in different ro--vibrational states.
Very accurate measurements, which have
recently become
available
for these systems
present an exciting challenge for
the ``non--Born-Oppenheimer Atomic and Molecular Quantum Mechanics."
Also calculations of
higher excited states,
where the coupling of the electronic and nuclear motions
can be significant,
will be performed and the nature and magnitude of the
coupling effect will be
analyzed. I should mentioned, we are one of a very few research
groups which have developed a practical methodology for
q.m. non--adiabatic calculations of molecular systems.

It should be noted that we are in no way restricted to conventional
systems.
Calculations involving ``exotic'' particles present no difficulty
since in general our methodology can treat collections of particles
with any given masses and charges fully non--adiabatically.
%dbk

Initial applications of the method with
analytically fitted PES's as the
interaction potentials
in the ro--vibrational Hamiltonian
will start with
``calibration cases"
such as the H$_3^+$, H$_2$O, HCN and LiCN systems and their
isotopomers.
These systems considered before by others
using different methodologies
and their calculated ro--vibrational spectra
were compared with very accurate experimental data.
Our calculations will be particularly relevant
in analysis of
recently aquired high--resolution
intersteller spectra.

For H$_3^+$ we will generate the
potential energy surface using
the variational approach with explicitly correlated
gaussians and with the optimization algorithm based
on analytical derivatives of the variational functional
with respect to the non--linear parameters of gaussians,
which we have recently developed.
For H$_2$O we will use the best existing potentials,
as well as the potential generated using our SSMRCC method.
Application of our approach to systems with more than three
nuclei are the most interesting since
other methods have been mostly limited to three-body cases.
We find particularly interesting
to investigate
the
ro-vibrationally hot elemental clusters, such as
C$_n$, Si$_n$, P$_n$, $n$=4,5 {\it etc.}, and their anions.
Calculations of highly excited ro-vibrational states near the
dissociation barrier of
these systems will reveal their structures and dynamics
at high temperatures.
Among other applications we will consider
the high--temperature
ro--vibrational structure
of the vinylidine--acetylene,
$HCCH \leftrightarrow CCH_2$, isomeric system.
We will also study
ro--vibrational spectra
of small molecular dimers such
as (HF)$_2$, (H$_2$O)$_2$, {\it etc.}


\newpage

\section{References}

\begin{thebibliography}{999}


\bibitem{1}
D.B. Kinghorn and L. Adamowicz,
The Electron Affinity of Hydrogen, Deuterium and Tritium:
A Non--Adiabatic Variational Calculation Using Explicitly-Correlated
Gaussian Basis Set, J. Chem. Phys.,
{\bf 106}, 4589 (1997).



\bibitem{2}
D. Gilmore, P.M. Kozlowski, D.B. Kinghorn and
L. Adamowicz, Analytic First Derivatives for
Explicitly-Correlated Multi-Center Gaussian Geminals,
Int. J. Quantum Chem. {\bf 63}, 991 (1997).


\bibitem{3}
D.B. Kinghorn and L. Adamowicz,
A New N--Body Potential and Basis Functions for
Variational Energy Calculations,
J.Chem.Phys., {\bf 106}, 8760 (1997).

\bibitem{4}
D. B. Kinghorn and L. Adamowicz, in {\em Pauling's Chemical
Bonding}, edited by Z.B. Maksic and W.J. Orville--Thomas,
Elsevier Science, 1998, in press.

\bibitem{5}
D.B. Kinghorn and L. Adamowicz,
A Correlated Basis Set for Non--Adiabatic Energy
Calculations of Diatomic Molecules,
J.Chem.Phys., in press.




\end{thebibliography}


%\newpage


%\section{Budget Justification}
%The following components of the prototype parallel computer system are
%budgeted whithin
%the requested amounts:

%\noindent
%Cluster server:
%Chasis [\$250], AUSU P2B-DS motherboard [\$450], (2) PII400 processors
%[\$680],
%9GB SCSI hard drive [\$470], 128MB PC100 SDRAM [\$170],
%17'' monitor [\$265], video card [\$22], 100Mb nic [\$19], 100Mb TX
%switch [\$229] --- SUBTOTAL \$2555

%\noindent
%Diskless Compute Node:
%Chasis [\$200], DFI P2XBLD motherboard [\$170], (2) PII350 processors
%[\$420],
%128MB PC100 SDRAM [\$170], 100Mb nic [\$19], misc. [\$75] --- SUBTOTAL
%\$1054

%\noindent
%Cluster server + two compute nodes + local sales tax --- TOTAL \$5000

\newpage




\begin{tabular}
[c]{|c|c|}\hline\hline
\multicolumn{2}{|c|}{Compute Nodes}\\\hline\hline
Item & Cost\\\cline{1-1}%
\multicolumn{1}{|l|}{California PC Products ATX Deskside Steel Chassis Model
8C8A00} & \multicolumn{1}{|l|}{\$160}\\\cline{1-1}%
\multicolumn{1}{|l|}{SPI 300W, Rev 2.01 ATX Power Supply} &
\multicolumn{1}{|l|}{\$80}\\\cline{1-1}%
\multicolumn{1}{|l|}{(2) 80mm Cooling Fan} & \multicolumn{1}{|l|}{\$40}%
\\\cline{1-1}%
\multicolumn{1}{|l|}{Teac 1.44MB Floppy} & \multicolumn{1}{|l|}{\$30}%
\\\cline{1-1}%
\multicolumn{1}{|l|}{S3 Video Card} & \multicolumn{1}{|l|}{\$25}\\\cline{1-1}%
\multicolumn{1}{|l|}{3.2 GB Ultra DMA Hard Disk} & \multicolumn{1}{|l|}{\$120}%
\\\cline{1-1}%
\multicolumn{1}{|l|}{ASUS P2B-D Motherboard} & \multicolumn{1}{|l|}{\$320}%
\\\cline{1-1}%
\multicolumn{1}{|l|}{(2) Intel PII-450 512K} & \multicolumn{1}{|l|}{\$1160}%
\\\cline{1-1}%
\multicolumn{1}{|l|}{Pentium II Fan, Heat Sink, grease} &
\multicolumn{1}{|l|}{\$30}\\\hline
\multicolumn{1}{|l|}{128MB Fast PC100 SDRAM CAS-2} &
\multicolumn{1}{|l|}{\$230}\\\hline
\multicolumn{1}{|l|}{DEC Tulip 100Mb Fast Ethernet NIC} &
\multicolumn{1}{|l|}{\$25}\\\hline
\multicolumn{1}{|l|}{Shipping} & \multicolumn{1}{|l|}{\$130}\\\hline
\multicolumn{1}{|r|}{SubTotal} & \multicolumn{1}{|l|}{\$2350}\\\hline
\multicolumn{1}{|r|}{24Units} & \multicolumn{1}{|l|}{\$56400}\\\hline
\end{tabular}
\newline 

\bigskip%

\begin{tabular}
[c]{|c|c|}\hline\hline
\multicolumn{2}{|c|}{Master Node}\\\hline\hline
Item & Cost\\\cline{1-1}%
\multicolumn{1}{|l|}{California PC Products ATX Deskside Steel Chassis Model
8C8A00} & \multicolumn{1}{|l|}{\$160}\\\cline{1-1}%
\multicolumn{1}{|l|}{Zippy/Emacs 400W Power Supply} &
\multicolumn{1}{|l|}{\$140}\\\cline{1-1}%
\multicolumn{1}{|l|}{(2) 80mm Cooling Fan} & \multicolumn{1}{|l|}{\$40}%
\\\cline{1-1}%
\multicolumn{1}{|l|}{Teac 1.44MB Floppy} & \multicolumn{1}{|l|}{\$30}%
\\\cline{1-1}%
\multicolumn{1}{|l|}{Millennium G200 8MB SGRAM AGP} &
\multicolumn{1}{|l|}{\$150}\\\cline{1-1}%
\multicolumn{1}{|l|}{IBM 9.1GB HammerHead 10,00RPM Ultra2 Wide LVD SCSI,
5.6ms, Hard Disk} & \multicolumn{1}{|l|}{\$790}\\\cline{1-1}%
\multicolumn{1}{|l|}{Ultra2 Wide LVD SCSI Cable} & \multicolumn{1}{|l|}{\$35}%
\\\cline{1-1}%
\multicolumn{1}{|l|}{ASUS P2B-DS Motherboard with Integrated Ultra2 Wide SCSI}%
& \multicolumn{1}{|l|}{\$520}\\\cline{1-1}%
\multicolumn{1}{|l|}{4/8GB Internal SCSI Travin Tape Drive} &
\multicolumn{1}{|l|}{\$390}\\\cline{1-1}%
\multicolumn{1}{|l|}{Yamaha CDR-4260t 6X Read / 4x Write CD-R / 2x ReWrite.
SCSI} & \multicolumn{1}{|l|}{\$320}\\\cline{1-1}%
\multicolumn{1}{|l|}{(2) Intel PII-450 512K} & \multicolumn{1}{|l|}{\$1160}%
\\\cline{1-1}%
\multicolumn{1}{|l|}{Pentium II Fan, Heat Sink, grease} &
\multicolumn{1}{|l|}{\$30}\\\hline
\multicolumn{1}{|l|}{256MB Fast PC100 SDRAM CAS-2} &
\multicolumn{1}{|l|}{\$460}\\\hline
\multicolumn{1}{|l|}{DEC Tulip 100Mb Fast Ethernet Card} &
\multicolumn{1}{|l|}{\$25}\\\hline
\multicolumn{1}{|l|}{Netgear Gigabit Ethernet Card} &
\multicolumn{1}{|l|}{\$315}\\\hline
\multicolumn{1}{|l|}{DigiView 19'' Monitor} & \multicolumn{1}{|l|}{\$500}%
\\\hline
\multicolumn{1}{|l|}{Keytronics keyboard} & \multicolumn{1}{|l|}{\$35}\\\hline
\multicolumn{1}{|l|}{Logitech Mouse Man Ergonomic 3-button Mouse} &
\multicolumn{1}{|l|}{\$25}\\\hline
\multicolumn{1}{|l|}{Shipping} & \multicolumn{1}{|l|}{\$175}\\\hline
\multicolumn{1}{|r|}{Total} & \multicolumn{1}{|l|}{\$5300}\\\hline
\end{tabular}
\newline
\begin{tabular}
[c]{|c|c|}\hline\hline
\multicolumn{2}{|c|}{Network Hardware}\\\hline\hline
Item & Cost\\
\multicolumn{1}{|l|}{BayStack 450-24T Switch (24 100Base-TX ports + 1 MDA and
Cascade slot} & \multicolumn{1}{|l|}{\$2325}\\
\multicolumn{1}{|l|}{BayStack 450-1SX-port 1000Base-SX Single PHY MDA} &
\multicolumn{1}{|l|}{\$980}\\
\multicolumn{1}{|l|}{Multimode Fiber Optic Duplex Cable 5ft} &
\multicolumn{1}{|l|}{\$40}\\
\multicolumn{1}{|l|}{CAT-5 Cables (24)} & \multicolumn{1}{|l|}{\$120}\\\hline
\multicolumn{1}{|l|}{Shipping} & \multicolumn{1}{|l|}{\$80}\\\hline
\multicolumn{1}{|r|}{Total} & \multicolumn{1}{|l|}{\$3545}\\\hline
\end{tabular}

\bigskip%
\begin{tabular}
[c]{|c|c|}\hline\hline
\multicolumn{2}{|c|}{Software}\\\hline\hline
Item & Cost\\
\multicolumn{1}{|l|}{Linux operating system distribution with Beowulf
modifications} & \multicolumn{1}{|l|}{\$0}\\
\multicolumn{1}{|l|}{DQS Distributed Que System for resource allocation and
job scheduling} & \multicolumn{1}{|l|}{\$0}\\
\multicolumn{1}{|l|}{MPICH and PVM message passing libraries} &
\multicolumn{1}{|l|}{\$0}\\
\multicolumn{1}{|l|}{ScLAPACK, PAPACK parallel subroutene libraries} &
\multicolumn{1}{|l|}{\$0}\\
\multicolumn{1}{|l|}{C, C++, F77 compilers} & \multicolumn{1}{|l|}{\$0}%
\\\hline
\multicolumn{1}{|l|}{Misc. cluster management and performance monitoring
tools} & \multicolumn{1}{|l|}{\$0}\\\hline
\multicolumn{1}{|l|}{HPF/F90 compiler (multiple licences and parallel suport)}%
& \multicolumn{1}{|l|}{\$4000}\\\hline
\multicolumn{1}{|r|}{Total} & \multicolumn{1}{|l|}{\$4000}\\\hline
\end{tabular}

\bigskip

Total cost for 24 node (48 processor) cluster \$69245
\end{document}





