% H2 paper from cgr1 calculations
% Donald B. Kinghorn
% University of Arizona
% Mon Mar 22 17:53:40 MST 1999

\documentclass[12pt]{article}
%\usepackage{aip}
%\usepackage{setspace}
\usepackage{amsmath}
\usepackage{amsfonts}
\usepackage{amssymb}


\begin{document}

\title{Improved non-adiabatic ground-state energy upper bound for dihydrogen}
\author{Donald B. Kinghorn and Ludwik Adamowicz\\
        Department of Chemistry\\
        University of Arizona\\
        Tucson AZ 85721}
\maketitle

\begin{abstract}
A new, highly accurate, value for the non-adiabatic energy 
of the dihydrogen ground state is reported, (-1.1640250232 hartree).
The calculations were performed with a direct non-adiabatic variational
approach using a new correlated {G}aussian basis set including powers of the
the inter-nuclear distance.    
\end{abstract}

%\section{Introduction}
We recently introduced a new correlated Gaussian basis set suitable
for high accuracy non-adiabatic calculations on diatomic molecules
\cite{Kinghorn99a}.
In that paper we gave a detailed description of the basis set including
formulas for matrix elements and energy gradient components together with 
information on implementation. We demonstrated the implementation by
including accurate results for the standard reference system H$_{2}^{+}$.
In this letter we continue the validation of this new basis by reporting
a new variational energy upper-bound for the ground state of dihydrogen 
molecule. This new bound should provide an accurate reference energy
for future non-adiabatic calculations and for evaluating the quality
of adiabatic plus non-adiabatic ``correction'' methodologies
that are based on the Born-Oppenheimer approximation. 

%\section{Non-adiabatic Hamiltonian}
In the non-adiabatic approach all particles are treated equally utilizing
their given masses and full interactions with all other particles. 
Without invoking any approximations, the total Hamiltonian 
can be separated into an operator representing the translational motion
of the center of mass and an operator representing the internal energy.
We perform this separation by making a transformation to an internal
reference frame with origin at particle one,
\begin{equation} \label{rdef}
\mathbf{r}=\left[
  \begin{array}[c]{c}
    \mathbf{r}_{1}\\
    \mathbf{r}_{2}\\
    \mathbf{r}_{3}
  \end{array}
  \right]  
  = \left[
    \begin{array}[c]{c}
      \mathbf{R}_{2}-\mathbf{R}_{1}\\
      \mathbf{R}_{3}-\mathbf{R}_{1}\\
      \mathbf{R}_{4}-\mathbf{R}_{1}\\
    \end{array}
  \right],
\end{equation}
where the $\mathbf{R}_{i}$ are the original particle coordinates.
This transformation to  internal coordinates together with the 
conjugate momentum transformation yields the non-adiabatic Hamiltonian
for the internal energy of a four particle system,
\begin{equation} \label{intham1}
H=-\frac{1}{2}\left(  \sum_{i}^{3}\frac{1}{\mu_{i}}\nabla_{i}^{2}
  +\sum_{i\neq j}^{3}\frac{1}{M_{1}}\nabla_{i}^{\prime}\nabla_{j} \right)
  +\sum_{i=1}^{3}\frac{q_{0}q_{i}}{r_{i}}  
  +\sum_{i<j}^{3}\frac{q_{i}q_{j}}{r_{ij}},
\end{equation}
here the $\mu_{i}$ are reduced masses, $M_{1}$ is the mass of particle 1,
(the coordinate reference particle), and $\nabla_{i}$ is the gradient with
respect to the $x,y,z$ coordinates $\mathbf{r}_{i}$. The potential energy is
the same as in the total Hamiltonian but is now written using internal
distance coordinates. The charges are mapped from the original particles as
$\{Q_{1},Q_{2},Q_{3},Q_{4}\}\mapsto\{q_{0},q_{1},q_{2},q_{3}\}$. 
In internal coordinates, distances are denoted,(using the standard 2-norm), 
$r_{ij}=\left\|  \mathbf{r}_{i}-\mathbf{r}_{j}\right\|  =$ 
$\left\|  \mathbf{R}_{i+1}-\mathbf{R}_{j+1}\right\|$ with 
$r_{j}=\left\|  \mathbf{r}_{j}\right\|  =
\left\|  \mathbf{R}_{j+1}-\mathbf{R}_{1}\right\|$.
More general information on the non-adiabatic Hamiltonian and the 
center of mass transformation can be found in the references\cite{Kinghorn99a}.

%\section{Basis set} 
The basis set consists of explicitly correlated Gaussian's multiplied by
powers of the inter-nuclear distance. 
The general form is, ( $^{\prime }$ represents vector/matrix transposition
and $\otimes $ is the Kronecker product symbol) 
\begin{equation} \label{basis}
 \phi _k =
  r_{1}^{m_k} \exp \left[ -\mathbf{r}^{\prime }
      \left( L_kL_k^{\prime }\otimes I_3\right)\mathbf{r}\right],
\end{equation}
where, for H$_2$ $ \mathbf{r}$ is a $9\times 1$ vector of internal 
Cartesian coordinates, $L_k$ is
an $3\times 3$ rank $3$ lower triangular matrix of nonlinear variation
parameters, and $I_3$ is the $3\times 3$ identity matrix. The Kronecker
product with the identity insures rotational invariance of the basis
functions (the $\phi _k$ are angular momentum eigen-functions with J=0). The
exponent parameters are written in Cholesky factored form, 
$L_kL_k^{\prime },$ 
to insure positive definiteness of the quadratic form in the exponential
thereby, insuring $L^2$ integrability of the basis functions.
For a more complete discussion of
this basis and derivation of the Hamiltonian matrix elements and derivatives
in matrix form see reference\cite{Kinghorn99a}.

Basis functions for the ground state wave function 
are obtained by symmetry
projecting the $\phi _k$ using a  projection operator
$\mathcal{P}$. Thus,
\begin{equation}
\mathcal{P}\phi_{k}=\sum_{P}\chi_{P}\,\,r_{1}^{m_{k}}\exp\left[  -r^{\prime
}\left(  \tau_{P}^{\prime}A_{k}\tau_{P}\otimes I_{3}\right)  r\right],
\end{equation}
where $\tau_{P} $ are the permutation matrices transforming the
internal coordinates, 
\begin{equation}
\left( 
\begin{array}{ccc}
1 & 0 & 0 \\ 
0 & 1 & 0 \\
0 & 0 & 1
\end{array} 
\right),
\left( 
\begin{array}{ccc}
-1 & 0 & 0 \\ 
-1 & 1 & 0 \\
-1 & 0 & 1
\end{array} 
\right),
\left( 
\begin{array}{ccc}
1 & 0 & 0 \\ 
0 & 0 & 1 \\
0 & 1 & 0
\end{array} 
\right),
\left( 
\begin{array}{ccc}
-1 & 0 & 0 \\ 
-1 & 0 & 1 \\
-1 & 1 & 0
\end{array} 
\right).
\end{equation}
The coefficients $\chi_{P}$ are from the matrix elements of the irreducible
representation for the desired state, and for the ground state are all ones.

%\section{Mass values and other constants}
The hydrogen nuclear mass was computed
using the atomic mass given in \emph{The 1993 atomic mass evaluation} of
Audi and Wapstra\cite{Audi93} and is given in table[\ref{results}].
We use quantum units in this work except where
otherwise noted. Thus, $\hbar =1,$ $m_e=1,$ energy is in hartree$\left(
=2R_\infty \right) $, and distance is in bohr. 

%\section{Implementation}
The wave functions for the ground state is obtained by minimizing
the Rayleigh quotient; 
\begin{equation}
E\left( a;c\right) =\min_{\left\{ a,c\right\} }\frac{c^{\prime }H(a)c}{
c^{\prime }S(a)c}  \label{energy}
\end{equation}
where,
$H\left( a\right) $ and $S\left( a\right) $ are the Hamiltonian and overlap
matrices, respectively, 
which are functions of the nonlinear parameters contained in the
basis set exponent matrices $L_k$. We write $a$ for the collection of these
nonlinear parameters and $c$ is the vector of linear coefficients in the basis
expansion of the wavefunction.
Our experience indicates that 
much more thorough optimization can be achieved
by letting the optimizer
simultaneously vary both the linear and nonlinear parameters in the
Rayleigh quotient rather than alternately solving the eigen-problem for
the $c$'s and only letting the optimizer vary the nonlinear parameters, $a$.
The optimization software employed was the package TN by Stephen Nash\cite
{NashTN} --- available from netlib\cite{netlib}. TN is a truncated Newton
method utilizing a user supplied gradient. The analytic gradient of the
energy functional was derived using matrix 
differential calculus\cite{Kinghorn95a,Kinghorn95b} and is given in 
the references\cite{Kinghorn99a}.

%\section{Results} 
Table~(\ref{results})
contains expectation values computed using our 
optimized non-adiabatic wave function constructed from
512 basis functions, $\phi_k$.
Included in
the table are expectation values for the Hamiltonian, 
$\langle H \rangle$, the kinetic, $\langle T \rangle$, and
potential energy, $\langle V \rangle$, the virial coefficient 
$\eta=-\langle V \rangle /2\langle T \rangle$, the squared norm
of the energy gradient, $\|g\|_{2}^{2}$, and
$r_{1},$ and  $r_{1}^{2}$. 
The values of the virial coefficient and the
gradient norm indicate the high level of optimization
obtained for this wave function. The energy we report for the non-adiabatic
ground state of H$_2$ is a new rigorous variational upper-bound. 

Our current development efforts involve generalizing the basis for 
polyatomic systems and writing parallel software to handle
the arduous task of optimizing many nonlinear parameters. 
With this parallel implementation we will be able to perform
direct non-adiabatic calculations on systems with more than two nuclei.
No-one has yet carried out such calculations.
 

%\begin{acknowledgement}
\vspace{.25in}
\noindent
\large{\bf{Acknowledgement}} \normalsize \\
This work was supported by the National Science Foundation.
%\end{acknowledgement}


\begin{table}[!p]
\caption{Expectation values for the dihydrogen non-adiabatic ground state
         using a 512 term correlated {G}aussian wave function.
         Energy in Hartree, distance in Bohr. 
\label{results}}

\begin{tabular}{lrl}
\hline \hline
$\langle H \rangle$ & $-1.1640250232$ & This work (H mass$=1836.152693$a.u.)\\

$\langle T \rangle$ & $ 1.1640250041$  & \\
$\langle V \rangle$ & $-2.3280500273$  & \\
$\eta$              & $ 1.0000000081$  & \\
$\|g\|_{2}^{2}$     & $ 4.074 \times 10^{-15}$  & \\
$\langle r_1 \rangle$ & $ 1.4487380001$                & \\
$\langle r_{1}^{2} \rangle$ & $  2.1270459595$         & \\
\hline
\multicolumn{3}{l}{Literature values for $\langle H \rangle$}\\
\hline
 $-1.16402413$ &\multicolumn{2}{l}{
        Bishop and Cheung ref. \cite{Bishop77b},(H mass$=1836.15$a.u)}\\ 
        &\multicolumn{2}{l}{Variational, 1070 basis functions} \\
 $-1.1640239$ &\multicolumn{2}{l}{
        Chen and Anderson ref. \cite{Chen95},(H mass not given)}\\
        &\multicolumn{2}{l}{Quantum Monte Carlo} \\
 $-1.164025018$ &\multicolumn{2}{l}{
        Wolniewicz ref. \cite{Wolniewicz95}, (H mass$=1836.1527$)}\\
        &\multicolumn{2}{l}{
        High accuracy adiabatic and non-adiabatic corrections}\\
\hline \hline
\end{tabular}
\end{table}

\begin{thebibliography}{10}

\bibitem{Kinghorn99a}
D.~B. Kinghorn and L.~Adamowicz,
\newblock J. Chem. Phys. {\bf 110}, 7166 (1999).

\bibitem{Kinghorn95a}
D.~B. Kinghorn,
\newblock Int. J. Quantum Chem. {\bf 57}, 141 (1996).

\bibitem{Kinghorn95b}
D.~B. Kinghorn and R.~D. Poshusta,
\newblock Int. J. Quantum Chem. {\bf 62}, 223 (1997).

\bibitem{Audi93}
G. Audi and A.~H. Wapstra, Nucl. Phys. A {\bf 565},  1  (1993).

\bibitem{NashTN}
S.~G. Nash,
\newblock SIAM J. Numer. Anal. {\bf 21}, 770 (1984).

\bibitem{netlib}
netlib can be accessed by ftp at netlib@ornl.gov and 
netlib@research.att.com or by World Wide Web access at http://www.netlib.org.

\bibitem{Bishop77b}
D.~M. Bishop and L.~M. Cheung,
\newblock Phys. Rev. A {\bf 18}, 1846 (1977).

\bibitem{Chen95}
B.~Chen and J.~B. Anderson,
\newblock J. Chem. Phys. {\bf 102}, 2802 (1995).

\bibitem{Wolniewicz95}
L.~Wolniewicz,
\newblock J. Chem. Phys. {\bf 103}, 1792 (1995).

\end{thebibliography}

\end{document}

