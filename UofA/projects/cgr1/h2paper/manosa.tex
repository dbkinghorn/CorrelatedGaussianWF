%%%%%%%%%%%%%%%%%% file manosa.tex %%%%%%%%%%%%%%%%%%%%
%                                                     %
%   Copyright (c) 1992, Optical Society of America.   %
%                                                     %
%%%%%%%%%%%%%%%%%% November 10, 1992 %%%%%%%%%%%%%%%%%%
%
%    From version 3.0 of the REVTeX macro package.
%
% This file is part of a compuscript toolbox distributed by
% the APS, OSA, and AIP in conjunction with
% the TeX author-prepared program.
%
% All rights not specifically granted are reserved.
% For more information, see the README file.
%
%Filename: manosa.tex
\documentstyle[manuscript,osa]{revtex}
\def\btt#1{{\tt$\backslash$#1}}
\def\thesection {\Roman{section} } %%OSA CHANGE: WAS arabic
%
\makeatletter
% run page numbers by "chapter"  %File manosa.tex \hskip4in
\def\thepage{2 -- \@arabic\c@page}
% these page numbers need a bit more width
\def\@pnumwidth{4em}
%\makeatother
%\pagestyle{myheadings}
%\markright{REV\TeX\ 3.0. \hskip2em  Release date: November 10, 1992}
%%
\newcommand{\MF}{{\large{\manual META}\-{\manual FONT}}}
\newcommand{\manual}{rm}  % Replace the manual font with rm (Roman)
\newcommand\bs{\char '134 }  % add backslash char to \tt font
%
\begin{document}

%\title{OSA Input Guide for REV\TeX\ based
%Author-Prepared Compuscripts}
\title{REV\TeX\ Information for OSA Authors}



\maketitle       %% ESSENTIAL IF TITLE, AUTHOR, OR DATE ARE TO APPEAR

\makeatletter
\global\@specialpagefalse
\def\@oddhead{REV\TeX\ 3.0\hfill Released November 10, 1992}
\let\@evenhead\@oddhead
\def\@oddfoot{\reset@font\rm\hfill \thepage
\hfill\llap{ \protect\copyright{} 1992,
Optical Society of America}} \let\@evenfoot\@oddfoot
\makeatother
 % % %
\begin{center}
Instructions to authors for preparing compuscripts
to be submitted to OSA journals in the REV\TeX\ 3.0 format.
\hskip.2in Release date: November 10, 1992. \\
 \copyright\ 1992, Optical Society of America.
\end{center}
\tableofcontents
\makeatletter
\def\@oddfoot{\reset@font\rm\hfill \thepage\hfill}
\let\@evenfoot\@oddfoot
\makeatother

\newpage

\baselineskip = .5\baselineskip  % single space the first half

\section{ ELECTRONIC MANUSCRIPT FILE SUBMISSION}

\begin{center} Quick-Reference Check List for Submission: \\
\end{center}
\begin{tabular}{|cl|}  \hline
$\,\,\,\,\,\,\,$Submission by & IBM-compatible formatted diskette
should include: \\
   \hline
   1. &   Cover letter with journal and manuscript
               identification, hardware$\,\,\,\,\,\,\,$ \\
      &   identification, and
               corresponding author information restated \\
   2. &   Diskette with filename(s), manuscript number, and \\
      &        first author's name written on label \\
   3. &   REV\TeX\ 3.0 file(s) on the diskette that match the final, \\
      &        accepted manuscript                    \\
   4. &   Paper copy of the final, accepted manuscript, with camera \\
      &        ready figures. \\
                       \hline
\end{tabular}
\vskip.5in


     The Optical Society of America is pleased to invite
electronic files in REV\TeX\ 3.0 from authors of journal
manuscripts.  At this time, the REV\TeX\ format can be converted by
OSA's commercial typesetters, and it is hoped that most files
submitted in this fashion will be usable. \\
%\\

     Electronic manuscript files should be submitted at the
conclusion of the peer review process.  The key is that, on
acceptance, the author's electronic file must be in the hands of
the OSA Manuscript Office staff.  The electronic version must be
that of the authors's final, accepted manuscript.  If it is not,
the file will simply not be used.         \\
%\\

   A cover letter containing the following information should
accompany any electronic file submission:
\begin{itemize}
  \begin{enumerate}

  \item   Article identification -- include journal name,
          manuscript number, title of paper, and the fact that
          this is a REV\TeX\ 3.0 electronic file submission

  \item   Computer information --  give the name of computer used
          and the density of the diskette

  \item   Corresponding author information -- telephone and
          facsimile numbers, plus an e-mail address if available,
          should be restated in the cover letter
  \end{enumerate}
\end{itemize}

     Failure to include the necessary information may preclude
the use of an author's file.  Also, if questions arise and the
author cannot be reached for an answer in a timely fashion, then
the file may not be used (OSA will not delay publication of an
author's work in this way).  \\
%\\

     The address for submission is given below for the following
four OSA journals:
\begin{quote}
  {\it  Journal of the Optical Society of America A \\
        Journal of the Optical Society of America B \\
        Applied Optics.                             \\
        Optics Letters                     \\        }
\end{quote}
Conventional mail delivery of
the author's file on diskette to the address below is the current
mode of receipt.  In the future, electronic mail options are
expected to be available as well.
 \begin{quote}
          Optical Society of America     \\
          Manuscript Office              \\
          2010 Massachusetts Ave., N.W.  \\
          Washington, D.C.  20036-1023   \\
 \end{quote}
\begin{tabbing}
       Telephones: \= (202) 416-1916 - Manuscript Office\\
                   \> (202) 416-1903 - Technical Assistance \\
       Facsimile : \> (202) 416-6120                 \\
       E-mail: \> (Internet) osamss@pinet.aip.org -- Manuscript Office\\
    \>  fharris@pinet.aip.org -- Technical Assistance %optmss@pinet.aip.org
\end{tabbing}

     A paper copy of the manuscript is still required with
submissions on diskette.  Copy editing will still take place on
paper, and then the marked paper plus electronic file will be
sent to one of OSA's typesetters. \\
%\\

     The typesetter will examine the author's file to determine
whether it will be easier and less costly to convert and utilize
the full file or just parts of the file or to rekey the
manuscript completely.  The author's compliance with the
stylistic directions of \LaTeX\ and REV\TeX\ along with the degree of
copy editing will be the main factors affecting this decision.
In all cases, the typesetter will identify the course of action
taken by including a feedback form that the author will receive
with the proof. \\
%\\

     It may be interesting to know that the Society's typesetters
do not use REV\TeX\ or \LaTeX\ for their actual typesetting.  They
currently use Xyvision or Arbortext, which are professional,
specialty systems used by the typesetting industry.


\section{ WHERE TO TURN FOR HELP}

     Authors are expected to know the basics of \TeX\ and \LaTeX\
before using REV\TeX.  Also, authors should fully review the
README file included in REV\TeX\ 3.0 before getting started.  But
if problems or questions specific to REV\TeX\ 3.0 arise, the
following staff person from the Optical Society will advise the
OSA author:
\begin{quote}
          Frank E. Harris                 \\
          Optical Society of America      \\
          2010 Massachusetts Ave., N.W.   \\
          Washington, D.C.  20036-1023
\end{quote}

\begin{tabbing}
\hskip.5in \=  Telephone: \=  (202) 416-1903  \\
 \> Facsimile: \> (202) 416-6120   \\
 \> E-Mail: \> fharris@pinet.aip.org (Internet)--for REV\TeX\ help only \\
 \> \>  fharris@aip.org -- other inquiries.  \\
{ To FTP for files: \\ }
     \> \> ftp aip.org  \\
     \> \> anonymous    \\
     \> \> (Your Internet address) \\
     \> \> cd revtex30  \\
     \> \> mget *
\end{tabbing}
\vskip.25in

     Some inquiries may be forwarded to OSA members who have
volunteered to assist in answering REV\TeX\ technical questions.
Through this member-assisted network, it is OSA's intent to
provide satisfactory answers to all REV\TeX-related questions that
OSA authors ask.   \\
%\\

     If you are willing to participate in this member-assistance
program for REV\TeX, please get in touch with Frank E. Harris,
at the above e-mail address. \\
%\\

     If you are new to \TeX\ and \LaTeX, some books that you might
find useful are
\begin{itemize}
\item     Paul W. Abrahams, {\TeX\  \it  for the Impatient,}
          (Addison-Wesley Publishing Company, Reading,
          Massachusetts, 1990)

\item     Leslie Lamport, {\it \LaTeX\ - A Document Preparation System,}
          ( Addison-Wesley Publishing
          Company, Reading, Massachusetts, 1986)
\end{itemize}



\section{ GETTING STARTED:  BASIC TEMPLATES AND OVERVIEW}

     This is a description of the components of REV\TeX\ 3.0 that
are specific to OSA's journals.  A brief map of what files are
relevant and an overview of use are provided.  The quick-
reference guide below is intended for advanced users of REV\TeX\
3.0, while more detailed how-to-use information is given in
Section {\bf IV}.
\begin{quote}
        {\bf   NOTE:  All users of REV\TeX\ 3.0
should fully review the README  file before getting started. MS-DOS
users can use the TYPE command.\\
               Type: TYPE README: MORE [carriage return]. } \\
\end{quote}

     Users of REV\TeX\ 3.0 for OSA journals will want to use the
following files:
\begin{tabbing}
\hskip.5in \= README\hskip.5in \= - \= Brief instructions on REV\TeX\ use\\
\> revtex.sty \> - \> Main style file for all physics societies       \\
\> osa.sty    \> - \> Society-specific style file for OSA journals    \\
\> osa10.sty  \> - \> Fonts and format style file for OSA journals    \\
\> osa12.sty   \> - \> Fonts and format style file for OSA manuscripts \\
\> osabib.sty  \>  - \> Society file for bibliography style            \\
\> template.tex  \> - \> OSA template for creating a manuscript        \\
\> manosa.tex\> - \> The OSA portion of the REV\TeX\ manual, part of
                     which shows \\
\> \> \> output and corresponding REV\TeX\ input on facing pages    \\
\> sample.tex\> - \> Short excerpts of the three manuscripts listed
                     below, (about 20\% \\
\> \> \> of each original paper), with corresponding input\\
\> josaa.tex \> - \> Sample JOSA A paper (excerpts)  \\
\> josab.tex  \> - \> Sample JOSA B paper (excerpts) \\
\> aplop.tex  \> - \> Sample Applied Optics paper (excerpts)
\end{tabbing}
\vskip.25in

OSA authors will want to ``\LaTeX'' and print sample.tex, and also
the more complete sample for the journal to which they are
submitting, josaa.tex, josab.tex, or aplop.tex.  Optics Letters
authors should refer to josaa.tex or
josab.tex for a style guide, and select the josaa option in the
documentstyle command for their manuscripts. \\
%\\

A valuable tool for authors new to REV\TeX\ is the raw input files
josaa.tex, josab.tex, and aplop.tex.  These may be imported into a
word processor and printed, so that input and REV\TeX\ output can
be compared. \\
%\\

     The OSA template (template.tex) is a document file set up
and ready to use for manuscript input.  It includes all the basic
section tags and formatting commands (macros) that are relevant
to an OSA manuscript.  For a list of all available macros, please
refer to Appendix B, and for a list of symbols, see Appendix A.
It may also be helpful to scan the manuscript example provided in
Section {\bf V} to find other macros that may be useful to a
particular application. \\
%\\

     The three style files (osa.sty, osa10.sty, and osa12.sty)
will interpret the macros in terms of special layouts and fonts
for OSA journals and thus will produce a properly formatted
manuscript when printed. The file osabib.sty handles
cross-referencing and bibliographic citations and makes sure
that these are formatted according to OSA style.

\subsection{ Quick-Reference Guide }

     The following quick-reference guide may be particularly
useful for advanced REV\TeX\ users.         \\
\begin{enumerate}
 \item   See Appendix A for a list of symbols.
 \item    See Appendix B for a list of all REV\TeX\ macros in
          addition to many useful \LaTeX\ macros.  [Do not create
          and use new macros.  Use only \LaTeX\ and REV\TeX\ macros
          so the file will be usable by OSA's typesetters.]
 \item  Sections, subsections and subsubsections are supported.  It
 is also possible to suppress section numbering by putting a star
 after each command, i.e., \btt{section*}\{your name\}.
  \begin{tabbing}
\hskip.5in \= Main section heading:\hskip.5in \= \btt{section} \\
  \>  First subheading:     \>   \btt{subsection}     \\
  \>  Second subheading:    \>   \btt{subsubsection}
  \end{tabbing} \vspace{-.15in}
 \item    Delimiter for in-line math:  \$
 \item    To display and automatically number an equation, start
          with \btt{begin\{equation\}}
          and finish with \btt{end\{equation\}}.
 \item    To display and automatically number a group of
          equations, use  \btt{begin\{eqnarray\}} and
          \btt{end\{eqnarray\}}.  To
          get each equation to line up under the = or $\leq $ or
          similar sign, surround the = sign in each equation with
          \& signs.
 \item    To number displayed equations manually, use
           \btt{eqnum\{thenumber\}}.  This option would be used for
          equation ($7^{\prime} $), etc.
 \item    To number equations using letters start with
   \btt{begin\{mathletters\}} and end with  \btt{end\{mathletters\}}.
 \item    Number by section:  Put the \btt{eqsecnum} command before
          the first section.
 \item  Citations for cross referencing equations and sections use
 the same commands.  Bibliographic citations have separate commands.\\
 \begin{tabbing}
\hskip.5in\= Tag for citing equations in text:\hskip1.3in\=\btt{ref\{tag\}}\\
\> Tag for equations to be cited:  \> \btt{label\{tag\}}\\
\> Tag for citing references in text:  \>  \btt{cite\{reftag\}} \\
\> Tag for citing references ``on the line''
       in text:   \>   \btt{onlinecite\{reftag\}}  \\
\> Tag for listing references:  \>      \btt{bibitem\{reftag\}}
         \end{tabbing}
 \item    Journal name shortcuts: See Table 5 in section {\bf V}.
 \item    Place figure captions at the end of your manuscript.
          Use the commands  \btt{begin\{figure\}} and
          \btt{end\{figure\}} to start and end each figure.
          Use the command \btt{caption\{your caption here\}}
          to create and automatically
          number the caption. To label figure captions use
        \btt{label\{figureName\}}.  Numbering is automatic.
 \item   Please place your tables at the end of your
 manuscript submission.  The typesetters will put them in the
 appropriate place within the journal. \\
 \begin{tabbing}
\hskip.5in \= Start the table environment with \hskip.2in
                   \= \btt{begin\{tabular\}}, \\
\> and end with\> \btt{end\{tabular\}}. \\
\>  Within the table environment, some standard options are: \\
\>  Caption and number:\>      \btt{caption\{caption here\}}       \\
\>  Begin tables:\>            \btt{begin\{table\}\{column data\}} \\
\>  End tables:\>             \btt{end\{table\}}                  \\
\>  Make a horizontal rule:\>  \btt{tableline}                     \\
\>  Column headings:\>         \btt{multicolumn\{\}\{\}}           \\
\>  Footnotes:\>               \btt{tablenote\{note here\}}
 \end{tabbing}
\end{enumerate}

\section{ DETAILED HOW-TO-USE INFORMATION}
\begin{quote}
               NOTE:  Do not create and use new macros.
               Use only \LaTeX\ and REV\TeX\ macros so
               the file will be usable to OSA's typesetters.
\end{quote}
\subsection{  Title, Authors, Affiliation, Abstract}

     The document template for OSA (template.tex) already
contains the basic macros for the early parts of any manuscript:
the title, author(s), affiliation(s), and abstract.

\subsection{  Text}

     Paragraphs always begin with a blank input line.  Unless a
hyphen is required in a word and that hyphen falls at the end of
a line as you type it, do not hyphenate a word at the end of a
line; REV\TeX\ will do this.  Continue to hyphenate modifiers
within a line of text, e.g., ``electro-optical devices."\\
%\\

     Use 2 single curly quotes for quotation marks around quoted
text (``xxx''), not straight quotes ("xxx").  For opening quotes
this is two octal 140 characters (hex 60, near the top left on
most keyboards); for closing quotes, this is two octal 047 (hex
27) characters.  \\
%\\

Don't use \btt{smallskip}, \btt{bigskip}, or any other vertical motion
commands.  Horizontal motion commands are unnecessary as well. \\
%\\

     Authors should avoid the use of specially designed "define
characters" and choose symbols from those shown in the \LaTeX\
User's Guide \& Reference Manual or in Appendix A of this REV\TeX\
Author's Guide.  There is no guarantee that a specially designed
definition will produce the desired results at the typesetter's
production facility.
     If a special symbol is required and not listed in the \LaTeX\
User's Guide \& Reference Manual or in Appendix A of this REV\TeX\
Author's Guide, please request special consideration in the cover
letter accompanying the file submittal.  The copy editor will
make note of it, and the typesetter will attempt to accommodate
the author.  {\it  Use of unusual characters is subject to
approval by the managing editor. }


\subsection{  Section Headings }

     Three levels of headings are provided in REV\TeX:  section,
subsection, and subsubsection.  Precede the section heading with
the * command to suppress the automatic numbering; e.g., \\
%\\

\hskip.5in      \btt{section*\{Introduction\}}  \\
%\\

To label a section heading for cross referencing use the
\btt{label} command after the heading; e.g., \\
%\\

\hskip.5in  \btt{section\{Introduction\}} \btt{label\{sec:intro\}}\\
%\\

\subsection{  Math Within Paragraphs }

     REV\TeX\ uses the delimiter \$ for any in-line math, e.g.,
the quantity, $a^z$, is obtained from the input, \verb+  $a^z$. +\\
%\\

     Another example of in-line math is\\
     %\\

\begin{tabbing}
\hskip.5in\= output:\= ... difference between $\langle J_z \rangle (t)$ and
          $\langle J-z \rangle_{\rm HF}(t) ...$  \\
   \>       \\
\> input:\> ... difference between \$\btt{langle} J\_z \btt{rangle}  \\
\> \> (t)\$ and \$\btt{langle} J\_z \btt{rangle}\_\{\btt{rm} HF\}(t) ...\$\\
\end{tabbing}

     Standard mathematical style conventions are followed for
in-line math, such as smaller point size for
superscripts/subscripts, appropriate use of roman, italic, greek,
and script fonts, and the use of special math symbols.  See
Appendix A for a list of available symbols. \\
%\\

     \TeX\ will take care of setting the point size appropriate for
variables and numbers in the superscript/subscript position. \\
%\\

     \TeX\ makes latin letters within math italic by default.
These are commonly used for variables.   To get the roman font,
commonly used for functions such as exp and erf, use the \btt{rm}
command. \\
%\\

\begin{tabbing}
\hskip.5in \= input:\hskip.5in \= \verb+ $... {\rm exp}(t^2 - t) ...$ + \\
\> \\
\> output: \>   $... {\rm exp}(t^2 - t) ...$ \\
\end{tabbing}
%\\

     \TeX\ will assume that you want the superscript or subscript
to consist of the first token (generally a single character or
command) following the \verb+^{hat} or _{en dash}+ unless you use curly
brackets to delimit the superscript/subscript.  It is safest to
use the curly brackets if unsure.  The curly brackets here also
serve to limit the scope of the \btt{rm} command.  Again, don't use
any vertical or horizontal motion commands in math.

\subsection{  Displayed Equations }

     The most common (and preferred) type of displayed equation
is a narrow, indented single-line equation, with an equation
number on the same line.  Try to set as many equations as you can
in this way.  Specifically, use a solidus instead of a built-up
fraction whenever possible.


\subsubsection{  Numbering displayed equations }

     REV\TeX\ 3.0 permits two methods for numbering equations.  You
can assign your own equation numbers or you can allow REV\TeX\ to
number for you.  Equation numbers are not mandatory, and numbered
and unnumbered equations may be intermixed. \\
%\\

     Use the command \btt{eqnum\{thenumber\}} to number on your own.
You can also use this command to produce a specific equation
number not normally obtainable, ($1^{\prime}$), for example.  \\
%\\

     For automatically numbered single-line and multiline
equations, use the equation and eqnarray environments.  You can
use the \verb+\[,\]+ commands and the eqnarray* environment for
unnumbered single-line and multiline equations, respectively.
The command \btt{nonumber} will suppress the numbering on a single
line of an eqnarray.        \\
%\\

     If you wish a series of equations to be a lettered sequence,
e.g., (1a), (1b), and (1c), just include the equations or
eqnarray within the mathletters environment.    \\
%\\

     Finally, to have REV\TeX\ number equations by section, use the
eqsecnum style option.  See the sample manuscript in Section {\it V}
for illustrations of these equation numbering options.

\subsubsection{  Cross referencing displayed equations }

     Authors will probably not cross reference every equation in
text.  When a numbered equation needs to be referred to in text
by its number, the \btt{label\{tag\}} and \btt{ref\{tag\}}
commands should be used.  The \btt{label} command is used within
the equation or the eqnarray line to be referenced. \\
%\\
\begin{quote}
input: \\   \btt{begin\{equation\}} \\
\verb"{\partial^2 \psi \over \partial x^2} +
{\partial^2 \psi \over \partial y^2} +
{\partial^2 \psi \over \partial z^2} = -
{\partial \psi \over \partial t}" \\
\btt{label\{schroedinger\}} \\
          \btt{end\{equation\}}    \\
 ... It follows from Eq.\btt{} (\btt{ref\{schroedinger\}}) that this is the
          case... \\
\end{quote}  \hskip.15in

output:

\begin{equation}
{\partial^2 \psi \over \partial x^2} +
{\partial^2 \psi \over \partial y^2} +
{\partial^2 \psi \over \partial z^2} = -
{\partial \psi \over \partial t}
\label{schroedinger}
\end{equation}

\begin{quote}
          ... It follows from Eq.\ (\ref{schroedinger}) that this is the
          case...
\end{quote}
%\\

     Please note the parentheses surrounding the command.  They
are necessary for proper output.  You can also label individual
lines in an eqnarray.  Numbers produced with \btt{eqnum} can also be
cross referenced; just follow the \btt{eqnum} command with a \btt{label}
command.  See section F.  Footnotes and References  for further
information.


\subsection{  Footnotes and References  }
\begin{tabular}{|cl|}  \hline
 \hskip.5in  Quick & Guide to References \\
 \hline
 1. & The \btt{bibitem} command begins a reference. \\
 2. & References must be listed in the reference section in the order \\
    & in which they are first cited, in text, figures, or tables. \\
 3. & References will automatically be numbered by REV\TeX\
        in the $\,\,\,\,\,\,\,\,$\\
    &   order in which they occur in the reference section, unless \\
    &   the author provides his/her own label.  \\
 \hline
\end{tabular}
\vskip.5in

     The list of references should appear after the main body of
the paper.  References must be numbered consecutively in the
order of their first citation.  Please refer to recent issues of
the OSA journals for current style.  The sample manuscript in
Section {\bf V} also gives some examples of a variety of reference
entries.   \\
%\\

     List the references in the reference section by using the
\btt{bibitem} command, and cite them in the text by using
the \btt{cite} or \btt{onlinecite} command.  A present-address
footnote should appear immediately above
the reference section.  If there are acknowledgments, the
present-address footnote should be the last item in the
acknowledgments section. \\
%\\

     Here is a sample reference.\\
%\\

\begin{tabbing}
\hskip.5in\= input: \hskip.5in \=\btt{bibitem\{homer91\}}G.
            Homer and B. T. Rogers, \\
   \> \>    \btt{ao} \{\btt{bf} 30,\} 5002-5004 (1991).\\
 \> \\
\> output: \>  1.   G. Homer and B. T. Rogers, \ao {\bf 30,} 5002-
               5004 (1991). \\
\end{tabbing}
%\\

     {homer91} is a tag.  It can be any string of letters and
numbers that you will easily associate with the reference.  This
tag will be used in text to tell \TeX\ what reference you want to
cite.  See the example below.\\
%\\

\begin{tabbing}
\hskip.5in\= input:\hskip.5in\= ... This has been noted
                          previously \btt{cite\{homer91\}}.\\
\> \\
\> output: \>  ... This has been noted previously$^1$. \\
\end{tabbing}
%\\

     The macro \btt{ao} in the above example expands to Appl. Opt.,
the standard abbreviation for Applied Optics.  OSA has provided
macros for the most common journal abbreviations used in OSA
publications.   The macros save typing and improve the consistent
spelling of references.   For a complete listing see Table 5 at
the end of the sample document in \\ Section {\it V}.
%\\

%\subsection{ Cross Referencing }

     REV\TeX\ has built-in features for autonumbering of section
headings, equations, tables, and figures.  Cross referencing
depends on the use of tags that are defined by the user.  Tags
are strings of characters that identify the equations, tables,
and figures for the purposes of the user and so that the user
doesn't have to know what number REV\TeX\ has automatically
assigned to the item.  The \btt{label} command is used to identify
tags for \TeX.                      \\
%\\

     You will need to \LaTeX\ the original file more than once to
ensure that the tags have been properly linked to appropriate
numbers.  If you add any tags, you will need to \LaTeX\ more than
once in subsequent work sessions.  \LaTeX\ will display an error
message that ends with

\begin{center} \ldots Rerun to get cross-reference right.
\end{center}

     If you see that message, \LaTeX\ the file again.  If the error
message appears after two \LaTeX ings, please check your labels.
You probably have referred to an item in text without tagging the
item.                \\
%\\

     You may not need to know (or care to know) all about what
\LaTeX\ is doing for autonumbering; however, you may want to know
that when you \LaTeX\ the file for the first time, an auxiliary
file with the .aux filename extension will be created that
connects numbers with their tags.  Subsequent \LaTeX ing accesses
the auxiliary file to put the proper number in the text.
% End of Cross Referencing subsection...


\subsection{  Figure Captions}

     Figure captions are a part of the electronic manuscript and
should appear after the references.  They should be input
sequentially in the order in which they are cited in the text;
REV\TeX\  will label and number the captions Fig. 1, Fig. 2, etc.
Please place the figure captions at the end of the manuscript. \\
%\\

     The \btt{label} command is used to cross reference figures
in the text.  This command is inserted after the text of the figure
caption and before the final curly bracket.\\
%\\

\begin{tabular}{ll}
input: &   \btt{figure}\{Text of first caption.\btt{label\{fig1\}\}}  \\
       &   \btt{figure}\{This is the second caption: high-pressure Xe- \\
       &   lamp spectrum as seen at the output of the Mach-Zehnder    \\
       &   interferometer\}.\btt{label\{fig2\}}                       \\
       &                                                              \\
output:  &    Fig. 1.  Text of first caption.                         \\
       &                                                              \\
       &      Fig. 2.  This is the second caption: high-pressure      \\
       &   Xe-lamp spectrum as seen at the output of the Mach-        \\
       &   Zehnder interferometer.                                    \\
       & \\
\end{tabular} \\

Figures are cited in text with the use of the \btt{ref} command. \\
%\\

\begin{tabular}{ll}
input: &   ... It can be seen from Fig. \btt{ref}\{fig1\} that the data\\
       &   are inconsistent with this conclusion...    \\
       & \\
output:&   ... It can be seen from Fig. 1 that the data are \\
       &   inconsistent with this conclusion... \\
       & \\
\end{tabular} \\
%\\


     Figures and illustrations are submitted as originals or
glossy prints.  Follow the rules elaborated in the Information
for Contributors sections that appear in most issues of OSA's
journals.

\subsection{  Tables }

     Tables are a part of the electronic manuscript and should
appear at the end of the file.  Every table must have a complete
title and the correct number of descriptive column headings.
Please set all tables within a \btt{begin\{tabular\}} and an
\btt{end\{tabular\}} command.  Each individual table must begin with
\btt{begin\{table\}}, and end with \btt{end\{table\}}. \\
%\\

     Tables are set to 15.25 cm wide in OSA's manuscript style of
REV\TeX.  Any usage of the \btt{narrowtext},  \btt{mediumtext},
and \btt{widetext} commands will simply be ignored, as these
are not relevant to OSA's manuscript style.  \\
%\\

     See examples of tables in a current OSA journal issue for
the placement of table lines.  The table commands will set single
horizontal lines appearing at the beginning and end of the table.
A single horizontal rule should be set after the column headings
with the use of the \btt{tableline} command.  Extra sets of column
headings within the table will require another \btt{tableline} to
separate the headings from the column entries.  Do not insert any
other horizontal or vertical lines in the body of the table.  \\
%\\

     Since tables are automatically numbered, the \btt{label} command
is used with the \btt{ref} command to cite tables in the text.  The
\btt{label} command should appear after the \btt{end\{tabular\}} and
before the \btt{end\{table\}} command. \\ \hskip.15in

\begin{center}    Special Table Considerations
\end{center}
\begin{enumerate}
\item  Numerical columns should align on the decimal point.
              Use the d alignment designator.  \\
\item   Use \$ delimiters for all math in a table (no
               displayed equation commands). \\
\item  Footnotes in a table will automatically be labeled
               a, b, c, etc.  \\
\item  Extra wide tables that will not fit into the
               15.25-cm width provided  can be compressed by
               using the \btt{squeezetable} command.
\end{enumerate}


\section{  INPUT AND OUTPUT FROM A SAMPLE MANUSCRIPT}

     The following pages illustrate a short annotated sample of
an OSA manuscript created with REV\TeX.  Both the input and output
are presented for comparison purposes. We welcome your suggestions
of sample formats or conventions
that might be added to this manuscript example in the future to
make it more useful to OSA's authors.  Please send your ideas to
\begin{quote}
          Frank E. Harris  \\
          Optical Society of America   \\
          2010 Massachusetts Ave., N.W.\\
          Washington, D.C.  20036-1023 \\
\end{quote}
\begin{quote}
     Telephone:  (202) 416-1903   \\
     Facsimile:  (202) 416-6102   \\
     E-Mail: fharris@aip.org (Internet) \\
\end{quote}


In addition to examining the following sample, many OSA authors
will find valuable examples in the files sample.tex, josaa.tex,
josab.tex, and aplop.tex.  Optics Letters
authors can refer to josaa.tex or
josab.tex for a style guide, and select the josaa option in the
documentstyle command for their manuscripts. \\
%\\

     The OSA template (template.tex) is a document file set up
and ready to use for manuscript input.  It includes all the basic
section tags and formatting commands (macros) that are relevant
to an OSA manuscript. \\
%\\


**************************************************************************


     The Optical Society of America thanks you for your participation
in the {\it REVTEX} electronic manuscript activity.  Your electronic
submissions will help the Society improve its capabilities and
competencies to serve the OSA technical community even better in the
future.  Thank you.



\newpage
\setcounter{eqletter}{0}       \setcounter{equation}{0}
\setcounter{figure}{0}
% \end{document}
% This is the file MANUAL2.TEX , MODIFIED TO TEST OSA MACROS
% from version 3.0 of the REVTeX macro package.
%                *** !!!!!!!! 3.0 !!!!!!!! ***
% This file is part of a compuscript toolbox distributed by
% the American Physical Society in conjunction with
% the \TeX\ author-prepared program.
%
% All rights not specifically granted are reserved.
%
% Copyright (c) 1992, Optical Society of America.
%
% For more information, see the README file.
%
%Filename: manual2.tex
%\documentstyle[manuscript,osa]{revtex}
%\def\btt#1{{\tt$\backslash$#1}}
%
%
%\newcommand{\MF}{{\large{\manual META}\-{\manual FONT}}}
%\newcommand{\manual}{rm}  % Replace the manual font with rm (Roman)
%\newcommand\bs{\char '134 }  % add backslash char to \tt font
%
%\begin{document}
%\title{Title of Manuscript}
\vskip.5in
\begin{center}{ \rm \Large \bf Title of Manuscript}\end{center}
%\author{A. A.  Author and B. B. Author}
\vskip.5in \begin{center} A. A.  Author and B. B. Author\end{center}
%\address{Authors' institution and/or address}
\begin{center}\it  Authors' institution and/or address\end{center}
%\author{C. C. Author}
\vskip.5in \begin{center} C. C. Author\end{center}
%\address{Second author institution and/or address}
\begin{center} \it  Second author institution and/or address\end{center}

%\maketitle       %% ESSENTIAL IF TITLE, AUTHOR, OR DATE ARE TO APPEAR

\begin{abstract}
In this version of the manual's sample document, each page of \TeX\
output will be followed by a page in \btt{verbatim} mode showing
the input that produced the facing page.   This provides examples
of almost  everything an OSA author needs to know to produce an
article.   The facing page format makes it especially convenient.
\end{abstract}


\section*{1. First-level heading:}
\label{sec:level1}

\baselineskip = 2\baselineskip
Here is the first sentence in Section  1, demonstrating
section cross-referencing. Note that this sample file was run
without the eqsecnum option selected.  OSA accepts equation numbers
of the form (1), (2), etc.  If you wish to use numbers of the form
(1.1), (1.2), etc., use the eqsecnum option.

The author will not know the received date when the compuscript
is first submitted; production will insert this. Every
article includes an abstract.  The abstract is a concise
summary of the work covered at length
in the main body of the article.
It is used for secondary publications and for information retrieval
purposes.  OSA will enter the received date.

\newpage
\baselineskip = .5\baselineskip  % single space the verbatim
\begin{verbatim}
% This is the file MANUAL2.TEX , MODIFIED TO TEST OSA MACROS
% from version 3.0 of the REVTeX macro package.
%                *** !!!!!!!! 3.0 !!!!!!!! ***
% This file is part of a compuscript toolbox distributed by
% APS and OSA in conjunction with the TeX author-prepared program.
% All rights not specifically granted are reserved.
% Copyright (c) 1992 Optical Society of America.
%
% For more information, see the README file.
% Filename: manual2.tex
\documentstyle[manuscript,osa]{revtex}
\def\btt#1{{\tt$\backslash$#1}}
\newcommand{\MF}{{\large{\manual META}\-{\manual FONT}}}
\newcommand{\manual}{rm}  % Replace the manual font with rm (Roman)
\newcommand\bs{\char '134 }  % add backslash char to \tt font
%
\begin{document}

\title{Title of Manuscript}
\author{A. A.  Author and B. B. Author}
\address{Authors' institution and/or address}

\author{C. C. Author}
\address{Second author institution and/or address}


\maketitle       %% ESSENTIAL IF TITLE, AUTHOR, OR DATE ARE TO APPEAR

\begin{abstract}
In this version of the manual's sample document, each page of \TeX\
output will be followed by a page in \btt{verbatim} mode showing
the input that produced the facing page.   This provides examples
of almost  everything an OSA author needs to know to produce an
article.   The facing page format makes it especially convenient.
\end{abstract}

\section{First-level heading:}
\label{sec:level1}

Here is the first sentence in Section \ref{sec:level1}, demonstrating
section cross-referencing. Note that this sample file was run
without the eqsecnum option selected.  OSA accepts equation numbers
of the form (1), (2), etc.  If you wish to use numbers of the form
(1.1), (1.2), etc., use the eqsecnum option.


The author will not know the received date when the compuscript
is first submitted; production will insert this. Every
article includes an abstract.  The abstract is a concise
summary of the work covered at length
in the main body of the article.
It is used for secondary publications and for information retrieval
purposes.  OSA will enter the received date.

\end{verbatim} \newpage
\baselineskip = 2\baselineskip  % back to double space

\subsection*{A. Second-level heading:}
\label{sec:level2}

Here is the first sentence in Section 1 A, demonstrating
section cross-referencing.
The commands \btt{section} and \btt{subsection} are used to start
sections and subsections. You should follow the section command with
the section title, enclosed in curly brackets.

If you wish to cross-reference a section,
follow the section command with a \btt{label\{ Your-section-name\}}
command. A blank input line tells \TeX\ that a new paragraph begins.

Reference citations in text use the command \btt{cite}.
In the reference section of this paper
each reference is ``tagged'' by a string (in curly brackets).
The proper form for citing in text is
\btt{cite}$\{${\it string}$\}$,
and the result is shown here.\cite{smith82,jones78}
We will cite  other people \cite{smith82,jonessmith80}
and journals here. We also cite other people again (Refs.\
\onlinecite{smith82} and \onlinecite{jonessmith80}).  To get
reference numbers that appear on the line, use the \btt{onlinecite}
command.

It is worth mentioning that  REV\TeX\ ``collapses'' lists
of reference numbers where possible.  We now cite
everyone together, \cite{smith82,jones78,jonessmith80} and once again
(Refs.\ \onlinecite{smith82,jones78,jonessmith80}).

\section*{2. Displayed equations}
\subsection*{A. Another second-level heading}
\subsubsection*{1. Third-level heading:}
\label{sec:level3}

Here is the first sentence in Section 2 A 1, demonstrating
section cross-referencing.
In \LaTeX\ there are many different ways to display equations, and a
few preferred ways are noted below.

Below we have indented, single-line equations with numbers; this is
the most common type of equation in {\bf OSA} journals:
\begin{equation}
\chi_+(p)\alt{\bf [}2|{\bf p}|(|{\bf p}|+p_z){\bf ]}^{-1/2}
\left(
\begin{array}{c}
|{\bf p}|+p_z\\
px+ip_y
\end{array}\right)\;,
\end{equation}

\newpage
\baselineskip = .5\baselineskip  % single space the verbatim
\begin{verbatim}
\subsection{Second-level heading:}
\label{sec:level2}

Here is the first sentence in Section \ref{sec:level2}, demonstrating
section cross-referencing.
The commands \btt{section} and \btt{subsection} are used to start
sections and subsections. You should follow the section command with
the section title, enclosed in curly brackets.

If you wish to cross-reference a section,
follow the section command with a \btt{label\{ Your-section-name\}}
command. A blank input line tells \TeX\ that a new paragraph begins.

Reference citations in text use the command \btt{cite}.
In the reference section of this paper
each reference is ``tagged'' by a string (in curly brackets).
The proper form for citing in text is
\btt{cite}$\{${\it string}$\}$,
and the result is shown here \cite{smith82,jones78}.
We will cite  other people \cite{smith82,jonessmith80}
and journals here. We also cite other people again (Refs.\
\onlinecite{smith82} and \onlinecite{jonessmith80}).  To get
reference numbers that appear on the line, use the \btt{onlinecite}
command.

It is worth mentioning that  REV\TeX\ ``collapses'' lists
of reference numbers where possible.  We now cite
everyone together, \cite{smith82,jones78,jonessmith80} and once again
(Refs.\ \onlinecite{smith82,jones78,jonessmith80}).


\section{Displayed equations}
\subsection{Another second-level heading}
\subsubsection{Third-level heading:}
\label{sec:level3}

Here is the first sentence in Section\ \ref{sec:level3}, demonstrating
section cross-referencing.
In \LaTeX\ there are many different ways to display equations, and a
few preferred ways are noted below.

Below we have indented, single-line equations with numbers; this is
the most common type of equation in {\bf OSA} journals:
\begin{equation}
\chi_+(p)\alt{\bf [}2|{\bf p}|(|{\bf p}|+p_z){\bf ]}^{-1/2}
\left(\begin{array}{c}
|{\bf p}|+p_z\\
px+ip_y
\end{array}\right)\;,
\end{equation}
\end{verbatim} \newpage
\baselineskip = 2\baselineskip  % back to double space
\begin{equation}
\left\{\openone234567890abc123\alpha\beta\gamma\delta%
1234556\alpha\beta{1\sum^{a}_{b}\over A^2}\right\}\label{one},
\end{equation}
Note the outline numeral one in Eq.\ (\ref{one}).
If the equation is a little wider, the equation number automatically
moves down to the next line:
\begin{equation}
\left\{abc1234567890abc1234\alpha\beta\gamma\delta%
1234556\alpha\beta{1\sum^{a}_{b}\over A^2}
abc1234567890abc1234\alpha\beta\gamma\delta%
1234556\alpha\beta{1\sum^{a}_{b}\over A^2}\right\}.
\end{equation}

When the \btt{label} command is used [cf. input
for Eq. (\ref{one})],
the equation can be referred to in text without your knowing the
equation number that \TeX\ will assign to it.

Math will be flush left by default, in OSA submissions.
It should allow longer equations to be displayed before line numbers
are displayed below.

\begin{equation}
\left\{ab12345678abc123456abcdef\alpha\beta\gamma\delta%
1234556\alpha\beta{1\sum^{a}_{b}\over A^2}\right\},
\end{equation}
\begin{equation}
\epsilon^\ast_\mu(p)\to c(V)D_V
\sum_\tau c^f_\tau \bar u(f)P_\tau
\gamma_\mu v( \bar f)\;
[\epsilon_jl_i\epsilon_i]_{\sigma_1}\chi_{\sigma_1}(p_1)\;.
\end{equation}


If you have a single-line equation that you don't want
numbered, you can use the \btt{[}, \btt{]} format:
\[g^+g^+ \rightarrow g^+g^+g^+g^+ \dots ~,~~q^+q^+\rightarrow
q^+g^+g^+ \dots ~. \]

\subsubsection*{2. Multiline equations}

Multiline equations are obtained by using the
\btt{begin$\{$eqnarray$\}$}, \btt{end$\{$eqnarray$\}$} format.
Use the \btt{nonumber}
command at the end of each line where you do not want a number:
\begin{eqnarray}
{\cal M}=&&ig_Z^2(4E_1E_2)^{1/2}(l_i^2)^{-1}
\delta_{\sigma_1,-\sigma_2}
(g_{\sigma_2}^e)^2\chi_{-\sigma_2}(p_2)\nonumber\\
&&\times
[\epsilon_jl_i\epsilon_i]_{\sigma_1}\chi_{\sigma_1}(p_1),  \\
\sum \vert M^{\rm viol}_g \vert ^2&=&g^{2n-4}_S(Q^2)~N^{n-2}
        (N^2-1)\nonumber \\
 & &\times \left( \sum_{i<j}\right)
  \sum_{\rm perm}
 {1 \over S_{12}}
 {1 \over S_{12}}\sum_\tau c^f_\tau~.
\end{eqnarray}

\newpage
\baselineskip = .5\baselineskip  % single space the verbatim
\begin{verbatim}
\begin{equation}
\left\{\openone234567890abc123\alpha\beta\gamma\delta%
1234556\alpha\beta{1\sum^{a}_{b}\over A^2}\right\}\label{one},
\end{equation}
Note the open one in Eq.\ (\ref{one}).
If the equation is a little wider, the equation number automatically
moves down to the next line:
\begin{equation}
\left\{abc1234567890abc1234\alpha\beta\gamma\delta1234556\alpha\beta%
{1\sum^{a}_{b}\over A^2}abc1234567890abc1234\alpha\beta\gamma\delta%
1234556\alpha\beta{1\sum^{a}_{b}\over A^2}\right\}.
\end{equation}
When the \btt{label} command is used [cf. input for Eq. (\ref{one})],
the equation can be referred to in text without your knowing the
equation number that \TeX\ will assign to it.

Math will be flush left by default, in OSA submissions.
It should allow longer equations to be displayed before line numbers
are displayed below.

\begin{equation}
\left\{ab12345678abc123456abcdef\alpha\beta\gamma\delta%
1234556\alpha\beta{1\sum^{a}_{b}\over A^2}\right\},
\end{equation}
\begin{equation}
\epsilon^\ast_\mu(p)\to c(V)D_V \sum_\tau c^f_\tau \bar
u(f)P_\tau \gamma_\mu v( \bar f)\;
[\epsilon_jl_i\epsilon_i]_{\sigma_1}\chi_{\sigma_1}(p_1)\;.
\end{equation}

If you have a single-line equation that you don't want
numbered, you can use the \btt{[}, \btt{]} format:
\[g^+g^+ \rightarrow g^+g^+g^+g^+ \dots ~,~~q^+q^+\rightarrow
q^+g^+g^+ \dots ~. \]

\subsubsection{Multiline equations}
Multiline equations are obtained by using the
\btt{begin$\{$eqnarray$\}$}, \btt{end$\{$eqnarray$\}$} format.
Use the \btt{nonumber}
command at the end of each line where you do not want a number:
\begin{eqnarray}
{\cal M}=&&ig_Z^2(4E_1E_2)^{1/2}(l_i^2)^{-1}
\delta_{\sigma_1,-\sigma_2}
(g_{\sigma_2}^e)^2\chi_{-\sigma_2}(p_2)\nonumber\\
&&\times [\epsilon_jl_i\epsilon_i]_{\sigma_1}\chi_{\sigma_1}(p_1), \\
\sum \vert M^{\rm viol}_g \vert ^2&=&g^{2n-4}_S(Q^2)~N^{n-2}
        (N^2-1)\nonumber \\
 & &\times \left( \sum_{i<j}\right)
  \sum_{\rm perm}
 {1 \over S_{12}}
 {1 \over S_{12}}\sum_\tau c^f_\tau~.
\end{eqnarray}
\end{verbatim} \newpage
\baselineskip = 2\baselineskip  % back to double space


If you wish to set a multiline equation without any line numbers,
you can use the \verb+\begin{eqnarray*}+,
\verb+\end{eqnarray*}+ format:
\begin{eqnarray*}
\sum \vert M^{\rm viol}_g \vert ^2&=&g^{2n-4}_S(Q^2)~N^{n-2}
        (N^2-1)\\
 & &\times \left( \sum_{i<j}\right)
 \left( \sum_{\rm perm}
 {1 \over S_{12}S_{23}S_{n1}}\right)
 {1 \over S_{12}}~.
\end{eqnarray*}
To obtain numbers not normally produced by the automatic numbering,
use the \verb+\eqnum{#1}+ command, where \verb+#1+ is the desired
equation number. For example, to get an equation number of
(\ref{eq:mynum}),
\begin{equation}
g^+g^+ \rightarrow g^+g^+g^+g^+ \dots ~,~~q^+q^+\rightarrow
q^+g^+g^+ \dots ~. \eqnum{2.7$'$}\label{eq:mynum}
\end{equation}

{\it A few notes on} \verb=\eqnum=.
The \verb+\eqnum+ must come before the \verb+\label+, if any.
The numbering set with \verb+\eqnum+ is {\it transparent} to the
automatic numbering in REV\TeX; therefore
you must know the number ahead of time and {\it must\/} make
sure that the number set with \verb+\eqnum+ stays in step
with the automatic numbering.
\verb+\eqnum+ works with both single-line and multiline equations.
You could, if you wished, do all the numbering in a paper
manually with \verb+\eqnum+.
Enclosing single-line and multiline equations in \btt{begin\{mathletters\}}
and \btt{end\{mathletters\}} will produce
a set of equations that are ``numbered'' with letters, as shown
in Eqs.\ (\ref{mlett:1}) and (\ref{mlett:2}) below:
\begin{mathletters}
\begin{equation}
\left\{abc123456abcdef\alpha\beta\gamma\delta%
1234556\alpha\beta{1\sum^{a}_{b}\over A^2}\right\},\label{mlett:1}
\end{equation}
\begin{eqnarray}
{\cal M}=&&ig_Z^2(4E_1E_2)^{1/2}(l_i^2)^{-1}
(g_{\sigma_2}^e)^2\chi_{-\sigma_2}(p_2)\nonumber\\
&&\times
[\epsilon_i]_{\sigma_1}\chi_{\sigma_1}(p_1).\label{mlett:2}
\end{eqnarray}
\end{mathletters}

\newpage
\baselineskip = .5\baselineskip  % single space the verbatim
\begin{verbatim}
If you wish to set a multiline equation without any line numbers,
you can use the \verb+\begin{eqnarray*}+,
\verb+\end{eqnarray*}+ format:
\begin{eqnarray*}
\sum \vert M^{\rm viol}_g \vert ^2&=&g^{2n-4}_S(Q^2)~N^{n-2}
        (N^2-1)\\
 & &\times \left( \sum_{i<j}\right)
 \left( \sum_{\rm perm}
 {1 \over S_{12}S_{23}S_{n1}}\right)
 {1 \over S_{12}}~.
\end{eqnarray*}
To obtain numbers not normally produced by the automatic numbering,
use the \verb+\eqnum{#1}+ command, where \verb+#1+ is the desired
equation number. For example, to get an equation number of
(\ref{eq:mynum}),
\begin{equation}
g^+g^+ \rightarrow g^+g^+g^+g^+ \dots ~,~~q^+q^+\rightarrow
q^+g^+g^+ \dots ~. \eqnum{2.7$'$}\label{eq:mynum}
\end{equation}

{\it A few notes on} \verb=\eqnum=.
The \verb+\eqnum+ must come before the \verb+\label+, if any.
The numbering set with \verb+\eqnum+ is {\it transparent} to the
automatic numbering in REV\TeX; therefore
you must know the number ahead of time and {\it must\/} make
sure that the number set with \verb+\eqnum+ stays in step
with the automatic numbering.
\verb+\eqnum+ works with both single-line and multiline equations.
You could, if you wished, do all the numbering in a paper
manually with \verb+\eqnum+.
Enclosing single-line and multiline equations in \btt{begin\{mathletters\}}
and \btt{end\{mathletters\}} will produce
a set of equations that are ``numbered'' with letters, as shown
in Eqs.\ (\ref{mlett:1}) and (\ref{mlett:2}) below:
\begin{mathletters}
\begin{equation}
\left\{abc123456abcdef\alpha\beta\gamma\delta%
1234556\alpha\beta{1\sum^{a}_{b}\over A^2}\right\},\label{mlett:1}
\end{equation}
\begin{eqnarray}
{\cal M}=&&ig_Z^2(4E_1E_2)^{1/2}(l_i^2)^{-1}
(g_{\sigma_2}^e)^2\chi_{-\sigma_2}(p_2)\nonumber\\
&&\times
[\epsilon_i]_{\sigma_1}\chi_{\sigma_1}(p_1).\label{mlett:2}
\end{eqnarray}
\end{mathletters}
\end{verbatim} \newpage
\baselineskip = 2\baselineskip  % back to double space

\subsection*{B. New Printing Characters for OSA}
The following are composite characters, created by short macros
included in the osa.sty style file.
\begin{eqnarray*}
\begin{array}{rrrr}
\,\,\,\,\,\, \succsim & \,\,\,\,\,\, \precsim &
\,\,\,\,\,\, \corresponds & \\
\,\,\,\,\,\, \lambdabar & \,\,\,\,\,\,  \slantfrac{1}{2}, &
\,\,\,\,\,\, \slantfrac{2}{3},& \,\,\, {\rm etc.}
\end{array}
\end{eqnarray*}
The macros to produce these special characters are shown in Table
\ref{table1}, at the end of this document.

\section*{3. Cross-referencing}
REV\TeX\ will automatically number sections, equations,
figure captions, and tables. In order to
reference them in text, use the \btt{label} and \btt{ref}
commands.

The \btt{label} command appears following a section heading
or within a equation, figure
caption, or table; the \btt{ref} command appears in text
where citation is to occur.  We will refer to the first
figure (Fig.~\ref{autonum}) here.
We can refer to the ``late figure'' also (Fig.~\ref{latefigure}).

References to figures: Fig.~\ref{autonum}, Fig.~\ref{latefigure}
and Fig.~\ref{reduced}.

References to tables:
Table \ref{table1},
Table \ref{table2},
Table \ref{table3},   and
Table \ref{latetable}.
{\bf OSA}
style requires that the initial citation of figures or tables
be in numerical order in text, so don't cite Fig.~\ref{reduced}
until you've cited Fig.~\ref{latefigure}.
See {\it Style and Notation
Guide}.

\section*{4. Tables}
In manuscript format, tables should be placed at the end of the submission,
after the figure captions. The tables that were once in this location
have been placed at the end of the document.


\newpage
\baselineskip = .5\baselineskip  % single space the verbatim
\begin{verbatim}
\subsection{New Printing Characters for OSA}
The following are composite characters, created by short macros
included in the OSA.STY style file.
\begin{eqnarray*}
\begin{array}{rrrr}
\,\,\,\,\,\, \succsim & \,\,\,\,\,\, \precsim &
\,\,\,\,\,\, \corresponds &\\
\,\,\,\,\,\, \lambdabar & \,\,\,\,\,\,  \slantfrac{1}{2} &
\,\,\,\,\,\, \slantfrac{2}{3},& \,\,\, {\rm etc.}
\end{array}
\end{eqnarray*}

The macros to produce these special characters are shown in Table
\ref{table1}, at the end of this document.

\section{Cross-referencing}
REV\TeX\ will automatically number sections, equations,
figure captions, and tables. In order to
reference them in text, use the \btt{label} and \btt{ref}
commands.

The \btt{label} command appears following a section heading
or within a equation, figure
caption, or table; the \btt{ref} command appears in text
where citation is to occur.  We will refer to the first
figure (Fig.~\ref{autonum}) here.
We can refer to the ``late figure'' also (Fig.~\ref{latefigure}).

References to figures: Fig.~\ref{autonum}, Fig.~\ref{latefigure}
and Fig.~\ref{reduced}.

References to tables:
Table \ref{table1},
Table \ref{table2},
Table \ref{table3},   and
Table \ref{latetable}.
{\bf OSA}
style requires that the initial citation of figures or tables
be in numerical order in text, so don't cite Fig.~\ref{reduced}
until you've cited Fig.~\ref{latefigure}.
See {\it Style and Notation
Guide}.

\section{Tables}
In manuscript format, tables should be placed at the end of the submission,
after the figure captions. The tables that were once in this location
have been placed at the end of the document.

\end{verbatim} \newpage
\baselineskip = 2\baselineskip  % back to double space


Tables must begin with a \verb+"\begin{table}"+ command.
This starts the table environment, which may contain several
tables. A \verb+"\begin{tabular}{}"+ is used to start the actual table.
\verb+"\end{tabular}"+ and \verb+"\end{table}+ are used to
end the table and the table environment.

The command \btt{caption\{Your Caption\}} is used to create a caption
for a table. It should be used just prior to the \btt{begin\{tabular\}\{\}}
command.

The second set of braces following \{tabular\} must contain
information to indicate the number of columns and the type of justification
you wish to use.  The information is conveyed by letters. Each letter
within the braces indicates one column in the table.  The value of the
letter indicates the kind of justification.

Some examples include c for centered, l for left justification,
r for right justification, and d for decimal, where numbers will
be lined up according to their decimal points.

\acknowledgments

 For the Acknowledgments section  use the command
\verb+\acknowledgments+ to produce the heading.  The section number
will not appear.

If you want to suppress section numbers (as in the Acknowledgments
section), the command \btt{section*} can be used. This has already
been done by the
\verb+\acknowledgments+ macro, so the \btt{section*} command is not
needed in this case.


%\appendix
\section*{Appendix: $\,\,$ A}

To start appendixes, you should use the command
\verb+\appendix+, followed by the command \verb+\section{}+.
Please note the equation numbers in an appendix:  Note that the
letters and numbers switch places and functions in the appendix.
\begin{eqnarray}
{\rm P} = mc, \eqnum{(A1)} \\
{\rm E} = mc^2.  \eqnum{(A2)}
\end{eqnarray}

\newpage
\baselineskip = .5\baselineskip  % single space the verbatim
\begin{verbatim}
Tables must begin with a \verb+"\begin{table}"+ command.
this starts the table environment, which may contain several
tables. A \verb+"\begin{tabular}{}"+ is used to start the actual table.
\verb+"\end{tabular}"+ and \verb+"\end{table}+ are used to
end the table and the table environment.

The command \btt{caption\{Your Caption\}} is used to create a caption
for a table. It should be used just prior to the \btt{begin\{tabular\}\{\}}
command.

The second set of braces following \{tabular\} must contain
information to indicate the number of columns and the type of justification
you wish to use.  The information is conveyed by letters. Each letter
within the braces indicates one column in the table.  The value of the
letter indicates the kind of justification.

Some examples include c for centered, l for left justification,
r for right justification, and d for decimal, where numbers will
be lined up according to their decimal points.

\acknowledgments

 For the Acknowledgments section  use the command
\verb+\acknowledgments+ to produce the heading.  The section number
will not appear.

If you want to suppress section numbers (as in the Acknowledgments
section), the command \btt{section*\{\}} can be used. This has already
been done by the
\verb+\acknowledgments+ macro, so the \btt{section*\{\}} command is not
needed in this case.

\appendix
\section{}

To start appendixes, you should use the command
\verb+\appendix+, followed by the command \verb+\section{}+.
Please note the equation numbers in an appendix:  Note that the
letters and numbers switch places and functions in the appendix.
\begin{eqnarray}
{\rm P} = mc, \\
{\rm E} = mc^2.
\end{eqnarray}
\end{verbatim} \newpage
\baselineskip = 2\baselineskip  % back to double space
Either the \btt{begin\{eqnarray\}} (as above) or the
\btt{begin\{equation\}} (as below)
command may be used to start the equations, in the appendix.  The
appropriate ending command must also be given, of course.

\begin{equation} {\rm E} = \case 1/2 mv^2. \label{appa}
\eqnum{(A3)}\end{equation}

\section*{Appendix: $\,\,$ B}
The following section demonstrates a book reference, a journal reference,
and a proceedings reference.  Each reference should begin with the
\btt{bibitem\{Name\}}  command.  \{Name\} should be a short name that
can be used with the \btt{cite\{Name\}} command in the text.

The first reference is a book reference. It shows the authors, the
{\it Title in Italics,} and the publisher, city, and year of publication
in parentheses.

The  second reference, for a journal, shows authors, ``title in quotes,''
journal name, {\bf Volume number in boldface,} complete page numbers
(beginning and ending), and year in parentheses.

The third reference is a proceedings reference.  The authors, ``title
in quotes,''  {\it proceedings name in italics,}
editors, (publisher and year in parenthesis), and volume and pages.

To produce {\it italics}, the command sequence
\{\btt{it\{Title in Italics\}} was used.  To produce bold type, the
command \{\btt{bf\{Number in boldface\}} was used.  To produce right
facing double quotes, 2 single quotes should be used.  To produce
left facing quotes, use 2 `left quote' marks.  On most IBM keyboards
this can be found near the number 1 at the top left corner of the
keyboard.

For greater convenience and accuracy, OSA has provided a set of standard
macros for some commonly used journal abbreviations.  They contain
special formatting commands to ensure correct spacing.  They also
save typing and ensure consistent spelling of the journal abbreviations.

See Table \ref{abbrev} for a list of journal abbreviation macros
supported by the OSA style option.  Other societies also use some
of these keystroke saving macros.

\newpage
\baselineskip = .5\baselineskip  % single space the verbatim
\begin{verbatim}
Either the \btt{begin\{eqnarray\}} (as above) or the
\btt{begin\{equation\}} (as below)
command may be used to start the equations, in the appendix.  The
appropriate ending command must also be given, of course.

\begin{equation} {\rm E} = \case 1/2 mv^2. \label{appa}\end{equation}

\section{}
The following section demonstrates a book reference, a journal reference,
and a proceedings reference.  Each reference should begin with the
\btt{bibitem\{Name\}}  command.  \{Name\} should be a short name that
can be used with the \btt{cite\{Name\}} command in the text.

The first reference is a book reference. It shows the authors, the
{\it Title in Italics,} and the publisher, city and year of publication
in parenthesis.

The  second reference, for a journal, shows authors, ``title in quotes,''
journal name, {\bf Volume number in boldface,} complete page numbers
(beginning and ending), and year in parentheses.

The third reference is a proceedings reference.  The authors, ``title
in quotes,''  {\it proceedings name in italics,}
editors, (publisher and year in parenthesis), and volume and pages.

To produce {\it italics}, the command sequence
\{\btt{it\{Title in Italics\}} was used.  To produce bold type, the
command \{\btt{bf\{Number in boldface\}} was used.  To produce right
facing double quotes, 2 single quotes should be used.  To produce
left facing quotes, use 2 `left quote' marks.  On most IBM keyboards
this can be found near the number 1 at the top left corner of the
keyboard.

For greater convenience and accuracy, OSA has provided a set of standard
macros for some commonly used journal abbreviations.  They contain
special formatting commands to ensure correct spacing.  They also
save typing and ensure consistent spelling of the journal abbreviations.

See Table \ref{abbrev} for a list of journal abbreviation macros
supported by the OSA style option.  Other societies also use some
of these keystroke saving macros.

\end{verbatim} \newpage
\baselineskip = 2\baselineskip  % back to double space


\begin{references}
\bibitem{smith82}F. Zernike and J. Midwinter, {\it Applied Nonlinear
Optics} (Wiley, New York, 1973).
\bibitem{jones78}K. W. Kirby and L. G. DeShazer ``Refractive indexes
of 14 nonlinear crystals isomorphic to ${\rm KH}_2{\rm PO}_4$,'' \josab
{\bf 4,} 1072-1078 (1987).
\bibitem{jonessmith80}C. C. Skiscim and B. L. Golden, ``Optimization
by simulated annealing: a preliminary computational study for the TSP,''
in {\it Proceedings of the 1983 Winter Simulation Conference,} S. Roberts,
J.Banks, and B Schmeiser, eds. (Institute of Electrical and Electronics
Engineers, New York, 1983), Vol.\ 1, pp. 523-535.
\end{references}

\newpage
\baselineskip = .5\baselineskip  % single space the verbatim
\begin{verbatim}

\begin{references}
\bibitem{smith82}F. Zernike and J. Midwinter, {\it Applied Nonlinear
Optics} (Wiley, New York, 1973).
\bibitem{jones78}K. W. Kirby and L. G. DeShazer, ``Refractive indexes
of 14 nonlinear crystals isomorphic to ${\rm KH}_2{\rm PO}_4$,'' \josab
{\bf 4,} 1072-1078 (1987).
\bibitem{jonessmith80}C. C. Skiscim and B. L. Golden, ``Optimization
by simulated annealing: a preliminary computational study for the TSP,''
in {\it Proceedings of the 1983 Winter Simulation Conference,} S. Roberts,
J.Banks, and B Schmeiser, eds. (Institute of Electrical and Electronics
Engineers, New York, 1983), Vol.\ 1, pp. 523-535.
\end{references}


\end{verbatim} \newpage
\baselineskip = 2\baselineskip  % back to double space


\begin{figure}
\caption{A figure caption.  The figure captions are automatically
numbered.\label{autonum}}
\end{figure}

\begin{figure}
\caption{The ``late figure.'' This figure was inserted when the paper
was finished.  Since the figures are automatically numbered,
no renumbering in text was necessary. All that
needed to be done was to type the caption in the
proper place and cite the figure in text.\label{latefigure}}
\end{figure}

\begin{figure}
\caption{A figure caption. Figures will be reduced to an appropriate
size by  the production staff of the journal.\label{reduced}}
\end{figure}

\begin{figure}
\caption{A figure caption.  The labels you give tables and figures
can be descriptive.} %%%%%
\label{figure4}%%%%
\end{figure}


\newpage
\baselineskip = .5\baselineskip  % single space the verbatim
\begin{verbatim}

\begin{figure}
\caption{A figure caption.  The figure captions are automatically
numbered.\label{autonum}}
\end{figure}

\begin{figure}
\caption{The ``late figure.'' This figure was inserted when the paper
was finished.  Since the figures are automatically numbered,
no renumbering in text was necessary. All that
needed to be done was to type the caption in the
proper place and cite the figure in text.\label{latefigure}}
\end{figure}

\begin{figure}
\caption{A figure caption. Figures will be reduced to an appropriate
size by  the production staff of the journal.\label{reduced}}
\end{figure}

\begin{figure}
\caption{A figure caption.  The labels you give tables and figures
can be descriptive.} %%%%%
\label{figure4}%%%%
\end{figure}


\end{verbatim} \newpage
\baselineskip = 2\baselineskip  % back to double space


\begin{table}
\caption{OSA special characters and the macros to produce them.
(This is a small table, which would occupy the width of a
narrow column in the finished article.
In manuscripts all tables
will be displayed with the same width you see here.
Also, table captions are automatically numbered.)}
\begin{tabular}{lr}
Macro & Character\\
\tableline
\verb+\corresponds+&$\corresponds$ \\
\verb+\lambdabar +&$\lambdabar$   \\
\verb+\succsim +&$\succsim $ \\
\verb+\precsim +&$\precsim$  \\
\verb+\slantfrac{1}{2}+\tablenote{Replace 1 and 2 with your numbers.}
&$\slantfrac{1}{2}$
\end{tabular}
\label{table1}
\end{table}



\begin{table}
\caption{A table with math characters.  Two alternative
occupations of special positions
by KMnCL$_3$ ions in the two space groups $D_{4h}^1$ and $D_{4h}^1$.
  For a special value of the $x$ and $y$ parameters, a
set of special positions may
split into two sets of special positions of higher symmetry.
This table's footnote was inserted using the ``tablenote"
command at the point of insertion.}
\begin{tabular}{ccccc}
 &\multicolumn{2}{c}{$D_{4h}^1$}&\multicolumn{2}{c}{$D_{4h}^5$}\\
\cline{2-3}   \cline{4-5}
 Ion&1st alternative&2nd alternative&lst alternative
&2nd alternative\\ \tableline
 K&$(2e)+(2f)$&$(4i)$&$(2c)+(2d)$&$(4f)$\\
 Mn&$(2g)$&$(a)+(b)+(c)+(d)$&$(4e)$&$(2a)+(2b)$\\
 Cl&$(a)+(b)+(c)+(d)$&$(2g)$+$(2h)$&$(a)+(2b)$&$(4e)$
\tablenote{The $z$ parameter of these positions is
   $z\sim\kern-1em\slantfrac{1}{4}$.}\\
 He&$(8r)$&$(4j)$&$(4g)$\\   %
 Ag& &$(4k)$& &$(4h)$\\
 \end{tabular}
 \label{table2}    % for cross references, only
 \end{table}
\bigskip

\newpage
\baselineskip = .5\baselineskip  % single space the verbatim
\begin{verbatim}
\begin{table}
\caption{OSA special characters and the macros to produce them.
(This is a small table, which would occupy the width of a
narrow column in the finished article.
In manuscripts all tables
will be displayed with the same width you see here.
Also, table captions are automatically numbered.)}
\begin{tabular}{lr}
Macro & Character\\
\tableline
\verb+\corresponds+&$\corresponds$ \\
\verb+\lambdabar +&$\lambdabar$   \\
\verb+\succsim +&$\succsim $ \\
\verb+\precsim +&$\precsim$  \\
\verb+\slantfrac{1}{2}+\tablenote{Replace 1 and 2 with your numbers.}
&$\slantfrac{1}{2}$
\end{tabular}
\label{table1}
\end{table}



\begin{table}
\caption{A table with math characters.  Two alternative
occupations of special positions
by KMnCL$_3$ ions in the two space groups $D_{4h}^1$ and $D_{4h}^1$.
  For a special value of the $x$ and $y$ parameters, a
set of special positions may
split into two sets of special positions of higher symmetry.
This table's footnote was inserted using the "tablenote"
command at the point of insertion.}
\begin{tabular}{ccccc}
 &\multicolumn{2}{c}{$D_{4h}^1$}&\multicolumn{2}{c}{$D_{4h}^5$}\\
 \cline{2-3}   \cline{4-5}
 Ion&1st alternative&2nd alternative&lst alternative
&2nd alternative\\ \tableline
 K&$(2e)+(2f)$&$(4i)$&$(2c)+(2d)$&$(4f)$\\
 Mn&$(2g)$&$(a)+(b)+(c)+(d)$&$(4e)$&$(2a)+(2b)$\\
 Cl&$(a)+(b)+(c)+(d)$&$(2g)$+$(2h)$&$(a)+(2b)$&$(4e)$
\tablenote{The $z$ parameter of these positions is
  $z\sim\kern-1em\slantfrac{1}{4}$.}\\
 He&$(8r)$&$(4j)$&$(4g)$\\   %
 Ag& &$(4k)$& &$(4h)$\\
 \end{tabular}
 \label{table2}    % for cross references, only
 \end{table}

\end{verbatim} \newpage
\baselineskip = 2\baselineskip  % back to double space

\begin{table}
\setdec 00.00
\caption{Another table.  This shows decimal and implied-decimal
alignments. Predecimal zeros are required according to OSA style.}
\begin{tabular}{lldddc}
One&Two&Three&Four&Five\\
\tableline
one&two&three&four&five\\
He\tablenote{Some tables require footnotes.}&2&\dec 2.772
& 45672.554 & 0.691 \\
C&Ca\tablenote{Some tables need more than one footnote.}
& 12537.649 & 37.663 &\dec 86.378 \\
\end{tabular}  %%
\label{table3}
\end{table}

\begin{table}
\caption{A ``late table.''  This table was added after most of the
paper had been completed. Since the tables are
automatically numbered, no renumbering in text was necessary. This
table was added to show the use of the d specifier for lining
things up. This time I am going to use the tablenotemark command in the
body of the table and the tablenotetext command at the end.  This
permits multiple citations of the same reference.}
\begin{tabular}{l|dddrd}
\multicolumn{2}{c}{Align by .}&
  \multicolumn{4}{c}{Multiple alignments}\\
\tableline
Glass &1.89\tablenotemark[1] &1.45 &23.66\tablenotemark[2] &0 &0.002\\
Water &1.00 &1.35 &100.00\tablenotemark[1] &344 &\tablenotemark[3]   \\
Wood  &0.87 &0.00\tablenotemark[1]\tablenotemark[2] &7.80 &45 &0.316\\
\end{tabular}
\label{latetable}
\tablenotetext[1]{Here we cite Ref.\ \onlinecite{jones78}.}
\tablenotetext[2]{By the way, these are fictitious numbers.}
\tablenotetext[3]{These are fictitious numbers also.}
\end{table}

\newpage
\baselineskip = .5\baselineskip  % single space the verbatim
\begin{verbatim}
\begin{table}
\setdec 00.00
\caption{Another table.  This shows decimal and implied-decimal
alignments. Predecimal zeros are required according to OSA style.}
\begin{tabular}{llcdcc}
One&Two&Three&Four&Five\\
\tableline
one&two&three&four&five\\
He\tablenote{Some tables require footnotes.}&2&\dec 2.772
& 45672.554 & 0.691 \\
C&Ca\tablenote{Some tables need more than one footnote.}
& 12537.649 & 37.663 &\dec 86.378 \\
\end{tabular}  %%
\label{table3}
\end{table}

\begin{table}
\caption{A ``late table.''  This table was added after most of the
paper had been completed. Since the tables are
automatically numbered, no renumbering in text was necessary. This
table was added to show the use of the d specifier for lining
things up. This time I am going to use the tablenotemark command in the
body of the table, and the tablenotetext command at the end.  This
permits multiple citations of the same reference.}
\begin{tabular}{l|dddrd}
\multicolumn{2}{c}{Align by .}&
  \multicolumn{4}{c}{Multiple alignments}\\
\tableline
Glass& 1.89\tablenotemark[1]& 1.45& 23.66\tablenotemark[2]& 0& 0.002\\
Water& 1.00& 1.35& 100.00\tablenotemark[1]& 344& \tablenotemark[3]  \\
Wood & 0.87& 0.00\tablenotemark[1]\tablenotemark[2]& 7.80& 45& 0.316\\
\end{tabular}
\label{latetable}
\tablenotetext[1]{Here we cite Ref.\ \onlinecite{jones78}.}
\tablenotetext[2]{By the way, these are fictitious numbers.}
\tablenotetext[3]{These are fictitious numbers also.}
\end{table}

\end{verbatim} \newpage

\begin{table}
\caption{OSA Journal abbreviations and the macros to produce them.
Unfortunately, in manuscript style this table will not fit on 1
page.  The input for this table starts below the last lines of output,
on the facing page.}
\begin{tabular}{ll}
Macro & Output\\
\tableline
\verb+\ao  + & \ao  \\
\verb+\ap  + & \ap  \\
\verb+\apl  + & \apl  \\
\verb+\apj  + & \apj  \\
\verb+\bell  + & \bell  \\
\verb+\jqe  + & \jqe  \\
\verb+\assp  + & \assp  \\
\verb+\aprop  + & \aprop  \\
\verb+\mtt  + & \mtt  \\
\verb+\iovs  + & \iovs  \\
\verb+\jcp  + & \jcp  \\
\verb+\jmo  + & \jmo  \\
\verb+\josa  + & \josa  \\
\verb+\josaa  + & \josaa  \\
\verb+\josab  + & \josab  \\
\verb+\jpp  + & \jpp  \\
\verb+\nat  + & \nat  \\
\verb+\oc  + & \oc  \\
\verb+\ol  + & \ol  \\
\verb+\pl  + & \pl  \\
\verb+\pra  + & \pra  \\
\verb+\prb  + & \prb  \\
\verb+\prl  + & \prl  \\
\verb+\pspie  + & \pspie  \\
\verb+\sjqe  + & \sjqe  \\
\verb+\vr  + & \vr  \\
\end{tabular}
\label{abbrev}
\end{table}


\baselineskip = .5\baselineskip  % single space the verbatim
\begin{verbatim}
\begin{table}
\caption{OSA Journal abbreviations and the macros to produce them.}
\begin{tabular}{ll}
Macro & Output\\
\tableline
\verb+\ao  + & \ao  \\
\verb+\ap  + & \ap  \\
\verb+\apl  + & \apl  \\
\verb+\apj  + & \apj  \\
\verb+\bell  + & \bell  \\
\verb+\jqe  + & \jqe  \\
\verb+\assp  + & \assp  \\
\verb+\aprop  + & \aprop  \\
\verb+\mtt  + & \mtt  \\
\verb+\iovs  + & \iovs  \\
\verb+\jcp  + & \jcp  \\
\verb+\jmo  + & \jmo  \\
\verb+\josa  + & \josa  \\
\verb+\josaa  + & \josaa  \\
\verb+\josab  + & \josab  \\
\verb+\jpp  + & \jpp  \\
\verb+\nat  + & \nat  \\
\verb+\oc  + & \oc  \\
\verb+\ol  + & \ol  \\
\verb+\pl  + & \pl  \\
\verb+\pra  + & \pra  \\
\verb+\prb  + & \prb  \\
\verb+\prl  + & \prl  \\
\verb+\pspie  + & \pspie  \\
\verb+\sjqe  + & \sjqe  \\
\verb+\vr  + & \vr  \\
\end{tabular}
\label{abbrev}
\end{table}

\end{document}
\end{verbatim}
\end{document}
% end of file manosa.tex

