\documentstyle[12pt,overcite]{article}
\setlength{\textwidth}{7in}
\setlength{\textheight}{9.25in}
\setlength{\oddsidemargin}{-.25in}
%
\setlength{\topmargin}{-.9in}
\def\Real{\rm I\hspace{-0.2em}R}
\def\nin{{\in \! \! \! \! \! /} \,}
\def\nequiv{{\equiv \! \! \! \! \! \! /} \,}


\newcommand{\II}{\mbox{$\scriptstyle \bf I$}}
\newcommand{\JJ}{\mbox{$\scriptstyle \bf J$}}
\newcommand{\KK}{\mbox{$\scriptstyle \bf K$}}
\newcommand{\LLL}{\mbox{$\scriptstyle \bf L$}}
\newcommand{\ii}{\mbox{$\scriptstyle \bf i$}}
\newcommand{\jj}{\mbox{$\scriptstyle \bf j$}}
\newcommand{\kk}{\mbox{$\scriptstyle \bf k$}}
\newcommand{\lll}{\mbox{$\scriptstyle \bf l$}}
\newcommand{\aaa}{\mbox{$\scriptstyle \bf a$}}
\newcommand{\bb}{\mbox{$\scriptstyle \bf b$}}
\newcommand{\cc}{\mbox{$\scriptstyle \bf c$}}
\newcommand{\dd}{\mbox{$\scriptstyle \bf d$}}
\newcommand{\AAA}{\mbox{$\scriptstyle \bf A$}}
\newcommand{\BB}{\mbox{$\scriptstyle \bf B$}}
\newcommand{\CC}{\mbox{$\scriptstyle \bf C$}}
\newcommand{\DD}{\mbox{$\scriptstyle \bf D$}}
\newcommand{\half}{\mbox{$\frac{1}{2}$}}
\newcommand{\mul}{\multicolumn{2}{c}{none}}
\newcommand{\qua}{\mbox{$\frac{1}{4}$}}
\newcommand{\six}{\mbox{$\frac{1}{6}$}}
\newcommand{\eight}{\mbox{$\frac{1}{8}$}}
\newcommand{\tfour}{\mbox{$\frac{1}{24}$}}
\newcommand{\onetw}{\mbox{$\frac{1}{120}$}}
\newcommand{\sfour}{\mbox{$\frac{1}{740}$}}
\newcommand{\npr}{\mbox{$\frac{1}{n!}$}}
\newcommand{\beq}{\begin{equation}}
\newcommand{\eeq}{\end{equation}}
\newcommand{\tripler}{T_{3}\!\left(^{a^{\prime}b^{\prime}
{\rm C}^{\prime}}_{{\rm I}^{\prime}j^{\prime}k^{\prime}}\right)}
\newcommand{\quadrupler}{T_{4}\!\left(^{a^{\prime}b^{\prime}
{\rm C}^{\prime}{\rm D}^{\prime}}_{{\rm I}^{\prime}{\rm J}^{\prime}
k^{\prime}l^{\prime}}\right)}
\newcommand{\pentupler}{T_{5}\!\left(^{a^{\prime}b^{\prime}
{\rm C}^{\prime}{\rm D}^{\prime}
{\rm E}^{\prime}}_{{\rm I}^{\prime}{\rm J}^{\prime}
{\rm K}^{\prime}
l^{\prime}m^{\prime}}\right)}
\newcommand{\sextupler}{T_{6}\!\left(^{a^{\prime}b^{\prime}
{\rm C}^{\prime}{\rm D}^{\prime}
{\rm E}^{\prime}{\rm F}^{\prime}}
_{{\rm I}^{\prime}{\rm J}^{\prime}
{\rm K}^{\prime}{\rm L}^{\prime}
m^{\prime}n^{\prime}}\right)}
% \newcommand{\hhline}{\hline \vspace{-4.8mm} \\ \hline}

\begin{document}

\setlength{\baselineskip}{1em}

%\pagestyle{empty}

\noindent
{\bf PROJECT SUMMARY}

This proposal requests support for development and implementation
of methods allowing study of molecular systems without
assuming the Born--Oppenheimer (BO) approximation regarding the
separability of the nuclear and electronic motions. In this
approach we will employ explicitly  
correlated cartesian gaussian functions.
My group has been working in the area of non--Born--Oppenheimer
atomic and molecular quantum mechanics 
for several years now and, encouraged by the 
results obtained so far, we would like to propose a
research
project where we consider 
a more general problem of stationary bound
quantum states of multi--particle
systems which interact through an isotropic potential.
The non-BO molecular system consisting of electrons and nuclei 
with
Coulombic repulsive and attractive potentials 
is an example of such a problem.
Another example, which we propose to study with
the new methodology, is the problem of ro--vibrational structures
of molecules and clusters. 
In this case the isotropic interaction potential is provided
in the form of the potential energy hypersurface (PES)
calculated for the system using electronic structure methods.
One can consider the PES of the 
ground or an excited electronic state in such a calculation.
Among the electronic structure methods which we will
use to calculate PES's are
the state--selective, multi--reference coupled cluster (SSMRCC)
method and the variational approach utilizing correlated gaussians 
and analytical derivatives of the variational functional with
respect to linear and non-linear parameters of the gaussian
expansion of the wave function.
Both methods have been developed in our group.
PES can be considered as the
isotropic (rotationally invariant) interaction potential
for the atoms, which form the molecule.
Unlike the Coulombic potential, which includes only two-body
components, the potential given by PES has N-body character and,
although in most cases it is dominated by two-body interactions,
the three- and higher-body contributions can be important.
%New functional forms for the multi--body expansion of the potential 
%energy
%surface  and for the basis function expansion of the ro--vibrational
%wave function to be used variational calculations of vibrational
%and rotational levels of N-atom systems
%are proposed.
In this project we propose that N--body
explicitly correlated gaussians with pre-multiplying factors consisting of
products of powers of internal distance coordinates are utilized in a dual
role to analytically represent the isotropic N--atom potential and the 
wave functions expressed in terms of internal cartesian coordinates. 
We propose to develop and implement the methodology for the general
N--body case. 

The initial non-Born-Oppenheimer 
application of the present methodology 
will be limited
to smaller atomic and molecular systems.
Following our recent non--adiabatic calculations
of the electron affinities of hydrogen, deuterium and tritium,
which produce results matching very well with the experimental
values, we will now consider smaller molecular systems.
In the first application calculations 
we will consider
hydrogen molecules and ions and their isotopomers,
H$_2^+$, H$_2$, H$_3$, {\it etc.}, 
%small dipole--bound anions, {\it e.g.}, LiH$^-$, LiD$^-$,
and other molecular systems, {\it e.g.}, HeH$^+$, HeH$_2^+$, HeH$_2$.
The aim of these calculations will be to determine 
electron affinities and electron detachment energies 
of these systems in different rovibrational states.
Very accurate measurements, which have 
recently become
available 
for these systems
present an exciting challenge for 
the ``non--Born-Oppenheimer Atomic and Molecular Quantum Mechanics."
Also calculations of 
higher excited states,
where the coupling of the electronic and nuclear motions
can be significant, 
will be performed and the nature and magnitude of the 
coupling effect will be 
analysed.

Initial applications of the method with 
analytically fitted PES's as the 
interaction potentials
%in the ro--vibrational Hamiltonian
will start with 
``calibration cases"
such as the H$_3^+$ and H$_2$O systems and their 
isotopomers.
These systems considered before by others
using different methodologies
and their calculated ro-vibrational spectra 
were compared with very accurate experimental data.
%For H$_3^+$ we will generate the 
%potential energy surface using 
%the variational approach with explicitly correlated
%gaussians and with the optimization algorithm based
%on analytical derivatives of the variational functional
%with respect to the non--linear parameters of gaussians,
%which we have recently developed.
%For H$_2$O we will use the best existing potentials,
%as well as the potential generated using our SSMRCC method.
Application of our approach to systems with more than three
nuclei are the most interesting since 
other methods have been mostly limited to three-body cases.
We find particularly interesting to investigate 
ro-vibrationally hot elemental clusters, such as
C$_n$, Si$_n$, P$_n$, $n$=4,5 {\it etc.}, and their anions.
Calculations of highly excited ro-vibrational states near the
dissociation barrier of
these systems will reveal their structures and dynamics 
at high temperatures.
Among other applications we will consider
the high--temperature
ro--vibrational structure  
of the vinylidine--acetylene,
$HCCH \leftrightarrow CCH_2$, isomeric system.
We will also study
ro--vibrational spectra 
of small molecular dimers such
as (HF)$_2$.

Finally, it is also our goal to
provide through the proposed program, an inspiring environment 
for the training of graduate and undergraduate students as
future scientists. This project, which combines fundamental elements of 
quantum mechanics and spectroscopy has a potential to appeal
to students who are 
concerned with 
fundamental properties of molecular systems.
The proposed research will provide a unique view of molecules,
their structures and properties without resorting to
the Born--Oppenheimer clamped-nucleus model.


\newpage



\setcounter{page}{1}

\begin{center}
{\Large \bf 
Non-Born-Oppenheimer and Ro--Vibrational Calculations of
Molecular Systems with r$_{ij}$ Dependent Gaussians.}

\end{center}


\vspace{2mm}

\noindent
{\bf I. RESULTS FROM PRIOR NSF SUPPORT.}


\noindent
{\bf "Multi--reference Coupled--cluster Theory. 
Theoretical Study on Singly- and Doubly--charged Elemental 
Cluster Anions."  Grant CHE-9300497,  \$164,700, 4/15/93-10/15/96.}

\vspace{3mm}

As of 3/26/98, the research funded by the above 
grant has resulted in 60 publications
\cite{A1,A2,A3,A4,A5,A6,A7,A8,%
A9,A10,A11,A12,A13,A14,A15,%
A16,A17,A18,%
A19,A20,A21,A22,A23,A24,A25,%
A26,A27,A28,A29,A30,A31,A32,A33,A34,%
A35,A36,A37,%
A38,A381,A382,%
A383,A384,A385,A3851,%
A3860,%
A3861,A3862,A3863,A3864,A3865,%
A3866,A3867,A3868,A3869,A3870,A3871,%
A3872,A3873,A3874,A3875}  
58 of which
have been already published or accepted for 
publication and two have been submitted.
The following research
results have been obtained with the support 
of the funding provided by NSF:
The state--selective, multi--reference coupled-cluster  
(SSMRCC) theory has been formulated based
on the doubly--exponential form of the wave function,  
with the reference in the active orbital space represented
by an exponentiated CC excitation operator  
acting on the formal reference determinant (Fermi vacuum), and
with the excitations to the non--active space 
also represented with the exponential of the virtual excitation
operator,
$|\Psi\rangle  = e^{T^{\rm (ext)}} | \Phi^{\rm (int)}
\rangle  = e^{T^{\rm (ext)}}
e^{T^{\rm (int)}} |0\rangle$.
This method,
which is geared towards very accurate calculations of
the potential energy surfaces for the ground and excited 
states of molecular systems,
has been computationally implemented at 
the SSMRCCSD(T) and SSMRCCSD(TQ) levels. 
Some other research projects less directly related 
to the development of the MRCC methodology, but having
the potential to provide certain theoretical 
solutions relevant to future work on the CC method, have also been
carried out.  
These include 
development and implementation of   
techniques for utilization of explicitly  correlated  gaussian 
functions  in molecular Born--Oppenheimer (BO)
and non--BO 
calculations and 
techniques to study  
large--amplitude vibrations in molecules and complexes.  

\vspace{4mm}

\noindent
{\bf II.   CONTRIBUTION TO DEVELOPMENT OF HUMAN RESOURCES.}

There have been a number of collaborators who 
have carried out the work described here. Their
contributions to the above--cited accomplishments have 
been immense and their involvement contributed to their
growth as independent scientists. The collective 
effort involving formulation of the theory, computational
implementation and application work should be credited 
to my five research associates, ten graduate students,
eight visiting scientists and three undergraduate students.  
It is due only to the devotion of these researchers that
we have been able to create a stimulating research environment 
conducive to teamwork and collaboration.

\vspace{2mm}

\noindent
{\bf Research Associates:}

Dr. {\em Piotr Piecuch}, recently appointed as an 
assistant professor at the Michigan State University, joined us for
one and a half year and worked on theoretical formulation and 
computational implementation of the fully--exponential
SSMRCC scheme and on developing improved 
methods for converging the CC amplitude equations;  
Dr. {\em Keya Ghose}, 
currently a Humboldt Fellow at the 
Lehrstuhl f\"{u}r Theoretische Chemie, Ruhr-Universit\"{a}t in Bochum,
Germany, stayed for a year with my group and 
worked on implementation of the recursively calculated
intermediates and on the property calculations;  
Dr. {\em  Zdenek Slanina}, 
now a professor at the Laboratory
of Molecular-Design Engineering,
Toyohashi University of Technology, Japan,
worked for a yaer in our group on carbon clusters, fullerenes and
related projects;  
Dr. {\em Genady Gutsev}, 
currently
at Jackson State University, 
spent two years in our group (first year of his stay was
funded by a grant from the CAST program) and did 
calculations on molecular and cluster anions; 
and Dr. {\em Don Kinghorn}, 
currently at the University of Arkansas with Prof. P. Pulay,
worked for one and a half years in my group 
on implementations of explicitly correlated gaussian functions
to non--BO atomic and molecular calculations and 
to the general N--body vibrational problem.

\vspace{2mm}

\noindent
{\bf Graduate Students:}

\noindent
Dr. {\em Nevin Oliphant}, currently 
with a computer company serving the Wall Street Stock Exchange,
made a significant contribution to the conceptual 
framework of the SSMRCC theory and its implementation;
Dr. {\em Pawel Kozlowski}, currently a postdoctoral research 
associate  
with Prof. T. Spiro at
Princeton University, developed the non--BO approach 
to multi--body systems and implemented explicitly
correlated gaussian functions to perform application 
calculations for those systems; 
Mr. {\em Z. (John) Zhang}
continued Pawel's work on correlated gaussians and 
implemented analytical derivatives for optimization of the
variational wave functions for atomic and molecular 
systems.  In this he has been helped by 
Mr. {\em Eric Schwegler} 
who, after completing his first year of the graduate program 
at Arizona, transferred to the University of
Minnesota; 
Mr. {\em Yasser Elkadi}
has been working on 
dipole--bound anionic states of molecules and cluster
systems; 
Dr. {\em William McCarthy}, 
who graduated 
with a Ph.D. in September 1996 and moved to the Utah
State University to join Prof. S. Aust's group,  
developed methodology for describing
large--amplitude vibrations in larger molecular systems; 
Mr. {\em Vadim Alexandrov} has been working on
development of a wave-packet time-propagation method which will
be applied to study intra- and inter-molecular proton transfer
reactions; 
Mr. {\em Doug Gilmore} before moving to Law School at Concorde,
New Hampshire,
developed a technique for
optimizing multi--center correlated gaussians
based on analytical gradients; 
Ms. {\em Dayle Smith} 
has studied 
intramolecular hydrogen--bondings
systems involving DNA bases and electron attachment to these
systems;   
Ms. {\em Sandra Blumhorst} 
has continued Gilmore's work on implementation
of explicitly correlated gaussians to molecular problems.

\vspace{2mm}
\noindent
{\bf Undergraduate Students:}

\noindent
Mr. {\em John Millam}, 
who worked in my group on metastable
states of helium and neon dimer anions,
after graduating from 
Rice, now works with Prof. B. Schlegel at 
Wayne State;
Mr. {\em Nathan Oyler}, 
currently a graduate student at the Department of
Chemistry, University of Washington, investigated 
dipole--bound anionic states of nucleic acid bases; and 
Ms. {\em Melissa Farrow} 
did some work on graphic interfaces of our programs.


\vspace{2mm}
\noindent
{\bf Visiting scientists:}

\noindent
Drs. {\em Ivan Ortega--Blake}  and {\em Humberto Saint--Martin} from 
UNAM at Cuernavaca, Mexico, visited 
Arizona several times for short visits
to work on the collaborative
project funded by International Programs at NSF; 
Dr. {\em Maciej Nowak} from the Institute of
Physics, Polish Academy of Sciences, visited 
my group for a month to work on a review article describing
our collaborative research on IR spectroscopy of 
the nucleic acid bases and their analogs; 
Prof. {\em Jong Lee} from Chonbuk National University,
South Korea, spent his sabbatical year in my group 
working on electronic excitations of chain carbon clusters;
Dr. {\em Andrzej Le\'{s}} from the Department of Chemistry, 
University of Warsaw, spent three months in Summer '94 with us and did
theoretical investigations of the phosphate hydrolysis 
process. 
Prof. {\em Andrzej Sobolewski} from the Institute of
Physics of the Polish Academy of Sciences, 
spent a year in my group working on dynamics of the proton
transfer reactions in molecules and complexes involving nucleic 
acid bases, and on photophysics of hydrated carbon
clusters; 
Dr. {\em Johan Smets}, from the University 
of Leuven, visited us twice for three--month periods (his stay
was supported by a grant from NATO) and worked 
on assignment of IR spectra of water complexes of nucleic
acid bases based on the theoretically predicted 
frequencies.  
Dr. {\em Sourav Pal} from the National Chemical Lab,
in Puna, India, spent a month in my group and 
worked on the SSMRCC response theory.
Dr. {\em Stepan Stepanian} spent six months in my group 
in '97 on a COBASE fellowship and calculated and
assigned experimental IR matrix--isolation 
spectra of pyrimidine--quinone and quinone--quinone dimers
and the spectra of some aminoacids and small peptides.
Last year we had a month--long visit in our group by Ms.
{\em Kristien Schoone} and Ms. {\em Riet Ramaekers} 
from the University of Leuven
who worked with us on the NATO project, and a three-month visit
of Ms. {\em Marta Fores} from University of Girona who worked
on calculations concerning excited-state proton transfer
reactions.

\vspace{4mm}
\noindent
{\bf III.  PROJECT DESCRIPTION}

\vspace{2mm}
\noindent
{\bf A. INTRODUCTION}

\vspace{2mm}
\noindent
{\bf A.1 Equivalent Treatment of Nuclei and Electrons}

In this proposal we discuss an approach to atomic and 
molecular quantum--mechanical calculations without assuming 
the clamped--nuclei approximation. Although such calculations
are still rare, the progress
in conceptual formulation of the theory in this area and, more
importantly, in development of necessary computational tools
has progressed to the point when non--Born--Oppenheimer
calculations may soon become possible for more extended molecular
systems. If it does happen, 
the problems of molecular spectroscopy,
which cannot be approached based 
on the concept of the Potential Energy Surface (PES),
will open to theoretical investigations.
It may also provide a new insight in such important notions of
chemistry as Chemical Bonding and Molecular Structure.

In an attempt to make the quantum mechanical calculations on 
molecular systems practical and to provide a more intuitive
interpretation of the computed results, it has long been a quest
in the electronic structure theory of molecules 
to establish a solid base for separating
the motion of light electrons from the motion of heavier nuclei.
It is believed that the original work of Born and Oppenheimer (BO)
\cite{BO1927} initiated the discussion by the analysis
of the diatomic case. Further works of Combes and Seiler
\cite{CS1980}, who managed with the use of singular perturbation
theory to resolve the problem of the diverging series which appeared
in the BO expansion, and particularly of Klein and coworkers
\cite{KM1992} who extended the formalism to polyatomic systems,
have brought the consideration of the topic to a level of commonly
accepted theory.

Apart from the further refinements of the theoretical grounds
for the BO approach, which is closely related to the notion
of the Potential Energy Surface (PES), there has been a continuing
interest in theoretically approaching molecular systems 
with a method which equivalently treats
the motions of both nuclei and electrons.
Here we propose to further advance the methodology 
in which treatment of the nuclear and electronic motions
in molecules departs entirely from the PES concept. 
It is particularly
interesting how in this type of approach the conventional concepts
of molecular structure and chemical bonding will be represented
and how different these representations will be 
from the representations
developed based on the BO approximation. In particular the concept
of chemical bonding, which at the BO level is an electronic 
phenomenon, will now be described as an effect derived from
collective dynamical behaviour of both electrons and nuclei.

Another motivation for developing the non--adiabatic approach to 
describe the states of molecules stems from the realization that
in order to reach ``spectroscopic" accuracy in quantum--mechanical
calculations ({\it i.e.}, error less than 1 $\mu$hartree), one needs
to account for the the coupling between
motions of electrons and nuclei. Modern experimental techniques,
such as gas--phase ion--beam spectroscopy, reach accuracy on the
order of 0.001~cm$^{-1}$. \cite{C} In order for quantum molecular mechanics
to continue providing assistance in resolving and assigning
experimental spectra, especially in studies of reaction dynamics,
work has to continue on development of more refined theoretical 
methodology, which accounts for non--adiabatic interactions.
Our recent calculations of electron affinities (EA) of hydrogen, deuterium
and tritium atoms \cite{A37} can serve as example of 
problems where very accurate experiments challenge the ability
of accurate non--adiabatic theoretical calculations 
to deliver high quality results.
Our calculated values of EAs for H and D of 6083.0983~cm$^{-1}$
and 6086.7126~cm$^{-1}$, respectively, are in excellent
agreement with the experimental results of Lineberger and
coworkers of 6082.99$\pm$0.15~cm$^{-1}$ and 6086.2$\pm$0.6~cm$^{-1}$,
respectively. Our tritium EA of 6087.9169~cm$^{-1}$ has not yet 
been experimentally verified. Similar calculations are now being
performed in our group 
on electron affinities of the Lithium isotopes.

We can generalize the non--adiabatic problem of nuclei and electrons
interacting with two-body Coulomb potentials
and formulate it as a more general problem of determining  
quantum states
of a system of particles interacting with 
isotropic many-body potentials. More
specifically, we are interested in two cases: 
Case 1, where the
particles are nuclei and electrons 
and
Case 2, where the systems consist of any type of particles
({\it e.g.}, atoms in a molecule or in a cluster)
and the interaction potential can be modeled by a
many-body expansion.
One example of the latter case is a calculation of the ro--vibrational
structure of a molecule, where the potential 
of the interaction between the atoms
is given by the PES determined using electronic structure
methods.
Both of these cases are modeled by the same general form of
the Hamiltonian as discussed below.
Another possible case, which may also be studied but is 
not included among the topics in this proposal, is the system
of nucleons interacting with short--range nuclear forces.

It is convenient in practical calculations to use an
analytical representation of the PES in terms of some
basis functions. In the following development we
assume that such a representation can be found.
We also discuss the use of correlated gaussians 
in constructing analytical representations of molecular PES's.

To model the physical systems, {\it i.e.}, 
write the Hamiltonian, particles are
considered to be non-relativistic, point masses,
$M_i$, interacting under
an isotropic potential. The total Hamiltonian then has the familiar form, 
\begin{equation}
H_{tot}=-\sum_i^N\frac{\nabla _{{\bf R}_i}^2}{2M_i}
+ V\left(R_{ij}, R_{ijk}, R_{ijkl}, \ldots%
\,;\,\,\,i<j<k<l<\ldots ,\,\,\,\,i=1...N\right).
\label{ham1}
\end{equation}
In the above Hamiltonian the potential can include two-particle
interactions, as well as effective interactions 
of three or more particles. 
In the case of the Coulomb potential, only pair interactions
will remain in the Hamiltonian. In ro--vibrational calculations,
the interaction potential between atoms, which is given by the
potential energy surface (PES) (in the present approach this will be 
represented by a isotropic analytical 
functional fit; here the term "isotropic"
is used to denote a "rotationally--invariant" ),
may require in some cases three- and higher multi--body
components. However, apart from the form of the potential, which
will be different for non--adiabatic calculations of electron--nucleus
systems and for ro--vibrational problems, the general approach
will be similar for the two cases, and this is the reason
for considering 
a unified approach to
both cases in this proposal. 
Moreover, a more general approach will extend the scope of applications
of the proposed methodology to wider spectrum of molecular problems.
The first step in this direction has been described in our recent papers
\cite{A382,A383}.


\vspace{2mm}
\noindent
{\bf A.2 Ro--vibrational Hamiltonian}


Advances in 
the atomic and molecular quantum mechanics
coupled with the phenomenal
increase in the computational power of computers has opened the door to the
development of ``{\it ab--initio} spectroscopy and dynamics.'' Using
advanced electronic structure 
computational methodologies 
it is now possible to generate accurate potential energy
surfaces 
for molecular systems with even a dozen or more atoms. However,
there is still much work to be done in developing methods for representing
these surfaces analytically to facilitate dynamics calculations. 
Historically, methods
for investigation of vibration--rotation energy levels have ranged from
simple harmonic oscillator rigid rotor models with perturbation extensions
based on the classical ro--vibrational Hamiltonians of Eckart\cite{Eckart35},
Wilson and Howard\cite{WilsonHoward36}, and the quantum mechanical version
of the Eckart Hamiltonian put forth by Watson\cite{Watson68}, to the
variational method of Sutcliffe and Tennyson\cite{ref:ch22}, and
the Morse oscillator based Hamiltonian, (MORBID), of Jensen\cite{Jensen88}.
There are successful computer 
programs, such
as LeRoy's program LEVEL 6.0\cite{LeRoy95} for diatomic systems,
Hutson's program BOUND\cite{Hutson93} for numerical 2- or 3--D
calculations of atom--diatom Van 
der Waals molecules, and Tennyson {\it et al.}
program suites DVR3D and TRIATOM,
to calculate energy levels, wave functions (w.f.) and 
dipole transition moments of triatomic systems \cite{ref:ten1}.
However, there has been very little done to extend these
approaches beyond triatomic systems.

The early 1980's work on floppy systems, such as $KCN$ 
\cite{ref:kcn1,ref:kcn2,ref:kcn3} and $CH_2$
\cite{ref:ch21,ref:ch22,ref:ch23}, showed that there were
practical problems in application of the Eckart--Watson Hamiltonian
for such cases. The difficulties result not only from
well--known singularity problems for linear geometries, but
also from some technical instabilities 
of remaining in the true domain of the problem \cite{ref:j22}.
Some of the problems were partially overcome with the development of
a series of formally exact, body--fixed Hamiltonians defined
in terms of internal molecular coordinates. The formal development
in this area,
which should be credited to
Sutcliffe and coworkers
\cite{ref:j36}, was further extended and automated
with the use of the algebraic language {\em REDUCE} by Handy
and coworkers \cite{ref:j27}. The disadvantage of this approach
is that different coordinates, and thus different Hamiltonians,
are required for the optimal treatment of each class of
vibrational problem. Also, the singularity problems did not
disappear completely,
although a solution to 
this problem, at least for triatomic systems, has been found with the
development of Hamiltonians in terms of ``flexible" coordinates
\cite{ref:j32,ref:j33,ref:j34,ref:j35,ref:j36,ref:j37}.
These coordinates enter the Hamiltonian only via the moment of
inertia terms. 

When the Hamiltonian of the problem is constructed, a wide variety
of solution strategies can be applied to obtain vibrational energy
levels and w.f.'s. The most popular approach, which has led
to the most accurate results, has been the use of a basis set
expansion, which requires mastering the art of choosing appropriate
basis functions. Here one should mention the works by Light and
coworkers \cite{ref:j25,ref:j45,ref:j46}, who introduced the
discrete variable representation (DVR) method based on  the 
finite--element approach. This method employs 
the conventional basis set
technique and, through a hierarchy of diagonalizations and truncations,
leads to a final secular problem with a greatly reduced dimension.

For systems with significant Coriolis coupling, one needs to
consider coupling between the vibrational and rotational motions.
The straightforward approach utilizing superposition of vibrational
and rotational basis functions in the variational calculations
leads to a secular problem which grows rapidly with total angular 
momentum, $L$. To circumvent this problem, Chen {\it et al.} and
Tennyson and Sutcliffe \cite{ref:j22,ref:j50}, proposed a
two--step variational method, where the fully coupled ro--vibrational
problem is expanded in terms of solutions of some suitable
``vibrational" problem. 
With this approach, which we will also use in our work, the 
rotational motion problem can be regarded as effectively solved.


As it is probably fair to regard the variational method for
calculating the ro--vibrational spectra of triatomic systems
from a given potential energy surface as a mature technique,
the general N--body ro--vibrational problem is still widely open
and unsolved.
The present work proposes a new coherent
methodology addressing this problem.
The proposed method does not require exotic
transformations or approximations of the Hamiltonian, 
it is conceptually
straight forward, and can be easily extended to larger systems.

\vspace{2mm}
\noindent
{\bf A.3 Analytic representations of potential surfaces}

There have been three main methods employed for analytically representing
potential energy surfaces; series expansions, interpolation schemes, and
many-body function representation. The power series approach includes the
early work of Dunham\cite{Dunham32a,Dunham32b} using expansions in $%
(R-R_e)/R_e$ and more recent work by Simons, Parr, and Finlan\cite{Simons73}
using $(R-R_e)/R$, Ogilvie\cite{Ogilvie81} using $2\left( R-R_e\right)
/\left( R+R_e\right) $, Thakkar\cite{Thakkar75} with $1-(R_e/R)^{-a-1}$ and
Huffaker\cite{Huffaker76} using an expansion with a Morse--like
variable $1-\exp \left[ -a\left( R-R_e\right) \right],$ 
\cite{Morse29} (where $a$ is a Dunham
constant). Better convergence radii can be obtained using rational
polynomials such as the Pad\'{e} approximants used in the study of several
di-atomic systems\cite{Jordan74,Jorish79,Sonnleitner81,Pardo86}. Another
type of expansion is to use a Taylor series obtained by numerical
differentiation to give a quadratic or quartic force field representation.
Lee, Taylor, and coworkers have obtained very
good results for low lying vibrational levels using accurate quartic force
fields\cite{Lee95a,Dateo94,Jorish79,Lee95b,%
ref:t12,ref:t13,ref:t14} for systems as large as ethylene
\cite{Martin95}. There is a hybrid approach using Morse functions and two
dimensional splines suggested by Koizumi, {\it et al.}, \cite{Koizumi91}
which has been described and used by Bentley
{\it et al.}
\cite{Bentley92}. This method is similar to the rotated Morse oscillator
method\cite{Bowman75,Connor75}. Interpolation schemes using splines or
piecewise polynomials have been used for fitting single and
multi-dimensional potential energy surfaces\cite
{Forsythe77,Malik80,Sathyamurthy75,Dunne87,Bruehl88}. The 
many--body function
representation has been explored most thoroughly by Murrell {\it et al.}
\cite{Murrell84},
Searles and von-Nagy-Felsobuki\cite{Searles93}, 
and Mezey. \cite{Mezey87}
The Sorbie-Murrell\cite
{Sorbie75} potential for $H_2O$ was used successfully by Whitehead and Handy%
\cite{Whitehead76} for vibrational band origin calculations. 


\vspace{2mm}
\noindent
{\bf B. Work Proposed}

{\bf B.1 Explicitly Correlated Gaussian Functions}

In the present work we propose to employ explicitly correlated
gaussian functions in variational calculations of 
non--adiabatic quantum states of systems 
consisting of nuclei and electrons, and of
ro--vibrational
energy levels of N--atom molecules and clusters. 
The application of correlated gaussian functions in
molecular calculations has gained momentum
following some impressively accurate 
works of Jeziorski, Szalewicz and
their coworkers on small molecular systems performed
with the use of these functions.
\cite{sz1}
The proposed procedure is derived from
the non--adiabatic approach, which we have pursued in our group
for the last few years, and based on several past applications
of correlated gaussians in Born--Oppenheimer (BO) 
and non--BO calculations on atomic
and molecular systems.
\cite{A2,A3,A6,A8,%
A14,%
A16,%
A20,A21,%
A27,%
A37,%
A38,A382,A383,%
kozlowski91,kozlowski92a,kozlowski92b,kozlowski92c}.
The procedure is also derived from the works 
of Poshusta and Kinghorn
concerning non--BO calculations
%There have been several highly accurate non-adiabatic variational
%calculations on atomic and exotic few particle systems using simple
%correlated gaussians\cite
\cite{Poshusta83,Kinghorn93,Kinghorn95b}
%,Kinghorn96a}.
Dr. Don Kinghorn was a postdoctoral research associate 
in my group in 1995-1997 and has
expressed an interest to work with us on the present project
if his current commitments permit.

The centerpiece of the 
proposed methodology is the multiple use of 
gaussian functions of the
general form, 
\begin{equation}
\phi _k=\prod_{i<j}r_{ij}^{m_{kij}}\exp \left[ -{\bf r}^{\prime
}(A_k\otimes I_3){\bf r}\right] .  
\label{basisfcn}
\end{equation}
These functions, hereafter referred to simply as ``$\phi _k$'', will serve
as multi-parameter expansion functions for analytically represented
potential energy surfaces and, with the addition of a rotational component
and appropriate symmetry projection, will be used as variational basis
functions for vibration-rotation (and electronic) w.f.'s. In $\phi
_k $ the term $\prod_{i<j}r_{ij}^{m_{kij}}$ is a product of ``distance''
coordinates, $r_{ij},$ raised to powers $m_{kij}$ (positive, negative
\cite{kpc}
or zero).
 The exponential component is an explicitly correlated gaussian with $%
{\bf r}$ representing a length $3n$ column vector of internal (relative)
coordinates (${\bf r}^{\prime }$ denotes the transpose of ${\bf r}$,
{\it i.e.}, a row vector), $A_k$ is an 
$N\times N$ symmetric matrix of exponent
parameters (usually positive semi-definite). The Kronecker product of $A_k$
with the $3\times 3$ identity matrix, $I_3$, insures rotational invariance
as will be shown in a later section. The matrix/vector form of $\phi _k$
allows us to exploit the powerful matrix differential calculus, described by
Kinghorn\cite{Kinghorn95a}, for deriving elegant and easily implementable
mathematical forms for integrals and derivatives required in variational
calculations.

The first proposed use for $\phi _k$ we will discuss is as expansion
functions for n-body potential energy surfaces or, in general, any potential
which is an isotropic 
(rotationally invariant)
function of internal coordinates. Our expansion will
utilize $\phi _k$ in the spirit of the many-body expansion described by
Murrell\cite{Murrell84}. Note: The $N$-particle 
Coulomb potential is a special
case of this expansion: Set $A_k=0$ and 
choose the appropriate $m_{kij}=-1$ or $0$. 
Also note, that Lennard-Jones type potentials are also special cases
of an expansion in $\phi _k.$

Utilizing $\phi _k$ as expansion functions for the potential energy operator
and as variational basis functions, both in the same 
internal cartesian coordinates $%
{\bf r}$, gives us a simple coherent method for computing energy
eigen-states without making unnecessary transformations and approximations to
the Hamiltonian. This flexibility in the $\phi _k$ will allow our procedures
to be used  
for $N$-body vibration-rotation calculations with
analytic representations of an Born-Oppenheimer or 
adiabatic potential energy surface.
Note: the $\phi _k$ are angular momentum eigen-functions
with eigen-value $J=0$; thus, modification of the basis to $Y_M^L\phi _k$
will yield higher angular momentum eigen-states when the factor $Y_M^L$ is
chosen as an angular momentum eigen-function for the desired state.
This is a very important property particularly in determination
of ro--vibrational states, since it allows separate variational
calculations of vibrational levels corresponding to different
values of the total angular momentum operator (different rotational states).
For example, we can calculate the vibrational spectrum of a 
molecule for states with $J=0$ separately from the calculation
of states with $J=1$ and with higher $J's$. 
Representation of higher angular eigen-states will be discussed in 
one of the following sections.



\vspace{2mm}
\noindent
{\bf B.2 Coordinates and the Hamiltonian}

One can express the Hamiltonian (\ref{ham1})
in terms of the coordinates of particles and
their relative distances as: 
\begin{equation}
H_{tot}=-\sum_i^N\frac{\nabla _{{\bf R}_i}^2}{2M_i}+V\left( \left\| 
{\bf R}_i-{\bf R}_j\right\| \,;\,\,\,i<j,\,\,\,\,i=1...N\right) .
\label{ham2}
\end{equation}
The particles are numbered from $1$ to $N$ with $M_i$ the mass of particle 
$i$, ${\bf R}_i=[X_i\,\,Y_i\,\,Z_i]^{\prime }$ a column vector of cartesian
coordinates for particle $i$ in the external, laboratory fixed, frame, $%
\nabla _{{\bf R}_i}^2$ the Laplacian in the coordinates 
of ${\bf R}_i$, and $\left\| {\bf R}_i-{\bf R}_j\right\| $ 
the distance between
particles $i$ and $j$. The total Hamiltonian (\ref{ham2}) is, of
course, separable into an operator describing the translational motion of
the center--of--mass and an operator describing the internal energy. This
separation is realized by a transformation to 
the center--of--mass and internal
(relative) coordinates.
Let ${\bf R}$ be the vector of particle coordinates in the laboratory
fixed reference frame:
\begin{equation}
{\bf R=}\left[ 
\begin{array}{c}
{\bf R}_1 \\ 
{\bf R}_2 \\ 
\vdots  \\ 
{\bf R}_N
\end{array}
\right] =\left[ 
\begin{array}{c}
X_1 \\ 
Y_1 \\ 
Z_1 \\ 
\vdots  \\ 
Z_N
\end{array}
\right] 
\end{equation}
Center--of--mass and internal coordinates are given by the transformation $T:%
{\bf R}\mapsto [{\bf r}_0^{\prime },{\bf r}^{\prime }]^{\prime }$,
\begin{equation}
T=\left[ 
\begin{array}{ccccc}
\frac{M_1}{m_0} & \frac{M_2}{m_0} & \frac{M_3}{m_0} & \cdots  & \frac{M_N}{%
m_0} \\ 
-1 & 1 & 0 & \cdots  & 0 \\ 
-1 & 0 & 1 & \cdots  & 0 \\ 
\vdots  & \vdots  & \vdots  & \ddots  & \vdots  \\ 
-1 & 0 & 0 & \cdots  & 1
\end{array}
\right] \otimes I_3  \label{Ttran}
\end{equation}
where $m_0=\sum_i^NM_i$. ${\bf r}_0$ is the vector of coordinates for the
center--of--mass and ${\bf r}$ is a length $3n=3\left( N-1\right) $ vector
of internal coordinates with respect to a reference frame with origin at
particle 1 (this particle is usually the heaviest particle in the system):
\begin{equation}
{\bf r=}\left[ 
\begin{array}{c}
{\bf r}_1 \\ 
{\bf r}_2 \\ 
\vdots  \\ 
{\bf r}_n
\end{array}
\right] =\left[ 
\begin{array}{c}
{\bf R}_2-{\bf R}_1 \\ 
{\bf R}_3-{\bf R}_1 \\ 
\vdots  \\ 
{\bf R}_N-{\bf R}_1
\end{array}
\right] .  \label{rdef}
\end{equation}
Using this coordinate transformation, and the conjugate momentum
transformation, the internal Hamiltonian for the problem 
we are considering
can be written as \cite{Kinghorn93,Kinghorn95b}: 
\begin{equation}
H=-\frac 12\left( \sum_i^n\frac 1{\mu _i}\nabla _i^2+\sum_{i\neq j}^n\frac
1{M_1}\nabla _i\cdot \nabla _j\right) +V\left(
r_{ij};\,\,\,i<j,\,\,\,\,\,i=0...n\right)   \label{intham1}
\end{equation}
where the $\mu _i$ are reduced masses, $M_1$ is the mass of particle 1 (the
coordinate reference particle), and $\nabla _i$ is the gradient with respect
to the $x,y,z$ coordinates ${\bf r}_i$. The potential energy is still a
function of the distance between particles but is now written using internal
distance coordinates, $r_{ij}=\left\| {\bf r}_i-{\bf r}_j\right\| =$ $%
\left\| {\bf R}_{i+1}-{\bf R}_{j+1}\right\| \,\,$with\thinspace $%
r_{0j}\equiv r_j=\left\| {\bf r}_j\right\| =\left\| {\bf R}_{j+1}-%
{\bf R}_1\right\| .$ The kinetic energy term in this Hamiltonian can be
written as a quadratic form in the length $3n$ vector gradient operator, $%
\nabla _{{\bf r}},$ the gradient with respect to the length $3n$ vector $%
{\bf r}$ of internal coordinates. This gives a compact matrix/vector form
of the Hamiltonian with the kinetic energy expressed as a quadratic form in
the gradient operator, 
\begin{equation}
H=-\nabla _{{\bf r}}^{\prime }\left( M\otimes I_3\right) \nabla _{{\bf%
r}}+V\left( r_{ij};\,\,\,i<j\,,\,\,\,\,i=0...n\right)   \label{ham}
\end{equation}
$M$ is an $n\times n$ matrix with $1/2\mu _i$ on the diagonal and $1/2M_1$
for off--diagonal elements. This is the Hamiltonian we 
will use in variational
energy calculations of ro--vibrational states. 

Our n-body potential expansion function, $\prod_{i<j}r_{ij}^{m_{kij}}\exp
\left[ -{\bf r}^{\prime }(A_k\otimes I_3){\bf r}\right] $, is written
using the scalar ``distance'' variables $\left\{ r_{ij}\right\} $ and the
internal coordinate vector variable ${\bf r.}$ These two sets of
variables both completely describe the ``geometry'' of a system. To make
this equivalence more obvious and to present alternative forms for the $\phi
_k$, we will show the interchangeability of these two sets of variables. $%
r_{ij}^m$ can be written as a function of ${\bf r}$ using the matrix $%
\left( J_{ij}\otimes I_3\right) $ with $J_{ij}$ defined as an $n\times n$
matrix with 1's in the $ii$ and $jj$ positions, -1 in the $ij$ and $ji$
positions and 0's elsewhere\cite{Poshusta83,Kinghorn95a}, 
\begin{equation}
r_{ij}^m=\left[ {\bf r}^{\prime }(J_{ij}\otimes I_3){\bf r}\right]
^{m/2}.  \label{rijJ}
\end{equation}
$r_{ij}^m$ can, equivalently, be written using the component vectors, $%
{\bf r}_i$, of ${\bf r}$ 
\begin{equation}
r_{ij}^m = \left[ {\bf r}_i^{\prime }{\bf r}_i+{\bf r}_j^{\prime }%
{\bf r}_j-2{\bf r}_i^{\prime }{\bf r}_j\right] ^{m/2} 
= \left[ r_i^2+r_j^2-2{\bf r}_i^{\prime }{\bf r}_j\right] ^{m/2}.
\end{equation}
Thus, using eqn(\ref{rijJ}), $\phi _k$ can be written purely in terms of the
vector variable ${\bf r}$, 
\begin{equation}
\phi _k=\prod_{i<j}\left[ {\bf r}^{\prime }(J_{ij}\otimes I_3){\bf r}%
\right] ^{\frac{m_{kij}}2}\exp \left[ -{\bf r}^{\prime }(A_k\otimes I_3)%
{\bf r}\right] .  \label{phir}
\end{equation}

Alternatively, $\phi _k$ can be expressed using the distance coordinates $%
\left\{ r_{ij}\right\} $. The quadratic form in the exponential of $\phi _k$
may be converted to $\left\{ r_{ij}\right\} $ variables as follows (we drop
the subscript $k$ for convenience):
\begin{equation}
{\bf r}^{\prime }(A\otimes I_3){\bf r} = \sum_{i,j}{\bf r}%
_i^{\prime }{\bf r}_j\,\,A_{ij} 
= {\sf tr}\left[ \left( {\bf r}_i^{\prime }{\bf r}_j\right) A\right] 
= {\sf tr}\left[ \left( r_{ij}^2\right) B\right]  
= \sum_{i,j}r_{ij}^2\,\,B_{ij}
\end{equation}
where ${\sf tr}\left[ \:\: {}\right] $ 
is the matrix trace operator, $\left( 
{\bf r}_i^{\prime }{\bf r}_j\right) $ is the $n\times n$ matrix of dot
products of the component vectors of ${\bf r},$ $\left( r_{ij}^2\right) $
is the $n\times n$ matrix of squared distance variables, and $B$ is a matrix
with elements given, in terms of the elements of an arbitrary matrix $A$, by
the transformation, 
\begin{equation}
B_{ij}=\left\{ 
\begin{array}{ll}
\frac 12\sum_{k=1}^{n}\left( A_{ik}+A_{kj}\right) , & i=j \\ 
-\frac 14\left( A_{ij}+A_{ji}\right) , & i\neq j
\end{array}
\right. .  \label{Btran}
\end{equation}
Hence, the $\phi _k$ can be written using only distance coordinates, 
\begin{equation}
\phi _k=\prod_{i<j}r_{ij}^{m_{kij}}\exp \left[ -{\sf tr}\left[ \left(
r_{ij}^2\right) B_k\right] \right] .
\end{equation}
If one has $\phi _k$ in terms of $B_k$, {\it i.e.} 
a function of $r_{ij},$ and
wishes to transform to $A_k,$ a function of ${\bf r}$, the following
relation can be used, 
\begin{equation}
A_{ij}=\left\{ 
\begin{array}{ll}
B_{ii}+\sum_{k\neq i,j}^{n}\left( B_{ik}+B_{kj}\right) , & i=j \\ 
-\left( B_{ij}+B_{ji}\right) , & i\neq j
\end{array}
\right. .  \label{Atran}
\end{equation}
If $\phi _k$ is to be square integrable, 
then $A_k$ must be positive definite. The simplest way to insure this
condition is to write $A_k$ in Cholesky factored form, 
\begin{equation}
A_k=L_kL_k^{\prime }, \; \; \; \;
L_k
{\rm \; \; lower  \; triangular \; and \;
rank \;}n.
\end{equation}

\vspace{2mm}
\noindent
{\bf B.3 N--Body Potential}

Consider the suitability of $\phi_k$ as an expansion function for potential
energy hypersurfaces. There are several requirements that a general n-body
potential needs to satisfy:

\begin{enumerate}


\item  It should be invariant to rotations 
and translations of the system.
Using the vector form, eqn(\ref{phir}), it is easy to show that $\phi _k$ is
rotationaly invariant, {\it i.e.}, 
invariant to any orthogonal transformation.
Let $U$ be any $3\times 3$ orthogonal matrix (any proper or improper
rotation in 3 space) then the action of $U$ on $\phi _k$ is to transform the
quadratic forms in the pre-multiplying factors and exponential factor as
(using the exponential factor as an example): 
\begin{eqnarray}
\left( \left( I_n\otimes U\right) {\bf r}\right) ^{\prime }(A_k\otimes
I_3)\left( I_n\otimes U\right) {\bf r} &=&{\bf r}^{\prime }\left(
I_n\otimes U^{\prime }\right) (A_k\otimes I_3)\left( I_n\otimes U\right) 
{\bf r} \\
&=&{\bf r}^{\prime }(A_k\otimes U^{\prime }U){\bf r} 
= {\bf r}^{\prime }(A_k\otimes I_3){\bf r}
\end{eqnarray}
leaving $\phi _k$ invariant. Hence, any expansion in $\phi _k$ will be
isotropic in ${\cal R}^3$.




\item  It should collapse to a suitable (n-m)-body potential as any m
particles are removed from the system.
To insure the correct asymptotic behavior of the potential as any particles
are adiabaticly removed from the system the many-body expansion described
by Murrell and co-workers\cite{Murrell84} can be used. In this expansion the
potential is written as a sum of $M$-body terms with $M$ ranging from 1 to $N
$ for an $N$ particle system, 
\begin{equation}
V=\sum_{\left\{ A\right\} }V_A+\sum_{\left\{ AB\right\}
}V_{AB}+\sum_{\left\{ ABC\right\} }V_{ABC}+\cdots 
+ V_{\underbrace{ABCD \cdots}_N}
%
%+ V\stackunder{N}{_{%
%\underbrace{ABCD\ldots }}}.
\end{equation}
$V_A$ is the energy of particle $A$ in the state that
would result by adiabaticly removing it from the system. The remaining
terms are the appropriate $M$-body terms for $M$ from 2 to $N$. There are $%
%\left(\begin{array}{ll} N \\ M \end{array} \right) =
%\binom NM=
\frac{N!}{M!\left( N-M\right) !}$ $M$-body terms in an $N$
particle system. This is best illustrated with a simple example. Consider a
system of three atoms labeled $A,\,B,\,C$, so that the distance from
particle $A$ to particle $B$ is $r_1$, $A$ to $C$ is $r_2$ and $B$ to $C$ is 
$r_{12}$. In our approach the $ABC$ 
potential will be represented by the following expression:
The model potential can then be written as: 
\begin{eqnarray}
V &=&V_A+V_B+V_C+V_{AB}+V_{AC}+V_{BC}+V_{ABC}  \nonumber \\
&=&V_A+V_B+V_C  \nonumber \\
&&+\sum_i c_i^{AB}r_1^{m_i^{AB}}e^{-B_i^{AB}r_1^2}+\sum_i
c_i^{AC}r_2^{m_i^{AC}}e^{-B_i^{AC}r_2^2}+\sum_i
c_i^{BC}r_{12}^{m_i^{BC}}e^{-B_i^{BC}r_{12}^2}  \nonumber \\
&&+\sum_i c_i^{ABC}r_1^{m_i^{ABC}}r_2^{n_i^{ABC}}r_{12}^{p_i^{ABC}}\exp \left[
-{\bf r}^{\prime }\left( A_i^{ABC}\otimes I_3\right) {\bf r}\right] 
\label{MbodyV}
\end{eqnarray}
where $V_A+V_B+V_C$ is the sum of the energy of the separated atoms, the
next three terms are expansions for the possible two-body terms and the last
term is an expansion for the full three-body term. This potential has the
correct asymptotic behavior as long as the exponent parameters are positive
definite. It is important to notice 
that by admitting negative powers of $r_i$ and $r_{ij}$ 
the expansion (\ref{MbodyV}) 
can effectively describe the $1/R^n, n > 0,$
decay of the two-body components of the molecular potential.
Also one notices
that the matrix elements
$\langle \phi_k | V | \phi_l \rangle$ are simple overlap integrals.


\item  
The function should approach infinity as any two like particles
approach each other.
%\subsubsection{Asymptotic behavior $r_{ij}\rightarrow 0$}
The correct asymptotic behavior of the potential as any $r_{ij}$ approach
zero is insured by including at least one negative value of the exponent $m$
in each of the two-body expansions when an expansion such as eqn(\ref{MbodyV}%
) is used. 


\item  
The function should be flexible enough to handle complicated behavior
such as multiple maxima and minima.
It is not expected that an expansion in $\phi _k$ will result in the most
compact analytic model function possible for a given potential but rather
that the $\phi _k$ will provide the flexibility needed to model complex
dynamical behavior, perhaps by using a larger 
number of the $\phi _k$ in an
expansion. A general $\phi _k$ term in an $n$-body component of an expansion
will contain one linear parameter and $n\left( n+1\right) /2$ non-linear
parameters and many terms could be used to represent this component.
Additionally, 
for each term there are the powers, $m_{ij},$ for the $r_{ij}$
pre-multiplying factors to be chosen. 
In our preliminary calculations of some sample three- and four-body
potentials, we found that the these pre--multiplying factors 
are essential to effectively describe the minima on the
potential energy hyper--surface.
Placing a power of an interparticle distance in front of the gaussian
function
shifts the maximum of the gaussian away from the origin.
Since this shift is dependent on the power, one can effectively 
model different interaction potentials with a few gaussians
(see an example in Section B.6);
the point being---
the $\phi _k$ provide a large pallet of terms for modeling a potential by
non-linear data fitting, but these terms will obviously need to chosen
carefully. This is the art of non-linear curve fitting. 

%The number of possible pre-multiplying
%factors for an $n$-body term with $Z$ different values of the powers $m$ is
%given by, 
%\begin{equation}
%%\textstyle
%{\# of terms }\left( \prod_{i<j}r_{ij}^{m_{kij}}\right) 
%%\textstyle{ }%
%=Z^{n\left( n+1\right) /2}.
%\end{equation}
%For example, there would be 4096 possible pre-multiplying factors for a
%4-body term if all possible values of $m_{ij}$ including -1, 0 ,1 and 2 are
%used. (Obviously one would not want to use all of these.) Thus, there could
%be a large number of parameters available for fitting the model potential to 
%{\it ab-initio} and/or experimental data. This gives the model potential
%using $\phi _k$ the advantage of great flexibility at the expense of,
%possibly, difficult optimization of the many parameters. 



\item  
The function should be differentiable.
The $\phi _k$ are infinitely differentiable. Therefore, any potentials
modeled as linear combinations of $\phi _k$ are also infinitely
differentiable and expressible in terms of the derivatives of the $\phi _k$.
Derivatives of the potential can be useful for fitting to, or predicting,
experimental force field data and for imposing fitting constraints based on
equilibrium geometries. Derivatives are also useful for characterizing
features of the potential such as local extrema and saddle points. Treating $%
\phi_k$ as a function of the vector of relative coordinates, ${\bf r,}$
as in eqn(\ref{phir}), the gradient vector and Hessian matrix can be derived
using matrix differential calculus\cite{Kinghorn95a}. Introducing an
over-bar notation for the Kronecker product with the $3\times 3$ identity, 
{\it i.e.} $A_k\otimes I_3=\bar{A}_k\,\,$, and defining $G=\left(
\sum_{i\leq j}m_{kij}\left( {\bf r}^{\prime }\bar{J}_{ij}{\bf r}%
\right) ^{-1}\bar{J}_{ij}\right) $, the gradient with respect to ${\bf r}
$ is then, 
\begin{equation}
\nabla _{{\bf r}}\phi _k=\left[ G-2\bar{A}_k\right] \phi _k\,{\bf r,}
\end{equation}
and the Hessian is given by, 
\begin{eqnarray}
\frac{\partial \nabla _{{\bf r}}\phi _k}{\partial {\bf r}^{\prime }}
&=&\left[ G+\left[ G-2\bar{A}_k\right] {\bf rr}^{\prime }\left[ G-2\bar{A}%
_k\right] -2\bar{A}_k\mathstrut \strut 
\begin{array}{ll}
&  \\ 
& 
\end{array}
\right.   \nonumber \\
&&\left. -2\sum_{i\leq j}m_{ij}\left( {\bf r}^{\prime }\bar{J}_{ij}%
{\bf r}\right) ^{-2}\bar{J}_{ij}{\bf rr}^{\prime }\bar{J}_{ij}\right] .
\end{eqnarray}

Two other derivatives which may be useful when fitting a potential 
or optimizing the non--linear parameters in the w.f.
are the
derivatives with respect to the non-linear parameters in the matrices $A_k$
and $B_k$, which can also be derived using matrix differential calculus
\cite{Kinghorn95a}. 
Derivatives of gaussians can be used to 
minimize the variational functional and to calculate
energy levels. We have shown the utility of such an approach
in algorithms for variational BO and non-BO calculations,
where we used first and second derivatives to guide
the Newton--Raphson optimization procedure
\cite{A8,A14,A21,A38,kozlowski92b}. 

\item  There should be a way to constrain the model potential to the
physical symmetries of the system.
A meaningful model potential will necessarily reflect the physical symmetry
of the system being modeled. Rotational symmetry has already been addressed,
(the $\phi _k$ are isotropic), leaving the problem of how to handle
permutational symmetry. The potential should be totally symmetric with
respect to exchange of like particles. This permutational symmetry can be
accounted for in the $\phi _k$ using a projection method. Consider a system
of $N$ particles invariant under the action of a group, $G$, represented by
a set, $\left\{ P_{\alpha \in G}\right\} ,$ of $N\times N$ permutation
matrices. A model potential represented as an expansion in $\phi _k$ is a
function of the $n=N-1$ component vectors of ${\bf r}$, the relative
coordinates. The permutation $P_\alpha $ acting on the $N$ particle
coordinates induces a transformation on the center--of--mass and relative
coordinates given by 
\begin{equation}
T\bar{P}_\alpha T^{-1}=I_3\oplus \bar{\tau}_\alpha,
\end{equation}
where $T$ is the transformation matrix given in eqn(\ref{Ttran}). 
The right hand side of this 
expression is the direct sum of the identity acting on the
center--of--mass coordinates, ${\bf r}_0$, and $\tau _\alpha ,$ which is an 
$n\times n$ ``permutation'' matrix acting on the component vectors of the
relative coordinate vector ${\bf r}$. The action of the permutation
represented by $P_\alpha $ on $\phi _k$ is then, 
\begin{equation}
P_\alpha \phi _k=\prod_{i<j}\left[ {\bf r}^{\prime }(\tau _\alpha
^{\prime }J_{ij}\tau _\alpha \otimes I_3){\bf r}\right] ^{\frac{m_{kij}}%
2}\exp \left[ -{\bf r}^{\prime }(\tau _\alpha ^{\prime }A_k\tau _\alpha
\otimes I_3){\bf r}\right] .
\end{equation}
The action of the totally symmetric representation of the group $G$ on $\phi
_k$ is thus induced by the projector $\sum_{\alpha \in G}\tau _\alpha .$
This method of symmetry projection on correlated gaussians is discussed in
more detail in the references\cite{Kinghorn93,Kinghorn95b,Poshusta83}.


\end{enumerate}

%\noindent As will be shown bellow, an expansion in $\phi _k$ can satisfy all
%of these requirements.
%\subsubsection{Rotational invariance}
%\subsubsection{Asymptotic behavior $r_{ij}\rightarrow \infty \,\,\,\,$(The
%many-body expansion)}
%\subsubsection{Flexibility of $\phi _k$}
%\subsubsection{Derivatives}
%\subsubsection{Symmetry}

\vspace{2mm}
\noindent
{\bf B.4 Variational Wave Function}

In vibration-rotation calculations,
the w.f.'s will be
expanded as symmetry projected linear combinations of the explicitly
correlated $\phi _k$ multiplied by an angular term, $Y_{LM}^k:$ 
\begin{equation}
\Psi_{LM\Gamma }={\cal P}_\Gamma \sum_k c_k Y_{LM}^k\phi_k.  \label{wf}
\label{Y1}
\end{equation}
Here $\phi _k$ are the explicitly correlated n-body gaussians given 
in eqn(%
\ref{basisfcn}), ${\cal P}_\Gamma $ is an appropriate permutational
symmetry projection operator for the desired state, 
$\Gamma $, and $Y_{LM}^k$
is a product of coupled solid harmonics labeled by the total angular
momentum quantum numbers $L$ and $M$.
Permutational symmetry is handled using projection methods in the same
manner as described for the potential expansion in the previous section.
%Again, the reader is referred to the references for details\cite
%{Poshusta83,Kinghorn93,Kinghorn95b}.

$Y_{LM}^k$ is a vector--coupled product of solid harmonics\cite{Biedenharn81}
given by the Clebsch-Gordon expansion, 
\begin{equation}
Y_{LM}^k=\sum_{\,\,\begin{array}{c} {\left\{ l_j,m_j\right\} }
\\ {m_1+\cdots
+m_n=M} \end{array}
}\left\langle LM;k\right. |\left. l_1m_1\cdots l_nm_n\right\rangle
%
%Y_{LM}^k=\sum_{\,\,\QATOPD. . {\left\{ l_j,m_j\right\} }{m_1+\cdots
%+m_n=M}}\left\langle LM;k\right. |\left. l_1m_1\cdots l_nm_n\right\rangle
\prod_j^n{\cal Y}_{l_jm_j},
\end{equation}
and the solid harmonics are given by:
\begin{equation}
{\cal Y}_{lm}\left( {\bf r}_j\right) =\left[ \frac{2l+1}{4\pi }\left(
l+m\right) !\,\left( l-m\right) !\right] ^{\frac 12}\sum_p\frac{\left(
-x_j-iy_j\right) ^{p+m}\left( x_j-iy_j\right) ^pz_j^{l-2p-m}}{2^{2p+m}\left(
p+m\right) !\,p!\,\left( l-m-2p\right) !}.
\end{equation}
The ${\cal Y}_{lm}\left( {\bf r}_j\right) $ are single particle
angular momentum eigen-functions in relative coordinates which transform the
same as spherical harmonics, {\it i.e.} have the same eigen-values. Since
the $\phi _k$ are angular momentum eigen-functions with zero total angular
momentum, the product with $Y_{LM}^k$ can be used in principle to obtain any
desired angular momentum eigen-state. Note the $k$ dependence of $Y_{LM}^k$;
this is included since there are many ways to couple the individual angular
momenta, $l_j$, 
to achieve the desired total angular momentum $L$ and it may
be necessary to include several sets of the $l_j$ in order to obtain a
realistic description of the w.f. Varga and Suzuki\cite{Varga95}
have recently
proposed representing the angular dependence of the w.f. using
a single solid harmonic whose argument contains additional variational
parameters, ${\bf u} = (u_1, u_2, \cdots, u_n)$:
\begin{equation}
\Psi_{LM\Gamma } = {\cal P}_\Gamma Y_{LM}({\bf v}) \sum_k c_k \phi_k,
\; \; {\rm with} \; \; 
{\bf v} = \sum_{i=1}^{n} u_i {\bf r}_i.
\label{Y2}
\end{equation}
There appears to be several advantages in doing this and we are
investigating the possibility of using this approach in our full N-body
implementation.
The strict separation of the angular and ``radial" variables is
eqns.(\ref{Y1}) and (\ref{Y2}) allows separate consideration of
the vibrational states with different total angular momentum
quantum number, $L$. The magnitude of the Coriolis coupling
for the particular $L$-state will determine whether the most
general form, eqn.(\ref{Y1}), or more simplified form,
eqn.(\ref{Y2}), of the total w.f. should be used. 

There have been several highly accurate non-adiabatic variational
calculations on atomic and exotic few particle systems using simple
correlated gaussians\cite
{Kinghorn93,Kinghorn95b,A3,A37,Varga96}.
By simple we mean
they only contain the exponential part of the $\phi _k,$ (no $r_{ij}$
pre-multipliers). However, attempts at non-adiabatic molecular calculations
have been plagued by problems with linear dependence in the basis during
energy optimizations. This problem occurs in calculations on atomic systems
also, but to a much lesser extent. We believe that we understand this
phenomena and that our new basis, 
including pre-multiplying powers of $r_{ij}$
will eliminate or at least drastically reduce the linear dependence
problems. Our reasoning is as follows: In systems with more than one heavy
particle ({\it e.g.}, two or more nuclei in a non-B.O. calculation,
or two or more atoms in a ro-vibrational calculation)
there will be a larger particle density away from the origin,
where one of the heavy particles is placed, than near the origin.
In the case of a system with three heavy particles, in addition
to a reduction of the probability density of two of the particles
when they approach the third one located at the coordinate origin,
there will be a density reduction when the two particles when 
they approach
each other.
That is, the w.f. will have peaks shifted
away from the origin and peaks shifted away from 
$r_{ij}$=0 points. There are three ways to account for this behavior in
the w.f. expressed in terms of 
correlated gaussians; 1) use correlated gaussians
with shifted centers, {\it i.e.}, 
$\exp [-\left( {\bf r-s}\right) ^{\prime }%
\bar{A}\left( {\bf r-s}\right) ]$; 2) Use near linearly dependent
combinations of simple correlated gaussians with large matched $\pm $ linear
coefficients; or 3) Use pre-multiplying powers of $r_{ij}$. The first option
is unacceptable since it results in a w.f. which no longer
represents a pure angular momentum state. The second option is what we
believe causes the linear dependence and numerical instability which we are
trying to avoid. The third option is what we are proposing. The linear
dependence that we have observed in our calculations using the simple
correlated gaussians looks, in some sense, like an attempt by the
optimization to include in the w.f. derivatives of the basis
functions with respect to the non-linear parameters. The near linear
dependent terms resemble numerical derivatives. Removal of these near linear
dependent terms has an adverse affect on the w.f.'s, as manifested by
poor energy results, but leaving them in leads to numerical instabilities
which hinder optimization or cause complete collapse of the eigen-solutions.
Now, derivatives of simple gaussians with respect to non-linear parameters,
elements of the matrices $A_k$, bring down pre-multiplying (even) powers of $%
r_{ij}$. Thus, explicitly including pre-multiplying powers of $r_{ij}$ in
the basis functions should add the needed flexibility to the basis in a
numerically stable way. Also, we expect the rate of convergence to be
improved by these pre-multiplying $r_{ij}^m$ terms in the same way that they
effect convergence in the Hyllerass basis. The $\phi _k$ are similar to the
Hyllerass basis functions with the Slater--type exponentials replaced by
fully correlated gaussian type exponentials. We believe that excellent
results for both case 1 and case 2 type calculations will be obtained using
our new variational basis functions. We have preliminary results to support
the above assertions (see the next section).

The proposed work requires new types of integrals to be derived
and implemented. Over the last few years we have made several
steps in this direction. We have derived multi--particle
Hamiltonian integrals with N--body 
general cartesian correlated gaussians
\cite{kozlowski92a}, we presented integrals involving
$J_z$ and $J^2$ operators \cite{A20}, and we recently derived
Hamiltonian matrix elements with correlated gaussians
containing high powers of coordinates  \cite{A382}.
This work will continue within this project.

\vspace{2mm}
\noindent
{\bf B.5 Variational Wave Function for Non-Adiabatic Calculations}

The construction of the wave function for a molecule 
to be used as an
starting guess in variational non--adiabatic calculations
can be based on a superposition of the electronic wave function
obtained in a BO calculation and the ro--vibrational wave function
obtained in calculation where a fit to the potential energy
surface was used as the interaction potential. Let us use the H$_2^+$
ion as an example. Using gaussians, the BO electronic wave function 
wave function can be represented as the following expansion:
\begin{equation}
\Psi^{BO} = 
\sum_i c_i^{BO} r_{ {\bf A}1}^{m_i^A} r_{ {\bf B}1}^{m_i^B}
{\rm exp} \left[ -\alpha_i^A r_{ {\bf A}1}^2 
- \alpha_i^B r_{ {\bf B}1}^2 \right],
\label{BO1}
\end{equation}
where the bold indeces indicate the centers, which are not allowed to move.
The ro--vibrational wave function (with $J=0$), which is obtained using the
approach described above with the ${\bf A}$ nucleus assumed to be
the center of the internal coordinate system, has the following form
(see the following section):
\begin{equation}
\Psi^{ro-vib} = 
\sum_i c_i^{ro-vib} r_{ {\bf A}B}^{m_i^{AB}} 
{\rm exp} \left[ -\alpha_i^{AB} r_{ {\bf A}B}^2 \right].
\label{BO2}
\end{equation}
Now we construct the superposition of the two functions:
\begin{equation}
\Psi^{non-BO} = 
\sum_{i,j} c_i^{ro-vib} c_j^{BO} 
r_{ {\bf A}B}^{m_i^{AB}}
r_{ {\bf A}1}^{m_j^A} r_{B1}^{m_j^B}
{\rm exp} 
\left[ -\alpha_i^{AB} r_{ {\bf A}B}^2 -\alpha_j^A r_{ {\bf A}1}^2 
- \alpha_j^B r_{B1}^2 \right],
\label{BO3}
\end{equation}
The above approach can easily be extended to systems with more particles,
as well as to excited states.



\vspace{2mm}
\noindent
{\bf B.6 An Illustrative Example}

We will provide a simple example of using $\phi _k$ as expansion functions
for potentials and as variational vibrational energy basis functions for 
the $^1\sum_g^{+}$ $H_2$ potential of Kolos and 
Wolniewicz \cite{Kolos65}. 
The purpose of these calculations was to test the ideas and general
procedure for using $\phi _k$. Angular factors where not included in these
trial calculations and thus, since the $\phi _k$ are angular momentum
eigen-functions with total angular momentum equal to zero, the results 
correspond to
ground and excited vibrational levels in the rotational ground state. 
The general procedure for obtaining vibrational energies involves three
steps: 1) obtain a high quality numerical potential, 2) fit the potential to
an expansion in $\phi _k,$ 3) solve the eigen-problem using $\phi _k$ as
basis functions for the vibrational w.f.'s. 
A least--squares procedure 
was implemented to fit an expansion in the form 
\begin{equation}
V\left( r\right) =c_0+\sum_ic_ir^{m_i}e^{-b_ir^2}
\end{equation}
to the numerical $H_2$ potential:
\begin{equation}
\min \left[ \chi \right] =\min_{\left\{ b_i\right\} }\left\| 
V\left( r \right) -V_{H_2}\left( r \right) \right\| .
\end{equation}
An expansion using 15 of the $\phi _k$
with values of the $r$ powers ranging from -1 to 4 yielded an excellent fit
with $\chi =7.5\times 10^{-6}.$ 

Next variational calculations 
for the vibrational energy levels were carried out
using our fitted potential and compared with the numerically integrated
results of Wolniewicz \cite{Wolniewicz66}. Agreement between the two
different treatments of the data were very good. Also, with our fitted
potential we were able to compute all of the 15 levels below the
disassociation limit, whereas, only the first 13 levels were obtainable with
the numerical integration. Our two highest vibrational levels agree within
two wave numbers of the results obtained by LeRoy\cite{LeRoy68},
thus giving us confidence in our result for these highest levels.
Table \ref{freqtab} compares our computed
frequencies, $\Delta = E_{v+1}-E_v$, 
with those of Wolniewicz \cite{Wolniewicz66}.
When a single term w.f. 
\begin{equation}
\Psi _0=Nr^{17}e^{-4.3496r^2}
\end{equation}
was used to calculate the ground vibrational state,
we obtain a zero point energy in error
by only 15 micro-hartrees (3.3$cm^{-1}$).
Worthnoticing is a large pre--exponential power of $r$.

%TCIMACRO{\TeXButton{B}{\begin{table}[tbp] \centering}}
%BeginExpansion
\begin{table}[tbp] \centering%
%EndExpansion
\begin{tabular}{cccr}
\hline\hline
$v$ & Wolniewicz & This work & $\Delta $ \\ \hline
\multicolumn{1}{r}{0} & 4163.46 & 4163.44 & -.02 \\ 
\multicolumn{1}{r}{1} & 3927.93 & 3927.94 & .01 \\ 
\multicolumn{1}{r}{2} & 3697.41 & 3697.41 &  \tt{<} .01 \\ 
%\multicolumn{1}{r}{2} & 3697.41 & 3697.41 & \TEXTsymbol{<}.01 \\ 
\multicolumn{1}{r}{3} & 3469.51 & 3469.53 & .02 \\ 
\multicolumn{1}{r}{4} & 3242.69 & 3242.66 & -.03 \\ 
\multicolumn{1}{r}{5} & 3014.68 & 3014.77 & .09 \\ 
\multicolumn{1}{r}{6} & 2782.66 & 2782.56 & .10 \\ 
\multicolumn{1}{r}{7} & 2543.24 & 2543.33 & .09 \\ 
\multicolumn{1}{r}{8} & 2292.69 & 2292.73 & .04 \\ 
\multicolumn{1}{r}{9} & 2025.66 & 2025.96 & .30 \\ 
\multicolumn{1}{r}{10} & 1735.53 & 1735.42 & -.11 \\ 
\multicolumn{1}{r}{11} & 1413.65 & 1413.44 & -.21 \\ 
\multicolumn{1}{r}{12} &  & 1047.12 &  \\ 
\multicolumn{1}{r}{13} &  & 618.35 &  \\ \hline
\end{tabular}
\caption{Vibrational frequencies, $ E_{v+1}-E_{v }$,  for $ H_2 $ computed
variationally
 using a fitted potential  and the numerically integrated results of
Wolniewicz.
 All values in $ cm^{-1} $ } \label{freqtab}%
%TCIMACRO{\TeXButton{E}{\end{table}}}
%BeginExpansion
\end{table}%
%EndExpansion

\vspace{2mm}
\noindent
{\bf B.7 Proposed Applications}


Following our previous non--adiabatic calculations of HD$^+$
and the positronium molecule and some other model systems
\cite{A3,A6,A16,A27,kozlowski91,%
kozlowski92b,kozlowski92c} and more 
recent non--adiabatic calculations
of the electron affinities of hydrogen, deuterium and tritium
\cite{A37},
we will now consider smaller atomic and molecular systems.
In the first application calculations 
we will consider
hydrogen molecules and ions and their isotopomers,
H$_2^+$, H$_2$, H$_3$, {\it etc.}, 
and other molecular anions, {\it e.g.}, HeH$^+$, HeH$_2^+$.
Very accurate measurements for these systems, which have recently become
available present an exciting challenge for 
the "non--Born-Oppenheimer Atomic and Molecular Quantum Mechanics."
The aim of these calculations will be to determine 
electron affinities and electron detachment energies 
of these systems in different ro-vibrational states.
Also calculations of 
higher excited states
where the coupling of the electronic and nuclear motions
can be significant, particularly for anions,
will be performed and the nature and magnitude of the 
coupling effect will be 
analysed.

Calculations of the bound nuclear motion states of molecules and
clusters are making a contribution to many areas of physical science
including combustion, astrophysics, planetary atmospheres and
chemical kinetics. Applications of ro--vibrational calculations
to the area of traditional high--resolution spectroscopy, {\it i.e.},
low--lying states, have been numerous. These include predicting and
assigning spectra, calculating transition intensities and 
generating data for the calculation of thermodynamic and emissivity
parameters. Ro-vibrational calculations have also provided links
between highly excited ro-vibrational states, reaction dynamics and
quantum chaology.

One of the most remarkable successes
for theory has been the work on the $H_3^{+}$
system. The first laboratory spectroscopic detection of  $H_3^{+}$,
reported by Oka \cite{ref:k35} was of ro--vibrational transitions in the
fundamental $\nu_2$ infrared band. An important ingredient in the
identification of the spectrum was the set of vibration--rotation
constants calculated by Carney and Porter \cite{ref:k36}. This first
laboratory measurement prompted searches for $H_3^{+}$ in the interstellar
medium, and several spectacular discoveries were made of the presence
of $H_3^{+}$ in the Jovian \cite{ref:k41}, Saturn \cite{ref:k44} and
Uranus \cite{ref:k45} aurora. There has been 
extensive theoretical
work on $H_3^{+}$ and its isotopomers by Tennyson and coworkers
\cite{ref:ten2,ref:ten3,ref:ten4,ref:ten5,ref:ten6,ref:ten7,ref:ten8,%
ref:ten9}, where a wide range of the ro--vibrational excitations were
calculated using BO and adiabatic PES's. We will use this 
work as a reference for ``calibration" calculations, which we will
first perform with our method once it is fully implemented. 
For these calculations we will apply the PES used by Tennyson, as well as
the PES generated with the use of correlated floating gaussian
geminals, which we will calculate with our new computer 
program \cite{A38}. This program employs analytical 
energy derivatives
with respect to the linear and non-linear 
variational parameters involved in the w.f. 
to minimize the total energy of the system.

In the next step we will consider another calibration case 
of ro--vibrational states of $H_2O$.
Due to the recent spectacular discovery of water on the Sun based
on the assignment of high--resolution IR spectra and simultaneous
laboratory measurement of high--temperature emission spectra 
\cite{ref:bern1}, 
some very accurate calculations have been done on 
ro-vibrational structure of this system
by Tennyson and coworkers
on the theoretical side \cite{ref:ten10,ref:ten11,ref:ten12,%
ref:ten13,ref:ten14}. Again in this case we will use PES, employed
by Tennyson and coworkers, as well as the PES which we will 
generate with the use of our state--selective, multi--reference
coupled cluster [SSMRCC(TQ)] method 
\cite{A1,A4,A11,A17,A22,A25}.


Application of our approach to systems with more than three
nuclei are the most interesting since 
other methods have been limited to three-body cases.
We find particularly interesting initiate studies
of hot elemental clusters and their anions such as
C$_n$, Si$_n$, P$_n$, $n$=4,5 {\it etc.}.
Calculations of highly excited ro-vibrational states near the
dissociation barrier of
these systems will reveal their structures and dynamics 
at high temperatures.

Application of our approach to systems with more than three
atoms are the most interesting since 
other methods have been limited to three-body cases.
We find particularly interesting to investigate
hot elemental clusters, such as
C$_n$, Si$_n$, P$_n$, $n$=4,5 {\it etc.}, and their anions.
Calculations of highly excited ro-vibrational states near the
dissociation barrier of
these systems will reveal their structures and dynamics 
at high temperatures.
The proposed method should work much better in
applications concerning such extremely floppy systems
then the traditional approaches, which usually start 
from the rigid rotor approximation in a body-fixed rotating 
coordinate system and then correcting it either using the
perturbation theory or the variational technique.

Among other applications we will consider
the high--temperature
ro--vibrational structure  
of the vinylidine--acetylene,
$HCCH \leftrightarrow CCH_2$, isomeric system
on its lowest singlet and triplet PES's. 
This interesting tautomeric system has been one of the few
four--atomic molecules identified in the interstellar medium
and its high--temperature isomerization has been of interest
to combustion chemistry. 
Recent theoretical
studies by Schaefer and coworkers \cite{ref:sch1,ref:sch2}
following experimental work of Dupre {\it et al.} \cite{ref:dup1}
on Zeeman anticrossing spectra of the acetylene $^{1}A_u$
state motivate the proposed investigation.
In this case we will use the
multi--reference coupled cluster method with the CASSCF
w.f. to generate PES's.

Finally, we propose to investigate the ro--vibrational
structures of some smaller H--bonded complexes, such as
(HF)$_2$ and (H$_2$O)$_2$ dimers. In recent work Klopper and
coworkers \cite{klopper} proposed a new six--dimension potential
for the HF dimer. We will fit this potential in terms of
correlated gaussians and will evaluate the ro--vibrational states
of the dimer. Particularly, the high excited vibrational states
near the dissociation limit of the dimer will be 
of interest to us.

\newpage

\vspace{2mm}
\noindent
{\bf C. FINAL REMARKS}

This work has proposed new methodology for representing potential energy
surfaces analytically and for performing variational energy calculations
utilizing correlated gaussian functions that have pre-multipliers consisting
of products of distance coordinates raised to variable powers, 
$\prod_{i<j}r_{ij}^{m_{kij}}\exp \left[ -{\bf r}^{\prime%
}(A_k\otimes I_3){\bf r}\right]$.
These versatile new dual--purpose basis functions will be
useful for many-body expansions of potential energy surfaces, 
as well as for
both adiabatic and non-adiabatic energy calculations. The functions $\phi _k$
are flexible enough to meet criteria necessary for global analytic
representation of n-body potential energy surfaces including correct
asymptotic behavior, differentiability, symmetry adaptation, {\it etc.}. 
These
functions can also be used as rapidly convergent variational basis functions
representing w.f.'s for adiabatic systems to obtain
vibration-rotation energy levels and for fully non-adiabatic energy
calculations utilizing the same procedures for both types of calculations.
We have prototyped our procedures for fitting potential energy surfaces and
for performing variational vibration calculations. The procedures were
effective for the two trial systems investigated, the Morse potential and
the numeric $H_2$ potential of Kolos and Wolniewicz\cite{Kolos65}.

Having successfully completed testing of our general procedures, we will
proceed with full implementation of an n-body program suite. Programs will
include tools to assist in fitting many-body expansions with $\phi _k$ to
numerical potential data and code for n-body variational energy
calculations. Useful tools would include routines for interpolating sparse
data sets and extrapolating to asymptotic limits in order to generate
adequate numerical data for global potential representation and routines for
many parameter linear and non-linear least squares surface fitting. The
n-body code for variational energy calculations utilizing $\phi _k$ as
expansion functions for the potential and as basis functions for the trial
w.f.'s will be useful for both adiabatic and non-adiabatic
calculations since the Coulomb potential is just a special case of a
many-body expansion in $\phi _k$. Integrals needed for matrix elements in
the variational calculations have been completed and work is being done on
more elegant methods for representing high angular momentum states.

There are issues in this project which can not be definitely
resolved at the present stage and only through further research 
the most optimal directions to carry on the development can be
selected. One such issue concerns non-linear optimization of the
wave function, which will depend critically on the ease and black-box
manner this type of optimization can be preformed in routine applications.
It is possible that the approach of contracting gaussian functions,
as it is done in the electron structure theory 
or an even-tempered scheme 
would be the most
effective way of avoiding costly non-linear optimizations.
Another issue concerns the lengths of the expansions for the
wave function and the interaction potential in terms of $\phi_k$
functions, which can become very challenging for more degrees
of freedom.



Ours is one of the few research 
programs focusing on the non--adiabatic 
Atomic and Molecular Quantum Mechanics
and, therefore, provides a unique 
training place for scientists in this area.  
Furthermore, the development of the techniques 
utilizing explicitly--correlated gaussian functions
has been carried out by only a few groups in the world.  
In the age of parallel computing, these
types of functions may facilitate an important 
step towards achieving improved accuracy in 
{\it ab--initio} calculations in both conventional 
Born--Oppenheimer and non--adiabatic 
molecular quantum mechanics.
However, several major breakthroughs are still needed 
in the technology of explicitly correlated gaussians
to achieve this goal.


The access to the parallel computer technology 
has been recognized by many to be an
essential factor of the leading research 
in the area of computational qunatum physics.  
The next few years will see a significant 
transition of many computational tasks from
vector or superscalar machines to massive 
parallel computers.  This transition will
involve enormous effort in reprogramming 
and in the development of new
parallel algorithms for the
molecular quantum mechanics.
It is essential to realize that only by investing this
effort was it will be possible to obtain the gains in performance 
that parallel hardware promises.  
%The
%transition to massively parallel 
%algorithms and implementations will be no easier, and in many
%respects will be more difficult. 
In the present proposal I describe the methodology 
utilizing explicitly-correlated gaussian
functions which we believe carries great 
break-through potential for describing 
states of correlated multiparticle systems much more 
accurately than the existing methodologies and is
ideally suited for parallel computer platforms.  



\pagebreak

%\vspace{2mm}
%\noindent
%{\bf Literature Cited}
%\vspace{2mm}

\begin{thebibliography}{999}


\bibitem{A1}   
N. Oliphant and L. Adamowicz, Multi--Reference Coupled Cluster 
Method Based on the Single
Reference Formalism, Int. Rev. Phys. Chem. {\bf 12}, 339 (1993).

\bibitem{A2}
E. Schwegler, P.M. Kozlowski and L. Adamowicz, 
Application of Explicitly--Correlated Gaussian
Functions for Calculations of the 
Ground State of the Beryllium Atom, J. Comp. Chem. {\bf 14}, 566 (1993).

\bibitem{A3}   
P.M. Kozlowski and L. Adamowicz, Non--Adiabatic 
Variational Calculations for the Ground State of
the Positronium Molecule, Phys. Rev. {\bf A 48}, 1903 (1993).

\bibitem{A4}   P. Piecuch, N. Oliphant and L. Adamowicz, 
State--Selective Multi--Reference Coupled Cluster Theory,
J. Chem. Phys. {\bf 99}, 1875 (1993).

\bibitem{A5}   Z. Slanina, F. Uhlik, J. Kurtz and L. Adamowicz, 
A Classification of 200 Isomerizations among 51
Isotopomers of $C_8$ (D2d), J. Radioanal. Nucl. Chem. {\bf 170}, 
373 (1993).

\bibitem{A6}   
P.M. Kozlowski and L. Adamowicz, Equivalent 
Quantum Approach to Nuclei and Electrons in
Molecules, Chem. Rev. {\bf 93}, 2007 (1993).

\bibitem{A7}   W.J. McCarthy, M.A. Roehrig, Qi-Qi Chen, 
G.H. Henderson, L. Adamowicz and S.G. Kukolich,
Microwave Measurements and {\it Ab-Initio} Dynamics of the Large 
Amplitude Motion of the Ring
Puckering in 2-Sulpholene, J. Chem. Phys. {\bf 99}, 7305 (1993).

\bibitem{A8}   
Z. Zhang, P.M. Kozlowski and L. Adamowicz, Newton-Raphson 
Optimization of the Explicitely--Correlated 
Gaussian Functions for Calculations of 
the Ground State of the $He$ Atom, J. Comp. Chem.
{\bf 15}, 54 (1994).

\bibitem{A9}   
Z. Slanina, F. Uhlik and L. Adamowicz, 
Classification of 486 Isomerizations Among 72 $^{12}C/^{13}C$
Isotopomers of Cyclic $C_7$, J. Radioanal. 
Nucl. Chem. {\bf 170}, 107 (1993).

\bibitem{A10}  
Z. Slanina, S.-L. Lee, J.-P. Francois, J. Kurtz, L. Adamowicz 
and  M. Smigel, A Non-Planar Cyclic
Minimum-Energy Structure of Singlet $C_9$, 
Mol. Phys. {\bf 81}, 1489 (1994).

\bibitem{A11}  
P. Piecuch and L. Adamowicz, State-Selective Multi-Reference 
Coupled Cluster Theory Employing
Single-Reference Formalism:  Implementation and Application 
to the H8 Model System, J. Chem. Phys.
{\bf 100(8)}, 1 (1994).

\bibitem{A12}  
P. Piecuch and L. Adamowicz, Solving the Single-Reference 
Coupled-Cluster Equations Involving
Highly Excited Clusters in Quasidegenerate Situations, 
J. Chem. Phys. {\bf 100}, 5857 (1994).

\bibitem{A13}  
P. Piecuch and L. Adamowicz, State-Selective 
Multi-Reference Coupled-Cluster Theory Using
Multiconfiguration Self-Consistent Field Orbitals:  A 
Model Study on H8, Chem. Phys. Lett. {\bf 221}, 121
(1994).

\bibitem{A14}  Z. Zhang and L. Adamowicz, Explicitly--Correlated 
Gaussian Functions with $r_{ij}^{2n}$ Factors for Calculations
of the Ground State of the Helium Atom, J. Comp. Chem. 
{\bf 15}, 893 (1994).

\bibitem{A15}  
Z. Slanina, S.-L. Lee, J.-P. Francois, J. Kurtz 
and L. Adamowicz, Inversion of the $C_8$ Non--Planar Ring,
Chem. Phys. Lett. {\bf 223}, 397 (1994).

\bibitem{A16}  P.M. Kozlowski and L. Adamowicz, Effective 
Non--Adiabatic Calculations on $HD^{+}$ with 
Explicitly--Correlated Gaussian Functions, 
Int. J. Quant. Chem. {\bf 55}, 245 (1995).

\bibitem{A17}  
P. Piecuch and L. Adamowicz, Breaking 
Bonds with the State--Selective Multi--Reference Coupled-Cluster 
Method, J. Chem. Phys. {\bf 102}, 898 (1995).

\bibitem{A18}  
A. Sobolewski  and L. Adamowicz, {\it Ab-Initio} 
Characterization of Electronically Excited States in
Highly Unsaturated Hydrocarbons, J. Chem. Phys. 
{\bf 102}, 394 (1995) .

\bibitem{A19}  
V. Alexandrov. P. Piecuch and L. Adamowicz, 
State-Selective Multi-Reference Coupled-Cluster Theory
Employing the Single-Reference Formalism:  Application 
to Excited States of H8, J. Chem. Phys.
{\bf 102(8)}, 3301 (1995).

\bibitem{A20}  
P. M. Kozlowski and L. Adamowicz, Matrix Elements 
for  $J_z$ and $J^2$  Operators over Explicitely-Correlated
Cartesian Gaussian Functions, 
Int. J. Quant. Chem. {\bf 55}, 367 (1995).

\bibitem{A21}  
Z. Zhang and L. Adamowicz, Newton-Raphson Optimization 
of the Explicitly-Correlated Gaussian
Functions for the Ground State of the Be Atom, 
Int. J. Quantum Chem.  {\bf 54}, 281 (1995).

\bibitem{A22}  
K.B. Ghose and L. Adamowicz, Use of Recursively 
Generated Intermediates in State--Selective
Multi--reference Coupled--Cluster Method:  
A Numerical Example, J. Chem. Phys. {\bf 103}, 9324 (1995).

\bibitem{A23}  
Z. Slanina, S.-L. Lee, M. Smigel, J. Kurtz and 
L. Adamowicz, Smaller Carbon Clusters: Linear, Cyclic,
Polyhedral, in SCIENCE AND TECHNOLOGY OF 
FULLERENE MATERIALS, Eds. P. Bernier, D.S.
Bethune, L.Y. Chiang, T.W. Ebbesen, 
R.M. Metzger and J.W. Mintmire, Materials Research Society,
Pittsburgh, 1995, 163-168 (1995).

\bibitem{A24}  
W.J. McCarthy, L. Lapinski, M.J. Nowak and 
L. Adamowicz, Anharmonic Contributions to the
Inversion Vibration in 2-Aminopyrimidine, 
J. Chem. Phys. {\bf 103}, 656 (1995).

\bibitem{A25}  
K.B. Ghose, P. Piecuch and L. Adamowicz, 
Improved Computational Strategy for the State-Selective
Coupled-Cluster Theory with Semi-Internal Triexcited Clusters:  
Potential Energy Surface of HF
Molecule, J. Chem. Phys. {\bf 103}, 9331 (1995).

\bibitem{A26}  
K.B. Ghose, P. Piecuch, S. Pal and L. Adamowicz, 
State-Selective Multireference Coupled-Cluster
Theory:  In pursuit of Property Calculation, 
J. Chem. Phys. {\bf 104}, 6582 (1996).

\bibitem{A27}  
P. M. Kozlowski and L. Adamowicz, Variation 
Calculations of the Two-Photon Annihilation Rate of the
Positronium Molecule, J. Phys. Chem. {\bf 100}, 6266 (1996).

\bibitem{A28}  
G. Gutsev, A. Le\'{s} and L. Adamowicz, The Electronic 
and Geometrical Structure of Aluminum Flouride
Anions, $AlF_n, n = 1-4$, and Electron Affinities of 
Their Neutrals, J. Chem. Phys. {\bf 100}, 8925 (1994).

\bibitem{A29}  
G.L. Gutsev and L. Adamowicz, The Structure of 
the CF Anion and the Electron Affinity of the $CF_4$
Molecule, J. Chem. Phys. {\bf 102}, 9309 (1995).

\bibitem{A30}  
G. Gutsev, A. Sobolewski and L. Adamowicz, 
Theoretical Study on the Structure of Acetonitrile
$(CH_3CH)$ and its Anion $CH_3CH^-$, Chem. Phys. 
{\bf 196}, 1 (1995).

\bibitem{A31}  
G.L. Gutsev and L. Adamowicz, Relationship Between 
the Dipole Moments and the Electron Affinities
for Some Polar Organic Molecules, Chem. Phys. Lett. {\bf 235}, 
377 (1995).

\bibitem{A32}  
G.L. Gutsev and L. Adamowicz, The Electronic and 
Geometrical Structure of Dipole--Bound Anions
Formed by Polar Molecules, J. Phys. Chem. 
{\bf 99}, 13412 (1995).

\bibitem{A33}  
G.L. Gutsev and L. Adamowicz, The Valence and 
Dipole--Bound States of the Cyanomethide Ion,
$CH_2CN$ , Chem. Phys. Lett. 246, 245 (1995).

\bibitem{A34}  
L. Adamowicz and J.-P. Malrieu, Multi--Reference 
Self--Consistent Size--Extensive State--Selective
Configuration Interaction, J. Chem. Phys. 
{\bf 105}, 9240 (1996).

\bibitem{A35}  
L. Adamowicz, R. Caballol, J.-P. Malrieu and J. Meller, 
A General Bridge Between Configuration
Interaction and Coupled--Cluster Methods:  A Multistate 
Solution, Chem. Phys. Lett.
{\bf 259}, 619 (1996).

\bibitem{A36}  
L. Adamowicz and J.-P. Malrieu, Multi-Reference Self-Consistent 
Size--Extensive Configuration
Interaction (CI) - A Bridge Between the Coupled-Cluster 
Method and the CI Method, in MODERN
IDEAS IN COUPLED-CLUSTER METHODS, R.J. Bartlett, 
Ed., World Scientific Publishing, pp. 307-332 (1997).

\bibitem{A37}  
D.B. Kinghorn and L. Adamowicz, 
The Electron Affinity of Hydrogen, Deuterium and Tritium:  
A Non--Adiabatic Variational Calculation Using Explicitly-Correlated 
Gaussian Basis Set, J. Chem. Phys.,
{\bf 106}, 4589 (1997).

\bibitem{A38}  
D. Gilmore, P.M. Kozlowski, D.B. Kinghorn and 
L. Adamowicz, Analytic First Derivatives for
Explicitly-Correlated Multi-Center Gaussian Geminals, 
Int. J. Quantum Chem. {\bf 63}, 991 (1997).

\bibitem{A381}
W.J. McCarthy, L. Lapinski, M.J. Nowak and L. Adamowicz,
Out--of--Plane Vibrations of $NH_2$ in 2--Aminopyrimidine, 
J.Chem.Phys., accepted for publication.  

\bibitem{A382}
D.B. Kinghorn and L. Adamowicz,
A New N--Body Potential and Basis Functions for
Variational Energy Calculations,
J.Chem.Phys., {\bf 106}, 8760 (1997).

\bibitem{A383}
D. B. Kinghorn and L. Adamowicz, in {\em Pauling's Chemical
Bonding}, edited by Z.B. Maksic and W.J. Orville--Thomas,
Elsevier Science, 1998, accepted for publication.

\bibitem{A384}
K.B. Ghose, L. Adamowicz and S. Pal,
The State Selective Coupled Cluster Method with
Restricted Sets of Triples and Quadruples:
Some Aspects of the Theory and its Recent Applications,
Int.J.Quantum Chemistry, submitted.

\bibitem{A385}
K.B. Ghose and L. Adamowicz,
The State--Selective Multi--Reference Coupled Cluster
Method with Restricted Sets of Triples and Quadruples:
Applications for the BH, CO and C$_2$ Molecules,
Chem.Phys.Lett., submitted.


\bibitem{A3851}
L. Adamowicz, P. Piecuch and K. Ghose, 
The State--Selective Multi--Reference
Coupled Cluster Method, Mol. Phys., 
accepted for publications.



\bibitem{A3860}
N.A. Oyler and L. Adamowicz, Theoretical Ab--Initio
Calculations of the Electron Affinity of 
Thymine, Chem. Phys. Lett. {\bf 219}, 223 (1994).


\bibitem{A3861}
G.H. Roehrig, N.A. Oyler and L. Adamowicz, 
Can Electron Attachment Change Tautomeric
Equilibrium of Guanine?, Chem. Phys. Lett. {\bf 225}, 265 (1994).


\bibitem{A3862}
G.H. Roehrig, N.A. Olyer
and L. Adamowicz, Electron Affinity of Adenine.  
Theoretical Study, J. Phys. Chem. {\bf 99},
14285 (1995). 


\bibitem{A3863}
J. Smets, W.J. McCarthy and 
L. Adamowicz, Dipole--Bound Electron
Attachment of Uracil--Water Complex.  Theoretical 
{\it Ab--Initio} Study, 
J. Phys. Chem., {\bf 100}, 14655 (1996).

\bibitem{A3864}
J. Smets, 
W.J. McCarthy and L. Adamowicz, Water Molecules
Enhances Dipole--Bound Electron Affinity of 1--Methyl--Cytosine, 
Chem. Phys. Lett. {\bf 256}, 360 (1996).

\bibitem{A3865}
Y. Elkadi and L. Adamowicz, 
Dipole--Bound Electron Attachment
to Ethylene  Glycol Dimer.  Theoretical {\it Ab--Initio} 
Study, Chem. Phys. Lett., {\bf 261}, 507 (1996).

\bibitem{A3866}
D.M.A. Smith, J. Smets, Y. Elkadi and 
L. Adamowicz, Methylation Reduces
Electron Affinity of Uracil.  {\it Ab--Initio} 
Theoretical Study, J. Phys. Chem.
{\bf 101}, 8123 (1997).

\bibitem{A3867}
J. Smets, D.M.A. Smith, Y. Elkadi 
and L. Adamowicz, {\it Ab--Initio} Theoretical Study
of Dipole--Bound Anions of Molecular 
Complexes.  Water Molecule Inhibits or
Enhances Electron Affinity of N,N-Dimethyl--Aminoadenine, 
Pol. J. Chem., accepted
for publication.

\bibitem{A3868}
J. Smets, D.M.A. Smith, Y. Elkadi and 
L. Adamowicz, The Search for Stable
Anions of Uracil Water Clusters.  {\it Ab--Initio} 
Theoretical Studies, J. Phys. Chem.
{\bf 101}, 9152 (1997).

\bibitem{A3869}
D.M.A. Smith, J. Smets, Y. Elkadi and 
L. Adamowicz, {\it Ab--Initio} Theoretical Study
of Dipole--Bound Anions of Molecular 
Complexes.  Water Trimer Anion, J. Chem.
Phys.
{\bf 107}, 5788 (1997).

\bibitem{A3870}
R. Ramaekers, D.M.A. Smith, Y. Elkadi and 
L. Adamowicz, Dipole--Bound Anion
of Hydrogen Fluoride Dimer.  Theoretical 
{\it Ab--Initio} Study, Chem. Phys. Lett.,
{\bf 277}, 269 (1997).

\bibitem{A3871}
R. Ramaekers, D.M.A. Smith, J. Smets 
and L. Adamowicz, Ab--Initio Theoretical
Study of Dipole--Bound Anions of Molecular 
Complexes. (HF)$_3^-$ and (HF)$_4^-$ Anions,
J. Chem. Phys.
{\bf 107}, 9475 (1997).

\bibitem{A3872}
V. Alexandrov, L. Adamowicz, and S. Stepanian,
Theoretical {\it Ab-initio} Study of OH Vibrational Band
in Gas-Phase Glycine Conformers, Chem.Phys.Lett.,
submitted.

\bibitem{A3873}
D.M.A. Smithd, J.Smeths, Y. Elkadi, and L. Adamowicz,
{\it Ab-initio} Theoretical Study of Dipole-Bound Anions
of Molecular Complexes. [H$_2$O \dots HCN]$^-$ and
[HCN $\dots$ H$_2$O]$^-$ Anions,
Chem.Phys.Lett., accepted for publication.

\bibitem{A3874}
V. Alexandrov, D.M.A. Smith, H. Rostkowska, M.J. Nowak, 
L. Adamowicz and W. McCarthy, 
Theoretical Study of the OH Stretching Band in 3-Hydroxy-2-Methyl-4-Pyrone,
J.Chem.Phys., accepted for publication.

\bibitem{A3875}
D.M.A. Smith, J. Smeths, Y. Elkadi, and L. Adamowicz,
{\it Ab-initio} Theoretical Study of Dipole-Bound Anions
of Molecular COmplexes. Water Tetramer Anions,
J.Chem.Phys., submitted.

%\bibitem{ref:A39}  
%L. Lapinski, M.J. Nowak, J. Fulara, A. Le\'{s} and 
%L. Adamowicz, Experimental and Theoretical Studies
%on the IR Spectra of 5--Methylcytosine,
%J. Phys. Chem. {\bf 94}, 6555 (1990).  


%\bibitem{ref:A40}  
%M.J. Nowak, L. Lapinski, H. Rostkowska, 
%A. Le\'{s}  and L. Adamowicz, Theoretical and Matrix-Isolation
%Experimental Study on 2(1H)--Pyridinethione/2--Pyridinethiol, 
%J. Phys. Chem., {\bf 94}, 7406 (1990).

%\bibitem{ref:A41}  
%M.J. Nowak, L. Lapinski, J. Fulara, A. Le\'{s}  and 
%L. Adamowicz, Theoretical and Infrared Matrix--Isolation Study on 
%4(3H)--Pyrimidinethione and 3(2H)--Pyridazinethione--Tautomerism and
%Phototautomerism, J. Phys. Chem. {\bf 95}, 2404 (1991).

%\bibitem{ref:A42}  
%J. Fulara, M.J. Nowak, L. Lapinski, A. Le\'{s}  and 
%L. Adamowicz, Theoretical and Matrix--Isolation
%Experimental Study of the Infrared Spectra of 5--Azauracil 
%and 6--Azauracil, Spectrochimica Acta {\bf 47A},
%595 (1991).

%\bibitem{ref:A43}  
%M. Nowak, L. Lapinski, Jan Fulara, A. Le\'{s}  and 
%L. Adamowicz, Matrix--Isolation IR Spectroscopy of
%Tautomeric Systems and its Theoretical Interpretation.  
%2--Hydroxy--Pyridine/2(1H)--Pyridinone, J. Phys.
%Chem. {\bf 96}, 1562 (1992).

%\bibitem{ref:A44}  
%A. Le\'{s}, L. Adamowicz, M.J. Nowak and L. Lapinski, 
%Theoretical Interpretation of the Gas Phase
%Equilibrium of 2--Hydroxypyridine/2(1H)--Pyridinone, 
%J. Mol. Structure {\bf 277}, 313 (1992).
%
%\bibitem{ref:A45}  
%L. Lapinski, M.J. Nowak, J. Fulara, A. Le\'{s}  and 
%L. Adamowicz, The Relation Between Structure and
%Tautomerism in Diazinones and Diazinethiones. An 
%Experimental Matrix--Isolation and Theoretical {\it Ab--Initio} Study, 
%J. Phys. Chem. {\bf 96}, 6250 (1992).

%\bibitem{ref:A46}  
%A. Le\'{s}, L. Adamowicz, M.J. Nowak and L. Lapinski, 
%Assignment of the Matrix--Isolation IR Spectra
%of Uracil and Tymine Based on New {\it Ab-Initio} Theoretical 
%Calculations, Spectrochim. Acta {\bf 48A} (10),
%1385 (1992).

%\bibitem{ref:A47}  
%H. Rostkowska, M.J. Nowak, L. Lapinski, 
%M. Bretner, T. Kulikowski, A. Le\'{s}  and L. Adamowicz,
%Infrared Spectra of 2--Thiocytosine and 5--Fluoro--2--Thiocytosine; 
%Experimental and {\it Ab-Initio} Studies,
%Spectrochim. Acta Part A, {\bf 49A}, 551 (1993).

%\bibitem{ref:A48}  
%H. Rostkowska, M.J. Nowak, L. Lapinski, M. Bretner, 
%T. Kulikowski, A. Le\'{s} and L. Adamowicz,
%Theoretical and Matrix--Isolation Experimental Studies 
%on 2--Thiocytosine and 5--Flouro--2--Thiocytosine,
%Biochimica et Biophysica Acta {\bf 1172}, 239 (1993).

%\bibitem{ref:A49}  
%H. Vranken, J. Smets, L. Lapinski, M.J. Nowak, 
%L. Adamowicz and G. Maes, Infrared Spectra and
%Tautomerism of Isocytosine.  An {\it Ab-Initio} and Matrix--Isolation 
%Study, Spectrochim. Acta Part A, {\bf 50A},
%875 (1994).

%\bibitem{ref:A50} 
%L. Lapinski, M.J. Nowak, A. Le\'{s}  and L. Adamowicz, 
%Photochemistry of Matrix--Isolated 4(3H)--Pyrimidinones, 
%J. Am. Chem. Soc. {\bf 116}, 1461 (1994).

%\bibitem{ref:A51}  
%A. Le\'{s}, L. Adamowicz, M.J. Nowak and L. Lapinski, 
%Concerted Biprotonic Tautomerism of 2--Hydroxypyridine, 
%J. Mol. Structure {\bf 312}, 157 (1994).

%\bibitem{ref:A52}  
%M.J. Nowak, A. Le\'{s}  and L. Adamowicz, Application 
%of {\it Ab-Initio} Quantum Mechanical Calculations
%to Assign Matrix--Isolation IR Spectra of Oxopyrimidines, 
%in Trends in Physical Chemistry, published
%by the Council of Scientific Information, India, 137-168 (1994).

%\bibitem{ref:A53}  
%L. Lapinski, M.J. Nowak, A. Le\'{s} and L. Adamowicz, 
%Comparison of {\it Ab--Initio} HF/6-31$G^{**}$, HF/6-31++$G^{**}$ 
%and MP2/6-31$^{**}$ IR Spectra of 4-Pyrimidinone Tautomers with 
%Matrix--Isolation Spectra,
%Vibrational Spectroscopy  {\bf 8}, 331 (1995).

%\bibitem{ref:A54}  
%L. Lapinski, D. Prusinowska, M.J. Nowak, 
%M. Bretner, K. Felczak, G. Maes and L. Adamowicz,
%Infrared Spectra of 6--aza--Thiouraciles:  
%An Experimental Matrix--Isolation and Theoretical {\it Ab-Initio}
%Study, Spectrochimica Acta {\bf 52}, 645 (1996).

%\bibitem{ref:A55}  
%D. Prusinowska, L. Lapinski, M.J. Nowak and 
%L. Adamowicz, Tautomerism, Phototautomerism and
%Infrared Spectra of Matrix--Isolated 2--Quinolinethione, 
%Spectrochemica Acta Part {\bf A51}, 1809 (1995).
%

%\bibitem{ref:A56}  
%M. Van Bael, K. Schoone, L. Houben,  J. Smets, 
%W. McCarthy, L. Adamowicz, M. Nowak and G. Maes,
%Matrix--Isolation FT--IR studies and {\it Ab-Initio} 
%Calculations of Hydrogen--bonded Complexes of Imidazole
%Comparison Between Experimental Data and 
%Different Theoretical Methods, J. Phys. Chem., submitted. 

%\bibitem{ref:A561}
%M. Rostkowska, M.J. Nowak, L. Lapinski, D. Smith, and L. Adamowicz,
%Molecular Structure and Infrared Spectra of
%3,4--Dihydroxy--3--Cyclobutene--1,2--Dione:
%Experimental Matrix--Isolation and Theoretical Hartree--Fock
%and Post Hartree--Fock Study, Spectrochimica Acta, 
%accepted for publication.  


%\bibitem{ref:A57}  
%A. Sobolewski and L. Adamowicz, Theoretical 
%Investigations of Proton Transfer Reaction in Hydrogen
%Bonded Complexes of Cytosine and Water, 
%J. Chem. Phys. {\bf 102}, 5708 (1995).

%\bibitem{ref:A58}  
%A. Sobolewski and L. Adamowicz, 
%Theoretical Investigations of the Excited--State Intra--molecular
%Proton Transfer Reaction in N--substituted--3--hydroxypyridinones, 
%Chem. Phys. Lett. {\bf 93}, 67 (1995).

%\bibitem{ref:A59}  
%A. Sobolewski and L. Adamowicz, Theoretical 
%Investigations of the Proton Transfer Reaction in
%Hydrogen--Bonded Complexes of Cytosine with HNO, 
%Chem. Phys. Lett. {\bf 234}, 94 (1995).

%\bibitem{ref:A60}  
%A. Sobolewski and L. Adamowicz, Theoretical 
%Investigations of the Proton Transfer Reaction in the
%Hydrogen--Bonded Complexes of 2--Pyrimidinone with Water, 
%J. Phys. Chem. {\bf 99}, 14277 (1995).

%\bibitem{ref:A61}  
%A. Sobolewski and L. Adamowicz, 
%Photophysics of 2--hydroxypyridine:  An {\it Ab-Initio} Study, J. Phys.
%Chem. {\bf 100}, 3933 (1996).

%\bibitem{ref:A62}  
%A. Sobolewski and L. Adamowicz, 
%Double--Proton Transfer in [2,2--Bipyridine]--3,3'--Diol:  
%An {\it Ab-Initio} Study, Chem. Phys. Lett. {\bf 252}, 33 (1996)


%\bibitem{ref:A63}  
%H. Saint--Martin, I. Ortega--Blake, A. Le\'{s}  and 
%L. Adamowicz, The Role of Hydration in the Hydrolysis
%of Pyrophosphate.  A Monte--Carlo Simulation 
%with Polarizable--type Interaction Potentials, Biochimica
%et Biophysica Acta {\bf 1207}, 12 (1994).

%\bibitem{ref:A64}  
%A. Le\'{s} and L. Adamowicz, {\it Ab-Initio} 
%Calculations of Biomolecules.  Proceedings of the CAM '94
%Physics Meeting in Cancun, Mexico, AIP 
%Conference Proceedings {\bf 342}, 190 (1995).


\bibitem{BO1927} M. Born and J. P. Oppenheimer, 
Zur Quantentheorie der Molekeln,
Annalen der Physik 
{\bf 84}, 457 (1927).

\bibitem{CS1980} J.M. Cobes and R. Seiler, Quantum Dynamics of Molecules,
ed. R. G.  Woolley, Plenum Press, New York, 1980, p. 435.

\bibitem{KM1992} M. Klein, A. Martinez, R. Seiler and X. Wang, Commun.
Math.Phys. {\bf 143}  607 (1992).

\bibitem{C}
A. Carington and R.A. Kennedy, Gas Phase Ion Chemistry;
Ed. M.T. Bowers, Academic Press, New York, vol.3, p.393.


\bibitem{ref:handy}
SPECTRO, written by J.F. Gaw, A. Willetts, W.H. Green, and N.C. Handy.

\bibitem{ref:ch21}
S. Carter and N. C. Handy, 
On the Calculation of Vibration--Rotation Energy Levels of
Quasi--Linear Molecules,
J.Mol.Spectrosc. {\bf 95}, 9 (1982).

\bibitem{ref:bowman}
J.M. Bowman and B. Gazdy, 
A Truncation/Recoupling Method for Basis Set Calculations of
Eigen-values and Eigen-functions,
J.Chem.Phys.
{\bf 94}, 454 (1991);
D.A. Jelski, R.H. Halay, and J.M. Bowman,
New Vibrational Self-Consistent Field Program
for Large Molecules,
J.Comp.Chem. {\bf 17}, 1645 (1996).


%\bibitem{ref:schwenke}
%D.W. Schwenke, 
%On the Computation of Ro--vibrational Energy Levels of
%Triatomic Molecules,
%Comput.Phys.Commun. {\bf 70}, 1 (1992).



%\bibitem{ref:podolski}
%B. Podolski, Quantum--Mechanically Correct Form of
%Hamiltonian Function for Conservative Systems,
%Phys.Rev. {\bf 32}, 812 (1928).

%\bibitem{ref:malloy}
%T.B. Malloy, J.Mol.Struct. {\bf 44}, 504 (1972).

%\bibitem{ref:wilson}
%E.B. Wilson, J.C. Decius, and P.C. Cross, in {\em
%Molecular Vibrations}, McGraw-Hill, New York 1972.


%\bibitem{ref:eckard}
%H.M. Pickett, 
%Vibration--Rotation Interactions and the Choice of
%Rotating Axes for Polyatomic Molecules,
%J.Chem.Phys. {\bf 56}, 1715 (1971).



%\bibitem{Bartlett95}
%R.~J. Bartlett,
%\newblock Coupled-cluster theory: An overview of recent developments,
%\newblock in {\em Modern Electronic Structure Theory}, edited by D.~R. Yarkony,
%World Scientific, Singapore, 1995.

\bibitem{Eckart35}
C.~Eckart,
Some Studies Concerning Rotating Axes and Polyatomic
Molecules,
\newblock Phys.Rev. {\bf 47}, 552 (1935).

\bibitem{WilsonHoward36}
E.~B. Wilson and J.~B. Howard,
The Vibration--Rotation Energy Levels
of Polyatomic Molecules,
\newblock J.Chem.Phys. {\bf 4}, 260 (1936).

\bibitem{Watson68}
J.~K.~G. Watson,
Simlification of the Molecular Vibration-Rotation
Hamiltonian,
Mol.Phys. {\bf 15}, 479 (1968);
{\it ibid.} 
The Vibration--Rotation Hamiltonian of Linear Molecules,
{\bf 19}, 465 (1970).


%\bibitem{SutcliffeTennyson87}
%B.~T. Sutcliffe and J.~Tennyson,
%\newblock J.Chem.Soc.Faraday Trans. 2 {\bf 83}, 1663
%(1987).

\bibitem{Jensen88}
P.~Jensen,
A new Morse Oscillator--Rigid Bender Internal
Dynamics (MORBID) Hamiltonian for Triatomic Molecules,
\newblock J.Mol.Spectr. {\bf 128}, 478 (1988).

\bibitem{LeRoy95}
R.~J. Le{R}oy,
\newblock {LEVEL 6.0} a 
computer program for bound and quasibound levels, and
calculating various expectation values and matrix elements,
\newblock Technical report, University of Waterloo.

\bibitem{Hutson93}
J.~M. Hutson,
\newblock {BOUND}: A program for calculating bound-state energies for weakly
bound molecular complexes, version 1 (1984) to version 5 (1993),
\newblock Computational Project No. 6 of the Science and Engineering Research
 Council, on Heavy Particle Dynamics, 1993.
%\newblock e-mail: J.M.Hutson@durham.ac.uk.

\bibitem{ref:ten1}
J. Tennyson, J. R. Henderson, and N. G. Fulton,
DVR3D: For the Fully Pointwise Calculations of Ro-vibrational
Spectra of Triatomic Molecules,
Comp.Phys.Comm. {\bf 86}, 175 (1995).



\bibitem{ref:kcn1}
J. Tennyson and B. T. Sutcliffe, 
{\it Ab--initio} Vibrational--Rotational Spectrum of Potasium Cyanide,
KCN,
Mol.Phys. {\bf 46}, 97 (1982).


\bibitem{ref:kcn2}
J. Tennyson and A. van der Avoird, 
{\it Ab--initio} Vibrational--Rotational Spectrum
of Potasium Cyanide, KCN. II. Large Amplitude Motions
and Ro-vibrational Coupling,
J.Chem.Phys. 
{\bf 76}, 5710 (1982).


\bibitem{ref:kcn3}
J. Tennyson and B. T. Sutcliffe, 
The {\it Ab--initio} Calculation of the 
Vibrational--Rotational Spectrum of Triatomic Systems in the  
Close--Coupling Approach with KCN and H$_2$Ne as Examples,
J.Chem.Phys. {\bf 77}, 4061 (1982).


\bibitem{ref:ch22}
B. T. Sutcliffe and J. Tennyson, 
Variational Methods for the Calculation of Ro-vibrational Energy
Levels of Small Molecules,
J.Chem.Soc.Faraday Trans. 2, 
{\bf 83}, 1663 (1987).


\bibitem{ref:ch23}
J. S. Lee and D. Secrest, 
A Calculation of the Rotation--Vibration States of He$_2$H$^+$,
J.Chem.Phys. {\bf 85}, 6565 (1986).

\bibitem{ref:j22}
C.-L. Chen, B. Maessen, and M. Wolfsberg, 
Variational Calculations of Rotational--Vibrational Energy
Levels of Water,
J.Chem.Phys. {\bf 83}, 1795 
(1985).

\bibitem{ref:j26}
B. T. Sutcliffe, in {\em Current Aspects of Quantum Chemistry}
(R. Carbo, ed.), Studies in Theoretical Chemistry, Vol. 21,
pp 99-125, Elsevier, New York (1982).

\bibitem{ref:j27}
N. C. Handy, 
The Derivation of Vibration Kinetic Energy
operator in Internal Coordinates,
Mol.Phys. {\bf 61}, 207 (1987).

\bibitem{ref:j32}
B. T. Sutcliffe and J. Tennyson, 
A Generalized Approach to the Calculation of Ro--vibrational Spectra
of Triatomic Molecules,
Mol.Phys. {\bf 58}, 1053 (1986).

\bibitem{ref:j33}
D. Estes and D. Secrest, 
The Vibration--Rotation Hamiltonian; A Unified
Treatment of Linear and Non--Linear Molecules,
Mol.Phys. {\bf 59}, 569 (1986).

\bibitem{ref:j34}
J. Makarewicz, 
Ro-vibrational Hamiltonian of a Triatomic Molecule in 
Local and Collective Internal Coordinates,
J.Phys. B: At.Mol.Opt.Phys. {\bf 21}, 1803 (1988).

\bibitem{ref:j35}
J. Makarewicz and W. Lodyga, 
Self--Consistent Internal Axes for a Rotating-Vibrating
Triatomic Molecules,
Mol.Phys. {\bf 64}, 899 (1988).

\bibitem{ref:j36}
J. M. Bowman, J. Zuniga, and A. Wierzbicki, 
Investigations of Transformed Mass--Selected Jacobi Coordinates
for Vibrations of Polyatomic Molecules with Applications to H$_2$O,
J.Chem.Phys. {\bf 90}, 2708 (1989).

\bibitem{ref:j37}
B. T. Sutcliffe and J. Tennyson, 
A General Treatment of Vibrational--Rotational Coordinates
For Triatomic Molecules,
Int.J.Quantum Chem. {\bf 39},
183 (1991).

\bibitem{ref:j25}
R. M. Whitnell and J. C. Light, J.Chem.Phys. 
Efficient Pointwise Representations for
Vibrational Wave Functions: Eigen-functions of H$_3^+$,
{\bf 90}, 1774 (1989).

\bibitem{ref:j45}
Z. Ba\u{c}i\'{c} and J. C. Light, 
Theoretical Methods for Ro-vibrational States of Floppy 
Molecules,
Ann.Rev.Phys.Chem. {\bf 40}, 
469 (1989).

\bibitem{ref:j46}
S. E. Choi and J. C. Light, 
Determination of the Bound and Quasibound States of Ar--HCl
van--der--Waals Complex:
Discrete Variable Representation Method,
J.Chem.Phys. {\bf 92}, 2129 (1990).

\bibitem{ref:j50}
J. Tennyson and B. T. Sutcliffe, 
Highly Rotationally Excited
States of Floppy Molecules: H$_2$D$^+$ with $J \leq 20$,
Mol. Phys. {\bf 58}, 1067 (1986).


\bibitem{Dunham32a}
J.~L. Dunham,
The Wentzel--Brillouin--Kramers Method of Solving
the Wave Equation,
\newblock Phys.Rev. {\bf 41}, 713 (1932).

\bibitem{Dunham32b}
J.~L. Dunham,
The Energy Levels of a Rotating Vibrator,
\newblock Phys.Rev. {\bf 41}, 721 (1932).

\bibitem{Simons73}
G.~Simons, R.~G. Parr, and J.~M. Finlan,
New Alternative to the Dunham Potential for Diatomic Molecules,
\newblock J.Chem.Phys. {\bf 59}, 3229 (1973).

\bibitem{Ogilvie81}
J.~F. Ogilvie,
A General Potential Energy Function for Diatomic Molecules,
\newblock Proc.Royal Soc. London A {\bf 378}, 287 (1981).

\bibitem{Thakkar75}
A.~Thakkar,
A New Generalized Expansion for the Potential Energy 
Curves of Diatomic Molecules,
\newblock J.Chem.Phys. {\bf 62}, 1693 (1975).

\bibitem{Huffaker76}
J.~N. Huffaker,
Diatomic Molecules as Morse Oscillators. 1. Energy Levels,
\newblock J.Chem.Phys. {\bf 64}, 3175 (1976).

\bibitem{Morse29}
P.~M. Morse,
Diatomic Molecules According to the Wave Mechanics. II. 
Vibrational Levels.
\newblock Phys.Rev. {\bf 34}, 57 (1929).

\bibitem{Jordan74}
K.~D. Jordan, J.~L. Kinsey, and R.~Silbey,
Use of Pade' Approximants in the Construction of 
Diabatic Potential Energy Curves for Ionic Molecules,
\newblock J.Chem.Phys. {\bf 61}, 911 (1974).

%\bibitem{Jorish79}
%V.~S. Jorish and N.~B. Shcherbak,
%\newblock Chem.Phys.Lett. {\bf 67}, 160 (1979).

\bibitem{Sonnleitner81}
S.~A. Sonnleitner, C.~L. Beckel, A.~J. Colucci, and E.~R. Scaggs,
Rational Fraction Representation of Diatomic 
Vibrational Potentials. V. The $^{3} \Sigma_g^+ $ State of H$_2^+$,
\newblock J.Chem.Phys. {\bf 75}, 2018 (1981).

\bibitem{Pardo86}
A.~Pardo, J.~J. Camacho, and J.~M.~L. Poyato,
The Pade' Approximant Method and its Applications to the Construction of
Potential Energy Curves for the Lithium Hydride Molecule,
\newblock Chem.Phys.Lett. {\bf 131}, 490 (1986).


\bibitem{Jorish79}
J.M.L. Martin, T.J. Lee, and P.R. Taylor,
An Accurate {\it Ab-initio} Quadratic Force Field for
Formaldehyde and its Isotopomers,
J.Mol.Spectr. {\bf 160}, 105 (1993).

%\bibitem{ref:t2}
%J.M.L. Martin, P.R. Taylor, J.T. Yustein, T.R. Burkholder, and L. Andrews,
%Pulsed Laser Evaporation of Boron/Carbon pellets:
%Infrared Spectra and Quantum Chemical Structures and Frequencies for 
%$BC_2$,
%J.Chem.Phys. {\bf 99}, 12 (1993).
%
%\bibitem{ref:t3}
%T.J. Lee, C.E. Dateo, B. Gazdy and J.M. Bowman,
%Accurate Quartic Force Fields and Vibrational Frequencies for HCN and HNC,
%J.Phys.Chem. {\bf 97}, 8937 (1993).

\bibitem{Dateo94}
C.E. Dateo, T.J. Lee, and D.W. Schwenke,
An Accurate Quartic Force Field and Vibrational Frequencies
for $HNO$ and $DNO$,
J.Chem.Phys. {\bf 101}, 5853 (1994).

%\bibitem{ref:t5}
%J.M.L. Martin and P.R. Taylor,
%{\it Ab-initio} Study of the Isoelectronic Molecules
%$BCN$, $BNC$, and $C_3$ Including Anharmonicity,
%J.Phys.Chem. {\bf 98}, 6105 (1994).

%\bibitem{ref:t6}
%J.M.L. Martin and P.R. Taylor,
%{\it Ab-initio} Study of the Molecules $BC$ and $B_2C$,
%J.Chem.Phys, {\bf 100}, 9002 (1994).

%\bibitem{ref:t7}
%J.M.L. Martin, P.R. Taylor,
%J.P. Fran\c{c}ois and R. Gijbels,
%{\it Ab-initio} Study of the Spectroscopy and Thermochemistry
%of the $C_2N$ and $CN_2$ molecules.
%Chem.Phys.Lett. {\bf 226}, 475 (1994).

%\bibitem{ref:t8}
%J.M.L. Martin, P.R. Taylor,
%J.P. Fran\c{c}ois and R. Gijbels,
%{\it Ab-initio} Study of the Spectroscopy, Kinetics and 
%Thermochemistry
%of the $BN_2$ molecule,
%Chem.Phys.Lett. {\bf 222}, 517 (1994).

\bibitem{Lee95a}
T.J. Lee, J.M.L. Martin and P.R. Taylor,
An Accurate {\it Ab-initio} Quartic Force Field and Vibrational
Frequencies for $CH_4$ and Isotopomers,
J.Chem.Phys. {\bf 102}, 254 (1995).

\bibitem{Martin95}
J.M.L. Martin, T.J. Lee, P.R. Taylor,
and J.P. Fran\c{c}ois,
The Anharmonic Force Field of Ethylene, $C_2H_4$, by Means of
Accurate {\it Ab-initio} Calculations,
J.Chem.Phys. {\bf 103}, 2589 (1995).

\bibitem{Lee95b}
T.J. Lee,
J.M.L. Martin, C.E. Dateo, and P.R. Taylor,
Accurate {\it Ab-initio} Quartic Force Fields,
Vibrational Frequencies, and Heats of Formation
for $FCN$, $FNC$, $ClCN$, and $ClNC$,
J.Phys.Chem. {\bf 99}, 15858 (1995).

\bibitem{ref:t12}
J.M.L. Martin and P.R. Taylor,
The Geometry, Vibrational Frequencies, and Total Atomization
Energy of Ethylene. A Calibration Study,
Chem.Phys.Lett. {\bf 248}, 336 (1996).

\bibitem{ref:t13}
J.M.L. Martin, D.W. Schwenke, T.J. Lee, and P.R. Taylor,
Is There Evidence for Detection of Cyclic $C_4$ in IR spectra?
An Accurate {\it ab-initio} Computed Quartic Force Field,
J.Chem.Phys. {\bf 104}, 4657 (1996).

\bibitem{ref:t14}
J.M.L. Martin and P.R. Taylor,
Structure and Vibrations of Small Carbon Clusters from 
Coupled--Cluster Calculations,
J.Phys.Chem. {\bf 100}, 6047 (1996).

%\bibitem{Lee95a}
%T.~J. Lee, J.~M.~L. Martin, and P.~R. Taylor,
%\newblock J.Chem.Phys. {\bf 102}, 254 (1995).

%\bibitem{Dateo94}
%C.~E. Dateo, T.~J. Lee, and D.~W. Schwenke,
%\newblock J.Chem.Phys. {\bf 101}, 5853 (1994).

%\bibitem{Martin93}
%J.~M.~L. Martin and T.~J. Lee,
%\newblock J.Mol.Spectr. {\bf 160}, 105 (1993).

%\bibitem{Lee95b}
%T.~J. Lee, J.~M.~L. Martin, C.~E. Dateo, and P.~R. Taylor,
%\newblock J.Phys.Chem. {\bf 99}, 15858 (1995).

%\bibitem{Martin95}
%J.~M.~L. Martin, T.~J. Lee, P.~R. Taylor, and J.-P. Fran\c{c}ois,
%\newblock J.Chem.Phys. {\bf 103}, 2589 (1995).

\bibitem{Koizumi91}
H.~Koizumi and G.~C. Schatz,
A Coupled Channel Study of HN$_2$ Unimolecular Decay Based on a
Global {\it Ab-initio} Potential Surface,
\newblock J.Chem.Phys. {\bf 95}, 4130 (1991).

\bibitem{Bentley92}
J.~A. Bentley, J.~M. Bowman, and B.~Gazdy,
A Global {\it ab--initio} Potential for HCN/HNC,
exact vibrational energies, and comparison to experiment,
\newblock Chem.Phys.Lett. {\bf 198}, 563 (1992).

\bibitem{Bowman75}
J.~M. Bowman and A.~Kuppermann,
A Semi-Numerical Approach to the Construction and Fitting of Triatomic 
Potential Energy Surfaces,
\newblock Chem.Phys.Lett. {\bf 34}, 523 (1975).

\bibitem{Connor75}
J.~N.~L. Connor, W.~Jakubetz, and J.~Manz,
Exact Quantum Transition Probabilities by the State Path Sum Method:
Collinear F + H$_2$ Reaction,
\newblock Mol.Phys. {\bf 29}, 347 (1975).

\bibitem{Forsythe77}
G.~E. Forsythe,
\newblock {\em Computer methods for mathematical computations},
\newblock Prentice-Hall, Englewood Cliffs, N.J., 1977.

\bibitem{Malik80}
D.~J. Malik, J.~Eccles, and D.~Secrest,
On Quantal Bound State Solutions and Potential Energy Surface Fitting.
A Comparison of the Finite Element Numerov-Cooley and Finite Difference 
\newblock J.Comp.Phys. {\bf 38}, 157 (1980).

\bibitem{Sathyamurthy75}
N.~Sathyamurthy and L.~M. Raff,
Quasiclassical Trajectory 
Studies Using 3D Spline Interpolation of {\it Ab--initio} Surfaces,
\newblock J.Chem.Phys. {\bf 63}, 464 (1975).

\bibitem{Dunne87}
S.~J. Dunne, D.~J. Searles, and E.~I. {von Nagy-Felsobuki},
{\it Ab--initio} 
Model of the Raman Spectrum of Li$_3^+$ : Breathe Mode Frequencies,
\newblock Spectrochim. Acta {\bf 43A}, 699 (1987).

\bibitem{Bruehl88}
M.~Bruehl and G.~Schatz,
Theoretical Studies of Collisional Energy Transfer in Highly-excited 
Molecules: Temperature and Potential Surface Dependence of Relaxation
in H$_2$, Ne, Ar + CS$_2$,
\newblock J.Phys.Chem. {\bf 92}, 7223 (1988).

\bibitem{Murrell84}
J.~N. Murrell, S.~Carter, 
S.~C. Farantos, P.~Huxley, and A.~J.~C. Varandas,
\newblock {\em Molecular Potential Energy Functions},
\newblock John Wiley, Chichester, 1984.

\bibitem{Sorbie75}
K.~S. Sorbie and J.~N. Murrell,
Analytical Potentials for Triatomic Molecules from Spectroscopic Data,
\newblock Mol.Phys. {\bf 29}, 1387 (1975).

\bibitem{Whitehead76}
R.~J. Whitehead and N.~C. Handy,
Variational Calculation of Low--Lying and Excited Vibrational Levels of
the Water Molecule,
\newblock J.Mol.Spectr. {\bf 59}, 459 (1976).


\bibitem{Carter82}
S.~Carter, I.~M. Mills, J.~N. Murrell, and A.~J.~C. Varandas,
Analytical Potentials for Triatomic Molecules IX. The Prediction of 
Anharmonic Force Constants from Potential Energy Surfaces Based on
Harmonic Force Fields and Dissociation Energies for SO$_2$ and O$_3$,
\newblock Mol.Phys. {\bf 45}, 1053 (1982).

\bibitem{Searles93}
D.~Searles and E.~{von Nagy-Felsobuki},
\newblock {\em Ab Initio Variational Calculations of Molecular
Vibrational-Rotational Spectra},
\newblock Springer-Verlag, Berlin, 1993.

\bibitem{Mezey87}
P.~G. Mezey,
\newblock {\em Potential Energy Hypersurfaces},
\newblock Elsevier, Amsterdam, 1987.



\bibitem{kozlowski91}
P. M. Kozlowski and L. Adamowicz, 
An Effective Method for Generating Nonadiabatic
Many--Body Wave Functions Using Explicitly Correlated
Gaussian--Type Functions,
J. Chem. Phys., {\bf 95}, 6681 (1991).

\bibitem{kozlowski92a}
P. M. Kozlowski and L. Adamowicz, 
Multi-Center and Multi-particle Integrals for
Explicitly Correlated Gaussian-Type Functions,
J. Comput. Chem., {\bf 13}, 602 (1992).


\bibitem{kozlowski92b}
P. M. Kozlowski and L. Adamowicz, 
Implementation of Analytical First Derivatives for
Evaluation of the Many--Body Non-Adiabatic Wave Function
with Explicitly Correlated Gaussian Functions,
J. Chem. Phys., {\bf 96}, 9013 (1992).

\bibitem{kozlowski92c}
P. M. Kozlowski and L. Adamowicz, 
Newton--Raphson Optimization of the Many Body
Non-Adiabatic Wave Function Expressed in Terms of Explicitly
Correlated Gaussian Functions,
J. Chem. Phys., {\bf 97}, 5063 (1992).


%\bibitem{ref:A3}   
%P.M. Kozlowski and L. Adamowicz, 
%Phys. Rev. {\bf A 48}, 1903 (1993).
%
%
%\bibitem{ref:A6}   
%P.M. Kozlowski and L. Adamowicz, 
%Chem. Rev. {\bf 93}, 2007 (1993).
%
%\bibitem{ref:A8}   
%Z. Zhang, P.M. Kozlowski and L. Adamowicz,
%J. Comp. Chem.
%{\bf 15}, 54 (1994).
%
%\bibitem{ref:A14}  Z. Zhang and L. Adamowicz, 
%J. Comp. Chem. {\bf 15}, 893 (1994).
%
%
%\bibitem{ref:A16}  P.M. Kozlowski and L. Adamowicz, 
%Int. J. Quant. Chem. {\bf 55}, 245 (1995).
%
%\bibitem{ref:A20}  
%P. M. Kozlowski and L. Adamowicz, 
%Int. J. Quant. Chem. {\bf 55}, 367 (1995).
%
%\bibitem{ref:A21}  
%Z. Zhang and L. Adamowicz, 
%Int. J. Quantum Chem.  {\bf 54}, 281 (1995).
%
%
%\bibitem{ref:A27}  
%P. M. Kozlowski and L. Adamowicz, 
%J. Phys. Chem. {\bf 100}, 6266 (1996).
%
%
%\bibitem{ref:A37}  
%D.B. Kinghorn and L. Adamowicz, 
%J. Chem. Phys., accepted for publication.
%
%\bibitem{ref:A38}  
%D. W. Gilmore, P.M. Kozlowski, D.B. Kinghorn and 
%L. Adamowicz,
%Int. J. Quantum Chem., accepted for publication.
%
%\bibitem{ref:A382}
%D.B. Kinghorn and L. Adamowicz,
%J.Chem.Phys., submitted.
%

\bibitem{kpc}
D.~B. Kinghorn and L. Adamowicz, we have recently derived
compact integral algorithms allowing for negative $m_{kij}$ powers
of $r_{ij}$ in the $\phi_k$ functions; unpublished result.


\bibitem{Kinghorn95a}
D.~B. Kinghorn,
Implementation of Gradient Formulas for Correlated
Gaussians: He, $^{\infty}$He, Ps$_2$, $^9$Be, and $^{\infty}$Be
test results,
\newblock Int.J.Quantum Chem. {\bf 57}, 141 (1996).

\bibitem{Kinghorn93}
D.~B. Kinghorn and R.~D. Poshusta,
Nonadiabatic Variational Calculations on Dipositronium
Using Explicitly Correlated Gaussian Basis Functions,
\newblock Phys.Rev. A {\bf 47}, 3671 (1993).

\bibitem{Kinghorn95b}
D.~B. Kinghorn and R.~D. Poshusta,
Density Matrices for Correlated Gaussians: Helium and
Dipositronium,
\newblock Int.J.Quantum Chem. {\bf 60}, 213 (1996).

\bibitem{Poshusta83}
R.~D. Poshusta,
Nonadiabatic Singer Polynomial Wave Functions for 
Three--Particle Systems,
\newblock Int.J.Quantum Chem. 
{\bf 24}, 65 (1983).

\bibitem{sz1}
R. Bukowski, B. Jeziorski, and K. Szalewicz,
New Effective Strategy of Generating Gaussian-type Geminal
Basis Sets for Correlation Energy Calculations,
J.Chem.Phys. {\bf 100}, 1366 (1994);
R. Bukowski, B. Jeziorski, S. Rybak, and K. Szalewicz,
Second-order Correlation energy for H$_2$O Using
Explicitly Correlated Gaussian Geminals,
J.Chem.Phys. {\bf 102}, 888 (1995);
R. Bukowski, B. Jeziorski, and K. Szalewicz,
Basis Set Superposition Problem in Interaction Energy
Calculations with Explicitly Correlated Bases:
Saturated Second- and Third-Order Energies for He$_2$,
J.Chem.Phys. {\bf 104}, 3306 (1996);
H.L. Williams, T. Korona, R. Bukowski, B. Jeziorski, and K. Szalewicz,
Helium Dimer Potential from Symmetry-Adapted
Perturbation Theory,
Chem.Phys.Lett. {\bf 262}, 431 (1996);
T. Korona, H.L. Williams, R. Bukowski,
B. Jeziorski, and K. Szalewicz,
Helium Dimer Potential from Symmetry Adapted Perturbation Theory
Calculations Using Large Gaussian Geminal and Orbital Basis Sets,
J.Chem.Phys. {\bf 106}, 5109 (1997);
and reference therein.


\bibitem{Biedenharn81}
L.~C. Biedenharn and J.~D. Louck,
\newblock {\em Angular Momentum in Quantum Physics. Theory and Application},
\newblock Encyclopedia of Mathematics and Its Applications, Addison-{W}esley,
Reading, {MA}, 1981.

\bibitem{Varga95}
K.~Varga and Y.~Suzuki,
Precise Solution of Few--Body Problems with the
Stochastic Variational Method on a Correlated Gaussian Basis,
\newblock Phys.Rev. C {\bf 52}, 2885 (1995).


\bibitem{Varga96}
K.~Varga and Y.~Suzuki,
Global Vector Representation of the Angular Motion of
Few--Particle Systems,
\newblock Phys.Rev. C  (1997),
\newblock in press.

\bibitem{Kolos65}
W.~Kolos and L.~Wolniewicz,
Potential--Energy Curves for the X$^{1} \Sigma _g^+$, 
b $^{3}\Sigma_u ^+,$
and C $^{1} \Pi_u$ States of the Hydrogen Molecule,
\newblock J.Chem.Phys. {\bf 43}, 2429 (1965).

\bibitem{Mathematica}
Wolfram Research, Inc., 
100 Trade Center Drive, Champaign, Illinois 61820-7237
USA,
\newblock {\em Mathematica}.

\bibitem{Wolniewicz66}
L.~Wolniewicz,
Vibrational--Rotational Study of the Electronic Ground State 
of the Hydrogen Molecule,
\newblock J.Chem.Phys. {\bf 45}, 515 (1966).

\bibitem{LeRoy68}
R.~J. Le{R}oy and R.~B. Bernstein,
Dissociation Energy and Vibrational Terms of Ground State 
(X$^{1} \Sigma _g ^+$) Hydrogen,
\newblock  J.Chem.Phys. {\bf 49}, 4312 (1968).


\bibitem{ref:k35}
T. Oka, 
Observation of the Infrared Spectrum of H$_3^+$, 
Phys.Rev.Lett. {\bf 45}, 531 (1980).

\bibitem{ref:k36}
G. D. Carney and R. N. Porter, 
{\it Ab--Initio} Prediction of the Rotation--Vibration 
Spectrum of H$_3^+$ and D$_3^+$,
Phys.Rev.Lett. {\bf 45}, 537 (1980);
H$_3^+$: Geometry Dependence of Electronic Properties, J.Chem.Phys. 
J.Chem.Phys. {\bf 60}, 4251 (1974);
H$_3^+$: {\it Ab--Initio} Calculation of the Vibration Spectrum,
J.Chem.Phys. {\bf 65}, 3547 (1976).

\bibitem{ref:k41}
P. Drossart, J.-P. Maillard, J. Caldwell, S. J. Kimm,
J. K. G. Watson, W. A. Majewski, J. Tennyson, S. Miller, S. K. Atreya,
J. T. Clarke, J. W. Waite, Jr., and R. Wagener, 
Detection of H$_3^+$ on Jupiter,
Nature {\bf 340},
539 (1989).

\bibitem{ref:k44}
T. R. Geballe, M.-F. Jagod and T. Oka, 
Detection of H$_3^+$ Infrared Emission Lines
in Saturn,
Astrophys. J. {\bf 408},
L108 (1993).

\bibitem{ref:k45}
L. Trafton, T. R. Geballe, S. Miller, J. Tennyson, and
G. E. Ballester, 
Detection of H$_3^+$ from Uranus,
Astrophys. J. {\bf 405}, 761 (1993).

\bibitem{ref:ten2}
L. Neale, S. Miller, and J. Tennyson, 
Spectroscopic Properties of the H$_3^+$ molecule:
a New Calculated Line List,
Astrophys. J. {\bf 464}, 516
(1996).

\bibitem{ref:ten3}
B. M. Dinelli, O. L. Polyansky and J. Tennyson, 
Spectroscopically Determined Born--Openheimer and
adiabatic surfaces for 
H$_3^+$, H$_2$D$^+$, D$_2$H$^+$, and D$_3^+$,
J.Chem.Phys. {\bf 103},
10433 (1995).

\bibitem{ref:ten4}
J. Tennyson, in {\em Physics World}, 
H$_3^+$: from Chaos to the Cosmos,
vol.{\bf 8}, p. 33 (1995).

\bibitem{ref:ten5}
O. L. Polyansky, B. M. Dinelli, C. R. Le Sueur, and J. Tennyson,
Asymmetric Adiabatic Correction to the Rotation--Vibration levels
of H$_2$D$^+$ and D$_2$H$^+$,
J.Chem.Phys. {\bf 102}, 9322 (1995).

\bibitem{ref:ten6}
J. Tennyson, 
Spectroscopy of H$_3^+$: Planets, Chaos and the Universe,
in {\em Reports on Progress in Physics}, {\bf 58},
421 (1995).

\bibitem{ref:ten7}
B. M. Dinelli, C. R. Le Sueur, J. Tennyson, and R. D. Amos,
{\it Ab--initio} Ro--vibrational Levels of H$_3^+$ beyond the
Born--Oppenheimer Approximation,
Chem.Phys.Lett. {\bf 232}, 295 (1995).

\bibitem{ref:ten8}
S. Miller, An Lam. Hoanh, and J. Tennyson,
What Astronomy Has Learn from Observation of H$_3^+$,
Can.J.Phys. {\bf 72}, 760 (1994).

\bibitem{ref:ten9}
J. Tennyson and S. Miller, 
H$_3^+$: from First Principles to Jupiter,
Contemp.Phys. {\bf 35}, 105 (1994).


\bibitem{ref:bern1}
L. Wallace, P. F. Bernath, W. Livingston, K. Hinkle, J. Busler,
B. Guo, and K. Zhang, 
Water on the Sun,
Science {\bf 268}, 1155 (1995).

\bibitem{ref:ten10}
F. Allard, P. H. Hauschildt, S. Miller, and J. Tennyson,
The Influence of H$_2$O Line Blanketing on the Spectra of
Cool Dwarf Stars,
Astroph.J.,Lett. {\bf 426}, L39 (1994).

\bibitem{ref:ten11}
C. D. Paulse and J. Tennyson, 
An Empirical Potential Energy Surface for Water Accounting for
States with High Angular Momentum,
J.Mol.Spectr. {\bf 168}, 313 (1994).

\bibitem{ref:ten12}
A. E. Lynas--Gray, S. Miller, and J. Tennyson,
Infrared Transition Intensities for Water: A Comparison of
{\it Ab--initio} and Fitted Dipole Moment Surfaces,
J.Mol.Spectr. {\bf 169}, 458 (1995).

\bibitem{ref:ten13}
J. H. Schryber, S. Miller, and J. Tennyson,
Computed Infrared Absorption Properties if Hot Water Vapour,
J.Quantit.Spectr.Radiat.Transf. {\bf 53}, 373 (1995).


\bibitem{ref:ten14}
N. F. Zobov, O. L. Polyansky, C. R. Le Sueur, and J. Tennyson,
Vibration--Rotation Levels of Water beyond the 
Born--Oppenheimer Approximation,
Chem.Phys.Lett. {\bf 260}, 381 (1996).


%\bibitem{ref:A1}   
%N. Oliphant and L. Adamowicz, 
%Int. Rev. Phys. Chem. {\bf 12}, 339 (1993).

%\bibitem{ref:A4}   
%P. Piecuch, N. Oliphant and L. Adamowicz, 
%J. Chem. Phys. {\bf 99}, 1875 (1993).
%
%\bibitem{ref:A11}  
%P. Piecuch and L. Adamowicz, 
%J. Chem. Phys.
%{\bf 100(8)}, 1 (1994).
%
%\bibitem{ref:A17}  
%P. Piecuch and L. Adamowicz, 
%J. Chem. Phys. {\bf 102}, 898 (1995).
%
%\bibitem{ref:A22}  
%K.B. Ghose and L. Adamowicz, 
%J. Chem. Phys. {\bf 103}, 9324 (1995).
%
%\bibitem{ref:A25}  
%K.B. Ghose, P. Piecuch and L. Adamowicz, 
%J. Chem. Phys. {\bf 103}, 9331 (1995).

\bibitem{ref:sch1} 
C. D. Sherrill, G. Vacek, Y. Yamaguchi, H. F. Schaefer, III, 
J. F. Stanton, and J. Gauss,
The A$^1_u$ State and the T$_2$ potential surface
of Acetylene: Implications for Triplet Perturbations in the
Fluorescence Spectra of teh A State,
J.Chem.Phys. {\bf 104}, 8507 (1996).

\bibitem{ref:sch2} 
G. Vacek, J. R. Thomas, B.J. DeLeeuw, Y. Yamaguchi, H. F. Schaefer, III, 
J. F. Stanton, and J. Gauss,
Isomerization Reactions of the Lowest Potential
Energy Hypersurface of Triplet Vinylidene and Triplet Acetylene,
J.Chem.Phys. {\bf 98}, 4766 (1993).

\bibitem{ref:dup1}
P. Dupre, P.G. Green, R.W. Field, 
Quantum Beat Spectroscopic Studies of Zeeman Anticrossing
in the A$^1_u$ State of the Acetylene Molecule
(C$_2$H$_2$),
Chem.Phys. {\bf 198}, 211 (1995). 

\bibitem{klopper}
W. Klopper, M. Quack, and M.A. Suhm,
A New {\it Ab--initio} Based Six--Dimensional Semi--Empirical Pair
Interaction Potential for HF,
Chem.Phys.Lett. {\bf 261}, 35 (1996).



\end{thebibliography}

\end{document}


