
\documentstyle[12pt,overcite]{article}

\setlength{\textwidth}{7in}
\setlength{\textheight}{9.0in}
\setlength{\oddsidemargin}{-.25in}
%
\setlength{\topmargin}{-.10in}
\def\Real{\rm I\hspace{-0.2em}R}
\def\nin{{\in \! \! \! \! \! /} \,}
\def\nequiv{{\equiv \! \! \! \! \! \! /} \,}

\newcommand{\half}{\mbox{$\frac{1}{2}$}}
\newcommand{\mul}{\multicolumn{2}{c}{none}}
\newcommand{\qua}{\mbox{$\frac{1}{4}$}}
\newcommand{\six}{\mbox{$\frac{1}{6}$}}
\newcommand{\tfour}{\mbox{$\frac{1}{24}$}}
\newcommand{\beq}{\begin{equation}}
\newcommand{\eeq}{\end{equation}}
\newcommand{\tripler}{T_{3}\!\left(^{a^{\prime}b^{\prime}
{\rm C}^{\prime}}_{{\rm I}^{\prime}j^{\prime}k^{\prime}}\right)}
\newcommand{\quadrupler}{T_{4}\!\left(^{a^{\prime}b^{\prime}
{\rm C}^{\prime}{\rm D}^{\prime}}_{{\rm I}^{\prime}{\rm J}^{\prime}
k^{\prime}l^{\prime}}\right)}
% \newcommand{\hhline}{\hline \vspace{-4.8mm} \\ \hline}


\begin{document}



\setlength{\baselineskip}{1em}


\noindent
{\bf Building a parallel supercomputer for molecular
quantum mechanical calculations}




%\begin{abstract}
\section{Abstract}
We propose the construction of a 
prototype high--performance / low--cost parallel computational platform
for a new type of quantum mechanical (q.m.) calculations on molecules
and clusters with explicitly correlated gaussian functions.
The methods for employing these novel basis functions
to both adiabatic and non--adiabatic q.m. calculations have
been under development in my research group for several 
years. This work has been coupled with extended computaional
implementation work which has recently culminated in an
integrated design
for parallel computational software and hardware
for performing practical calculations on non--adiabatic
and ro--vibrational behavior of chemical systems.
The computational hardware will consist of a Linux ``Beowulf'' cluster built 
from low cost commodity PC components running the Linux operating system
connencted with switched fast ethernet. 
Building of a  parallel computing platform 
represents a new direction in the research effort of our group at the University of Arizona. 
It also represents a novel direction in high performance scientific computing. 
Our research group has the necessary interdisciplinary skills to
construct and fully utilize this high performance computational 
platform.   
At present we propose to build a low cost prototype
system consisting of a cluster server and two diskless compute nodes
providing six  processors for parallel program execution. This system will be used
to demonstrate the feasibilty of our approach for
presentation to external funding agencies. 
In the proposal we provide background
information related to the design of the parallel cluster and 
to component specifications
and cost estimates.
%\end{abstract}

\newpage


\section{Introduction}

The performance of commodity personal computer (PC) components has been
steadily increasing while at the same time prices for these components have
been decreasing. Today it's possible to construct a computer with off the
shelf components for arround \$500 that will out perform a \$40,000
workstation
of just a few years ago. The availability of high quality, no cost, open
source, network operating system components and tools for parallel software
development together with high performance low cost PC components coupled with
inexpensive high speed networking hardware has made PC clustering, by far, the
most cost effective means of ``super computing''.
The most widespread implementation of this type of PC clustering originated
with the ``Beowulf Project''( http://beowulf.gsfc.nasa.gov/beowulf.html ) at
the Center of Excellence in Space Data and Information Sciences (CESDIS)
in
1994 ( http://cesdis.gsfc.nasa.gov/ ). Since then the Beowulf project
development has spread to research labs and universities around the world.
Today the term Beowulf cluster loosely refers to any PC clustering based on
the freely available, open source, Linux operating system. The software tools,
information and expertise for building Beowulf class PC clusters is readably
available from numerous sites on the web. A few good starting points are:
http://www.beowulf.org, http://www.linux.org/, http://www.redhat.com, and for
parallel computing in general the 
IEEE Computer Society's ParaScope 
site http://www.computer.org/parascope/.

\section{Proposed system}

The parallel computing cluster we are proposing will utilize high end PC
components running RedHat Linux with Beowulf kernel modifications and network
drivers together with tools for parallel software development. The compute
nodes will communicate over a private, switched, 100Mb ethernet LAN connected
to a cluster master node via 1000Mb ethernet. The master node will server as
the primary network gateway. The compute nodes will each contain two cpu's and
the networking hardware will allow us to scale the cluster in units of 12 or
24 nodes for a maximum of 192 nodes ( 284 cpu's ). We are initially planning
on building a 12 node system. The presnet request is made to provide
funding to purchase 10 notes.

The quantum chemistry programs we have developed (and are developing) readily
lend themselves to coarse grained parallelism which scales well and is ideally
suited for implementation on large parallel clusters. We are currently running
large, high accuracy calculations which require weeks or months of cpu time
with well optimized efficient algorithms. Our only logical recourse to
increase productivity is to go parallel. We have recently assembled a small 4
cpu test system and are currently working on parallel implementations of our
quantum chemistry codes. Our experiences so far are very encouraging and have
given use great motivation for this project. We are now confident that we have
the human resources necessary to build 
and fully utilize the system we are proposing.

The specifications for our system will now be given along with cost estimates.
It should be noted that PC components are commodities and prices fluctuate
with supply and demand, and new technology. However, the trend is for prices
to move downward while quality and performance increase. Therefore, the
specifications will likely change by the time you read this, and, this change
will most likely be for higher performance components at the same total system cost.

Note that components are chosen not on the basis of lowest cost but rather
highest reliability and performance. Component failure in the cluster is
minimized by paying attention to the details.%


\section{Scientific goals to be accomplished}



The workstation cluster, which we propose to build, will
be empolyed to perform calculations with the 
the methods allowing study of molecular systems without
assuming the Born--Oppenheimer (BO) approximation regarding the
separability of the nuclear and electronic motions, which we have
developed in my research group.\cite{1,2,3,4}
The approach can be also applied to 
a more general problem of stationary bound
quantum states of multi--particle
systems which interact through an isotropic potential.
The non-BO molecular system consisting of electrons and nuclei 
with
Coulombic repulsive and attractive potentials 
is an example of such a problem.
Another example
is the problem of ro--vibrational structures
of molecules and clusters. 
In this case the isotropic interaction potential is provided
in the form of the potential energy hypersurface (PES)
calculated for the system using electronic structure methods.
We will consider PES'es of the 
ground or an excited electronic state in such a calculation.
Unlike the Coulombic potential, which includes only two-body
components, the potential given by PES has N--body character and,
although in most cases it is dominated by two-body interactions,
the three- and higher-body contributions can be important.
%New functional forms for the multi--body expansion of the potential 
%energy
%surface  and for the basis function expansion of the ro--vibrational
%wave function to be used variational calculations of vibrational
%and rotational levels of N-atom systems
%are proposed.
In this project we propose that N--body
explicitly correlated gaussians with pre-multiplying factors consisting of
products of powers of internal distance coordinates are utilized in a dual
role to analytically represent the isotropic N--atom potential and the 
wave functions expressed in terms of internal cartesian coordinates. 
We propose to develop and implement the methodology for the general
N--body case. 

The initial non-Born-Oppenheimer 
application of the present methodology 
will focus on
small atomic and molecular systems.
Following our recent non--adiabatic calculations
of the electron affinities of hydrogen, deuterium and tritium,
as well as different isotopes of the Li atom,
which produced results matching very well with the experimental
values, we will now consider small molecular systems.
In the first calculations 
we will study
hydrogen molecules and ions and their isotopomers,
H$_2^+$, H$_2$, H$_3$, {\it etc.}, 
%small dipole--bound anions, {\it e.g.}, LiH$^-$, LiD$^-$,
and other molecular systems, {\it e.g.}, HeH$^+$, HeH$_2^+$, HeH$_2$.  
The aim of these calculations will be to determine 
electron affinities and electron detachment energies 
of these systems in different ro--vibrational states.
Very accurate measurements, which have 
recently become
available 
for these systems
present an exciting challenge for 
the ``non--Born-Oppenheimer Atomic and Molecular Quantum Mechanics."
Also calculations of 
higher excited states,
where the coupling of the electronic and nuclear motions
can be significant, 
will be performed and the nature and magnitude of the 
coupling effect will be 
analyzed.

%dbk
%It should be noted that we are in no way restricted to conventional
%systems.
%Calculations involving ``exotic'' particles present no difficulty 
%since in general our methodology can treat collections of particles
%with any given masses and charges fully non--adiabatically.  
%dbk

%Initial applications of the method with 
%analytically fitted PES's as the 
%interaction potentials
%in the ro--vibrational Hamiltonian
%will start with 
%``calibration cases"
%such as the H$_3^+$ and H$_2$O systems and their 
%isotopomers.
%These systems considered before by others
%using different methodologies
%and their calculated ro-vibrational spectra 
%were compared with very accurate experimental data.
%
%For H$_3^+$ we will generate the 
%potential energy surface using 
%the variational approach with explicitly correlated
%gaussians and with the optimization algorithm based
%on analytical derivatives of the variational functional
%with respect to the non--linear parameters of gaussians,
%which we have recently developed.
%For H$_2$O we will use the best existing potentials,
%as well as the potential generated using our SSMRCC method.
%Application of our approach to systems with more than three
%nuclei are the most interesting since 
%other methods have been mostly limited to three-body cases.
%We find particularly interesting 
%to investigate
%the
%ro-vibrationally hot elemental clusters, such as
%C$_n$, Si$_n$, P$_n$, $n$=4,5 {\it etc.}, and their anions.
%Calculations of highly excited ro-vibrational states near the
%dissociation barrier of
%these systems will reveal their structures and dynamics 
%at high temperatures.
%Among other applications we will consider
%the high--temperature
%ro--vibrational structure  
%of the vinylidine--acetylene,
%$HCCH \leftrightarrow CCH_2$, isomeric system.
%We will also study
%ro--vibrational spectra 
%of small molecular dimers such
%as (HF)$_2$, (H$_2$O)$_2$, {\it etc.}



\begin{thebibliography}{999}


\bibitem{1}
D.B. Kinghorn and L. Adamowicz,
The Electron Affinity of Hydrogen, Deuterium and Tritium:
A Non--Adiabatic Variational Calculation Using Explicitly-Correlated
Gaussian Basis Set, J. Chem. Phys.,
{\bf 106}, 4589 (1997).



\bibitem{2}
D. Gilmore, P.M. Kozlowski, D.B. Kinghorn and
L. Adamowicz, Analytic First Derivatives for
Explicitly-Correlated Multi-Center Gaussian Geminals,
Int. J. Quantum Chem. {\bf 63}, 991 (1997).


\bibitem{3}
D.B. Kinghorn and L. Adamowicz,
A New N--Body Potential and Basis Functions for
Variational Energy Calculations,
J.Chem.Phys., {\bf 106}, 8760 (1997).

\bibitem{4}
D. B. Kinghorn and L. Adamowicz, in {\em Pauling's Chemical
Bonding}, edited by Z.B. Maksic and W.J. Orville--Thomas,
Elsevier Science, 1998, in press.

\bibitem{5}
D.B. Kinghorn and L. Adamowicz,
A Correlated Basis Set for Non--Adiabatic Energy
Calculations of Diatomic Molecules,
J.Chem.Phys., in press.




\end{thebibliography}


\newpage


\section{Budget Justification}
The following components of the prototype parallel computer system are budgeted whithin
the requested amounts:

\noindent 
Cluster server: 
Chasis [\$250], AUSU P2B-DS motherboard [\$450], (2) PII400 processors [\$680],
9GB SCSI hard drive [\$470], 128MB PC100 SDRAM [\$170], 
17'' monitor [\$265], video card [\$22], 100Mb nic [\$19], 100Mb TX switch [\$229] --- SUBTOTAL \$2555

\noindent
Diskless Compute Node:
Chasis [\$200], DFI P2XBLD motherboard [\$170], (2) PII350 processors [\$420],
128MB PC100 SDRAM [\$170], 100Mb nic [\$19], misc. [\$75] --- SUBTOTAL \$1054

\noindent
Cluster server + two compute nodes + local sales tax --- TOTAL \$5000


\end{document}

