% keywords: bibtex latex tex bibliography references
%
\documentstyle[seg,manuscript]{revtex} % This line gives a manuscript style.
%
%\documentstyle[seg]{revtex} % Uncomment this line for a preprint style.
%
\def\BibTeX{\rm B{\sc ib}\TeX}
\begin{document}
\bibliographystyle{seg}

\title{Referencing and bibliographies \\ in SEG publications made easy}
\author{Martin Karrenbach}
\address{Stanford Exploration Project, 366 Mitchell, Stanford, CA 94305, U.S.A.}
\maketitle

\begin{abstract}
When preparing a paper for publication, authors usually struggle with 
the format and style of references to related articles.
First they have to make sure that previous publications are cited correctly
in the body of the paper. Second they have to make sure that the complete
list of cited publications obeys the stylistic demands of the SEG
editor. 

To make it as easy as possible for the author and also to enforce 
stylistic consistency we require that the {\tt Bibliography} environment
offer by \LaTeX\  is used.  SEG provides guidelines and examples 
of citations in this article.
\end{abstract}

\section{Philosophy}
We emphatically encourage authors to take advantage of the \BibTeX\ 
style file that the SEG editor provides. If you are using \TeX\  and 
\LaTeX\  already, you should also have \BibTeX\  available to you.
If you do not, we encourage you to get hold of the latest \TeX\ 
distribution -- \BibTeX\  is included.

\BibTeX\  saves the authors of scientific articles a lot of work,
by converting \BibTeX\  database entries into bibliographic entries
used in  \LaTeX\ .
Those get automatically included in your paper using the correct style.
All references will automatically get sorted and correctly numbered if
there are more than one paper from the same author per year.
The appropriate style will be chosen depending on the type of citation
such as articles, books, proceedings, theses or talks.

SEG currently provides all Geophysics articles up to 1992 on CDROM,
also in the previous year SEG published indices of Geophysics, 
The Leading Edge, SEG Abstracts, SEG Books and the 
publications of associated geophysical organizations.
All those data bases can be easily converted into
\BibTeX\  files. The Stanford Exploration Project has used this method
since 1991 and is happy to provide SEG with those tools.

You do not need to rely on outside \BibTeX\  databases, but you can 
easily create your own \BibTeX\  files. At the end of this section
the file {\tt segmaster.bib} is included as an example.
For more details have a look in Lamport's \LaTeX\ guide. \cite{lamport}

\section{How to use bibliographies in \LaTeX\  }
\LaTeX\  allows to refer to published material using the 
{\tt $\backslash$cite\{ \}} command. The SEG style file augments that
by another command {\tt $\backslash$shortcite\{ \}}, 
which only displays the year and not the author names.
For example to cite the first article in the bibliography that is shown below
\cite{lan66}, use the command \verb+\cite{lan66}+. The citation key {\tt lan66}
is the second argument to the \verb+\bibitem[tag]{citation key}+ command.
The second type of citation: \shortcite{lan66} can be invoked, using
\verb+\shotcite{lan66}+.
Those two commands govern how you want to cite material in your text.
At the end of your paper there has to be a list of all cited articles.
The standard way in \LaTeX\ is to have all cited entries listed in
the {\tt thebibliography} environment.
After the CONCLUSIONS and ACKNOWLEDGMENTS, the reference list will appear
by typing:

{\tighten
\begin{verbatim}
\begin{thebibliography}{0}
\bibitem[Landes, 1966]{lan66}
Landes, K. K., 1966, A scrutiny of the abstract, {II}: AAPG Bull., {\bf 50}, 
1992.
\bibitem[Lindsey, 1993]{lindsey94}
Lindsey, J. P, Jr., Instructions to authors: Geophysics, {\bf 58}, 2-9.
\end{thebibliography}
\end{verbatim}
}

The result will be the following:
{\tighten
\begin{thebibliography}{0}
\bibitem[Landes, 1966]{lan66}
Landes, K. K., 1966, A scrutiny of the abstract, {II}: AAPG Bull., {\bf 50}, 
1992.

\bibitem[Lindsey, 1993]{lindsey94}
Lindsey, J. P, Jr., Instructions to authors: Geophysics, {\bf 58}, 2-9.

\end{thebibliography}
}
This requires the author to type in all cited articles. Possibly repeating
it for other publications which require slight style changes. 
An alternative way with \BibTeX\  is shown in the next section 
(which avoids typing those in).


\section{How to use {{\BibTeX\ }} to create bibliographies}

In contrast to the previous section you could have put the two entries
in the following form in a file {\tt segmaster.bib}:
{\tighten
\begin{verbatim}
@ARTICLE{lan66,
   author = {K. K. Landes},
   journal = {AAPG Bull.},
   title = {A scrutiny of the abstract, {II}},
   volume = {50},
   page= {1992}
   year = {1966}
}
@ARTICLE{lindsey94,
   author = {J. P. Jr. Lindsey},
   journal = {Geophysics},
   title = {Instructions to authors},
   volume = {58},
   pages = {2-9},
   year = {1993}
}
\end{verbatim}
}
All entries start with some type identifier, such as {\tt @ARTICLE}, that
will describe what kind of publication it is. The first item after that
is the citation key {\tt lan66}. This is the key you use to refer to the
entry and with which you cite the document in your paper (\verb+\cite{lan66}+).
Following that are descriptive keywords, like {\tt author, title, 
volume, etc.}.
These keywords will vary with the type of publication.
For more details look at Lamport's \LaTeX guide \shortcite{lamport} and the
file {\tt segmaster.bib}, which contains examples how to write your own 
\BibTeX\ entries.

Instead of typing the bibliography into your paper (as the last section did)
you now have merely to put a single line:
{\bf
\begin{verbatim}
\bibliography{segmaster}
\end{verbatim}
}
That will tell \LaTeX\ to  insert the bibliography in your paper.
The argument to the {\bf \verb+\bibliography{...,...}+} 
can be a single file name
({\tt segmaster}) or a comma separated list of 
filenames ({\tt segmaster,mybib}).
\BibTeX\  assumes those files to have the extension {.bib}.

To get all the references resolved  you have to run your paper through:
{\tt latex}, then through {\tt bibtex}. 
{\tt latex} creates a {\tt .aux} file.
{\tt bibtex} uses that {\tt .aux} file and creates a {\tt .bbl} file.
This bibliography file {\tt .bbl} contains all the entries that you would
have originally had to type in by hand.
The biggest advantage is however, that you do not have to worry about
stylistic errors. It will be correct -- the SEG style file takes care of that.
To get this stylistic behavior, your document must contain the command
{\bf\verb+\bibliographystyle{seg}+}.
Once you have the {\tt .bbl} file created, 
you do not have to run {\tt bibtex} again; 
you just use {\tt latex} the same way as usual.
%
You only have to rerun {\tt bibtex} again, 
if you want to change (add or delete) cited papers in your manuscript.

%
%Look at the listings in this paper
%if you want to know how this paper and the bibliography file was written.

\section{Examples of Citations}

This section shows many examples (including the ones given in 
``Instructions to the authors'') of citing relevant published material.
The {\tt $\backslash$cite} command takes as argument a {\tt key}, 
that identifies a bibliography entry. It is easiest if you compare 
the following citations
with the original bibliography file, which is listed as an appendix.

I am citing a book in the full form \cite{lamport} , 
while here I am citing only the year of the book \shortcite{lamport}.
I was using the command: \verb+\cite{lamport}+  and then 
\verb+\shortcite{lamport}+.
The entry in the \BibTeX\ file is as follow:

{\tighten
\begin{verbatim}
@BOOK{lamport,
   author = {L[eslie] Lamport},
   title = {\LaTeX\  user's guide \& reference manual},
   year = {1985},
   publisher ={Addison-Wesley Publishing Company}
}
\end{verbatim}
}

Here I cite an  article the same way, once full \cite{segarticle} and 
once just the year \shortcite{segarticle}.
Articles with more than two authors are cited using the correct style
\cite{magazine3}.
%
%Note that if you want to keep something literal in the title enclose it with
%\{\}, that will keep it from being lowercased.
%
This is the citation of a  magazine article \cite{segmagazine},
a M.Sc. thesis \cite{segms},
a patent \cite{patent},
a book \cite{segbook},
an article in a book \cite{inbook},
an  SEG paper \cite{segabstract}
and finally an  SEG talk \cite{segtalk}.
 Here I am citing multiple papers in one citation: 
\cite{segtalk,segabstract}, using 
\verb+\cite{segtalk,segabstract}+. Be sure not to leave spaces in the 
comma separated list, \verb+\cite{segtalk,\ segabstract}+ will NOT work.
Letters will be added automatically to the year if an author has two 
publications in the same year, such as the articles by Claerbout 
\shortcite{Claerbout.sep.77.245,Claerbout.sep.77.19}.

To let you see all referenced articles, I put 
\verb+\bibliography{segmaster}+ here. This bibliography contains all
references given in the ``Instruction to the authors'', that appears
in the first issue of each volume of Geophysics.

\nocite{lamport}

% uncomment the following line when using bibtex
%\bibliography{segmaster}

% comment the following lines when using bibtex
%%%%%%%%%%%%%%%%%%%%%%%%%
%%% this is included from the easybib.bbl file that was generated by bibtex
\begin{thebibliography}{}

\bibitem[\protect\citeauthoryear{Anstey}{1976}]{patent}
Anstey, N., 1976, Seismic delineation of oil and gas reservoirs using borehole
  geophones: Canadian Patents 1 106 957 and 1 114 937.

\bibitem[\protect\citeauthoryear{Baker and Carter}{1972}]{inbook}
Baker, D.~W., and Carter, N.~L., 1972, Seismic velocity anisotropy calculated
  for ultramafic minerals and aggregates, {\it in} Heard, H.~C., Borg, I.~V.,
  Carter, N.~L.,  and Raleigh, C.~B., Eds., Flow and fracture of rocks: Am.
  Geophys. Union, Geophys. Mono., 16,  157--166.

\bibitem[\protect\citeauthoryear{Constable}{1986}]{segabstract}
Constable, S.~C., 1986, Offshore electromagnetic surveying techniques: 56th
  Ann. Internat. Mtg., Soc. Expl. Geophys., Expanded Abstracts,  81--82.

\bibitem[\protect\citeauthoryear{Davis and Rabinowitz}{1975}]{segbook}
Davis, P.~J., and Rabinowitz, P., 1975, Methods of numerical integration:
  Academic Press Inc.

\bibitem[\protect\citeauthoryear{Hubbard}{1979}]{segtalk}
Hubbard, T.~P., 1979, Deconvolution of surface recorded data using vertical
  seismic profiles: Presented at the 49th Ann. Internat. Mtg., Soc. Expl.
  Geophys.

\bibitem[\protect\citeauthoryear{Lamport}{1985}]{lamport}
Lamport, L., 1985, {\LaTeX\ user's guide \& reference manual}: Addison-Wesley
  Publishing Company.

\bibitem[\protect\citeauthoryear{Landes}{1966}]{lan66}
Landes, K.~K., 1966, A scrutiny of the abstract, {II}: AAPG Bull., {\bf 50},
  1992.

\bibitem[\protect\citeauthoryear{Lindsey}{1988}]{segmagazine}
Lindsey, J.~P., 1988, Measuring wavelet phase from seismic data: The Leading
  Edge, {\bf 7}, no. 7, 10--16.

\bibitem[\protect\citeauthoryear{Lodha}{1974}]{segms}
Lodha, G.~S., 1974, Quantitative interpretation of ariborne electromagnetic
  response for a spherical model: Masters's thesis, University of Toronto.

\bibitem[\protect\citeauthoryear{Loveridge \bgroup et al.\egroup
  }{1984}]{magazine3}
Loveridge, M.~M., Parkes, G.~E., Hatton, L.,  and Worthington, M.~H., 1984,
  Effects of marine source array directivity on seismic data and source
  signature deconvolution: First Break, {\bf 2}, no. 7, 16--23.

\bibitem[\protect\citeauthoryear{Zonge and Wynn}{1975}]{segarticle}
Zonge, K.~L., and Wynn, J.~C., 1975, {EM} coupling, its intrinsic value, its
  removal, and the cultural coupling problem: Geophysics, {\bf 40}, 831--850.

\bibitem[\protect\citeauthoryear{Claerbout}{1993a}]{Claerbout.sep.77.19}
Claerbout, J.~F., 1993a, 3-d local-monoplane annihilator: Stanford Exploration
  Project Report, {\bf 77}, 19--26.

\bibitem[\protect\citeauthoryear{Claerbout}{1993b}]{Claerbout.sep.77.245}
Claerbout, J.~F., 1993b, Steep-dip deconvolution: Stanford Exploration Project
  Report, {\bf 77}, 245--256.

\end{thebibliography}
%%%%%%%%%%%%%%%%%%%%%%%%%%

\appendix
\section*{Listing}
I list the complete bibliography file, that is used by this paper.
This file could be used to create many different reference styles.
Please compare these entries to the previous bibliographies.

{\tighten
\begin{verbatim}
@ARTICLE{lan66,
   author = {K[] K[] Landes},
   journal = {AAPG Bull.},
   title = {A scrutiny of the abstract, {II}},
   volume = {50},
   pages={1992},
   year = {1966}
}
@ARTICLE{lindsey93,
   author = {J. P. Jr. Lindsey},
   journal = {Geophysics},
   title = {Instructions to authors},
   volume = {58},
   pages = {2-9},
   year = {1993}
}
@ARTICLE{segarticle,
   author = {K[] L[] Zonge and J[] C[] Wynn},
   journal = {Geophysics},
   title = {{EM} coupling, its intrinsic value, its removal, 
             and the cultural coupling problem},
   volume = {40},
   pages = {831--850},
   year = {1975}
}
@ARTICLE{segmagazine,
   author = {J[] P[] Lindsey},
   journal = {The Leading Edge},
   title = {Measuring wavelet phase from seismic data},
   volume = {7},
   number = {7},
   pages = {10-16},
   year = {1988}
}
@BOOK{segbook,
   author = {P[] J[] Davis and P[] Rabinowitz},
   title = {Methods of numerical integration},
   year = {1975},
   publisher ={Academic Press Inc.}
}
@INBOOK{inbook,
    author={D. W. Baker and N. L. Carter},
    year={1972},
    editor={ H. C. Heard and I. V. Borg and N. L. Carter and C. B. Raleigh},
    title={Seismic velocity anisotropy calculated for ultramafic minerals
            and aggregates},
    booktitle={Flow and fracture of rocks},
    publisher={Am. Geophys. Union},
    series={Geophys. Mono.},
    volume={16},
    pages={157-166}
}
@MASTERSTHESIS{segms,
   author = {G[unter] S[ergei] Lodha},
   title = {Quantitative interpretation of ariborne 
            electromagnetic response for a spherical model},
   school = {University of Toronto},
   year = {1974}
}
@INPROCEEDINGS{segtalk,
   author={T. P. Hubbard},
   howpublished = {talk},
   year=  {1979},
   title ={Deconvolution of surface recorded data using 
           vertical seismic profiles},
   meeting={49th Ann. Internat. Mtg.},
   publisher={Soc. Expl. Geophys.},
}
@INPROCEEDINGS{segabstract,
   author={S[ullivan] C[lay] Constable},
   year=  {1986},
   title ={Offshore electromagnetic surveying techniques},
   meeting={56th Ann. Internat. Mtg.},
   booktitle={Expanded Abstracts},
   publisher={Soc. Expl. Geophys.},
   pages={81-82}
}
@ARTICLE{patent,
  author={N. Anstey}, 
  title={Seismic delineation of oil and gas reservoirs using 
         borehole geophones}, 
  year={1976}, 
  journal={Canadian Patents 1 106 957 and 1 114 937}
}
@PHDTHESIS{segphd,
   author = {J[oe] Dellinger},
   title = {Anisotropic Wave Propagation},
   school = {Stanford University},
   year = {1990}
}

@ARTICLE{magazine3,
  author={M. M.  Loveridge  and G. E.  Parkes  
          and L.   Hatton  and M. H.  Worthington }, 
  title={Effects of Marine Source Array Directivity on 
         Seismic Data and Source Signature Deconvolution}, 
  year=1984, 
  journal={First Break}, 
  volume=2, 
  number=7, 
  pages={16-23}
}
@BOOK{lamport,
   author = {L[eslie] Lamport},
   title = {{\LaTeX\  user's guide \& reference manual}},
   year = {1985},
   publisher ={Addison-Wesley Publishing Company}
}
\end{verbatim}
}
\end{document}
