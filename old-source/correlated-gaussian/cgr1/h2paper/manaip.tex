%%% file: manaip.tex from REVTeX 3.1
% Copyright (c) 1992, American Institute of Physics
\documentstyle[aps]{revtex}
\makeatletter
%run page numbers by chapter
\def\thepage{3-\@arabic\c@page}
%these page numbers need abit more width
\def\@pnumwidth{2em}
\def\REVTeX{REV\TeX}
\makeatother
\begin{document}
\twocolumn
\title{REV\TeX\ Information For AIP Authors\\
\vskip 1pc
Instructions to authors for preparing compuscripts to be submitted to\\
AIP journals in REV\TeX\ 3.1 format}

\maketitle
\tableofcontents

\makeatletter
\global\@specialpagefalse
\def\@oddhead{\REVTeX{} 3.1\hfill Released September 1996}
\let\@evenhead\@oddhead
% run page numbers by "chapter", with copyright for first page
\def\@oddfoot{\reset@font\rm\hfill \thepage\hfill
\ifnum\c@page=1
  \llap{\protect\copyright{} 1996
  American Institute of Physics}%
\fi
} \let\@evenfoot\@oddfoot
\makeatother

\section{Introduction}

The American Institute of Physics has been using author-prepared
REV\TeX\ files to produce author proofs since 1992. Author files
are converted into code that can be used by Xyvision, AIP's
composition system. Therefore, you may have noticed that these
compuscripts look exactly like conventionally processed articles.

It is because of this conversion that it is imperative that the
guidelines as set forth in the REV\TeX\ input guide are followed
exactly. Failure to comply with these guidelines may result in
rejection of your file, and your manuscript will then be typeset from
scratch.

At this time, all compuscripts follow conventional processing
schedules. There are no accelerated schedules for compuscripts,
nor are there any page charge reductions.





\section{Participating Journals }
\label{partjourn}
AIP has established compuscript procedures for the following
journals:

\begin{itemize}

\item The Astronomical Journal\footnote{{\it AJ} authors are
asked to use AAS\TeX\, an author markup package prepared by the
American Astronomical Society. For more information on AAS\TeX\,
send e-mail to {\tt aastex-help@aas.org}. For instructions on how
to submit a file electronically, send an empty e-mail message to
{\tt aj-instruct@aas.org}.}
\item Applied Physics Letters
\item Chaos
\item Computers in Physics
\item Journal of Applied Physics
\item Journal of the Acoustical Society of America
\item Journal of Chemical Physics
\item Journal of Mathematical Physics
\item Journal of Rheology
\item Journal of Physical and Chemical Reference Data
\item Journal of Vacuum Science \& Technology A \& B\footnote{Authors
wishing to submit a REV\TeX\ file to these journals may of course
do so, but please be aware that this society prefers the use of AIP's
Word/WordPerfect author toolkit. Information regarding this toolkit
may be found on the AIP homepage
 \protect({\tt http://www.aip.org/compuscripts.info.html}).
 The toolkit is available via anonymous ftp from ftp.aip.org in
 the directory{\tt /ems}.}
 \item Medical Physics\footnote{See {\it JVA} \& {\it JVB}.}
 \item Noise Control Engineering
 \item Powder Diffraction
 \item Physics of Fluids
 \item Physics of Plasmas
 \item Publications of the Astronomical Society of the Pacific\footnote{%
 Authors may submit either AAS\TeX\/, \LaTeX\/, or \TeX\ files. Send an
 empty e-mail message to {\tt pasp-instruct@aas.org} for more information.}
 \item Review of Scientific Instruments

 \end{itemize}

\section{Where to turn for help}
\subsection{Procedural questions}
\label{home}
For more detailed instructions on how to submit your REV\TeX\ file,
please see the {\it Information For Contributors} page in the journal
you wish to publish in. Information can also be found on various
journal homepages (most are accessible from the AIP homepage; URL:
{\tt http://www.aip.org}).

Please remember this ``golden rule'' of submission: Do not send a file
electronically or on diskette to the editorial office unless (a) your
article has been accepted for publication or (b) you are sure that
the editorial office accepts files prior to acceptance of your paper.

%In this section are listed both AIP contacts and editorial office contacts.
%Authors can always contact AIP with any questions, technical or procedural,
%surrounding the compuscript program.  However, answers may be obtained more
%quickly by contacting the editorial office in question, depending on what
%stage the manuscript is in and whether the question is procedural or
%technical (involving the use of REV\TeX ).

\subsection{REV\TeX\ questions}

Technical questions involving the use of REV\TeX\/, transmitting files
electronically, procedural questions, etc., can be answered 
by AIP staff. Please contact:

\begin{verse}
Liz Belmont\\
Compuscript Liaison, AIP\\
e-mail: esubs@aip.org\\
phone: 516-576-2454
\end{verse}

\subsection{Prior to manuscript acceptance}
%\subsubsection{Procedural questions}

Some editorial offices accept original submissions via e-mail,
but most do not. Contact the editorial office of the journal you
wish to submit to for instructions on how to send your original
file electronically. If you have already sent a hard copy of the
manuscript to the editorial office, DO NOT send a file until you
have been instructed to do so.

{\bf Note:} When corresponding with the editorial office, {\it always}
include an e-mail address. This will help to expedite matters when
the time comes for you to send the file electronically.






\subsection{After manuscript acceptance}

Files may be submitted to AIP or the editorial office via e-mail
or on diskette (PC or Mac, preferably high density diskettes). The
editorial office will send you instructions on how and when to
submit your file with your acceptance letter (or, depending on the
editorial office, an e-mail message notifying you of acceptance).

You will receive notification with your page proofs as to whether
or not your file was used by AIP, along with an author feedback
form, outlining problems we had with the file, or perhaps offering
suggestions on how to more easily prepare files in the future.



\subsection{Editorial Office Contacts}
\label{contacts}

Editorial office contacts are listed below (editor, editorial
assistant, phone number,  and e-mail address).


\bigskip

\noindent       Applied Physics Letters

\begin{verse}
                Dr. Nghi Q. Lam (Diane Kurtz)\\
                708-252-4200\\
                apl@anl.gov
 \end{verse}


\noindent      Astronomical Journal$^*$

\begin{verse}
             Dr. Paul Hodge (Chaim Rosemarin)\\
             206-685-2150\\
             astroj@astro.washington.edu\\
             $^*$Do not use REV\TeX .  See Sec. \ref{partjourn}.
\end{verse}

\noindent      Chaos

\begin{verse}
             Dr. David Campbell\\
             (Janis Bennett, AIP; janis@aip.org; 516-576-2403)\\
             217-333-3760\\
             dkc@faust.physics.uiuc.edu
\end{verse}

\noindent      Computers in Physics

\begin{verse}
             Dr. Lewis Holmes\\
             (Patricia Daukantas: 202-745-1895)\\
             301-209-3003\\
             lh4@aip.org
\end{verse}

\noindent      Journal of Applied Physics

\begin{verse}
             Dr. Steven Rothman (Diane Kurtz)\\
             708-252-4200\\
             jap@anl.gov
\end{verse}

\noindent      The Journal of the Acoustical Society of America

\begin{verse}
             Dr. Daniel Martin\\
             513-231-5278
\end{verse}

\noindent      The Journal of Chemical Physics

\begin{verse}
             Dr. John C. Light (Mitty Collier)\\
             312-702-7067\\
             light@jcp.uchicago.edu
\end{verse}


\noindent      Journal of Mathematical Physics

\begin{verse}
             Dr. Roger G. Newton\\
             (Penny Brigham, 812-855-3576)\\
             jmp@indiana.edu
\end{verse}

\noindent    Journal of Physical and Chemical Reference Data

\begin{verse}
             Dr. Jean Gallagher/Dr. Malcolm Chase\\
             (Julian Ives)\\
             301-975-2204\\
             jpcrd@nist.gov
\end{verse}


\noindent      Journal of Rheology

\begin{verse}
               Dr. Morton Denn (Elizabeth Frey)\\
               510-642-0176\\
               jor@cchem.berkeley.edu
\end{verse}


\noindent      Journal of Vacuum Science and Technology A

\begin{verse}
             Dr. Gerald Lucovsky (Becky York)\\
             919-248-1861\\
             jvst@mcnc.org
\end{verse}

\noindent      Journal of Vacuum Science and Technology B

\begin{verse}
             Dr. Gary McGuire (Becky York)\\
             919-248-1861\\
             jvst@mcnc.org
\end{verse}

\noindent      Medical Physics
%ADD E-MAIL ADDRESS
\begin{verse}
             Dr. John S. Laughlin\\
             (Linda Addonisio, 212-639-7414)\\
             addonisio@mpcs.mskcc.org
\end{verse}

\noindent     Noise Control Engineering$^*$   

\begin{verse}
             Dr. David K. Holger\\
             515-294-6240\\
             nz.dk@iastate.edu\\
             $^*$ Word/WordPerfect toolkit preferred. See Sec. \ref{partjourn}.
\end{verse}

\noindent      Publications of the Astronomical Society of the Pacific$^*$

\begin{verse}
             Dr. Howard Bond (Denise Dankert)\\
             410-338-4958\\
             pasp@stsci.edu\\
             $^*$ AAS\TeX\/, La\TeX\/, \TeX\ preferred and accepted.
             See Sec. \ref{partjourn}.
\end{verse}

\noindent      Physics of Fluids 

\begin{verse}
             Dr. Andreas Acrivos (Cheryl Pahaham)\\
             212-283-0962\\
             pfa@lev.engr.ccny.cuny.edu
\end{verse}

\noindent      Physics of Plasmas

\begin{verse}
             Dr. Ronald C. Davidson (Sandra Schmidt)\\
             609-243-2424\\
             physplas@pppl.gov
\end{verse}

\noindent      Powder Diffraction

\begin{verse}
             Deane K. Smith (Mary Rossi)\\
             814-865-5782\\
             smith@psumrl1.psu.edu
\end{verse}

\noindent      Review of Scientific Instruments

\begin{verse}
             Dr. Thomas H. Braid (Lynne Welsh)\\
             708-252-8236\\
             rsi@anl.gov
\end{verse}

\section{REV\TeX\ Tips}

You may have discovered that, although you followed all the 
instructions in the REV\TeX\ input guide, 
your file was found to be
unusable by AIP. Of course there will be circumstances beyond anyone's
control (damaged disks, corrupt e-mail messages, etc.) that will render
a file unusable. But there are some things that you, the author, can
do before you send AIP the file to ensure that it {\it is} accepted and used to
produce your page
proofs.

AIP has developed a robust conversion to translate your file to
our Xyvision page make-up system. This conversion was written
based on the APS REV\TeX\ input guidelines and the REV\TeX\
template. If those instructions are followed exactly, and the
correct REV\TeX\ tags used, then the file will be converted without
any problems.
But we do recognize that you may
be using your REV\TeX\ file to produce preprints, so you may find
yourself ``tweaking'' fonts or spacing, or even creating new
symbols. However, for our
conversion to work properly, it is absolutely necessary that you send
AIP a ``clean'' file to work with.

Here are some things you can do to help us.

\subsection{Author-Defined Macros}

Not all author-defined macros are ``bad.'' This used to be the
case, but owing to the efforts of AIP's programmers, we are now
able to globally expand ``keystroke-saving'' macros.

``Keystroke-saving'' macros are macros
defined via the \verb+\def+ and \verb+\newcommand+ family of commands.
For example, ``\verb+\be+'' would expand to ``\verb+\begin{equation}+''
and would be defined in the 
beginning of the file as \verb+\def\be{\begin{equation}}+.

So-called ``bad'' macros are custom macros that effect low-level
formatting. It is not necessary to worry about the finer points of
typesetting. Your file is being translated into another typesetting
language, and no amount of moving and kerning of superscripts and
subscripts and creating open-face fonts (for example) will change the way the
final printed page looks like.

If you are going to use your own keystroke-saving macros, please
list all of them at the beginning of the file, before the
\verb+\begin{document}+
line. Do not sprinkle your defs (or redefine the same macros!) throughout
your paper. Again, the macro expander will not work unless all author-defined
macros are present at the beginning of the file.

The macro expander does not support Plain \TeX\ commands at this time.
Do not use \verb+\else, \fi, \if, \input, \mathop, \mathrel,
\special+. As for La\TeX\ constructions, \linebreak
please do not use \verb+ \newenvironment, \newfont,
\newtheorem, \include+.
 Do not redefine labels, do not
reset counters, etc. 

\subsection{Style files}

At this time, there are no AIP journal-specific style files. Please
use one of the {\it Physical Review} style files that most closely resembles the style
you are trying to achieve. Do not use any private style files, or
``ersatz'' AIP style files found at various FTP sites. AIP does
not support these files.

The easiest thing to do is to use the following documentstyle line.
\begin{center}
\begin{verbatim}
\documentstyle[preprint,prb,aps]{revtex}
\end{verbatim}
\end{center}
This documenstyle line will be sufficient for most journals printed
by AIP.
Using this documentstyle line will give you a preprint (12 point,
one column) manuscript, with superscripted reference citation numbers.
(If you wish to number your equation numbers by section, use the
eqsecnum style; use amssymb, amsfonts for symbols and fonts, etc.)

\subsection{Plain \TeX\ and La\TeX\ files}

Some authors will put the REV\TeX\ documentstyle line at the beginning
of a regular, ``non-REV\TeX\/''  \LaTeX\ file and send it 
in as a bona fide REV\TeX\
file. We ask that authors please do not do this, since it still requires
us to go into your file and (a) edit the file into a ``real'' REV\TeX\
file or (b) reject the file outright due to the amount of extra editing
time it would take for us to make the file usable.

Use the REV\TeX\ template. If you use the template and
follow it implicitely, the odds of your file being used to
produce your proofs will greatly increase. (See the APS input 
guide for more information.)

Do not use Plain \TeX\ to typeset equations or tables. Use \LaTeX\ and REV\TeX\
commands. 

\subsection{Cross-referencing commands}

One of the most important things you can do is use cross-referencing
commands that REV\TeX\ and \LaTeX\ provide for \verb+\section, \bibitem,
and \cite+. Use \verb+\ref+ and \verb+\label+ for tagging figures
and tables. 

Not only will you get automatic numbering, but this markup will be
translated by AIP's conversion software to create ``SGML-like'' hyperlinks, 
which
will then be used in electronic journal products such as on-line journals
and CD-ROM's.

\subsection{Roman versus {\it italic}}

One of the most prevalent problems we encounter with author files 
is the incorrect uses of roman and italic characters. 
As a rule, all math variables should be typed as italic characters
in the file. Roman functions should be set in roman font. Units of
measure should be roman also. All too often, a file has been found to
be unusable for production of author proofs because of the number of
fonts changes that would have to be made to the file by the keyboarder.

For example, if an author
typesets math variables as  roman characters throughout a file, it
means that a keyboarder must go through the file and change each
character by hand, one by one. At this point, then, it is much easier
to typeset the article from scratch, and we may have to reject your file.

Conversely, some authors will typeset units of measure or abbreviations
for elements in italic fonts. This time, the keyboarder would have to change
all the italics to roman fonts, and once again, the file 
would have to be rejected.

Here is an example of a chronic problem. The author has typeset the
elements as italics:
\[
Ga_{1-x}As_{1-y}
\]

The above {\it should} be typeset as shown below:
\[
{\rm Ga}_{1-x}{\rm As}_{1-y}
\]

Another example: ``20 $\mu m$'' is wrong. ``20 $\mu$m'' is correct.


Most math variables (with very few exceptions) must be typeset as
italic characters. 

Very often, authors will set roman functions as italic; these must
all be changed to roman. Use \verb+\sin, \cos, \tan+, etc., 
not \verb+$sin$, $cos$, $tan$+, etc.

\subsection{Equations}

Do not attempt to break equations unnecessarily. Xyvision will do this
automatically. For the most part, eqnarrays should only be used for
matrices. Do not use Plain \TeX\ commands (i.e., \verb+$$+ and
\verb+\eqno+).

\subsection{References}

Frequently seen errors are: 
\begin{itemize}
\item Missing and/or extra punctuation (usually missing serial commas
between authors' names).
\item Spacing errors: these include not enough space between the
authors' initials, awkward spacing of journal names, etc.
\item Font problems: Journal titles should be roman, not italic;
book titles should be italic, not roman; journal volume numbers should be
emphasized in bold face, etc.
\end{itemize}

\subsection{The best advice and the final word...}

The best advice we can give you is, when you are in doubt about some
stylistic question, check the latest issue of the journal you wish to
publish in. Although most of the AIP-owned journals follow the same
style rules, there are some society-owned journals whose style differs 
slightly (or somewhat greatly!)
from the AIP style. Therefore, the final word may be found in the
journal itself.

However, if you still have a question even after consulting the journal,
you should contact the editorial office or AIP. Although the AIP Compuscript
Liaison may be able to help you with these queries, you might want to
consult with the Chief Production Editor in charge of the journal. A
list of phone numbers and e-mail addresses may be found on the AIP
homepage (see Sec. \ref{home}).

\bigskip
\bigskip
\section{Other packages that can be used with REV\TeX}

\begin{itemize}

\item
BIBTEX can be used to produce the bibliography section.  Use the guidelines
for {\em Physical Review}. But please remember to add the resulting
*.bbl file to your main file.

\item  PostScript graphics files: At this time, AIP does not
accept PostScript files. However, look for AIP to announce an
author-prepared electronic graphics program in the near future. Until  
that time, please send hard copies of your figures along with the
final hard copy version of your article to the editorial office.

\end{itemize}

\section{The Latest REV\TeX\ Information}

The latest information regarding the AIP Compuscript Program
may be found on the AIP homepage. Point your browser to
\begin{center}
\begin{verbatim}
http://www.aip.org/epub/compuscripts.info.html
\end{verbatim}
\end{center}


\end{document}


% end of file manaip.tex

