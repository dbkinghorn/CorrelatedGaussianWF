%% This document created by Scientific Word (R) Version 3.0

\documentclass[12pt]{article}
\usepackage{graphicx}
\usepackage{aip}
\usepackage{setspace}
\usepackage{amsmath}
\usepackage{amsfonts}
\usepackage{amssymb}
%TCIDATA{OutputFilter=latex2.dll}
%TCIDATA{CSTFile=LaTeX article (bright).cst}
%TCIDATA{Created=Fri Oct 23 17:38:26 1998}
%TCIDATA{LastRevised=Fri Nov 13 11:07:17 1998}
%TCIDATA{<META NAME="GraphicsSave" CONTENT="32">}
%TCIDATA{<META NAME="DocumentShell" CONTENT="Journal Articles\Standard LaTeX Article">}
%TCIDATA{Language=American English}
\newtheorem{theorem}{Theorem}
\newtheorem{acknowledgement}[theorem]{Acknowledgement}
\newtheorem{algorithm}[theorem]{Algorithm}
\newtheorem{axiom}[theorem]{Axiom}
\newtheorem{case}[theorem]{Case}
\newtheorem{claim}[theorem]{Claim}
\newtheorem{conclusion}[theorem]{Conclusion}
\newtheorem{condition}[theorem]{Condition}
\newtheorem{conjecture}[theorem]{Conjecture}
\newtheorem{corollary}[theorem]{Corollary}
\newtheorem{criterion}[theorem]{Criterion}
\newtheorem{definition}[theorem]{Definition}
\newtheorem{example}[theorem]{Example}
\newtheorem{exercise}[theorem]{Exercise}
\newtheorem{lemma}[theorem]{Lemma}
\newtheorem{notation}[theorem]{Notation}
\newtheorem{problem}[theorem]{Problem}
\newtheorem{proposition}[theorem]{Proposition}
\newtheorem{remark}[theorem]{Remark}
\newtheorem{solution}[theorem]{Solution}
\newtheorem{summary}[theorem]{Summary}
\newenvironment{proof}[1][Proof]{\textbf{#1.} }{\ \rule{0.5em}{0.5em}}

\doublespacing

\begin{document}


\title{A correlated basis set for non-adiabatic energy calculations on diatomic molecules}
\author{Donald B. Kinghorn and Ludwik Adamowicz\\Department of Chemistry\\University of Arizona\\Tucson AZ 85721}
\maketitle
\begin{abstract}
A new explicitly correlated basis set suitable for non-adiabatic energy
calculation on small diatomic molecules is presented. The basis functions
consist of correlated Gaussians multiplied by powers of the internuclear
distance. N-body formulas for Hamiltonian matrix elements and energy gradient
components are derived and presented along with a discussion of the
non-adiabatic Hamiltonian and symmetry considerations. A sample calculation is
presented for the ground state energy of the benchmark system H$_{2}^{+}$ in
which rapid convergence to near exact results was observed.
\end{abstract}



\section{Introduction}

An interesting Quantum Mechanical problem that is almost entirely untouched,
is that of computing molecular energy levels fully non-adiabatically, without
recourse to semi-classical constructs\textit{.} More specifically, finding
high accuracy variational\ solutions of the stationary state Schr\"{o}denger
equation including all electronic and nuclear degrees of freedom with
explicitly correlated wave functions.

The Born-Oppenheimer (BO) approximation is the center of the computational and
conceptual framework of conventional quantum theory of molecular systems and
there is no denying it's usefulness. However, the absence of a simple
alternative has lead to the semi-classical constructs of potential energy
hypersurface and equilibrium geometry largely supplanting the quantum
mechanical principles of delocalization and expectation value. Additionally,
there are times when the BO approximation fails completely.

When the perturbation strength, $\kappa$, is large, BO concepts become
inadequate. ($\kappa$ is the fourth root of the ratio of light to heavy
particle masses). Adiabatic corrections to the zero order BO energy of
Hydrogen containing molecules can be appreciable, ($\kappa=.15)$. Systems
containing exotic particles such as positrons, ($\kappa=1),$ and mesons,
($\kappa=.26),$ may not converge at all in the BO approximation. For exotic
systems and non-rigid (floppy) molecules the BO notion of equilibrium geometry
may be meaningless. In any case, achieving the highest possible accuracy for
variational energy bounds requires a non-adiabatic approach with full
correlation between all particles.

The present work proposes a new explicitly correlated basis set for performing
high accuracy non-adiabatic variational energy calculations on small
``diatomic'' systems. The basis functions are of the form, $\phi_{k}%
=r_{1}^{m_{k}}\exp[-\mathbf{r}^{\prime}\bar{A}_{k}\mathbf{r}]$. These basis
functions are correlated Gaussians with a pre-multiplying factor $r_{1}%
^{m_{k}},$ the $m$-th power of the distance between the two heavy particles,
hence the approach is restricted to diatomics with $\sigma$ electrons. The
exponential factor provides full correlation between all system particles.

The reasoning that lead to this form of the basis functions is as follows:
There have been several highly accurate non-adiabatic variational calculations
on atomic and exotic few particle systems using simple correlated
Gaussians\cite{Kinghorn93,Kinghorn95b,Kozlowski93b,Kinghorn97a,Varga96}. By
simple we mean that they only contain the exponential part of the $\phi_{k},$
(no $r_{1}$ pre-multipliers). However, attempts at non-adiabatic molecular
calculations have been plagued by problems with linear dependence in the basis
during energy optimizations. This problem occurs in calculations on atomic
systems also, but, to a much lesser extent. We feel that we understand this
problem and our initial trials with the new basis including pre-multiplying
powers of $r_{1}$ has eliminated or at least drastically reduced the linear
dependence problems. We believe the reason for the success with the new basis
is due to the fact that in diatomics there will be large particle density away
from the origin in relative coordinates (when one of the nuclei is taken as
the coordinate origin). That is, the wave function will have peaks shifted
away from the origin. There are three ways to account for this behavior in the
wave function using correlated Gaussians; 1) use correlated Gaussians with
shifted centers, i.e. $\exp[-\left(  \mathbf{r-s}\right)  ^{\prime
}\bar{A}\left(  \mathbf{r-s}\right)  ],$ 2) Use near linearly dependent
combinations of simple correlated Gaussians with large matched $\pm$ linear
coefficients, or 3) Use pre-multiplying powers of $r_{1}$. The first option is
unacceptable since it results in a wave function which no longer represents a
pure angular momentum state. The second option is what we believe causes the
linear dependence and numerical instability which we are trying to avoid. The
third option is what we have implemented. The linear dependence that we have
observed in our calculation using the simple correlated Gaussians looks, in
some sense, like an attempt by the optimization to include in the wave
function derivatives of the basis functions with respect to the non-linear
parameters. The near linear dependent terms resemble numerical derivatives.
Removal of these near linear dependent terms has an adverse affect on the wave
functions, as manifested by poor energy results, but leaving them in leads to
numerical instabilities which hinder optimization or cause complete collapse
of the eign-solutions. Now, derivatives of simple Gaussians with respect to
non-linear parameters, elements of the matrices $A_{k}$, bring down
pre-multiplying (even) powers of $r_{ij}$. Thus, explicitly including
pre-multiplying powers of $r_{ij}$ in the basis functions should add the
needed flexibility to the basis in a numerically stable way. This is indeed
what we have observed by including $r_{1}^{m_{k}}$ terms in our trials with
diatomic systems.

The rest of this manuscript presents our non-adiabatic method for variational
energy calculations on diatomic systems utilizing this new basis. We begin
with a presentation of the non-adiabatic Hamiltonian including the
transformation to center of mass and relative coordinates. This is followed by
a detailed examination of the basis functions. A discussion of symmetry
considerations follows including the projection method we use for obtaining
pure states. Formulas for Hamiltonian and overlap matrix elements are then
presented. Energy gradient components are given since this is essential for
thorough optimization of the many non-linear parameters contained in a
variational trial function in this basis. We then validate the methodology by
showing preliminary results for the benchmark system H$_{2}^{+}.$\ \ This is
followed by conclusions and proposed future work.

\section{Non-Adiabatic Hamiltonian}

Consider a system of $N$ particles with masses $\{M_{1},M_{2},...,M_{N}\}$ and
charges $\{Q_{1},Q_{2},...,Q_{N}\}$. If we take the particles to be charged
point masses with coulombic interactions. The non-relativistic Hamiltonian has
the familiar form,
\begin{equation}
H_{tot}=-\sum_{i}^{N}\frac{\nabla_{\mathbf{R}_{i}}^{2}}{2M_{i}}+\sum_{i<j}%
^{N}\frac{Q_{i}Q_{j}}{\left\|  \mathbf{R}_{i}-\mathbf{R}_{j}\right\|
}\,,\label{ham1}%
\end{equation}
where $\mathbf{R}_{i}=[X_{i}\,\,Y_{i}\,\,Z_{i}]^{\prime}$ is a column vector
of Cartesian coordinates for particle $i$ in the external, laboratory fixed,
frame, $\nabla_{\mathbf{R}_{i}}^{2}$ the Laplacian in the coordinates of
$\mathbf{R}_{i}$, and $\left\|  \mathbf{R}_{i}-\mathbf{R}_{j}\right\|  $ the
distance between particles $i$ and $j$ ($\left\|  ...\right\|  $ is the
standard 2-norm. Also, we use a back tick, ('), to denote the matrix/vector
transpose.) If the particles are a mixture of nuclei and electrons, the
standard treatment is to force a separation of this Hamiltonian into nuclear
and electronic components using the Born-Oppenheimer approximations. For the
non-adiabatic approach no such approximations are made. All particles are
treated equally utilizing their given masses and full interaction with all
other particles. There is of course one simplification of the total
Hamiltonian that does not involve any approximations, namely, the separation
of the translational degrees of freedom.

The total Hamiltonian, eqn(\ref{ham1}), is separable into an operator
describing the translational motion of the center of mass and an operator
describing the internal energy. This separation is realized by a
transformation to center of mass and internal (relative) coordinates. We now
present this transformation to establish the coordinate framework within which
our new basis set will be utilized.

Let $\mathbf{R}$ be the vector of particle coordinates in the laboratory fixed
reference frame,
\begin{equation}
\mathbf{R=}\left[
\begin{array}
[c]{c}%
\mathbf{R}_{1}\\
\mathbf{R}_{2}\\
\vdots\\
\mathbf{R}_{N}%
\end{array}
\right]  =\left[
\begin{array}
[c]{c}%
X_{1}\\
Y_{1}\\
Z_{1}\\
\vdots\\
Z_{N}%
\end{array}
\right]  .
\end{equation}
Center of mass and internal coordinates are given by the transformation
$T:\mathbf{R}\mapsto\lbrack\mathbf{r}_{0}^{\prime},\mathbf{r}^{\prime
}]^{\prime}$
\begin{equation}
T=\left[
\begin{array}
[c]{ccccc}%
\frac{M_{1}}{m_{0}} & \frac{M_{2}}{m_{0}} & \frac{M_{3}}{m_{0}} & \cdots &
\frac{M_{N}}{m_{0}}\\
-1 & 1 & 0 & \cdots & 0\\
-1 & 0 & 1 & \cdots & 0\\
\vdots & \vdots & \vdots & \ddots & \vdots\\
-1 & 0 & 0 & \cdots & 1
\end{array}
\right]  \otimes I_{3},\label{Ttran}%
\end{equation}
where $m_{0}=\sum_{i}^{N}M_{i}$. $\mathbf{r}_{0}$ is the vector of coordinates
for the center of mass and $\mathbf{r}$ is a length $3n=3\left(  N-1\right)  $
vector of internal coordinates with respect to a reference frame with origin
at particle 1
\begin{equation}
\mathbf{r=}\left[
\begin{array}
[c]{c}%
\mathbf{r}_{1}\\
\mathbf{r}_{2}\\
\vdots\\
\mathbf{r}_{n}%
\end{array}
\right]  =\left[
\begin{array}
[c]{c}%
\mathbf{R}_{2}-\mathbf{R}_{1}\\
\mathbf{R}_{3}-\mathbf{R}_{1}\\
\vdots\\
\mathbf{R}_{N}-\mathbf{R}_{1}%
\end{array}
\right]  .\label{rdef}%
\end{equation}
(The Kronecker product with the $3\times3$ identity matrix preserves the
isotropy of the transformation. The Kronecker product,or matrix outer product,
is defined in the references\cite{Kinghorn95a} or any advanced linear algebra
text.)
Using this coordinate transformation, and the
conjugate momentum transformation, the internal Hamiltonian can be written
as,
\begin{equation}
H=-\frac{1}{2}\left(  \sum_{i}^{n}\frac{1}{\mu_{i}}\nabla_{i}^{2}+\sum_{i\neq
j}^{n}\frac{1}{M_{1}}\nabla_{i}^{\prime}\nabla_{j}\right)  +\sum_{i<j}%
^{n}\frac{q_{i}q_{j}}{r_{ij}},\label{intham1}%
\end{equation}
where the $\mu_{i}$ are reduced masses, $M_{1}$ is the mass of particle 1,
(the coordinate reference particle), and $\nabla_{i}$ is the gradient with
respect to the $x,y,z$ coordinates $\mathbf{r}_{i}$. The potential energy is
the same as in the total Hamiltonian but is now written using internal
distance coordinates. The charges are mapped as $\{Q_{1},Q_{2},...,Q_{N}%
\}\mapsto\{q_{0},q_{1},...,q_{n}\}$ and the summation in the potential is from
$i=0$ to $n$ instead of from $1$ to $N$ as in the total Hamiltonian. Thus, in
internal coordinates, distances are denoted, $r_{ij}=\left\|  \mathbf{r}%
_{i}-\mathbf{r}_{j}\right\|  =$ $\left\|  \mathbf{R}_{i+1}-\mathbf{R}%
_{j+1}\right\|  \,\,$with\thinspace$r_{0j}\equiv r_{j}=\left\|  \mathbf{r}%
_{j}\right\|  =\left\|  \mathbf{R}_{j+1}-\mathbf{R}_{1}\right\|  .$

The kinetic energy term in the internal Hamiltonian, eqn(\ref{intham1}),
includes what is usually referred to as the mass polarization, i.e. the second
summation in eqn(\ref{intham1}). For convenience in the integral calculations
we write the total kinetic energy as a quadratic form in the length $3n$
vector gradient operator, $\nabla_{\mathbf{r}},$ with respect to the length
$3n$ vector $\mathbf{r}$ of internal coordinates. Thus we have for our
non-adiabatic Hamiltonian,
\begin{equation}
H=-\nabla_{\mathbf{r}}^{\prime}\left(  M\otimes I_{3}\right)  \nabla
_{\mathbf{r}}+\sum_{i<j}^{n}\frac{q_{i}q_{j}}{r_{ij}},\label{ham}%
\end{equation}
where $M$ is an $n\times n$ matrix with $1/2\mu_{i}$ on the diagonal and
$1/2M_{1}$ for the off diagonal elements. It is also possible to write the
potential term as a vector operator\cite{Kinghorn95a} however we have no
compelling reason to do so here. We make no further transformations or
approximations to this Hamiltonian. More information on the center of mass
separation and form of the Hamiltonian, eqn(\ref{ham}), can be found in the
references\cite{Kinghorn93,Kinghorn95b}.\qquad\ 

\section{Basis Set}

The Hamiltonian presented above is valid for any number of particles with any
masses and charges. The basis set we are presenting is, however, designed for
use in the case where exactly two particles have much larger masses than the
rest as is the case in diatomic molecules. If we label the two heavy particles
as one and two then the distance between these particles is given by $r_{1}=$
$\left\|  \mathbf{R}_{1}-\mathbf{R}_{2}\right\|  $ and the basis set we are
presenting is given by,
\begin{equation}
\phi_{k}=r_{1}^{m_{k}}\exp\left[  -\mathbf{r}^{\prime}\overline{A_{k}%
}\mathbf{r}\right]  .\label{basis1}%
\end{equation}
It is the inclusion of the factor $r_{1}^{m_{k}}$ which gives this basis its
stability and rapid convergence characteristics for diatomic molecules
compared with simple correlated Gaussians. The matrix of the quadratic form in
the exponential is denoted using an overbar to represent a Kronecker product
with the $3\times3$ identity matrix. This matrix $\overline{A_{k}}$ is
positive definite and symmetric to insure that the basis is square integrable,
\textit{i.e.} admit a norm. Thus we write $\overline{A_{k}}$ in a Cholesky
factored form as follows,\qquad%
\begin{align}
\overline{A_{k}} &  =\mathbf{r}^{\prime}(A_{k}\otimes I_{3})\mathbf{r}\\
&  =L_{k}L_{k}^{\prime}\text{ ,\thinspace\thinspace\thinspace\thinspace
\thinspace\thinspace\thinspace\thinspace}L_{k}\text{ lower triangular and rank
}n.\nonumber
\end{align}
The elements of the lower triangular matrix $L_{k}$ are the non-linear
variation parameters used to optimize the basis in energy calculations.

The quadratic form in the exponential of $\phi_{k}$ may be written using
internal distance variables, $\left\{  r_{ij}\right\}  ,$ as follows (we drop
the subscript $k$ for convenience):
\begin{align}
\mathbf{r}^{\prime}(A\otimes I_{3})\mathbf{r} &  =\sum_{i,j}\mathbf{r}%
_{i}^{\prime}\mathbf{r}_{j}\,\,A_{ij}\\
&  =\operatorname{tr}\left[  \left(  \mathbf{r}_{i}^{\prime}\mathbf{r}%
_{j}\right)  A\right]  \\
&  =\operatorname{tr}\left[  \left(  r_{ij}^{2}\right)  B\right]  \\
&  =\sum_{i,j}r_{ij}^{2}\,\,B_{ij}\,,
\end{align}
where $\operatorname{tr}\left[  {}\right]  $ is the matrix trace operator,
$\left(  \mathbf{r}_{i}^{\prime}\mathbf{r}_{j}\right)  $ is the $n\times n$
matrix of inner products of the component vectors of $\mathbf{r},$ $\left(
r_{ij}^{2}\right)  $ is the $n\times n$ matrix of squared distance variables,
and $B$ is a matrix with elements given, in terms of the elements of $A$, by
the transformation,
\begin{equation}
B_{ij}=\left\{
\begin{array}
[c]{ll}%
\frac{1}{2}%
%TCIMACRO{\dsum \limits_{t=1}^{n}}%
%BeginExpansion
{\displaystyle\sum\limits_{t=1}^{n}}
%EndExpansion
\left(  A_{it}+A_{tj}\right)  , & i=j\\
-\frac{1}{4}\left(  A_{ij}+A_{ji}\right)  , & i\neq j
\end{array}
\right.  .\label{Btran}%
\end{equation}
Conversely, given $B_{k}$, the transform to $A_{k},$ is given by,
\begin{equation}
A_{ij}=\left\{
\begin{array}
[c]{ll}%
B_{ii}+%
%TCIMACRO{\dsum \limits_{k\neq i,j}^{n}}%
%BeginExpansion
{\displaystyle\sum\limits_{k\neq i,j}^{n}}
%EndExpansion
\left(  B_{ik}+B_{kj}\right)  , & i=j\\
-\left(  B_{ij}+B_{ji}\right)  , & i\neq j
\end{array}
\right.  .\label{Atran}%
\end{equation}
Also, $r_{1}^{m}$ can be written as the square root of a quadratic form in
$\mathbf{r}$ using the matrix $\left(  J_{11}\otimes I_{3}\right)  $ with
$J_{11}$ defined as an $n\times n$ matrix with 1 in the $1,1$ position, and
0's elsewhere\cite{Poshusta83,Kinghorn95a},
\begin{equation}
r_{1}^{m}=\left[  \mathbf{r}^{\prime}(J_{11}\otimes I_{3})\mathbf{r}\right]
^{m/2}.\label{rijJ}%
\end{equation}
\ Hence, $\phi_{k}$ can be written using distance coordinates or internal
Cartesian coordinates,
\begin{equation}
\phi_{k}=r_{1}^{m_{k}}\exp\left[  -\operatorname{tr}\left[  \left(  r_{ij}%
^{2}\right)  B_{k}\right]  \right]  ,\label{basisrij}%
\end{equation}
or
\begin{equation}
\phi_{k}=\left[  \mathbf{r}^{\prime}(J_{11}\otimes I_{3})\mathbf{r}\right]
^{\frac{m_{k}}{2}}\exp\left[  -\mathbf{r}^{\prime}(A_{k}\otimes I_{3}%
)\mathbf{r}\right]  .\label{basisr}%
\end{equation}
The form of $\phi_{k}$ in eqn(\ref{basisrij}) is perhaps more amenable to
interpretation while the form in eqn(\ref{basisr}) is more useful for
evaluating matrix elements.

\section{Symmetry}

Before moving on to integral and gradient formulas we comment briefly on the
symmetry properties of the basis set and Hamiltonian. Any coordinate
transformation that commutes with the Hamiltonian constitutes a symmetry
element. The most common of these are translational, rotational and
permutational symmetry. Translational symmetry is of no concern for the
internal Hamiltonian. The Kronecker products with the $3\times3$ identity
matrix $I_{3}$ in equation (\ref{basisr}) insure rotational invariance of the
basis functions. The $\phi_{k}$ are angular momentum eigenfunctions with total
angular momentum $J=0$. The handling of permutational symmetry is outlined
briefly in what follows.

\subsection{\bigskip Induced Symmetry Transformations}

Permutations of real particles induce transformations on internal coordinates.
Let $P$ be a permutation of real particles, then, under the center of mass
transformation (\ref{Ttran}) the permutation $P$ induces the transformation
$TPT^{-1}\otimes I_{3}$ on center of mass and internal coordinates. $P$
interchanges particles with the same mass and thus leaves the center of mass
unchanged, it follows that $TPT^{-1}\otimes I_{3}$ is a direct sum of the
identity transformation on the center of mass and an induced transformation on
internal coordinates
\begin{equation}
TPT^{-1}\otimes I_{3}=\left(  I_{1}\oplus\tau_{P}\right)  \otimes I_{3}%
=I_{3}\oplus\left(  \tau_{P}\otimes I_{3}\right)  .\label{TPT}%
\end{equation}
Thus, $P$ transforms basis functions as,
\begin{equation}
P\phi_{k}=r_{1}^{m_{k}}\exp\left[  -r^{\prime}\left(  \tau_{P}^{\prime}%
A_{k}\tau_{P}\otimes I_{3}\right)  r\right]  .
\end{equation}
Note that $r_{1}^{m_{k}}$ is invariant under any permutation on systems of
particles for which $\phi_{k}$ is a valid basis.

The $n\times n$ matrices $\tau_{P}^{^{\prime}}$ are not in general elementary
permutation matrices for example: Consider the three particle system,
H$_{2}^{+},$ with the two protons taken as particles 1 and 2, then, as an
example, the permutation $P_{(12)}$ which exchanges the protons is given by,
the permutation matrix
\begin{equation}
P_{(12)}=\left(
\begin{array}
[c]{ccc}%
0 & 1 & 0\\
1 & 0 & 0\\
0 & 0 & 1
\end{array}
\right)  ,
\end{equation}
which is transformed by $T$ as
\begin{equation}
TP_{\left(  12\right)  }T^{-1}\otimes I_{3}=\left[
\begin{array}
[c]{ccc}%
1 & 0 & 0\\
0 & -1 & 0\\
0 & -1 & 1
\end{array}
\right]  \otimes I_{3}=I_{3}\oplus\left(  \tau_{\left(  12\right)  }\otimes
I_{3}\right)  .
\end{equation}
Thus,
\begin{equation}
\tau_{\left(  12\right)  }=\left[
\begin{array}
[c]{cc}%
-1 & 0\\
-1 & 1
\end{array}
\right]
\end{equation}
is not an elementary permutation matrix.

Energy eigenstates of the internal Hamiltonian transform irreducibly under the
induced symmetry group. Since the basis functions, $\phi_{k},$ are explicitly
correlated, pure state eigenfunctions are achieved by symmetry projection on
the basis functions. Symmetry projectors on real particle coordinates are
transformed to projectors on relative particle coordinates using the
transformation given above. Thus, a symmetry projector, $\mathcal{P}$, for an
irreducible representation of the permutational symmetry group of a system is
given by,
\begin{equation}
\mathcal{P}=\sum_{P}\chi_{P}P\,\,\mapsto\,\,\sum_{P}\chi_{P}\tau_{P}.
\end{equation}
Hence, $\mathcal{P}$ acts on $\phi_{k}$ as
\begin{equation}
\mathcal{P}\phi_{k}=\sum_{P}\chi_{P}\,\,r_{1}^{m_{k}}\exp\left[  -r^{\prime
}\left(  \tau_{P}^{\prime}A_{k}\tau_{P}\otimes I_{3}\right)  r\right]  .
\end{equation}
The coefficients $\chi_{P}$ are from the matrix elements of the irreducible
representation for the desired state, and in the special case of one
dimensional representations, can be taken as the characters of the representation.

Computational effort for computing matrix elements with symmetry projected
basis functions can be reduced by a factor equal to the order of the group by
exploiting commutation of the symmetry projectors with the Hamiltonian and
identity operators. In general
\begin{equation}
\left\langle \mathcal{P}\phi_{k}\right|  H\left|  \mathcal{P}\phi
_{l}\right\rangle =\left\langle \phi_{k}\right|  H\left|  \mathcal{P}%
^{\dagger}\mathcal{P}\phi_{l}\right\rangle
\end{equation}
thus, symmetry projection need only be performed on the ket. Typically
projection operators are Hermitian and essentially idempotent: $\mathcal{P}%
^{\dagger}\mathcal{P}\varpropto\mathcal{P}$ in any case we will simply write
$\mathcal{P}$ for ket projector.

\section{Matrix Elements}

Evaluation of Hamiltonian matrix elements is the heart of this work. The
matrix elements needed are of the form
\[
O_{kl}=\left\langle \phi_{k}\right|  O\left|  \mathcal{P}\phi_{l}\right\rangle
=\sum_{P}\chi_{P}\left\langle \phi_{k}\right|  O\left|  r_{1}^{m_{k}}%
\exp\left[  -r^{\prime}\left(  \tau_{P}^{\prime}A_{l}\tau_{P}\otimes
I_{3}\right)  r\right]  \right\rangle \,,
\]
where $O$ is the identity operator, the kinetic energy operator or the
potential energy operator. We will not write the summation over terms in the
symmetry projector in the formulas that follow and simply note that these
terms are accounted for by making the following substitution in the integral
formulas.
\begin{equation}
A_{l}\mapsto\tau_{P}^{\prime}A_{l}\tau_{P}=\tau_{P}^{\prime}L_{l}\left(
\tau_{P}^{\prime}L_{l}\right)  ^{\prime}%
\end{equation}
No other modification to the integral formulas need be made.

To simplify the presentation that follows we define
\begin{equation}
f_{k}=\exp\left[  -\mathbf{r}^{\prime}\overline{A_{k}}\mathbf{r}\right]  .
\end{equation}
Integrals and derivatives in a basis consisting of $f_{k}$ have been derived
in detail by Kinghorn\cite{Kinghorn95a,Kinghorn95b} using a powerful matrix
differential calculus. Many of the ``tricks'' used in that work are used in
the present work. Therefore, only the essential steps in formula derivations
will be presented here. The reader is referred to the references given above
for more detail. Before continuing with the integral and derivative
derivations a few needed notational definitions are presented.

\subsection{Notation}

The vec operator transforms a matrix into a vector by stacking the columns of
the matrix one underneath the other. Let $A$ be an $m\times n$ matrix and
$\mathbf{a_{j}}$ its $j^{th}$ column, then $\,\mathrm{vec}\,A$ 
is the $mn\times1$ vector
\begin{equation}
\,\mathrm{vec}\,A=\left[
\begin{array}
[c]{c}%
\mathbf{a_{1}}\\
\mathbf{a_{2}}\\
\vdots\\
\mathbf{a_{n}}%
\end{array}
\right]  .
\end{equation}

An operator similar to vec is the vech, ``vector half'', operator. Let $A$ be
a square $n\times n$ matrix . Then $\,\mathrm{vech}\,A$ is the $n\left(
n+1\right)  /2\times1$ vector obtained by stacking the lower triangular
elements of $A$. For example, if $n=3$,
\begin{equation}
\,\mathrm{vech}\,A=\left[
\begin{array}
[c]{c}%
a_{11}\\
a_{21}\\
a_{31}\\
a_{22}\\
a_{32}\\
a_{33}%
\end{array}
\right]  .
\end{equation}
For symmetric $X$, $\,\mathrm{vech}\,X$ contains the independent elements of
$X$.

We have already defined $J_{11}$ however, in general, $J_{ij}$ is defined as
the $n\times n$ matrix,
\begin{equation}
J_{ij}=\left\{
\begin{array}
[c]{ll}%
E_{ii} & i=j\\
E_{ii}+E_{jj}-E_{ij}-E_{ji} & i\neq j
\end{array}
\right.  \,,\label{Jij}%
\end{equation}
where $E_{ij}$ is the $n\times n$ matrix with $1$ in its $ij^{th}$ position
and $0$'s elsewhere.

Other notation used; $\,\mathrm{diag}\,A$ is the diagonal $n\times n$ matrix
consisting of the diagonal elements of the square matrix $A.$ The trace of $A
$ is denoted $\,\mathrm{tr}\,A,$ the determinant of $A$ is denoted $|A|$.
Other notation will be introduced as needed.

\subsection{Overlap matrix element}

To begin let $\left(  A_{k}\otimes I_{3}\right)  +\left(  A_{l}\otimes
I_{3}\right)  =\left(  A_{k}+A_{l}\right)  \otimes I_{3}=A_{kl}\otimes
I_{3}=\bar{A}_{kl}$, then,
\begin{equation}
\left\langle f_{k}\right.  |\left.  f_{l}\right\rangle =\pi^{3n/2}\left|
\overline{A_{kl}}\right|  ^{-1/2}=\pi^{3n/2}\left|  A_{kl}\right|  ^{-3/2}.
\end{equation}
The overlap in $\phi_{k}$ can be expressed as the expectation value of
$r_{1}^{m_{k}+m_{l}}$ in the $f_{k}$ basis,
\begin{align}
\left\langle \phi_{k}\right.  |\left.  \phi_{l}\right\rangle  &  =\left\langle
f_{k}\right|  r_{1}^{m_{k}+m_{l}}\left|  f_{l}\right\rangle \label{ov1}\\
&  =\frac{2}{\sqrt{\pi}}\,\Gamma\left(  \frac{m_{k}+m_{l}}{2}+\frac{3}%
{2}\right)  \,\left\langle f_{k}\right.  |\left.  f_{l}\right\rangle
\,\,\mathrm{tr}\left[  J_{11}A_{kl}^{-1}\right]  ^{\frac{m_{k}+m_{l}}{2}%
}\nonumber\\
&  =\frac{2}{\sqrt{\pi}}\,\Gamma\left(  \frac{m_{k}+m_{l}}{2}+\frac{3}%
{2}\right)  \,\left\langle f_{k}\right.  |\left.  f_{l}\right\rangle \left(
A_{kl}^{-1}\right)  _{11}^{\frac{m_{k}+m_{l}}{2}}.\nonumber
\end{align}
Note: This formula is valid for any values of $m_{k}+m_{l}$ $\geqslant-2.$
Using this result the normalized overlap matrix element is,
\begin{align}
S_{kl} &  =\frac{\left\langle \phi_{k}\right.  |\left.  \phi_{l}\right\rangle
}{\left(  \left\langle \phi_{k}\right.  |\left.  \phi_{k}\right\rangle
\left\langle \phi_{l}\right.  |\left.  \phi_{l}\right\rangle \right)  ^{1/2}%
}\label{Skl}\\
&  =\gamma_{1}\left(  m_{k},m_{l}\right)  \,2^{3n/2}\left[  \left(
\frac{\left(  A_{kl}^{-1}\right)  _{11}}{\left(  A_{k}^{-1}\right)  _{11}%
}\right)  ^{m_{k}}\left(  \frac{\left(  A_{kl}^{-1}\right)  _{11}}{\left(
A_{l}^{-1}\right)  _{11}}\right)  ^{m_{l}}\left(  \frac{\left\|
L_{k}\right\|  \left\|  L_{l}\right\|  }{\left|  A_{kl}\right|  }\right)
^{3}\right]  ^{1/2}\,,\nonumber
\end{align}
where, $\left\|  L_{k}\right\|  $ is the absolute value of the determinant of
the Cholesky factor of $A_{k},$ and
\begin{equation}
\gamma_{1}\left(  m_{k},m_{l}\right)  =\frac{2^{\frac{m_{k}+m_{l}}{2}}%
\,\Gamma\left(  \frac{m_{k}+m_{l}}{2}+\frac{3}{2}\right)  }{\,\left[
\Gamma\left(  m_{k}+\frac{3}{2}\right)  \Gamma\left(  m_{l}+\frac{3}%
{2}\right)  \right]  ^{1/2}}%
\end{equation}
is a constant that can be pre-computed for a range of $m$ values to avoid
evaluating gamma functions in the inner loop of the energy code.

\subsection{Kinetic Energy Matrix Element}

The kinetic energy can be reduced to a sum of derivatives of overlap integrals
when we use the gradient form of the operator;
\begin{align}
T_{kl} &  =\left\langle \phi_{k}\right|  -\nabla_{\mathbf{r}}^{\prime}\left(
M\otimes I_{3}\right)  \nabla_{\mathbf{r}}\left|  \phi_{l}\right\rangle \\
&  =\left\langle \nabla_{\mathbf{r}}^{\prime}\,r_{1}^{m_{k}}f_{k}\right|
\overline{M}\left|  \nabla_{\mathbf{r}}\,r_{1}^{m_{l}}f_{l}\right\rangle
.\nonumber
\end{align}
Now, recalling that $r_{1}^{m}=\left[  \mathbf{r}^{\prime}(J_{11}\otimes
I_{3})\mathbf{r}\right]  ^{m/2},$ the differential of $\phi_{k}$ with respect
to the vector $\mathbf{r}$ is, after simplification,
\begin{equation}
d\phi_{k}=\mathbf{r}^{\prime}\phi_{k}\left(  m_{k}r_{1}^{-2}\overline{J_{11}%
}-2\overline{A_{k}}\right)  d\mathbf{r.}%
\end{equation}
Hence,
\begin{align}
T_{kl} &  =\left\langle \phi_{k}\right|  \mathbf{r}^{\prime}\left(  m_{k}%
r_{1}^{-2}\overline{J_{11}}-2\overline{A_{k}}\right)  \overline{\,M}\,\left(
m_{l}r_{1}^{-2}\overline{J_{11}}-2\overline{A_{l}}\right)  \mathbf{r}\left|
\phi_{l}\right\rangle \\
&  =m_{k}\,m_{l}\,M_{11}\left\langle \phi_{k}\right|  \,r_{1}^{-2}\left|
\phi_{l}\right\rangle \nonumber\\
&  -2m_{k}\left\langle \phi_{k}\right|  \,r_{1}^{-2}\,\mathbf{r}^{\prime
}\overline{J_{11}}\,\overline{\,M}\,\overline{A_{l}}\mathbf{r\,}\left|
\phi_{l}\right\rangle \nonumber\\
&  -2m_{l}\left\langle \phi_{k}\right|  \,r_{1}^{-2}\,\mathbf{r}^{\prime
}\overline{\,A_{k}}\,\overline{M}\,\,\overline{J_{11}}\,\mathbf{r\,}\left|
\phi_{l}\right\rangle \nonumber\\
&  +4\left\langle \phi_{k}\right|  \,\mathbf{r}^{\prime}\overline{\,A_{k}%
}\,\overline{M}\,\,\,\overline{A_{l}}\mathbf{r\,}\left|  \phi_{l}\right\rangle
.\nonumber
\end{align}
The quadratic forms inside theses integrals can be written as the
$\mathrm{vec}\,\left[  \,\right]  $ of the rank one matrix $\mathbf{rr}%
^{\prime}$ which can further be written in terms of a derivative with respect
to $\overline{A_{kl}}$ and then moved outside of the
integral\cite{Kinghorn95a}. For example,
\begin{align}
\left\langle \phi_{k}\right|  \,r_{1}^{-2}\,\mathbf{r}^{\prime}\overline
{J_{11}}\,\overline{\,M}\,\overline{A_{l}}\mathbf{r\,}\left|  \phi
_{l}\right\rangle  &  =\left\langle \phi_{k}\right|  \,r_{1}^{-2}%
\,\mathrm{vec}\,\left[  \mathbf{rr}^{\prime}\right]  \mathbf{\,}\left|
\phi_{l}\right\rangle \mathrm{vec}\,\left[  \overline{J_{11}}\,\overline
{\,M}\,\overline{A_{l}}\right]  \\
&  =\frac{\partial}{\partial\left(  \mathrm{vec}\overline{A_{kl}}\right)
^{\prime}}\left\langle \phi_{k}\right|  \,r_{1}^{-2}\,\mathbf{\,}\left|
\phi_{l}\right\rangle \mathrm{vec}\,\left[  \overline{J_{11}}\,\overline
{\,M}\,\overline{A_{l}}\right]  .\nonumber
\end{align}
With the other integrals having a similar form. Then, making use of the
overlap formula, eqn(\ref{ov1}), the derivatives with respect to
$\mathrm{vec}\overline{A_{kl}}$ are,
\begin{equation}
\frac{\partial}{\partial\left(  \mathrm{vec}\overline{A_{kl}}\right)
^{\prime}}\left\langle \phi_{k}\right|  \,r_{1}^{-2}\,\mathbf{\,}\left|
\phi_{l}\right\rangle =\frac{1}{2}\left(  \mathrm{vec}\,\left[  \overline
{\,A_{kl}}\,^{-1}\right]  \right)  ^{\prime}+\frac{m_{k}+m_{l}}{6}%
\mathrm{tr}\left[  J_{11}A_{kl}^{-1}\right]  ^{-1}%
\end{equation}
and
\begin{equation}
\frac{\partial}{\partial\left(  \mathrm{vec}\overline{A_{kl}}\right)
^{\prime}}\left\langle \phi_{k}\right.  |\left.  \phi_{l}\right\rangle
=\frac{1}{2}\left(  \mathrm{vec}\,\left[  \overline{\,A_{kl}}\,^{-1}\right]
\right)  ^{\prime}.
\end{equation}
Then after substitution, rearrangement and simplification, the kinetic energy
matrix elements are given by,
\begin{align}
T_{kl} &  =6S_{kl}\,\mathrm{tr}\left[  A_{kl}^{-1}A_{k}MA_{l}\right]
+2S_{kl}\left(  A_{kl}^{-1}\right)  _{11}^{-1}\label{Tkl}\\
&  \times\left[  \frac{m_{k}m_{l}M_{11}}{\left(  m_{k}+m_{l}+1\right)  }%
-m_{k}\left(  A_{kl}^{-1}A_{l}MA_{l}A_{kl}^{-1}\right)  _{11}-m_{l}\left(
A_{kl}^{-1}A_{k}MA_{k}A_{kl}^{-1}\right)  _{11}\right]  .\nonumber
\end{align}

\subsection{Potential Energy Matrix Elements}

We will derive the potential energy components by finding the integral
$R_{kl}^{ij}=\left\langle 1/r_{ij}\right\rangle .$ There are two cases,
$m_{k}+m_{l}$ even and $m_{k}+m_{l}$ odd.

First some simplifying definitions. Let,
\begin{align}
a &  =\mathrm{tr}\left[  J_{11}A_{kl}^{-1}\right]  \label{adef}\\
b &  =\mathrm{tr}\left[  J_{ij}A_{kl}^{-1}\right]  \label{bdef}\\
c &  =\mathrm{tr}\left[  J_{11}A_{kl}^{-1}J_{ij}A_{kl}^{-1}\right]
,\label{cdef}%
\end{align}
which for implementation purposes can be written as,%

\begin{equation}
a=\left(  A_{kl}^{-1}\right)  _{11}%
\end{equation}%
\begin{equation}
b=\left\{
\begin{array}
[c]{ll}%
\left(  A_{kl}^{-1}\right)  _{ij} & i=j\\
\left(  A_{kl}^{-1}\right)  _{ii}+\left(  A_{kl}^{-1}\right)  _{jj}-2\left(
A_{kl}^{-1}\right)  _{ij} & i\neq j
\end{array}
\right.
\end{equation}
and
\begin{equation}
c=\left\{
\begin{array}
[c]{ll}%
\left(  A_{kl}^{-1}\right)  _{1i}^{2} & i=j\\
\left(  \left(  A_{kl}^{-1}\right)  _{1i}+\left(  A_{kl}^{-1}\right)
_{j1}\right)  ^{2} & i\neq j
\end{array}
\right.  .
\end{equation}

\subsubsection{$m_{k}+m_{l}$ Even}

Let $m_{k}+m_{l}=2p$ with $p=0,1,2,...$ then,
\begin{align}
R_{kl}^{ij} &  =\left\langle \phi_{k}\right|  \,1/r_{ij}\,\mathbf{\,}\left|
\phi_{l}\right\rangle \\
&  =\left\langle f_{k}\right|  \,r_{1}^{2p}/r_{ij}\,\mathbf{\,}\left|
f_{l}\right\rangle .
\end{align}
Using an integral transformation for $1/r_{ij}\,,$%
\begin{equation}
R_{kl}^{ij}=\frac{2}{\sqrt{\pi}}\int_{0}^{\infty}\left\langle \phi_{k}\right|
\,\exp\left[  -t^{2}\,\,\mathbf{r}^{\prime}\overline{J_{ij}}\,\mathbf{r}%
\right]  \,\mathbf{\,}\left|  \phi_{l}\right\rangle \,dt,
\end{equation}
followed by a differential transformation for $r_{1}^{2p},$ and integration
over $\mathbf{r}$, we have,
\begin{align}
R_{kl}^{ij} &  =\left.  \frac{2}{\sqrt{\pi}}\left(  -1\right)  ^{p}\int
_{0}^{\infty}\left\langle f_{k}\right|  \,\frac{\partial^{p}}{\partial u^{p}%
}\exp\left[  -u\,\,\mathbf{r}^{\prime}\overline{J_{11}}\,\mathbf{r}\right]
\exp\left[  -t^{2}\,\,\mathbf{r}^{\prime}\overline{J_{ij}}\,\mathbf{r}\right]
\,\mathbf{\,}\left|  f_{l}\right\rangle \,dt\right|  _{u=0}\nonumber\\
&  =\left.  \frac{2}{\sqrt{\pi}}\left(  -1\right)  ^{p}\frac{\partial^{p}%
}{\partial u^{p}}\int_{0}^{\infty}\int_{-\infty}^{\infty}\exp\left[
-\mathbf{r}^{\prime}\left(  \overline{A_{kl}}+u\,\,\overline{J_{11}}%
\,+t^{2}\,\,\overline{J_{ij}}\right)  \,\mathbf{r}\right]  \,\mathbf{\,}%
\,d\mathbf{r}\,dt\right|  _{u=0}\nonumber\\
&  =\left.  \frac{2}{\sqrt{\pi}}\left(  -1\right)  ^{p}\frac{\partial^{p}%
}{\partial u^{p}}\,\left\langle f_{k}\right.  |\left.  f_{l}\right\rangle
\int_{0}^{\infty}\left|  I_{n}+u\,\,J_{11}A_{kl}^{-1}\,+t^{2}\,\,J_{ij}%
\,A_{kl}^{-1}\right|  ^{-3/2}\mathbf{\,}\,\,dt\right|  _{u=0}.
\end{align}
Now, since $J_{11}$ and $J_{ij}$ are rank one, we can write the determinant in
the integral above as a polynomial in the traces we defined at the beginning
of this section and then integrate over $t$ yielding,
\begin{align}
R_{kl}^{ij} &  =\left.  \frac{2}{\sqrt{\pi}}\left(  -1\right)  ^{p}%
\frac{\partial^{p}}{\partial u^{p}}\,\left\langle f_{k}\right.  |\left.
f_{l}\right\rangle \int_{0}^{\infty}\left(  1+u\,a\,+t^{2}b+ut^{2}\left(
ab-c\right)  \right)  ^{-3/2}\mathbf{\,}\,\,dt\right|  _{u=0}\nonumber\\
&  =\left.  \frac{2}{\sqrt{\pi}}\left(  -1\right)  ^{p}\frac{\partial^{p}%
}{\partial u^{p}}\,\left\langle f_{k}\right.  |\left.  f_{l}\right\rangle
\left(  1+au\right)  ^{-1}\left(  b+u\left(  ab-c\right)  \right)
^{-1/2}\mathbf{\,}\,\,\right|  _{u=0}.
\end{align}
Then, differentiating $p$ times with respect to $u,$ setting $u$ to zero and
simplifying gives the final result,
\begin{equation}
R_{kl}^{ij}=S_{kl}\frac{\gamma_{2}\left(  p\right)  }{\sqrt{b}}\left[
1+\sum_{q=1}^{p}\gamma_{3}\left(  q\right)  \left(  1-\frac{c}{ab}\right)
^{q}\right]  ,\label{Rkleven}%
\end{equation}
where
\begin{equation}
\gamma_{2}\left(  p\right)  =\frac{\Gamma\left(  p+1\right)  }{\Gamma\left(
p+3/2\right)  }\label{gamma2}%
\end{equation}
and
\begin{equation}
\gamma_{3}\left(  q\right)  =\frac{\Gamma\left(  q+1/2\right)  }{\Gamma\left(
q+1\right)  \Gamma\left(  1/2\right)  }.\label{gamma3}%
\end{equation}

\subsubsection{$m_{k}+m_{l}$ Odd}

The odd case involves an additional integral transformation which
unfortunately makes these terms somewhat more complicated. The integral
evaluation is similar to that for the even case and proceeds as follows.

Let $m_{k}+m_{l}=2p-1$ with $p=1,2,3,...$ then,
\begin{align}
R_{kl}^{ij} &  =\left\langle \phi_{k}\right|  \,1/r_{ij}\,\mathbf{\,}\left|
\phi_{l}\right\rangle \nonumber\\
&  =\left\langle f_{k}\right|  \,r_{1}^{2p-1}/r_{ij}\,\mathbf{\,}\left|
f_{l}\right\rangle \nonumber\\
&  =\left\langle f_{k}\left|  \frac{r_{1}^{2p}}{r_{1}r_{ij}}\right|
\,\mathbf{\,}f_{l}\right\rangle .
\end{align}
Making transformations as for the even case, but with the addition of an extra
integral transformation for $1/r_{1}$ we obtain,%
\begin{align}
R_{kl}^{ij} &  =\left(  \frac{2}{\sqrt{\pi}}\right)  ^{2}\left(  -1\right)
^{p}\int_{0}^{\infty}\int_{0}^{\infty}\left\langle f_{k}\right|
\,\frac{\partial^{p}}{\partial u^{p}}\exp\left[  -u\,\,\mathbf{r}^{\prime
}\overline{J_{11}}\,\mathbf{r}\right]  \nonumber\\
&  \times\left.  \exp\left[  -v^{2}\,\,\mathbf{r}^{\prime}\overline{J_{11}%
}\,\mathbf{r}\right]  \exp\left[  -t^{2}\,\,\mathbf{r}^{\prime}\overline
{J_{ij}}\,\mathbf{r}\right]  \,\mathbf{\,}\left|  f_{l}\right\rangle
\,dv\,dt\right|  _{u=0}\nonumber\\
&  =\frac{4}{\pi}\left(  -1\right)  ^{p}\frac{\partial^{p}}{\partial u^{p}%
}\int_{0}^{\infty}\int_{0}^{\infty}\int_{-\infty}^{\infty}\nonumber\\
&  \left.  \exp\left[  -\mathbf{r}^{\prime}\left(  \overline{A_{kl}%
}+u\,\,\overline{J_{11}}\,+v^{2}\,\,\overline{J_{11}}+t^{2}\,\,\overline
{J_{ij}}\right)  \,\mathbf{r}\right]  \,\mathbf{\,}\,d\mathbf{r}%
\,dv\,dt\right|  _{u=0}\nonumber\\
&  =\frac{4}{\pi}\left(  -1\right)  ^{p}\frac{\partial^{p}}{\partial u^{p}%
}\,\left\langle f_{k}\right.  |\left.  f_{l}\right\rangle \int_{0}^{\infty
}\int_{0}^{\infty}\nonumber\\
&  \left.  \left|  I_{n}+\left(  u+v^{2}\right)  \,\,J_{11}A_{kl}^{-1}%
\,+t^{2}\,\,J_{ij}\,A_{kl}^{-1}\right|  ^{-3/2}\mathbf{\,}\,\,dv\,dt\right|
_{u=0}.
\end{align}
Again, reducing the determinant to a polynomial in traces and integrating
gives,
\begin{align}
R_{kl}^{ij} &  =\frac{4}{\pi}\left(  -1\right)  ^{p}\frac{\partial^{p}%
}{\partial u^{p}}\,\left\langle f_{k}\right.  |\left.  f_{l}\right\rangle
\int_{0}^{\infty}\int_{0}^{\infty}\nonumber\\
&  \left.  \left(  1+\left(  u+v^{2}\right)  \,a\,+t^{2}b+ut^{2}\left(
ab-c\right)  \right)  ^{-3/2}\mathbf{\,}\,dv\,dt\right|  _{u=0}\nonumber\\
&  =\frac{4}{\pi}\left(  -1\right)  ^{p}\frac{\partial^{p}}{\partial u^{p}%
}\,\left\langle f_{k}\right.  |\left.  f_{l}\right\rangle \int_{0}^{\infty
}\nonumber\\
&  \left.  \left(  1+a\left(  u+v^{2}\right)  \right)  ^{-1}\left(  b+\left(
u+v^{2}\right)  \left(  ab-c\right)  \right)  ^{-1/2}\mathbf{\,}%
\,\,\,dv\right|  _{u=0}\nonumber\\
&  =\frac{4}{\pi}\left(  -1\right)  ^{p}\frac{\partial^{p}}{\partial u^{p}%
}\,\left\langle f_{k}\right.  |\left.  f_{l}\right\rangle \left(  c\left(
1+au\right)  \right)  ^{-1/2}\nonumber\\
&  \left.  \times\sin^{-1}\left[  \left(  \frac{c}{a\left(  b+u\left(
ab-c\right)  \right)  }\right)  ^{-1/2}\right]  \right|  _{u=0}\mathbf{\,}\,.
\end{align}
Differentiating and simplifying gives the final result,
\begin{align}
R_{kl}^{ij} &  =\frac{2}{\sqrt{\pi}}S_{kl}\sqrt{a}\nonumber\\
&  \times\left[  \gamma_{3}\left(  p\right)  \frac{\sin^{-1}\left[
\sqrt{\frac{c}{ab}}\right]  }{\sqrt{c}}+\frac{1}{2\sqrt{ab-c}}\sum_{q=1}%
^{p}\sum_{t=0}^{q-1}\frac{1}{q}\gamma_{3}\left(  p-q\right)  \gamma_{3}\left(
t\right)  \left(  1-\frac{c}{ab}\right)  ^{q-t}\right]  .\label{Rklodd}%
\end{align}
The inclusion of the arcsine and the double summation in this formula
unfortunately complicates these odd power terms compared to the even power
case. Our implementation required twice as long for energy calculations when
odd $m_{k}$ were included in the basis set expansion due to the complexity of
this formula. Furthermore, we found that variation of near optimal $m_{k}$ by
plus or minus one had negligeable effect on energy convergence. Therefore in
our implementation utilizing gradient formulas for energy optimization we
excluded the odd power case.

\section{Energy Gradient}

The integral formulas above are sufficient for performing energy calculations
and constructing wave functions. However, optimization of the many non-linear
parameters contained in the exponent matrices $\ L_{k}$ demands excessive
computational resources if an approximate numerical energy gradient is used.
Several orders of magnitude of computational effort can be saved by utilizing
analytic gradients. Additionally, more thorough optimization can be achieved
due to increased accuracy of analytic gradients compared with numerical
approximations. We feel that analytic gradient formulas are essential for
practical utilization of the new basis functions. To this end, the energy
gradient formulas are now presented.

We begin with the derivative of the secular equation with respect to energy
eigenvalues. For a detailed derivation and background on matrix differential
calculus see the references\cite{Kinghorn95a,Kinghorn95b}.

The secular equation,
\begin{equation}
\left(  H-\epsilon S\right)  c=0,
\end{equation}
defines the energy $\epsilon$ as an implicit function of the $N\times N$
matrices $H$ and $S,$ where $N$ is the number of basis functions. $H$ and $S$
are themselves functions of the $Nn\left(  n+1\right)  /2\times1$ vector
$L=\left[  \left(  \mathrm{vech}\,L_{1}\right)  ^{\prime},\cdots,\left(
\mathrm{vech}\,L_{N}\right)  ^{\prime}\right]  $ of nonlinear exponential
parameters contained in the matrices $L_{k}$; recall that $A_{k}=L_{k}%
L_{k}^{\prime}$. The energy gradient with respect to $L$ is then
\begin{equation}
g=\nabla_{L}\epsilon=\frac{1}{c^{\prime}Sc}\left(  \frac{\partial
\,\mathrm{vech}\,H}{\partial L^{\prime}}-\epsilon\frac{\partial\,\mathrm{vech}%
\,S}{\partial L^{\prime}}\right)  ^{\prime}\left(  \,\mathrm{vech}\,\left[
2cc^{\prime}-\,\mathrm{diag}\,cc^{\prime}\right]  \right)  .\label{grad}%
\end{equation}

The matrix $\left(  \partial\,\mathrm{vech}\,H/\partial L^{\prime}%
-\epsilon\,\,\partial\,\mathrm{vech}\,S/\partial L^{\prime}\right)
=\partial\,\mathrm{vech}\,\left(  H-\epsilon S\right)  /\partial L^{\prime}$
in the gradient above is sparse and has dimension $N\left(  N+1\right)
/2\times Nn\left(  n+1\right)  /2$. This sparse matrix together with the
eigenvector $c$ can be collapsed to a dense partitioned form with dimension
$N\times Nn\left(  n+1\right)  /2$,
\begin{equation}
G=\left[
\begin{array}
[c]{cccc}%
c_{1}^{2}\frac{\partial\left(  H-\epsilon S\right)  _{11}}{\partial\left(
\,\mathrm{vech}\,L_{1}\right)  ^{\prime}} & 2c_{1}c_{2}\frac{\partial\left(
H-\epsilon S\right)  _{12}}{\partial\left(  \,\mathrm{vech}\,L_{2}\right)
^{\prime}} & \cdots & 2c_{1}c_{N}\frac{\partial\left(  H-\epsilon S\right)
_{1N}}{\partial\left(  \,\mathrm{vech}\,L_{N}\right)  ^{\prime}}\\
2c_{2}c_{1}\frac{\partial\left(  H-\epsilon S\right)  _{21}}{\partial\left(
\,\mathrm{vech}\,L_{1}\right)  ^{\prime}} & c_{2}^{2}\frac{\partial\left(
H-\epsilon S\right)  _{22}}{\partial\left(  \,\mathrm{vech}\,L_{2}\right)
^{\prime}} & \cdots & 2c_{2}c_{N}\frac{\partial\left(  H-\epsilon S\right)
_{2N}}{\partial\left(  \,\mathrm{vech}\,L_{N}\right)  ^{\prime}}\\
\vdots & \vdots & \ddots & \vdots\\
2c_{N}c_{1}\frac{\partial\left(  H-\epsilon S\right)  _{N1}}{\partial\left(
\,\mathrm{vech}\,L_{1}\right)  ^{\prime}} & 2c_{N}c_{2}\frac{\partial\left(
H-\epsilon S\right)  _{N2}}{\partial\left(  \,\mathrm{vech}\,L_{2}\right)
^{\prime}} & \cdots &  c_{N}^{2}\frac{\partial\left(  H-\epsilon S\right)
_{NN}}{\partial\left(  \,\mathrm{vech}\,L_{N}\right)  ^{\prime}}%
\end{array}
\right]  .
\end{equation}
With $G$ defined as above the gradient can be computed by summing over the
rows of $G$. That is
\[
g_{i}=\frac{1}{c^{\prime}Sc}\sum_{j}G_{ji}%
\]

The non-zero terms in the matrices $\partial\,\mathrm{vech}\,H/\partial
a^{\prime}$ and $\partial\,\mathrm{vech}\,S/\partial a^{\prime}$ are contained
in the $1\times n\left(  n+1\right)  /2$ vectors $\partial\,H_{kl}%
/\partial\left(  \,\mathrm{vech}\,L_{k}\right)  ^{\prime}$ and $\partial
\,S_{kl}/\partial\left(  \,\mathrm{vech}\,L_{k}\right)  ^{\prime},$and also in
$\partial\,H_{kl}/\partial\left(  \,\mathrm{vech}\,L_{l}\right)  ^{\prime}$
and $\partial\,S_{kl}/\partial\left(  \,\mathrm{vech}\,L_{l}\right)  ^{\prime
}$. Matrix derivatives depend on the arrangement of elements in a matrix
variable, therefore, since symmetry projection on kets effectively reorders
elements of the exponent matrix $L_{l},$ the formulas for derivatives with
respect to the exponent matrices in the ket, $\,\mathrm{vech}\,\left[
L_{l}\right]  ,$ are different than those with respect to the exponent
matrices in the bra,$\,\mathrm{vech}\,\left[  L_{k}\right]  $. ( Recall that
$\tau_{P}^{\prime}A_{l}\tau_{P}=\tau_{P}^{\prime}L_{l}L_{l}^{\prime}\tau
_{P}=\tau_{P}^{\prime}L_{l}\left(  \tau_{P}^{\prime}L_{l}\right)  ^{\prime}$
). Also, the matrix derivatives for the diagonal blocks of $G$ are complicated
somewhat by the symmetry projection on the kets. However, they can be computed
using the following relationship, for example,
\begin{equation}
\frac{\partial H_{kk}}{\partial\left(  \,\mathrm{vech}\,L_{k}\right)
^{\prime}}=\left.  \frac{\partial H_{kl}}{\partial\left(  \,\mathrm{vech}%
\,L_{k}\right)  ^{\prime}}\right|  _{l=k}+\left.  \frac{\partial H_{kl}%
}{\partial\left(  \,\mathrm{vech}\,L_{l}\right)  ^{\prime}}\right|  _{l=k}.
\end{equation}
Thus, only two sets of formulas for the derivatives need be computed.

Matrix elements are scalar valued matrix functions of the exponent matrices
$\ L_{k}.$ Therefore, the appropriate mathematical tool for finding
derivatives is the matrix differential calculus\cite{Kinghorn95a}. Using this,
the derivations are non-trivial but straight forward. Since it is beyond the
scope of this paper to give a detailed discussion of the matrix differential
calculus, only the final results form the derivations will be presented. The
reader wishing to derive these formulas, or other matrix derivatives, is
referred to the paper cited above and references therein.

Note: only one term in the symmetry projection will be represented. As was the
case for the integral formulas, the symmetry terms require the substitution
$A_{l}\mapsto\tau_{P}^{\prime}A_{l}\tau_{P}=\tau_{P}^{\prime}L_{l}\left(
\tau_{P}^{\prime}L_{l}\right)  ^{\prime}$ or more generally, $L_{l}\mapsto
\tau_{P}^{\prime}L_{l}.$ \ This is required for derivatives with respect to
both $\mathrm{vech}\,\left[  L_{k}\right]  $ and $\mathrm{vech}\,\left[
L_{l}\right]  .$ The derivatives with respect to $\mathrm{vech}\,\left[
L_{l}\right]  $ will require further modification and this will be noted in
the formulas below.

\subsection{Overlap Derivatives}

Using the normalized overlap formula, eqn(\ref{Skl}), the derivative with
respect to the non-zero terms of the lower triangular matrix $L_{k}$ is,
\begin{align}
\frac{\partial S_{kl}}{\partial\left(  \,\mathrm{vech}\,L_{k}\right)
^{\prime}}  &  =\frac{3}{2}S_{kl}\,\mathrm{vech}\,\left[  \left(  \,L_{k}%
^{-1}\right)  ^{\prime}-2A_{kl}^{-1}L_{k}\right]  ^{\prime}\nonumber\\
&  +S_{kl}\frac{m_{k}}{\left(  A_{k}^{-1}\right)  _{11}}\mathrm{vech}\,\left[
\,A_{k}^{-1}J_{11}A_{k}^{-1}L_{k}\right]  ^{\prime}\nonumber\\
&  -S_{kl}\frac{m_{k}+m_{l}}{\left(  A_{kl}^{-1}\right)  _{11}}\mathrm{vech}%
\,\left[  \,A_{kl}^{-1}J_{11}A_{kl}^{-1}L_{k}\right]  ^{\prime}.
\end{align}
For the derivative with respect to the elements $\mathrm{vech}\,L_{l}$ we
account for the symmetry terms by making a multiplication by $\tau_{P}$ \ in
addition to the substitutions described above. Thus,
\begin{align}
\frac{\partial S_{kl}}{\partial\left(  \,\mathrm{vech}\,L_{l}\right)
^{\prime}}  &  =\frac{3}{2}S_{kl}\,\mathrm{vech}\,\left[  \tau_{P}\left(
\,L_{l}^{-1}\right)  ^{\prime}-2\tau_{P}A_{kl}^{-1}L_{l}\right]  ^{\prime
}\nonumber\\
&  +S_{kl}\frac{m_{l}}{\left(  A_{l}^{-1}\right)  _{11}}\mathrm{vech}\,\left[
\,\tau_{P}A_{l}^{-1}J_{11}A_{l}^{-1}L_{l}\right]  ^{\prime}\nonumber\\
&  -S_{kl}\frac{m_{k}+m_{l}}{\left(  A_{kl}^{-1}\right)  _{11}}\mathrm{vech}%
\,\left[  \,\tau_{P}A_{kl}^{-1}J_{11}A_{kl}^{-1}L_{l}\right]  ^{\prime}.
\end{align}

\subsection{Kinetic Energy Derivatives}

The kinetic energy gradient components are obtained by differentiating
eqn(\ref{Tkl}) with respect to $\mathrm{vech}\,L_{k}$ and $\mathrm{vech}%
\,L_{l}.$ First, to simplify the derivative expressions, lets define,
\begin{equation}
T_{2}=\frac{m_{k}m_{l}M_{11}}{\left(  m_{k}+m_{l}+1\right)  }-m_{k}\left(
A_{kl}^{-1}A_{l}MA_{l}A_{kl}^{-1}\right)  _{11}-m_{l}\left(  A_{kl}^{-1}%
A_{k}MA_{k}A_{kl}^{-1}\right)  _{11}\,,
\end{equation}
which is a component of the kinetic energy matrix element. Then,
\begin{align}
\frac{\partial T_{kl}}{\partial\left(  \,\mathrm{vech}\,L_{k}\right)
^{\prime}} &  =\frac{\partial S_{kl}}{\partial\left(  \,\mathrm{vech}%
\,L_{k}\right)  ^{\prime}}\frac{T_{kl}}{S_{kl}}+2S_{kl}\left[
\begin{array}
[c]{c}%
\,\\
\,
\end{array}
\right.  \nonumber\\
&  +6\mathrm{vech}\left[  A_{kl}^{-1}A_{l}MA_{l}A_{kl}^{-1}L_{k}\right]
^{\prime}\nonumber\\
&  +2\left(  A_{kl}^{-1}\right)  _{11}^{-2}T_{2}\mathrm{vech}\left[
A_{kl}^{-1}J_{11}A_{kl}^{-1}L_{k}\right]  ^{\prime}\nonumber\\
&  -2\left(  A_{kl}^{-1}\right)  _{11}^{-1}m_{l}\mathrm{vech[}A_{kl}%
^{-1}J_{11}A_{kl}^{-1}A_{k}MA_{l}A_{kl}^{-1}L_{k}+\nonumber\\
&  \quad\quad\quad\,\quad A_{kl}^{-1}A_{l}MA_{k}A_{kl}^{-1}J_{11}A_{kl}%
^{-1}L_{k}]\nonumber\\
&  +2\left(  A_{kl}^{-1}\right)  _{11}^{-1}m_{k}\mathrm{vech[}A_{kl}%
^{-1}J_{11}A_{kl}^{-1}A_{l}MA_{l}A_{kl}^{-1}L_{k}+\nonumber\\
&  \quad\quad\quad\quad A_{kl}^{-1}A_{l}MA_{l}A_{kl}^{-1}J_{11}A_{kl}%
^{-1}L_{k}]\left.
\begin{array}
[c]{c}%
\,\\
\,
\end{array}
\right]  ,
\end{align}
and the derivative with respect to $\mathrm{vech}\,L_{l}$ including the
symmetry projector term $\tau_{P}$ is given by,
\begin{align}
\frac{\partial T_{kl}}{\partial\left(  \,\mathrm{vech}\,L_{l}\right)
^{\prime}} &  =\frac{\partial S_{kl}}{\partial\left(  \,\mathrm{vech}%
\,L_{l}\right)  ^{\prime}}\frac{T_{kl}}{S_{kl}}+2S_{kl}\left[
\begin{array}
[c]{c}%
\,\\
\,
\end{array}
\right.  \nonumber\\
&  +6\mathrm{vech}\left[  \tau_{P}A_{kl}^{-1}A_{k}MA_{k}A_{kl}^{-1}%
L_{l}\right]  ^{\prime}\nonumber\\
&  +2\left(  A_{kl}^{-1}\right)  _{11}^{-2}T_{2}\mathrm{vech}\left[  \tau
_{P}A_{kl}^{-1}J_{11}A_{kl}^{-1}L_{l}\right]  ^{\prime}\nonumber\\
&  -2\left(  A_{kl}^{-1}\right)  _{11}^{-1}m_{k}\mathrm{vech[}\tau_{P}%
A_{kl}^{-1}J_{11}A_{kl}^{-1}A_{l}MA_{k}A_{kl}^{-1}L_{l}+\nonumber\\
&  \quad\quad\quad\,\quad\tau_{P}A_{kl}^{-1}A_{k}MA_{l}A_{kl}^{-1}J_{11}%
A_{kl}^{-1}L_{l}]\nonumber\\
&  +2\left(  A_{kl}^{-1}\right)  _{11}^{-1}m_{l}\mathrm{vech[}\tau_{P}%
A_{kl}^{-1}J_{11}A_{kl}^{-1}A_{k}MA_{k}A_{kl}^{-1}L_{l}+\nonumber\\
&  \quad\quad\quad\quad\tau_{P}A_{kl}^{-1}A_{k}MA_{k}A_{kl}^{-1}J_{11}%
A_{kl}^{-1}L_{l}]\left.
\begin{array}
[c]{c}%
\,\\
\,
\end{array}
\right]  .
\end{align}

\subsection{Potential Energy Derivatives}

The purpose of the gradient formulas is to enhance optimization efforts and we
have found that well optimized wave functions utilizing a mixture of even and
odd powers, $m_{k},$ did not produce results any better than using even powers
only. Hence, for this, and reasons that were stated at the end of the section
describing the potential energy matrix elements, we will not derive the
gradient terms for the potential energy matrix elements that utilize odd
$m_{k}.$

Using the even $m_{k}$ potential energy integral formula, eqn(\ref{Rkleven}),
the definitions for $a,\,b$ and $c$ given in eqns(\ref{adef},\ref{bdef}%
,\ref{cdef}), and the definitions for $\gamma_{2}$ and $\gamma_{3}$ given in
eqns(\ref{gamma2},\ref{gamma3}), the gradient terms for the potential, with
$m_{k}+m_{l}=2p,$ are given by,
\begin{align}
\frac{\partial R_{kl}^{ij}}{\partial\left(  \,\mathrm{vech}\,L_{k}\right)
^{\prime}} &  =\frac{\partial S_{kl}}{\partial\left(  \,\mathrm{vech}%
\,L_{k}\right)  ^{\prime}}\frac{R_{kl}^{ij}}{S_{kl}}-\gamma_{2}\left(
p\right)  \,S_{kl}\left[
\begin{array}
[c]{c}%
\,\\
\,
\end{array}
\right.  \nonumber\\
&  \frac{R_{kl}^{ij}}{2b}\frac{\partial b}{\partial\left(  \,\mathrm{vech}%
\,L_{k}\right)  ^{\prime}}-b^{-1/2}\left[
\begin{array}
[c]{c}%
\,\\
\,
\end{array}
\right.  \nonumber\\
&  \sum_{q=1}^{p}\gamma_{3}\left(  q\right)  \,q\,\left(  1-\frac{c}%
{ab}\right)  ^{q-1}\frac{c}{\left(  ab\right)  ^{2}}\nonumber\\
&  \times\left(  \frac{\partial a}{\partial\left(  \,\mathrm{vech}%
\,L_{k}\right)  ^{\prime}}b+a\frac{\partial b}{\partial\left(  \,\mathrm{vech}%
\,L_{k}\right)  ^{\prime}}-\frac{ab}{c}\frac{\partial c}{\partial\left(
\,\mathrm{vech}\,L_{k}\right)  ^{\prime}}\right)  \left.
\begin{array}
[c]{c}%
\,\\
\,
\end{array}
\right]  \left.
\begin{array}
[c]{c}%
\,\\
\,
\end{array}
\right]  ,
\end{align}
where
\begin{equation}
\frac{\partial a}{\partial\left(  \,\mathrm{vech}\,L_{k}\right)  ^{\prime}%
}=-2\,\mathrm{vech}\,\left[  A_{kl}^{-1}J_{11}A_{kl}^{-1}L_{k}\right]  ,
\end{equation}%

\begin{equation}
\frac{\partial b}{\partial\left(  \,\mathrm{vech}\,L_{k}\right)  ^{\prime}%
}=-2\,\mathrm{vech}\,\left[  A_{kl}^{-1}J_{ij}A_{kl}^{-1}L_{k}\right]
\end{equation}
and
\begin{align}
\frac{\partial c}{\partial\left(  \,\mathrm{vech}\,L_{k}\right)  ^{\prime}} &
=-2\,\mathrm{vech}\,\left[  A_{kl}^{-1}J_{11}A_{kl}^{-1}J_{ij}A_{kl}^{-1}%
L_{k}\right]  \nonumber\\
&  -2\,\mathrm{vech}\,\left[  A_{kl}^{-1}J_{ij}A_{kl}^{-1}J_{11}A_{kl}%
^{-1}L_{k}\right]  .
\end{align}
The derivative with respect to $\mathrm{vech}\,L_{l}$ has the same form as the
above but with $L_{k}$ replaced by $\tau_{P}^{\prime}L_{l}$ and each of the
expressions $\mathrm{vech}\,\left[  A_{kl}^{-1}\cdots\right]  $ replaced by
$\mathrm{vech}\,\left[  \tau_{P}A_{kl}^{-1}\cdots\right]  .$

\section{Implementation and a Sample Calculation}

Programming a variational energy code using our formulas for matrix elements
and gradient components is greatly simplified by the matrix vector form of our
derivations. This is largely due to our use of the matrix differential
calculus without which integral and derivative evaluation would have been
extremely difficult (if not impossible). Implementation of these formulas is
straight forward taking advantage of the matrix intrinsics in Fortran90. The
matrix programming language MATLAB\cite{MATLAB} has also proven to be valuable
for testing and debugging program components since with MATLAB our formulas
can essentially be coded as written. All formulas have been checked for
consistency against multiple implementations and known special cases
($m_{k}=0$). All gradient components have been verified by finite differences. 

Optimization of nonlinear parameters in wavefunctions constructed from this
new basis is essential for good results. It is the availability of our
gradient formulas that makes use of this basis practical. The optimization
software used for our test calculations was the package TN written by Stephen
Nash\cite{NashTN}. TN is a truncated Newton package where the quadratic
subproblem is solved approximately using a linear conjugate gradient method.
(Source code available from netlib\cite{netlib}.) Performance with this
optimization code has been good, but we are experimenting with other
optimization methodologies including tensor methods (fourth order methods)
various preconditioning schemes and integer programming for optimization of
the powers $m_{k}$.

There are very few non-adiabatic variational energy calculation reported in
the literature for molecular systems. The results of Bishop and
Chung\cite{Bishop77} for H$_{2}^{+}$ are highly accurate making this system an
essential benchmark for new calculations of this type. To validate our
methodology we present some preliminary calculations on this benchmark system.
We should stress however that we are not limited to three particle systems and
our implementation is for general N-particle diatomic systems. The practical
limitation on the number of particles is from the number of terms needed for
proper symmetry projection, which is on the order of $N_{e}!,$ where $N_{e}$
\ is the number of electrons.

Table (\ref{H2pRes}), present results for the non-adiabatic ground state
energy of H$_{2}^{+}$ utilizing from 4 to 128 of our new basis functions for
the expansion of the wave function. The values reported in the table are: the
energy in Hartrees, the squared norm of the gradient, $g^{\prime}g$, and the
absolute value of one minus the virial coefficient, $\left|  1-\eta\right|  $.
The virial coefficient, $\eta=-\left\langle V\right\rangle /2\left\langle
T\right\rangle ,$ was computed as a simple check on the scale optimally of the
wave function. The nuclear mass was the same as that used in Bishop and
Chung's\cite{Bishop77} calculations. Bishop and Chung's results utilized a 356
term expansion in an elaborate correlated three particle basis set in elliptic
coordinates and we consider this result to be exact. Results are reported to
$1\times10^{-10}$ Hartree.

Energy convergence for our new basis was much better than we had expected.
This is evidenced by the rapid decrease in energy with the size of the
expansion as seen in table(\ref{H2pRes}). The high level of optimization is
indicated by the small values of the square of the gradient norms and virial
coefficients. An expansion with only 16 terms is in error by less than a
wavenumber, (3.36micro-Hartree = .737$cm^{-1}$). We attribute this to the high
level of optimization attainable using our analytic gradient formulas and the
importance of including the $r_{1}^{m_{k}}$ pre-multiplying factors.

The powers $m_{k}$ were determined by trail and error with a single basis
function calculation giving an energy minimum with $m_{k}=24$. This large
power for $r_{1}$ was not unexpected based on our experience with adiabatic
calculations of the vibrational levels of H$_{2}$\cite{Kinghorn97b}. For the
larger wave function expansions the $m_{k}$ were distributed between 10 and 30
with most values near 24.%

%TCIMACRO{\TeXButton{B}{\begin{table}[p] \centering}}%
%BeginExpansion
\begin{table}[p] \centering
%EndExpansion%
\begin{tabular}
[c]{llll}\hline\hline
N & energy/Hartree & $g^{\prime}g$ & $\left|  1-\eta\right|  $\\\hline
4 & -.5964107306 & $2.8\times10^{-14}$ & $3.9\times10^{-9}$\\
16 & -.5971357056 & $4.7\times10^{-15}$ & $1.3\times10^{-8}$\\
64 & -.5971390546 & $8.6\times10^{-13}$ & $7.7\times10^{-9}$\\
128 & -.5971390610 & $2.2\times10^{-13}$ & $1.7\times10^{-9}$\\
exact\cite{Bishop77} & -.5971390625 &  & \\\hline\hline
\end{tabular}
\caption{
H$_{2}^{+}%
$  non-adiabatic ground state energy calculations using from 4 to 128 basis functions.
$g^{\prime}%
g$, the squared norm of the gradient, is an indicator of optimization convergence and
$\left| 1-\eta\right
| $, the magnitude of one minus the virial coefficient, is an indicator of scale optimality
of the wave function.
\label{H2pRes}}%
%TCIMACRO{\TeXButton{E}{\end{table}}}%
%BeginExpansion
\end{table}%
%EndExpansion

\section{Conclusion}

We have presented formulas for integrals and derivatives of a new correlated
basis set suitable for high accuracy non-adiabatic variational energy
calculations of small diatomic molecules. These formulas together with
consideration of the non-adiabatic Hamiltonian and symmetry concerns provides
the foundation for implementation. We have carried out an implementation and
demonstrated the quality of results attainable by computing high accuracy
results for the benchmark system H$_{2}^{+}$. With our basic methodology
validated we will refine our procedures to include additional expectation
values and proceed with more extensive calculations on small diatomic systems.

The implementation of our energy code is general for N-particle diatomic
systems, but as stated earlier practical limits are imposed by the large
number of terms in the symmetry projection operators for systems with more
than a few electrons. With current computational resources we should be able
to compute energy levels for diatomic systems with as many as five electrons.
This offers many exciting possibilities since past non-adiabatic methods for
diatomics have been limited to 1 and 2 electron systems.

One of our successes is the high level of optimization that can be obtained
using our gradient formulas. However, regardless of the method used, the
optimization of large non-linear problems is formidable and often
unpredictable. For our trail H$_{2}^{+}$ calculations optimization proceeded
without notable difficulty. For this system it was fairly easy to generate
good starting points for the optimization by trial and error. Optimization
results for problems with large numbers of non-linear parameters can often
depend greatly on the initial guess for these parameters. Poor initial guesses
can lead to slow convergence or worse, numerical instability. A possibility
for generating high quality initial guesses is to generate optimization
parameters based on a superposition of the electronic wave function obtained
in a standard BO calculation using Gaussian basis functions and the
ro--vibrational wave function obtained from a variational calculation with a
basis of the form\cite{Kinghorn97b} $r^{m_{k}}e^{-a^{2}r^{2}}$. \ Initial
guesses are then generated from the collection of exponential parameters and
$m_{k}$ values.

To meet the demands for cpu time required for these large optimization
problems we are working on a parallel implementation of our programs. The data
structures and algorithms we use to implement our theoretical results are
readily implementable in a data parallel fashion allowing us to take advantage
of scalable parallelism using clusters of workstations and dedicated
computational nodes. We have worked out the technical details for building and
administering a ``Beowulf''\cite{beowulfwww} class parallel Linux cluster
composed of high end PC components. The combination of high performance/low
cost parallel hardware and data parallel algorithms will provide an excellent
computational environment for our non-adiabatic calculations.

To extend our methods beyond diatomics and to include more correlation in our
basis set we are working on integrals and derivatives for a more general basis
set, $\prod_{i<j}r_{ij}^{m_{kij}}\exp\left[  -\mathbf{r}^{\prime}(A_{k}\otimes
I_{3})\mathbf{r}\right]  .$ We expect the rate of convergence to be improved
by the inclusion of all $r_{ij}^{m}$ terms in the same way that they effect
convergence in the Hyllerass basis. Also, Varga and Suzuki\cite{Varga95} have
proposed a method for representing the angular dependence of the wave function
using a single solid harmonic whose argument contains additional variational
parameters, there appears to be several advantages in doing this and we are
investigating the possibility of using this approach to extend our basis set
beyond $J=0$ angular momentum states. The addition of general products of
powers of $r_{ij}$ factors with solid harmonics for higher angular momentum
states would give us a fully correlated basis set capable of representing
non-adiabatically all electronic, vibrational and rotational energy levels of
general N-body systems. The ``tricks'' learned form the present derivations
for the diatomic case represent an important step toward this goal.

%\begin{acknowledgement}
\vspace{.25in}
\noindent
\large{\bf{Acknowledgement}} \normalsize \\
This work was supported by the National Science Foundation.
%\end{acknowledgement}

\bibliographystyle{AIP}
\bibliography{4-PREFS,HMHE,MCALC,VibRot}
\end{document}
