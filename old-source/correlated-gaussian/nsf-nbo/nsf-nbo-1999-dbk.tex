\documentstyle[12pt,overcite]{article}
\setlength{\textwidth}{7in}
\setlength{\textheight}{9.35in}
\setlength{\oddsidemargin}{-.25in}
\setlength{\topmargin}{-.2in}


\begin{document}

\setlength{\baselineskip}{1em}

%\pagestyle{empty}

\noindent
{\bf PROJECT SUMMARY}

This proposal requests support for development and implementation
of methods allowing very accurate 
study of molecular systems without
assuming the Born--Oppenheimer (BO) approximation regarding the
separability of the nuclear and electronic motions. In this
approach we will employ explicitly  
correlated cartesian gaussian functions,
which very effectively described 
quantum states of systems consisting of 
interacting particles (nuclei and electrons).
The very high accuracy of the method to achieved
by employing the variational method and optimizing
both linear and non--linear parameters of the
wave function using analytical derivatives of the
the energy functional with respect these parameters. 
My group has been working in the area of 
non--Born--Oppenheimer (non--BO)
atomic and molecular quantum mechanics 
for several years now and, encouraged by the 
results obtained so far, we would like to propose a
project the development of the nBO methods
will be extended to calculations of excited states and
relativistic effects.
The methodology will be applied to molecular systems
with up to three nuclei and six electrons and
results will be generated, whose accuracy exceeds
the accuracy enabled by currently used techniques.
This enhanced precision of the calculations 
we will be able to address the disrepencies which
still persist between the theoretical and experimental data.
We will also generate new theoretical data
concerning properties and systems, 
which has not yet been experimentally investigated
with the precision used for some smaller systems like
H$_2^+$, H$_2$, and their isotopomers.
Based on our preliminary results we are confident
that the technique employing correlated gaussians
offers a breakthrough allowing extension of the
rigorous fully non--adiabatic approach
to systems with more than two
electrons. 
Such calculations have never been accomplished.

In 
stationary bound
quantum states of 
systems 
consisting of electrons and nuclei
deviations from the BO approximation
are rare and the non--BO effects
are rather small. 
In low--energy states only  
ro--vibrational level are excited.
In the first approximation  these levels can be
described as different states of the motion
of the nuclei whose interaction is described by the
isotropic ({\it i.e.} rotationally invariant)
potential energy hypersurface (PES).
PES can be calculated with the desired precision using
an electronic structure method.
Among the electronic structure methods which we
use are
the state--selective, multi--reference coupled cluster (SSMRCC)
method and the variational approach utilizing correlated gaussians.
Both methods have been developed in our group. 
and analytical derivatives of the variational functional with
respect to linear and non--linear parameters of the gaussian
expansion of the wave function.
Both methods have been developed in our group.
Unlike the Coulombic potential, which includes only two--body
components, the potential given by PES has N--body character and,
although in most cases it is dominated by two--body interactions,
the three- and higher--body contributions can be important.
In our non--BO calculations we will determine the
wave functions of the ro--vibrational states using
gaussian fits to PES, and we will use these functions
to construct first approximations to the non--BO wave functions.


The application of the non--BO 
methodology 
will focus on
small atomic and molecular systems.
We will start
hydrogen molecules, and their ions,
H$_2$, H$_3^+$, H$_3$, H$_3^-$, H$_4^+$, H$_4$, {\it etc.},
and their isotopomers. 
small dipole--bound anions, {\it e.g.}, LiH$^-$, LiD$^-$,
molecular systems involving 
helium and hydrogen atoms, {\it e.g.}, HeH$^+$, HeH$_2^+$, HeH$_2$,
lithium--hydrogen systems, LiH$_2$ and LiH$_2^-$,
and lithium dimer, Li$_2$.  
The aim of these calculations will be to determine 
electron affinities, electron detachment energies,
ionization potentials, excitation energies and transition probabilities 
for different ro--vibrational states.
Very accurate measurements, which have 
recently become
available 
for these systems
present an exciting challenge for 
the ``non--Born-Oppenheimer Molecular Quantum Mechanics."
Also calculations of 
higher excited states,
where the coupling of the electronic and nuclear motions
can be significant, 
will be performed and the nature and magnitude of the 
coupling will be analyzed

%dbk
It should be noted 
that we are in no way restricted to conventional systems.
Calculations involving ``exotic'' particles present no difficulty 
since in general our methodology can treat collections of particles
with any given masses and charges fully non--adiabatically.  
%dbk
 

Finally, it is also our goal to
provide through the proposed program, an inspiring environment 
for the training of graduate and undergraduate students as
future scientists. This project, which combines elements of 
quantum mechanics and spectroscopy has a potential to appeal
to students who are 
concerned with 
fundamental properties of molecular systems.
The proposed research will provide a unique view of molecules,
their structures and properties without resorting to
the clamped-nucleus model.


\newpage



\setcounter{page}{1}

\begin{center}
{\Large \bf 
Non--Born--Oppenheimer Calculations of
Molecular Systems with r$_{ij}$ Dependent Gaussians.}

\end{center}


\vspace{2mm}

\noindent
{\bf I. RESULTS FROM PRIOR NSF SUPPORT.}


\noindent
{\bf "Multi-reference State-selective Coupled-cluster Theory."
Grant CHE-9734821, \$210,000, 5/5/98-11/4/01.} 

\vspace{3mm}

As of 9/10/99, the research funded by the above 
grant has resulted in 66 publications
\cite{A1,A2,A3,A4,A5,A6,A7,A8,%
A9,A10,A11,A12,A13,A14,A15,%
A16,A17,A18,%
A19,A20,A21,A22,A23,A24,A25,%
A26,A27,A28,A29,A30,A31,A32,A33,A34,%
A35,A36,A37,%
A38,A381,A382,%
A383,A384,A385,A3851,%
A3860,%
A3861,A3862,A3863,A3864,A3865,%
A3866,A3867,A3868,A3869,A3870,A3871,%
A3872,A3873,A3874,A3875,%
A3876,A3877,A3878,A3879,A3880,A3881}  
62 of which
have been already published or accepted for 
publication and four have been submitted.
The following research
results have been obtained with the support 
of the funding provided by NSF:
The state--selective, multi--reference coupled-cluster  
(SSMRCC) theory has been formulated based
on the doubly--exponential and semi--linear forms 
of the wave function.
The reference functions in this approach 
is either represented as 
an exponentiated CC operator  
acting on the formal reference determinant (Fermi vacuum)
or as a CASSCF wave function.
The excitations to the non--active orbital space 
are represented by an exponential CC
operator,
$|\Psi\rangle  = e^{T^{\rm (ext)}} | \Phi^{\rm (int)}
\rangle$, where $\Phi^{\rm (int)}$
is either 
$\Phi^{\rm (int)} = e^{T^{\rm (int)}} |0\rangle$ 
or 
$\Phi^{\rm (int)} = C^{\rm (int)} |0\rangle$ and
$C^{\rm (int)} |0\rangle$ is the CASSCF wave function.
This method,
which is geared towards very accurate calculations of
the potential energy surfaces for the ground and excited 
states of molecular systems,
has been computationally implemented at 
the SSMRCCSD(T) and SSMRCCSD(TQ) levels. 
  

\vspace{4mm}

\noindent
{\bf II.  CONTRIBUTION TO DEVELOPMENT OF HUMAN RESOURCES.}

There have been a number of collaborators who 
have carried out the work described here. Their
contributions to the above--cited accomplishments have 
been immense and their involvement contributed to their
growth as independent scientists. The collective 
effort involving formulation of the theory, computational
implementation and application work should be credited 
to my five research associates, ten graduate students,
eight visiting scientists and three undergraduate students.  
It is due only to the devotion of these researchers that
we have been able to create a stimulating research environment 
conducive to teamwork and collaboration.

\vspace{2mm}

\noindent
{\bf Research Associates:}

Dr. {\em Piotr Piecuch}, recently appointed as an 
assistant professor at the Michigan State University, joined us for
one and a half year and worked on theoretical formulation and 
computational implementation of the fully--exponential
SSMRCC scheme and on developing improved 
methods for converging the CC amplitude equations;  
Dr. {\em Keya Ghose}, 
after completing her Ph.D. degree with Prof.S.Pal in Puna, India,
stayed for a year with my group and 
worked on implementation of the recursively calculated
intermediates and on the property calculations;  
Dr. {\em  Zdenek Slanina}, 
now a professor at the Laboratory
of Molecular-Design Engineering,
Toyohashi University of Technology, Japan,
worked for a year in our group on carbon clusters, fullerenes and
related projects;  
Dr. {\em Genady Gutsev}, 
currently
at Jackson State University, 
spent two years in our group (first year of his stay was
funded by a grant from the CAST program) and did 
calculations on molecular and cluster anions; 
and Dr. {\em Don Kinghorn}, 
%dbk 
has worked for two and a half years in my group
(interrupted by
a short stay at the University of Arkansas with Prof. P. Pulay) 
on implementations of explicitly correlated gaussian functions
to non--BO atomic and molecular calculations and 
to the general N--body vibrational problem and has just completed
work on the key integral and gradient formulas for the work 
presented here.
%dbk

\vspace{2mm}

\noindent
{\bf Graduate Students:}

\noindent
Dr. {\em Pawel Kozlowski}, currently an assistant professor
in the Department of Chemistry, 
University of Louisville, Kentucky,
made a significant contribution to the conceptual
framework of the SSMRCC theory and its implementation;
developed the non--BO approach
to multi--body systems and implemented explicitly
correlated gaussian functions to perform application
calculations for those systems;
Dr. {\em Nevin Oliphant}, currently 
with a computer company serving the Wall Street Stock Exchange,
made a significant contribution to the conceptual 
framework of the SSMRCC theory and its implementation;
developed the non--BO approach 
to multi--body systems and implemented explicitly
correlated gaussian functions to perform application 
calculations for those systems; 
Mr. {\em Z. (John) Zhang}
continued Pawel's work on correlated gaussians and 
implemented analytical derivatives for optimization of the
variational wave functions small atoms;
Mr. {\em Yasser Elkadi}
has been working on 
dipole--bound anionic states of molecules and cluster
systems; 
Dr. {\em William McCarthy}, 
who graduated 
with a Ph.D. in September 1996 and moved to the Utah
State University to join Prof. S. Aust's group,  
developed methodology for describing
large--amplitude vibrations in larger molecular systems; 
Mr. {\em Vadim Alexandrov} has been working on
development of a wave--packet time--propagation method which will
be applied to study intra- and inter-molecular proton transfer
reactions; 
Mr. {\em Doug Gilmore} before moving to Law School at Concorde,
New Hampshire,
developed a technique for
optimizing multi--center correlated gaussians
based on analytical gradients; 
Ms. {\em Dayle Smith} 
has studied 
intra-molecular hydrogen--bonding
systems involving DNA bases and electron attachment to these
systems;   
%dbk
Mr. {\em Maurice Cafiero} 
and Mr. {\em Ismail Al-Jihad} are two talented and 
enthusiastic new graduate students who
joining my group and are starting their research. 
%dbk

\vspace{2mm}
\noindent
{\bf Undergraduate Students:}

\noindent
Mr. {\em Nathan Oyler}, 
currently a graduate student at the Department of
Chemistry, University of Washington, investigated 
dipole--bound anionic states of nucleic acid bases; and 
%dbk
Mr. {\em Chris Tarsitano} currently a graduate student at the 
University of Chicago 
worked in  
on three--body vibrational calculation using the methodology
developed in our group.
%dbk

\vspace{2mm}
\noindent
{\bf Visiting scientists:}

\noindent
Prof. {\em Jong Lee} from Chonbuk National University,
South Korea, spent his sabbatical year in my group 
working on electronic excitations of chain carbon clusters; 
Dr. {\em Johan Smets}, currently with Proctor and Gamble
corporation in Brussels,
visited us twice for three--month periods (his stay
was supported by a grant from NATO) and worked 
on assignment of IR spectra of water complexes of nucleic
acid bases based on the theoretically predicted 
frequencies.  
Dr. {\em Stepan Stepanian} visited us twice
for  six months  
in '97 on a COBASE fellowship 
and in '98 using  NATO funding and calculated and
assigned experimental IR matrix--isolation 
spectra of pyrimidine--quinone and quinone--quinone dimers
and the spectra of some aminoacids and small peptides.
In '97 we had a month--long visit in our group by Ms.
{\em Kristien Schoone} and Ms. {\em Riet Ramaekers} 
from the University of Leuven
who worked with us on the NATO project, and a three--month visit
of Ms. {\em Marta Fores} from University of Girona who worked
on calculations concerning excited-state proton transfer
reactions.
In '98 Dr. Vladimir Ivanov from Kharkov, Ukraine, spent six
months in my group and worked on the MRCC theory.

\vspace{4mm}
\noindent
{\bf III.  PROJECT DESCRIPTION}

\vspace{2mm}
\noindent
{\bf A. INTRODUCTION}

\vspace{2mm}
\noindent
{\bf A.1 Equivalent Treatment of Nuclei and Electrons}



Soon after the Schr\"{o}dinger equation was introduced in 1926
several works appeared dealing with the fundamental problem of the
nuclear motion in molecules 
Very soon after the relativistic equations have been introduced
for one and two--electron systems.
The experiments
on the Lamb shift stimulated derivation of the expersion
for fine energy corrections related to mass--velocity effects,
radiative effects, etc.. \cite{k8}
Following this development questions were raised, whether an
{\it ab initio} approach, in which only the values of
fundamental constants are taken from experiments, is capable
of reproducing experimental results with the precision
which matches that of the experimental techniques. At first
the questions concerned
the accuracy of the calculations, but soon they were extended
to testing the model of the molecular electron structure
provided by the non--relativistic and relativistic quantum mechanics.


In order to answer these questions accurate experimental
and theoretical results were needed for representative
molecular systems. Theoreticians, for obvious reasons, have
favored very simple systems, such as the hydrogen molecular
anion for their calculations. However, with only one electron,
this system did not provide a proper test case for the molecular
quantum mechanical methods due to the absence of the
electron correlation. Therefore the two--electron hydrogen
molecule has served as the system on which the fundamental
laws of quantum mechanics have been first tested.
The theoretical and computational work described  in
this proposal will allow us to extend the set of very
accurately calculated molecules using non--adiabatic
and relativistic approach to systems with six electrons
and three nuclei. Thus new ground will be opened to
the fruitful competition and mutual stimulation of
theory and experiment.



In this proposal we discuss an approach to atomic and 
molecular quantum--mechanical calculations without assuming 
the clamped--nuclei approximation. Although such calculations
are still rare, the progress
in conceptual formulation of the theory in this area and, more
importantly, in development of necessary computational tools
has progressed to the point where non--Born--Oppenheimer
calculations may soon become possible for more extended molecular
systems. When this happens, 
the problems of molecular spectroscopy
which are not adequately described 
by the concept of the Potential Energy Surface (PES)
will open to theoretical investigations.
It may also provide a new insight in such important notions of
chemistry as Chemical Bonding and Molecular Structure.

In an attempt to make the quantum mechanical calculations on 
molecular systems practical and to provide a more intuitive
interpretation of the computed results, it has long been a quest
in the electronic structure theory of molecules 
to establish a solid base for separating
the motion of light electrons from the motion of heavier nuclei.
It is believed that the original work of Born and Oppenheimer (BO)
\cite{BO1927} initiated the discussion by the analysis
of the diatomic case. Further works of Combes and Seiler
\cite{CS1980}, who managed with the use of singular perturbation
theory to resolve the problem of the diverging series which appeared
in the BO expansion, and particularly of Klein and coworkers
\cite{KM1992} who extended the formalism to polyatomic systems,
have brought the consideration of the topic to a level of commonly
accepted theory.

Apart from the further refinements of 
the BO approach
there has been a continuing
interest in theoretically approaching molecular systems 
with a method which treats
the motions of both nuclei and electrons equivalently.
We propose to further advance this non--adiabatic methodology 
in which treatment of the nuclear and electronic motions
in molecules departs entirely from the PES concept. 
It is particularly
interesting how in this type of approach the conventional concepts
of molecular structure and chemical bonding will be represented
and how different these representations will be 
from the representations
developed based on the BO approximation. In particular the concept
of chemical bonding, which at the BO level is an electronic 
phenomenon, will now be described as an effect derived from
collective dynamical behavior of both electrons and nuclei.

Another motivation for developing the non--adiabatic approach to 
describe the states of molecules stems from the realization that
in order to reach ``spectroscopic" accuracy in quantum--mechanical
calculations ({\it i.e.}, error less than 1 $\mu$hartree), one needs
to account for the the coupling between
motions of electrons and nuclei. Modern experimental techniques,
such as gas--phase ion--beam spectroscopy, reach accuracy on the
order of 0.001~cm$^{-1}$ (5 nano-hartree).\cite{C} 
In order for quantum molecular mechanics
to continue providing assistance in resolving and assigning
experimental spectra, especially in studies of reaction dynamics,
work has to continue on development of more refined theoretical 
methodology, which accounts for non--adiabatic interactions.

\vspace{3mm}
\noindent
{\bf A.2 Preliminary Non--Adiabatic Results 
Obtained by our Research Group}
\vspace{2mm}

\vspace{2mm}
{\bf A.2.1 H$^-$/H, D$^-$/D, T$^-$/T}

Our recent calculations of electron affinities (EA) of hydrogen, deuterium
and tritium atoms \cite{A37} can serve as example of 
problems where very accurate experiments challenge the ability
of accurate non--adiabatic theoretical calculations 
to deliver high quality results. 
We were able to provide the needed accuracy.
Our calculated values of EAs for H and D of 6083.0983~cm$^{-1}$
and 6086.7126~cm$^{-1}$, respectively, are in excellent
agreement with the experimental results of Lineberger and
coworkers of 6082.99$\pm$0.15~cm$^{-1}$ and 6086.2$\pm$0.6~cm$^{-1}$,
respectively. Our tritium EA of 6087.9169~cm$^{-1}$ has not yet 
been experimentally verified. 


\vspace{2mm}
{\bf A.2.2 Li$^-$/Li System and its Isotopomers} 


Similar calculations as for the H$^-$/H isotopomer series
are currently being performed for  
for the Lithium isotopes.
%dbk
Our preliminary non--adiabatic 
Li$^-$ calculations 
with 384--term wave function
have given new energy upper bound equal to
-7.5007396158~hartree (this energy was calculated
with infinite mass of the lithium nucleus), which is
over 210~micro--hartree lower in energy than that
given by Chung and Fullbright.\cite{Chung92} 
For the $^7$L$^-$ isotop the with the 384--term wave function
we get the energy equal to -7.500128967~hartree and this value 
is the first fully non--adiabatic result on this system we 
are aware of. (These are not final results.)
%dbk


\vspace{2mm}
{\bf A.2.3 H$_2$ and its Isotopomers}

In one of the recent issues of {\em Phys.Rev.Lett.}\cite{ad1} 
we reported calculations which produce new non--adiabatic
variational upper bound for the ground state energy of
the H$_2$ molecule (-1.1640250232~hartree). This result
was obtained with a 512--term wave function and is the highest
accuracy result ever reported for this fundamental molecular system.
We have also calculated a new non--adiabatic upper bound
for D$_2$ equal to  -1.167168803~hartree.
For HD with a 512--term expansion we obtained -1.165471893~hartree.
These results affirm that the our correlated gaussian
basis set with squared inter--particle distances included in the
gaussian exponents and with powers of these distances included
as pre--exponential multipliers can produce results 
which supersede those obtained before provided that 
non--linear besis set parameters are fully optimized.

 
\vspace{2mm}
{\bf A.2.3 LiH}

The main advantage of our approach is that it is not
limited to two electron homonuclear diatomics, as are the
methods used by others. For example, using a modest--size
wave function of 64 gaussians we have recently performed
a non--adiabatic calculation on LiH (four--electron system).
which resulted in the non--adiabatic energy for the 
ground state of -8.057934~hartree.
The fast rate of the convergence of the calculation
indicates that a basis set of around 1000 basis functions
should be enough to produce the non--adaibatic energy
accurate to a few nano-hartree. 


%\vspace{3mm}
%\noindent
%{\bf A.3 General Non--Adiabatic Hamiltonian
%\vspace{2mm}


%To model the physical systems, {\it i.e.}, 
%write the Hamiltonian, particles are
%considered to be non--relativistic, point masses,
%$M_i$, interacting under
%an isotropic potential. The total Hamiltonian then has the familiar form, 
%\begin{equation}
%H_{tot}=-\sum_i^N\frac{\nabla _{{\bf R}_i}^2}{2M_i}
%+ V\left(R_{ij}, R_{ijk}, R_{ijkl}, \ldots%
%\,;\,\,\,i<j<k<l<\ldots ,\,\,\,\,i=1...N\right).
%\label{ham1}
%\end{equation}
%In the above Hamiltonian the potential can include two--particle
%interactions, as well as effective interactions 
%of three or more particles. 
%In the case of the Coulomb potential, only pair interactions
%will remain in the Hamiltonian. In ro--vibrational calculations,
%the interaction potential between atoms, which is given by the
%potential energy surface (PES) 
%may require three- and higher multi--body
%components. However, apart from the form of the potential, which
%can have different forms in different problems,
%the general approach to solving the Schr\"{o}dinger
%equation remains the same since the isortopic character
%of the potential leads to a similar approach
%is separating the internal motion from the motion
%of the center of mass of the system.
%In result, the technique developed here, although formally
%developed for nucleus--electron case with pair Coulomb
%potentials 
%can be extended 
%to wider spectrum of molecular problems.
%The first step in this direction has been described in our recent papers
%\cite{A382,A383}.



\vspace{3mm}
\noindent
{\bf B. Work Proposed}

{\bf B.1 Explicitly Correlated Gaussian Functions}

In the present work we propose to employ explicitly correlated
gaussian functions in variational calculations of 
non--adiabatic quantum states of systems 
consisting of nuclei and electrons.
The application of correlated gaussian functions in
molecular calculations has gained momentum
following some impressively accurate 
works of Jeziorski, Szalewicz and
their coworkers on small molecular systems performed
with the use of these functions.
\cite{sz1}
The proposed procedure is derived from
the non--adiabatic approach, which we have pursued in our group
for the last few years, and based on several past applications
of correlated gaussians in Born--Oppenheimer (BO) 
and non--BO calculations on atomic
and molecular systems.
\cite{A2,A3,A6,A8,%
A14,%
A16,%
A20,A21,%
A27,%
A37,%
A38,A382,A383,%
kozlowski91,kozlowski92a,kozlowski92b,kozlowski92c}.
The procedure is also derived from the works 
of Poshusta and Kinghorn
concerning non--BO calculations
%There have been several highly accurate non-adiabatic variational
%calculations on atomic and exotic few particle systems using simple
%correlated gaussians\cite
\cite{Poshusta83,Kinghorn93,Kinghorn95b}
%,Kinghorn96a}.
Dr. Don Kinghorn was a research associate 
in my group in 1995--1997 and a yaer ago he 
rejoined us to continue his work on correlated
gaussians. He has completed derivations of key integral and
gradient formulas for the basis set presented below and has
writen high quality code including full gradient optimizations 
for non--adiabatic diatomic and atomic calculations. 

The centerpiece of the 
proposed methodology is the use of explicitly 
correlated gaussian basis functions of the
general form, 
\begin{equation}
\phi _k=\prod_{i<j}r_{ij}^{m_{kij}}\exp \left[ -{\bf r}^{\prime
}(A_k\otimes I_3){\bf r}\right] .  
\label{basisfcn}
\end{equation}
These functions, hereafter referred to simply as ``$\phi_k$'', 
with addition of rotational components
and appropriate symmetry projections, will be used as variational basis
functions for w.f.'s describing internal non--adiabatic
states of multi-particle systems. In $\phi_k$ 
the term $\prod_{i<j}r_{ij}^{m_{kij}}$ is a product of ``distance''
coordinates, $r_{ij},$ raised to powers $m_{kij}$ (positive, negative
\cite{kpc}
or zero).
The exponential component is an explicitly correlated gaussian with 
${\bf r}$ representing a length $3n$ column vector of internal (relative)
coordinates (${\bf r}^{\prime }$ denotes the transpose of ${\bf r}$,
{\it i.e.}, a row vector), $A_k$ is an 
$N\times N$ symmetric matrix of exponent
parameters (positive definite). The Kronecker product of
$A_k$
with the $3\times 3$ identity matrix, $I_3$, insures rotational invariance
as will be shown in a later section. The matrix/vector form of $\phi _k$
allows us to exploit the powerful 
matrix differential calculus, described by
Kinghorn\cite{Kinghorn95a}, for deriving 
elegant and easily implementable
mathematical forms for integrals and 
derivatives required in variational
calculations.
Note that 
the $\phi _k$ are angular momentum eigen-functions
with eigen--value $J=0$; thus, 
modification of the basis to $Y_M^L\phi _k$
will yield higher angular momentum 
eigen--states when the factor $Y_M^L$ is
chosen as an angular momentum 
eigen--function for the desired state.
This is a very important property 
allowing separation of variational calculations 
for vibrational--electronic states 
corresponding to different
values of the total angular 
momentum operator (different rotational states).
Representation of higher angular 
eigen--states will be discussed in 
one of the following sections.



\vspace{2mm}
\noindent
{\bf B.2 Coordinates and the Hamiltonian}

One can express the non-relativistic Hamiltonian
in terms of the coordinates of particles and
their relative distances as: 
\begin{equation}
H_{tot}=-\sum_i^N\frac{\nabla _{{\bf R}_i}^2}{2M_i}+V\left( \left\| 
{\bf R}_i-{\bf R}_j\right\| \,;\,\,\,i<j,\,\,\,\,i=1...N\right) .
\label{ham2}
\end{equation}
The particles are numbered from $1$ to $N$ with $M_i$ the mass of particle 
$i$, ${\bf R}_i=[X_i\,\,Y_i\,\,Z_i]^{\prime }$ a column vector of cartesian
coordinates for particle $i$ in the external, laboratory fixed, frame, $%
\nabla _{{\bf R}_i}^2$ the Laplacian in the coordinates 
of ${\bf R}_i$, and $\left\| {\bf R}_i-{\bf R}_j\right\| $ 
the distance between
particles $i$ and $j$. The total Hamiltonian (\ref{ham2}) is, of
course, separable into an operator describing the translational motion of
the center--of--mass and an operator describing the internal energy. This
separation is realized by a transformation to 
the center--of--mass and internal
(relative) coordinates.
Let ${\bf R}$ be the vector of particle coordinates in the laboratory
fixed reference frame:
\begin{equation}
{\bf R=}\left[ 
\begin{array}{c}
{\bf R}_1 \\ 
{\bf R}_2 \\ 
\vdots  \\ 
{\bf R}_N
\end{array}
\right] =\left[ 
\begin{array}{c}
X_1 \\ 
Y_1 \\ 
Z_1 \\ 
\vdots  \\ 
Z_N
\end{array}
\right] 
\end{equation}
Center--of--mass and internal coordinates are given by the transformation $T:%
{\bf R}\mapsto [{\bf r}_0^{\prime },{\bf r}^{\prime }]^{\prime }$,
\begin{equation}
T=\left[ 
\begin{array}{ccccc}
\frac{M_1}{m_0} & \frac{M_2}{m_0} & \frac{M_3}{m_0} & \cdots  & \frac{M_N}{%
m_0} \\ 
-1 & 1 & 0 & \cdots  & 0 \\ 
-1 & 0 & 1 & \cdots  & 0 \\ 
\vdots  & \vdots  & \vdots  & \ddots  & \vdots  \\ 
-1 & 0 & 0 & \cdots  & 1
\end{array}
\right] \otimes I_3  \label{Ttran}
\end{equation}
where $m_0=\sum_i^NM_i$. ${\bf r}_0$ is the vector of coordinates for the
center--of--mass and ${\bf r}$ is a length $3n=3\left( N-1\right) $ vector
of internal coordinates with respect to a reference frame with origin at
particle 1 (this particle is usually the heaviest particle in the system):
\begin{equation}
{\bf r=}\left[ 
\begin{array}{c}
{\bf r}_1 \\ 
{\bf r}_2 \\ 
\vdots  \\ 
{\bf r}_n
\end{array}
\right] =\left[ 
\begin{array}{c}
{\bf R}_2-{\bf R}_1 \\ 
{\bf R}_3-{\bf R}_1 \\ 
\vdots  \\ 
{\bf R}_N-{\bf R}_1
\end{array}
\right] .  \label{rdef}
\end{equation}
Using this coordinate transformation, and the conjugate momentum
transformation, the internal Hamiltonian for the problem 
we are considering
can be written as \cite{Kinghorn93,Kinghorn95b}: 
\begin{equation}
H=-\frac 12\left( \sum_i^n\frac 1{\mu _i}\nabla _i^2+\sum_{i\neq j}^n\frac
1{M_1}\nabla _i\cdot \nabla _j\right) +V\left(
r_{ij};\,\,\,i<j,\,\,\,\,\,i=0...n\right)   \label{intham1}
\end{equation}
where the $\mu _i$ are reduced masses, $M_1$ is the mass of particle 1 (the
coordinate reference particle), and $\nabla _i$ is the gradient with respect
to the $x,y,z$ coordinates ${\bf r}_i$. The potential energy is still a
function of the distance between particles but is now written using internal
distance coordinates, $r_{ij}=\left\| {\bf r}_i-{\bf r}_j\right\| =$ $%
\left\| {\bf R}_{i+1}-{\bf R}_{j+1}\right\| \,\,$with\thinspace $%
r_{0j}\equiv r_j=\left\| {\bf r}_j\right\| =\left\| {\bf R}_{j+1}-%
{\bf R}_1\right\| .$ The kinetic energy term in this Hamiltonian can be
written as a quadratic form in the length $3n$ vector gradient operator, $%
\nabla _{{\bf r}},$ the gradient with respect to the length $3n$ vector $%
{\bf r}$ of internal coordinates. This gives a compact matrix/vector form
of the Hamiltonian with the kinetic energy expressed as a quadratic form in
the gradient operator, 
\begin{equation}
H=-\nabla _{{\bf r}}^{\prime }\left( M\otimes I_3\right) \nabla _{{\bf%
r}}+V\left( r_{ij};\,\,\,i<j\,,\,\,\,\,i=0...n\right)   \label{ham}
\end{equation}
$M$ is an $n\times n$ matrix with $1/2\mu _i$ on the diagonal and $1/2M_1$
for off--diagonal elements. This is the Hamiltonian we 
will use in variational non--adiabatic calculations.



\vspace{2mm}
\noindent
{\bf B.3 Relativistic Corrections}



In the conventional approach to account for relativistic 
effects the 
fixed--nucleus approximation has been assumed.
The most streighforward way to account for the 
electronic relativisitic
effects in the calculations is by using the perturbation theory.
To calculate the lowest--order relativistic corrections to the
energy one usually starts with the Breit equation,\cite{k15}
and in the Pauli approximation one gets contributions to the 
relativistic electronic hamiltonian ilustrated below using the
corrections for H{_2}:
\begin{equation}
H^{\prime} = \sum_{j=1}^{5} H_j,
\end{equation}
where
\begin{equation}
H_1 = -(\alpha^2/8)(\Delta_1^2 + \Delta_2^2)
\end{equation}
is the relativistic correction due to the variation of mass with velocity,
\begin{equation} 
H_2 = (\alpha^2/2) r_{12} | [\Delta_1 \Delta_2 +
r_{12}^{-2} \vec{r_{ij}} \cdot (\vec{r_{12}} \cdot \Delta_1)\Delta_2]
\end{equation}
is the relativistic correction to the interaction
between the electrons (retardation),
\begin{equation}
H_3 = -(i \alpha^2/2) \{ [ \vec{F_1} \times \Delta_1 +
2 r_{12}^{-3}  \vec{r_{12}} \times \Delta_2 ] \cdot \vec{s_1}
+ [\vec{F_2} \times \Delta_2 + 2 r_{12}^{-3} \vec{r_{21}}
\times \Delta_1] \times \vec{s_2} \}
\end{equation}
represents the spin--orbit coupling which venishes for $\Sigma$
states, and $\vec{F_i} = -\Delta_i V$,
\begin{equation}
H_4 = (\alpha / 4) (\Delta_1 \vec{F_1} + \Delta_2 \vec{F_2})
\end{equation}
is a term  characteristic of the Dirac tehory,
\begin{equation}
H_5 = \alpha^2 \{ -(8\pi/3)(\vec{s_1} \cdot \vec{s_2})
\delta^{(3)}(\vec{r_{12}}) +
r_{12}^{-3} [\vec{s_1} \cdot \vec{s_2} -
3 r_{12}^{-2} (\vec{s_1} \cdot \vec{r_{12}}^{-2})
(\vec{s_1} \cdot \vec{r_{12}} ) ] \}
\end{equation}
represents the spin--orbit interaction.

The relativistic contribution to the energy is obtained as the 
expectation value of $H_{rel}$ calculated with a non--relativistic
wave function. All the above corrections are of the order of
$\alpha^2$, where $\alpha$ is the fine structure constant.
The esimated value for the relativistic contribution for H_{2} is 
equal to -2.398~cm$^{-1}$ for the equilibrium distance,\cite{kk} and
its contribution to the dissociation energy amounts to
-0.524~cm$^{-1}$.

From quantum electrodynamics high--order radiative corrections can 
be derived and for a two--electron system one gets:\cite{k15}
\begin{equation}
\Delta E_{rad} = (\frac{3}{6})
\alpha^3 Z \overline{\delta^{(3)} (\vec{r_1})}
[{\rm log}(mc^2/K_0) + (\frac{19}{30} - {\rm log}2]
+ (\frac{28}{3} \alpha^3 \overline{\delta^{(3)}(\vec{r_{12}})}
{\rm log} \alpha,
\end{equation}
where log$K_0$ is defined by:
\begin{equation}
[\sum_{n} \vec{p_{0n}} \cdot \vec{p_{n0}}(E_n - E_0)]
{\rm log} (K_0 / Z^2) =
\sum_n \vec{p_{0n}} \cdot \vec{p_{n0}} 
(E_n - E_0) {\rm log} |(E_n - E_0)/Z^2|
\end{equation}
and is known as the Bethe logarithm. 
The estimated value of the radiative corrections for H_{2} at  the 
equilibrium internuclear separation is 0.718~cm$^{-1}$,
and they contribute to the dissociation energy -0.174~cm$^{-1}$.
\cite{kk}

In the present approach in calculating relativistic 
corrections we can lift the fixed--nucleus approximation
and consider the relativistic affect using
the fully non--adiabatic wave function. 
In the first approximation only the electron 
relativistic effects can be considered, but our
approach also allows for including electron-nucleus
contributions and purely nuclear contributions.
The contribution to the relativistic effects from
the coupling of the nuclear--electron motion has not been
considered in the past is spite of the fact that the
presence of velocity--dependent terms
precludes the rigorous separation of electronic
and nuclear motion (even in the absence of external field).
This is due to the dependence of the relativistic
effects on higher powers of $\frac{v}{c}$ - though
this ratio is much smaller for nuclei than for
electrons. 



\vspace{2mm}
\noindent
{\bf B.4 Variational Wave Function}


Our expansion for the wave function, $\prod_{i<j}r_{ij}^{m_{kij}}\exp
\left[ -{\bf r}^{\prime }(A_k\otimes I_3){\bf r}\right] $, is written
using the scalar ``distance'' variables $\left\{ r_{ij}\right\} $ and the
internal coordinate vector variable ${\bf r.}$ These two sets of
variables both completely describe the ``geometry'' of a system. To make
this equivalence more obvious and to present alternative forms for the $\phi
_k$, we will show the interchangeability of these two sets of variables. $%
r_{ij}^m$ can be written as a function of ${\bf r}$ using the matrix $%
\left( J_{ij}\otimes I_3\right) $ with $J_{ij}$ defined as an $n\times n$
matrix with 1's in the $ii$ and $jj$ positions, -1 in the $ij$ and $ji$
positions and 0's elsewhere\cite{Poshusta83,Kinghorn95a}, 
\begin{equation}
r_{ij}^m=\left[ {\bf r}^{\prime }(J_{ij}\otimes I_3){\bf r}\right]
^{m/2}.  \label{rijJ}
\end{equation}
$r_{ij}^m$ can, equivalently, be written using the component vectors, $%
{\bf r}_i$, of ${\bf r}$ 
\begin{equation}
r_{ij}^m = \left[ {\bf r}_i^{\prime }{\bf r}_i+{\bf r}_j^{\prime }%
{\bf r}_j-2{\bf r}_i^{\prime }{\bf r}_j\right] ^{m/2} 
= \left[ r_i^2+r_j^2-2{\bf r}_i^{\prime }{\bf r}_j\right] ^{m/2}.
\end{equation}
Thus, using eqn(\ref{rijJ}), $\phi _k$ can be written purely in terms of the
vector variable ${\bf r}$, 
\begin{equation}
\phi _k=\prod_{i<j}\left[ {\bf r}^{\prime }(J_{ij}\otimes I_3){\bf r}%
\right] ^{\frac{m_{kij}}2}\exp \left[ -{\bf r}^{\prime }(A_k\otimes I_3)%
{\bf r}\right] .  \label{phir}
\end{equation}

Alternatively, $\phi _k$ can be expressed using the distance coordinates $%
\left\{ r_{ij}\right\} $. The quadratic form in the exponential of $\phi _k$
may be converted to $\left\{ r_{ij}\right\} $ variables as follows (we drop
the subscript $k$ for convenience):
\begin{equation}
{\bf r}^{\prime }(A\otimes I_3){\bf r} = \sum_{i,j}{\bf r}%
_i^{\prime }{\bf r}_j\,\,A_{ij} 
= {\sf tr}\left[ \left( {\bf r}_i^{\prime }{\bf r}_j\right) A\right] 
= {\sf tr}\left[ \left( r_{ij}^2\right) B\right]  
= \sum_{i,j}r_{ij}^2\,\,B_{ij}
\end{equation}
where ${\sf tr}\left[ \:\: {}\right] $ 
is the matrix trace operator, $\left( 
{\bf r}_i^{\prime }{\bf r}_j\right) $ is the $n\times n$ matrix of dot
products of the component vectors of ${\bf r},$ $\left( r_{ij}^2\right) $
is the $n\times n$ matrix of squared distance variables, and $B$ is a matrix
with elements given, in terms of the elements of an arbitrary matrix $A$, by
the transformation, 
\begin{equation}
B_{ij}=\left\{ 
\begin{array}{ll}
\frac 12\sum_{k=1}^{n}\left( A_{ik}+A_{kj}\right) , & i=j \\ 
-\frac 14\left( A_{ij}+A_{ji}\right) , & i\neq j
\end{array}
\right. .  \label{Btran}
\end{equation}
Hence, the $\phi _k$ can be written using only distance coordinates, 
\begin{equation}
\phi _k=\prod_{i<j}r_{ij}^{m_{kij}}\exp \left[ -{\sf tr}\left[ \left(
r_{ij}^2\right) B_k\right] \right] .
\end{equation}
If one has $\phi _k$ in terms of $B_k$, {\it i.e.} 
a function of $r_{ij},$ and
wishes to transform to $A_k,$ a function of ${\bf r}$, the following
relation can be used, 
\begin{equation}
A_{ij}=\left\{ 
\begin{array}{ll}
B_{ii}+\sum_{k\neq i,j}^{n}\left( B_{ik}+B_{kj}\right) , & i=j \\ 
-\left( B_{ij}+B_{ji}\right) , & i\neq j
\end{array}
\right. .  \label{Atran}
\end{equation}
If $\phi _k$ is to be square integrable, 
then $A_k$ must be positive definite. The simplest way to insure this
condition is to write $A_k$ in Cholesky factored form, 
\begin{equation}
A_k=L_kL_k^{\prime }, \; \; \; \;
L_k
{\rm \; \; lower  \; triangular \; and \;
rank \;}n.
\end{equation}

Let us
consider the suitability of $\phi_k$ as an expansion function for
the variational wave functions for the non--adiabatic calculations. 
One can make the following observations:

\begin{enumerate}


\item  
If $\phi_k$'s are to describe the $J=0$ state they
should be invariant to rotations 
of the system.
Using the vector form, 
eqn(\ref{phir}), it is easy to show that $\phi _k$ is
invariant under any orthogonal transformation
({\it e.g.} rotation).
Let $U$ be any $3\times 3$ 
orthogonal matrix (any proper or improper
rotation in 3 space) then the action 
of $U$ on $\phi_k$ is to transform the
quadratic forms in the pre--multiplying 
factors and exponential factor as
(using the exponential factor as an example): 
\begin{eqnarray}
\left( \left( I_n\otimes U\right) {\bf r}\right) ^{\prime }(A_k\otimes
I_3)\left( I_n\otimes U\right) {\bf r} &=&{\bf r}^{\prime }\left(
I_n\otimes U^{\prime }\right) (A_k\otimes I_3)\left( I_n\otimes U\right) 
{\bf r} \\
&=&{\bf r}^{\prime }(A_k\otimes U^{\prime }U){\bf r} 
= {\bf r}^{\prime }(A_k\otimes I_3){\bf r}
\end{eqnarray}
leaving $\phi_k$ invariant. Hence, any expansion in $\phi_k$ will be
isotropic in ${\cal R}^3$.




\item  
The function should be differentiable.
The $\phi_k$ are infinitely 
differentiable. Therefore, any wave function 
represented as linear combinations 
of $\phi_k$ are also infinitely
differentiable and expressible in terms 
of the derivatives of the $\phi_k$.
This property is extremely useful for
variational optimization of the wave function,
which can be greatly accelerated, if 
the energy functional used in the optimization
can be directly differentiated with respect
to the non--linear parameters of the basis 
functions.
Easily implementable formulas for energy  gradient 
components can be derived
using matrix differential 
calculus\cite{Kinghorn95a}.
For example, using the normalized overlap formula, for the diatomic
implementation of our basis functions, the derivative with
respect to the elements of the lower triangular matrix $L_{k}$, 
(the Cholesky factors of the exponent matrix) is,
\begin{align}
\frac{\partial S_{kl}}{\partial\left(  \,\mathrm{vech}\,L_{k}\right)
^{\prime}}  &  =\frac{3}{2}S_{kl}\,\mathrm{vech}\,\left[  \left(  \,L_{k}%
^{-1}\right)  ^{\prime}-2A_{kl}^{-1}L_{k}\right]  ^{\prime}\nonumber\\
&  +S_{kl}\frac{m_{k}}{\left(  A_{k}^{-1}\right)  _{11}}\mathrm{vech}\,\left[
\,A_{k}^{-1}J_{11}A_{k}^{-1}L_{k}\right]  ^{\prime}\nonumber\\
&  -S_{kl}\frac{m_{k}+m_{l}}{\left(  A_{kl}^{-1}\right)  _{11}}\mathrm{vech}%
\,\left[  \,A_{kl}^{-1}J_{11}A_{kl}^{-1}L_{k}\right]  ^{\prime}.
\end{align}
The complete set of formulas for matrix elements and gradient
components can be
found in our paper introducing the diatomic form of the basis set 
\cite{kinghorn99a}.
We have shown the utility of such an approach
in algorithms for variational BO and non--BO calculations,
where we used first and second derivatives to guide
the Newton--Raphson optimization procedure
\cite{A8,A14,A21,A38,kozlowski92b}. 

\item  
The expansion of the wave function in terms
of $\phi_k$ has to properly
reflect the permutational symmetry
of the considered state. 
Since the $\phi _k$'s are isotropic
the problem of 
how to handle
permutational symmetry is
taken care of by appropriate symmetry projection operators. 
Consider a system
of $N$ particles  
whose permutational symmetry is given by
the group, $G$, represented by
a set, $\left\{ P_{\alpha \in G}\right\} ,$ of $N\times N$ permutation
matrices. A wave function represented as an expansion in $\phi _k$ is a
function of the $n=N-1$ component vectors of ${\bf r}$, the relative
coordinates. The permutation $P_\alpha $ acting on the $N$ particle
coordinates induces a transformation on the center--of--mass and relative
coordinates given by 
\begin{equation}
T\bar{P}_\alpha T^{-1}=I_3\oplus \bar{\tau}_\alpha,
\end{equation}
where $T$ is the transformation matrix given in eqn(\ref{Ttran}). 
The right hand side of this 
expression is the direct sum of the identity acting on the
center--of--mass coordinates, ${\bf r}_0$, and $\tau _\alpha ,$ which is an 
$n\times n$ ``permutation'' matrix acting on the component vectors of the
relative coordinate vector ${\bf r}$. The action of the permutation
represented by $P_\alpha $ on $\phi _k$ is then, 
\begin{equation}
P_\alpha \phi _k=\prod_{i<j}\left[ {\bf r}^{\prime }(\tau _\alpha
^{\prime }J_{ij}\tau _\alpha \otimes I_3){\bf r}\right] ^{\frac{m_{kij}}%
2}\exp \left[ -{\bf r}^{\prime }(\tau _\alpha ^{\prime }A_k\tau _\alpha
\otimes I_3){\bf r}\right] .
\end{equation}
The action of a representation of the group $G$ on $\phi
_k$ is thus induced by the projector $\sum_{\alpha \in G}\tau _\alpha .$
This method of symmetry 
projection on correlated gaussians is discussed in
more detail in the 
references\cite{Kinghorn93,Kinghorn95b,Poshusta83}.


\end{enumerate}


In calculations of rotationally excited states
the non--adiabatic w.f.'s will be
expanded as symmetry projected linear combinations of the explicitly
correlated $\phi _k$ multiplied by an angular term, $Y_{LM}^k:$ 
\begin{equation}
\Psi_{LM\Gamma }={\cal P}_\Gamma \sum_k c_k Y_{LM}^k\phi_k.  \label{wf}
\label{Y1}
\end{equation}
Here $\phi_k$ are the explicitly correlated n-body gaussians given 
in eqn(%
\ref{basisfcn}), ${\cal P}_\Gamma $ is an appropriate permutational
symmetry projection operator for the desired state, 
$\Gamma $, and $Y_{LM}^k$
is a product of coupled solid harmonics labeled by the total angular
momentum quantum numbers $L$ and $M$.
%Permutational symmetry is handled using projection methods in the same
%manner as described for the potential expansion in the previous section.
%Again, the reader is referred to the references for details\cite
%{Poshusta83,Kinghorn93,Kinghorn95b}.

$Y_{LM}^k$ is a vector--coupled product 
of solid harmonics\cite{Biedenharn81}
given by the Clebsch--Gordon expansion, 
\begin{equation}
Y_{LM}^k=\sum_{\,\,\begin{array}{c} {\left\{ l_j,m_j\right\} }
\\ {m_1+\cdots
+m_n=M} \end{array}
}\left\langle LM;k\right. |\left. l_1m_1\cdots l_nm_n\right\rangle
%
%Y_{LM}^k=\sum_{\,\,\QATOPD. . {\left\{ l_j,m_j\right\} }{m_1+\cdots
%+m_n=M}}\left\langle LM;k\right. |\left. l_1m_1\cdots l_nm_n\right\rangle
\prod_j^n{\cal Y}_{l_jm_j},
\end{equation}
and the solid harmonics are given by:
\begin{equation}
{\cal Y}_{lm}\left( {\bf r}_j\right) =\left[ \frac{2l+1}{4\pi }\left(
l+m\right) !\,\left( l-m\right) !\right] ^{\frac 12}\sum_p\frac{\left(
-x_j-iy_j\right) ^{p+m}\left( x_j-iy_j\right) ^pz_j^{l-2p-m}}{2^{2p+m}\left(
p+m\right) !\,p!\,\left( l-m-2p\right) !}.
\end{equation}
The ${\cal Y}_{lm}\left( {\bf r}_j\right) $ are single particle
angular momentum eigen--functions 
in relative coordinates which transform the
same as spherical harmonics, {\it i.e.} 
have the same eigen--values. Since
the $\phi _k$ are angular momentum 
eigen--functions with zero total angular
momentum, the product with $Y_{LM}^k$ 
can be used in principle to obtain any
desired angular momentum eigen--state. 
Note the $k$ dependence of $Y_{LM}^k$;
this is included since there are many 
ways to couple the individual angular
momenta, $l_j$, 
to achieve the desired total angular momentum $L$ and it may
be necessary to include several sets of the $l_j$ in order to obtain a
realistic description of the w.f. Varga and Suzuki\cite{Varga95}
have recently
proposed representing the angular dependence of the w.f. using
a single solid harmonic whose argument 
contains additional variational
parameters, ${\bf u} = (u_1, u_2, \cdots, u_n)$:
\begin{equation}
\Psi_{LM\Gamma } = {\cal P}_\Gamma Y_{LM}({\bf v}) \sum_k c_k \phi_k,
\; \; {\rm with} \; \; 
{\bf v} = \sum_{i=1}^{n} u_i {\bf r}_i.
\label{Y2}
\end{equation}
There appears to be several advantages in doing this and we are
investigating the possibility of using 
this approach in our full N--body
implementation.
The strict separation of the angular and ``radial" variables in
eqns.(\ref{Y1}) and (\ref{Y2}) allows separate consideration of
the vibrational states with different total angular momentum
quantum number, $L$. The magnitude of the Coriolis coupling
for the particular $L$--state will determine whether the most
general form, eqn.(\ref{Y1}), or more simplified form,
eqn.(\ref{Y2}), of the total w.f. should be used. 

There have been several highly accurate non--adiabatic variational
calculations on atomic and exotic few particle systems using simple
correlated gaussians\cite
{Kinghorn93,Kinghorn95b,A3,A37,Varga96}.
By simple we mean
they only contain the exponential part of the $\phi _k,$ (no $r_{ij}$
pre-multipliers). However, attempts at non-adiabatic molecular calculations
have been plagued by problems with linear dependence in the basis during
energy optimizations. This problem occurs in calculations on atomic systems
also, but to a much lesser extent. We believe that we understand this
phenomena and have shown\cite{kinghorn99a} that our new basis, 
including pre-multiplying powers of $r_{ij}$
will eliminate or at least drastically reduce the linear dependence
problems. Our reasoning is as follows: In systems with more than one heavy
particle ({\it e.g.}, two or more nuclei in a non-B.O. calculation,
or two or more atoms in a ro-vibrational calculation)
there will be a larger particle density away from the origin,
where one of the heavy particles is placed, than near the origin.
In the case of a system with three heavy particles, in addition
to a reduction of the probability density of two of the particles
when they approach the third one located at the coordinate origin,
there will be a density reduction when the two particles when 
they approach
each other.
That is, the w.f. will have peaks shifted
away from the origin and peaks shifted away from 
$r_{ij}$=0 points. There are three ways to account for this behavior in
the w.f. expressed in terms of 
correlated gaussians; 1) use correlated gaussians
with shifted centers, {\it i.e.}, 
$\exp [-\left( {\bf r-s}\right) ^{\prime }%
\bar{A}\left( {\bf r-s}\right) ]$; 2) Use near linearly dependent
combinations of simple correlated gaussians with large matched $\pm $ linear
coefficients; or 3) Use pre-multiplying powers of $r_{ij}$. The first option
is unacceptable since it results in a w.f. which no longer
represents a pure angular momentum state. The second option is what we
believe causes the linear dependence and numerical instability which we are
trying to avoid. The third option is what we are proposing. The linear
dependence that we have observed in our calculations using the simple
correlated gaussians looks, in some sense, like an attempt by the
optimization to include in the w.f. derivatives of the basis
functions with respect to the non-linear parameters. The near linear
dependent terms resemble numerical derivatives. Removal of these near linear
dependent terms has an adverse affect on the w.f.'s, as manifested by
poor energy results, but leaving them in leads to numerical instabilities
which hinder optimization or cause complete collapse of the eigen-solutions.
Now, derivatives of simple gaussians with respect to non-linear parameters,
elements of the matrices $A_k$, bring down pre-multiplying (even) powers of 
$r_{ij}$. Thus, explicitly including pre-multiplying powers of $r_{ij}$ in
the basis functions adds the needed flexibility to the basis in a
numerically stable way. Also, the rate of convergence is 
improved by these pre-multiplying $r_{ij}^m$ terms in the same way that they
effect convergence in the Hyllerass basis. The $\phi _k$ are similar to the
Hyllerass basis functions with the Slater--type exponentials replaced by
fully correlated gaussian type exponentials. We believe that excellent
results for non-adiabatic calculations will be obtained using
our new variational basis functions. 
%We have preliminary results to support
%the above assertions (see the next section).

The proposed work requires new types of integrals to be derived
and implemented. Over the last few years we have made several
steps in this direction. We have derived multi--particle
Hamiltonian integrals with N--body 
general cartesian correlated gaussians
\cite{kozlowski92a}, we presented integrals involving
$J_z$ and $J^2$ operators \cite{A20}, and we recently derived
Hamiltonian matrix elements with correlated gaussians
containing high powers of coordinates  \cite{A382}.
This work will continue within this project.

\vspace{2mm}
\noindent
%{\bf B.5 Variational Wave Function for Non-Adiabatic Calculations}
{\bf B.5 Superposition of Electronic and Ro--vibrational Wave Functions}
The construction of the wave function for a molecule 
to be used as an
starting guess in variational non--adiabatic calculations
can be based on a superposition of the electronic wave function
obtained in a BO calculation and the ro--vibrational wave function
obtained in calculation where a fit to the potential energy
surface is used as the interaction potential. Let us use the H$_2^+$
ion as an example. Using gaussians, the BO electronic wave function 
wave function can be represented as the following expansion:
\begin{equation}
\Psi^{BO} = 
\sum_i c_i^{BO} r_{ {\bf A}1}^{m_i^A} r_{ {\bf B}1}^{m_i^B}
{\rm exp} \left[ -\alpha_i^A r_{ {\bf A}1}^2 
- \alpha_i^B r_{ {\bf B}1}^2 \right],
\label{BO1}
\end{equation}
where the bold indices indicate the centers, which are not allowed to move.
The ro--vibrational wave function (with $J=0$), which is obtained using the
approach described above with the ${\bf A}$ nucleus assumed to be
the center of the internal coordinate system, has the following form
(see the following section):
\begin{equation}
\Psi^{ro-vib} = 
\sum_i c_i^{ro-vib} r_{ {\bf A}B}^{m_i^{AB}} 
{\rm exp} \left[ -\alpha_i^{AB} r_{ {\bf A}B}^2 \right].
\label{BO2}
\end{equation}
Now we construct the superposition of the two functions:
\begin{equation}
\Psi^{non-BO} = 
\sum_{i,j} c_i^{ro-vib} c_j^{BO} 
r_{ {\bf A}B}^{m_i^{AB}}
r_{ {\bf A}1}^{m_j^A} r_{B1}^{m_j^B}
{\rm exp} 
\left[ -\alpha_i^{AB} r_{ {\bf A}B}^2 -\alpha_j^A r_{ {\bf A}1}^2 
- \alpha_j^B r_{B1}^2 \right],
\label{BO3}
\end{equation}
The above approach can easily be extended to systems with more particles,
as well as to excited states
%dbk
and should provide good initial guesses for optimizing wave functions.
%dbk



\vspace{2mm}
\noindent
{\bf B.7 Proposed Applications}


Following our previous non--adiabatic calculations of HD$^+$
and the positronium molecule and some other model systems
\cite{A3,A6,A16,A27,kozlowski91,%
kozlowski92b,kozlowski92c} and more 
recent non--adiabatic calculations
of the electron affinities of hydrogen, deuterium and tritium
\cite{A37},
we will now consider the following atomic and molecular systems:



\vspace{2mm}
\noindent
{\bf B.7.1 Hydrogen Molecule and its Isotopomers. A Test
Case for Quantum Non--Adiabatic and Relativistic Calculations}



In the first applications 
we will consider
hydrogen molecules and ions and their isotopomers.
In his review entitled "Hydrogen Molecule. Test of Quantum Chemistry"
\cite{kk} Kolos discusses to observable quantities which have
been used to test the agreement between the experiment and theory.
They are the dissociation energy and the ionization potential.
For the former quantity the most recent theoretical result
for H$_2$, which was obtained based on very accurate Born-Oppenheimer
energies perturbatively corrected for non--adaiabatic effects
and also corrected for the relativistic effects calculated
ealier is equal to D$_{\rm o}$=36118.051~cm$^{-1}$. The remaining
descrpancy with the experimental value of W.C. Stwalley\cite{k29}
equal to 36118.6$\pm$0.5~cm$^{-1}$ is in the range of the
experimental uncetaintity.
However, the in the series H$_2$, HD and D$_2$ while the
desrepancies for H$_2$ and HD are positive, the desrepancy
with the latest experimental measurment for D$_2$ of
Eyler\cite{k42,k42p}
is slightly negative.
The questions which Kolos raises
in his paper the is a room for improvement of the theoretical
value and he concludes that some improvments can be made
in the following contributions:(1) the relativistic contributions,
which were calculated 30 years ago and the wavefunction
employed at that time was obtained in the Born--Oppenheimer
approximation and was not very accurate according to the present
day standards; (2) the radiative corrections whose values
was only estimated before\cite{k16}; (3) the adiabatic corrections
were calculated using the perturbation approach with Born--Oppenheimer
wave function and not with a fully non--adiabatic wave function.

The experimental determinations of IP have been made
by Rydberg--series extrapolation, either of the singlet
$np$ states or of high-l states, and this approach is still
successfully used to improve the accuracy of experimental IP.
Recently, however, a new approach has been introduced by Glab and
Hess\cite{k52} and developed  Eyler and coworkers.\cite{k53}
It is a two step procedure in which first the E, F state
is excited and, subsequently, transitions to the
Rydberg $np$ states are measured.  Using this approach Glab
and Hessler\cite{k52} have reduced the experimental error, and
McCormick {\it et al.}\cite{k53} have still increased the accuracy
and obtained IP =124417.524$\pm$0.015~cm$^{-1}$
In the last years the two main groups involved in experimental
determination of H$_2$ IP are the groups at the Yale University
and the University of Delaware (Gilligan, Eyler), and groups
at Universite de Paris-Sud and at the National Research Council
of Canada (Jungen, Dabrowski, Herzberg and Vervloet),\cite{k57}
and these groups continue producing new improved experimental
values of IP. The accuracy of the measurments is most remarkable, its
limits, however, have still not been reached and it is
expected\cite{k42p} that the experimental error will be reduced
to almost 0.001~cm$^{-1}$, making IP's of H$_2$ and its
isotopomers one of the most accuratelly measured quantity.
Using the  calculated
results for H$_2$, HD and D$_2$ by Kolos and Rychlewski,\cite{k12}
and very accurate H$_2^+$ energy of Moss,\cite{k58}
which includes the recent accurate value of the Lamb shift
correction,\cite{k59} the following results were obtained
for the IP's of H$_2$, HD and D$_2$, respectively:
124417.471~cm$^{-1}$, 124558.465~cm$^{-1}$, and 124745.377~cm$^{-1}$.
\cite{kk}
These can be compared to the best experimental values:
124417.484$\pm$0.017~cm$^{-1}$ and 124417.507$\pm$0.012~cm$^{-1}$
for H2,\cite{k42,k57} 124568.481$\pm$0.012~cm$^{-1}$ for HD,\cite{k42}
and 124745.353$\pm$0.024~cm$^{-1}$ for D$_2$.
As one can see the agreement between the theory and the
experiment is fair but not ideal and some discrepancy between
the two sets are still present. These can be due to the
non--adiabatic, relativistic and radiative corrections
calculated with still not sufficient accuracy (particularly the
latter two). Better calculations of the corrections using
a fully non--adiabatic approach can produce values, which can
be more trustworthy. This, in turn, may lead to new more refine
measurements.
Some room for improvement in the experimental
values may be expected based on the analysis
of the descrepencies between the theoretical and experimental
results. While the discepancies for H$_2$ and HD are positive
and equal to 0.013$\pm$0.017~cm$^{-1}$ (0.036$\pm$0.012~cm$^{-1}$)
and 0.016$\pm$0.012~cm$^{-1}$, for $D_2$ it is negative
and equal to -0.024$\pm$0.024~cm$^{-1}$.
Since  a refinement of the radiative corrections are expected
to bering a uniform improvement for all three isotopomers, if
it improves the egreement for H$_2$ and HD it will destroy
it for D$_2$ and {\it vice versa}. Thus, The source of the discrpency
may be in the non--adaiabatic or/and relativistic corrections.
Since, the smallest currently used correction to IP of H$_2$
is about eight times larger than the present--day experimental error,
the IP offers a very good and stringent test of the
calculated values of the non--adiabatic, relativistic and
radiative corrections.

Spectroscopy of the hydrogen molecule has been of fundamental
interest since the early days of quantum mechnics.
The masses of the nuclei give rise to more noticable
deviations from the Born-Oppeheimer approximation
and thus this system  may serve as a test for an
understanding of these effects. While, as described in the
previous section the $X^1\Sigma_g^+$ ground state of H$_2$
and its isotopomers have been calculated with an accuracy
of 0.01~cm$^1$, this level of accuracy has not yet been
achieved for excited states. The excited states of the
$1\Sigma_g^+$ symmetry were investigated with increasing
accuracy over the last decades\cite{w1,w2}
Calculations for the lowest vibrational levels in the second adiabatic state
of $X^1\Sigma_g^+$ symmetry, the $EF^1\Sigma_g^+$ state,
now have reached agreement with experiment within 0.1~cm$^{-1}$.
It can be anticipated that the non--adiabatic
effects increase when molecule is excited, since
the motion of the nuclei is subject to more shellow
potentials due to more delocalized electrons.
Excited state potentials of H$_2$ show complicated
shapes as a function of the internuclear distance  due
to avoiding crossings between diabatic energy curves.
A celebrated example is a double--well potentail of the
$EF^1\Sigma_g^+$ state. The third and fourth adiabatic states of
the $^1\Sigma_g^+$ symmetry, denoted as $GK^1\Sigma_g^+$
and $H\overline{H}^1\Sigma_g^+$,
have also  pronounced double--well structures.
None of these excited states has been ever calculated
within a fully non--adiabatic calculations.


The case of the HD isotopomer is qualitatively different
from that of the symmetric H$_2$ and D$_2$ molecules;
in the non--adiabatic approach states of {\it gerade}
and {\it ungerade} symmetry mix, thus lowering the
electronic symmetry from $D_{\infty h}$ to $C_{\infty v}$.
The observation of a very weak vibrational spectrum in HD
provided evidence for a static electric dipole moment
and a deviation from the inversion--symmetric charge
distribution in the ground state.
In a recent work\cite{w28} a strong $g-u$
symmetry breaking was reported for the $H\overline{H}^1\Sigma_g^+$
state of HD. No fully non--adiabatic study has been
conducted on the symmetry mixing phenomenon in HD in
excited states.




\vspace{2mm}
\noindent
{\bf B.7.2 
H$_3^+$, H$_3$, H$_3^-$, and their Isotopomers}



One of the most remarkable successes
for theory has been the work on the $H_3^{+}$
system. The first laboratory spectroscopic detection of  $H_3^{+}$,
reported by Oka \cite{ref:k35} was of ro--vibrational transitions in the
fundamental $\nu_2$ infrared band. An important ingredient in the
identification of the spectrum was the set of vibration--rotation
constants calculated by Carney and Porter \cite{ref:k36}. This first
laboratory measurement prompted searches for $H_3^{+}$ in the interstellar
medium, and several spectacular discoveries were made of the presence
of $H_3^{+}$ in the Jovian \cite{ref:k41}, Saturn \cite{ref:k44} and
Uranus \cite{ref:k45} aurora. There has been 
extensive theoretical
work on $H_3^{+}$ and its isotopomers by Tennyson and coworkers
\cite{ref:ten2,ref:ten3,ref:ten4,ref:ten5,ref:ten6,ref:ten7,ref:ten8,%
ref:ten9}, where a wide range of the ro--vibrational excitations were
calculated using BO and adiabatic PES's. Some other
more recent experiments has also appeared./cite{mccall,stark} 
There has been also some very accurate experimental work
done on the neutral H$_3$ and its 
isotopomers,\cite{azinovic,muller} and on the H$_3^-$ anion./cite{rob}
Our accurate non--adiabatic calculations of the excitation
and fragmentation energies of the H$_3$ molecule its
ions and isotopomers may extend the testing ground for
sub--wavenumber theoretical--experimental approaches to 
three atom system.



\vspace{2mm}
\noindent
{\bf B.7.3
LiH$^-$ Anion}

LiH$-$ is the smallest stable molecular anion.
Adiabatic electron affinities were recently reported for 
the LiH and LiD ($^7$Li) systems with the precision
of 0.012~eV.\cite{sarkas}
The experimentally determined electron affinities led to
anion dissociation energy values of 2.017$\pm$0.021~eV 
for LiH ($^7$Li)
and 2.034$\pm$0.021 for LiD$^-$.
LiH$^-$ isotopomer systems will be use to study
the non--adiabatic effects on the electron attachment
energy.



\vspace{2mm}
\noindent
{\bf C. Building a "Beowulf" Super COmputer}

The calculations we are proposing will place heavy demands on computational
resources. The optimization of variational wave functions with
respect to many non--linear parameters requires large amounts of
cpu time even when our efficient data structures and analytic gradients
are used. However, the data structures and algorithms we use to implement
our theoretical results are readily implementable in a data parallel fashion
allowing us to take advantage of scalable  parallelism using clusters
of workstations and dedicated computational nodes. We have worked out
the technical details for building and administering
a ``Beowulf''\cite{beowulfwww} class
parallel Linux cluster and are currently waiting on the shipment of
components for our prototype cluster. Each node
will consist of a dual processor ``box'' of high end
PC components including Pentium II 400 MHz processors and
256MB of 8ns 100MHz memory. Communication will be over
switched 100Mb Ethernet. Parallel clusters of this type
scale well and we expect to greatly expand our initial setup.
To take advantage of this parallel computing machinery we are  currently
utilizing ``High Performance Fortran'' \cite{HPFwww}
and PVM \cite{PVMwww} for code development. The combination of these
robust development tools and the high performance/low cost parallel hardware
will provide an excellent computational environment for our researchers.


vspace{2mm}
\noindent
{\bf D. FINAL REMARKS}

The functions $\phi _k = 
\prod_{i<j}r_{ij}^{m_{kij}}\exp \left[ -{\bf r}^{\prime%
}(A_k\otimes I_3){\bf r}\right]$
are flexible enough to meet criteria necessary for global analytic
representation of a molecular non--adiabatic wave function, {\it i.e.} 
correct
asymptotic behavior, differentiability, symmetry adaptation, {\it etc.}. 
We have prototyped our procedures in the calculations mentioned before. 
Having successfully completed testing of our general procedures, we will
proceed with full implementation of an 
n--body program suite. 
Integrals needed for matrix elements in
the variational calculations have been 
completed and work is being done on
more elegant methods for representing 
high angular momentum states.

There are issues in this project which cannot be definitely
resolved at the present stage and only through further research 
the most optimal directions to carry on the development can be
selected. One such issue concerns non--linear optimization of the
wave function, which will depend 
critically on the ease and black--box
manner this type of optimization 
can be performed in routine applications.
It is possible that the approach of 
contracting gaussian functions,
as it is done in the electron structure theory 
or an even--tempered scheme 
would be the most
effective way of avoiding costly non--linear optimizations.
Another issue concerns the lengths of the expansions for the
wave function in terms of $\phi_k$
functions, which can become very challenging for more degrees
of freedom.



Ours is one of the few research 
programs focusing on the non--adiabatic 
Atomic and Molecular Quantum Mechanics
and, therefore, provides a unique 
training place for scientists in this area.  
Furthermore, the development of the techniques 
utilizing explicitly--correlated gaussian functions
has been carried out by only a few groups in the world.  
In the age of parallel computing, these
types of functions may facilitate an important 
step towards achieving improved accuracy in 
{\it ab initio} calculations in both conventional 
Born--Oppenheimer and non--adiabatic 
molecular quantum mechanics.
However, several major breakthroughs are still needed 
in the technology of explicitly correlated gaussians
to achieve this goal.


\pagebreak

%\vspace{2mm}
%\noindent
%{\bf Literature Cited}
%\vspace{2mm}

\begin{thebibliography}{999}


\bibitem{A1}   
N. Oliphant and L. Adamowicz, Multi--Reference Coupled Cluster 
Method Based on the Single
Reference Formalism, Int. Rev. Phys. Chem. {\bf 12}, 339 (1993).

\bibitem{A2}
E. Schwegler, P.M. Kozlowski and L. Adamowicz, 
Application of Explicitly--Correlated Gaussian
Functions for Calculations of the 
Ground State of the Beryllium Atom, J. Comp. Chem. {\bf 14}, 566 (1993).

\bibitem{A3}   
P.M. Kozlowski and L. Adamowicz, Non--Adiabatic 
Variational Calculations for the Ground State of
the Positronium Molecule, Phys. Rev. {\bf A 48}, 1903 (1993).

\bibitem{A4}   P. Piecuch, N. Oliphant and L. Adamowicz, 
State--Selective Multi--Reference Coupled Cluster Theory,
J. Chem. Phys. {\bf 99}, 1875 (1993).

\bibitem{A5}   Z. Slanina, F. Uhlik, J. Kurtz and L. Adamowicz, 
A Classification of 200 Isomerizations among 51
Isotopomers of $C_8$ (D2d), J. Radioanal. Nucl. Chem. {\bf 170}, 
373 (1993).

\bibitem{A6}   
P.M. Kozlowski and L. Adamowicz, Equivalent 
Quantum Approach to Nuclei and Electrons in
Molecules, Chem. Rev. {\bf 93}, 2007 (1993).

\bibitem{A7}   W.J. McCarthy, M.A. Roehrig, Qi-Qi Chen, 
G.H. Henderson, L. Adamowicz and S.G. Kukolich,
Microwave Measurements and {\it Ab-Initio} Dynamics of the Large 
Amplitude Motion of the Ring
Puckering in 2-Sulpholene, J. Chem. Phys. {\bf 99}, 7305 (1993).

\bibitem{A8}   
Z. Zhang, P.M. Kozlowski and L. Adamowicz, Newton-Raphson 
Optimization of the Explicitely--Correlated 
Gaussian Functions for Calculations of 
the Ground State of the $He$ Atom, J. Comp. Chem.
{\bf 15}, 54 (1994).

\bibitem{A9}   
Z. Slanina, F. Uhlik and L. Adamowicz, 
Classification of 486 Isomerizations Among 72 $^{12}C/^{13}C$
Isotopomers of Cyclic $C_7$, J. Radioanal. 
Nucl. Chem. {\bf 170}, 107 (1993).

\bibitem{A10}  
Z. Slanina, S.-L. Lee, J.-P. Francois, J. Kurtz, L. Adamowicz 
and  M. Smigel, A Non-Planar Cyclic
Minimum-Energy Structure of Singlet $C_9$, 
Mol. Phys. {\bf 81}, 1489 (1994).

\bibitem{A11}  
P. Piecuch and L. Adamowicz, State-Selective Multi-Reference 
Coupled Cluster Theory Employing
Single-Reference Formalism:  Implementation and Application 
to the H8 Model System, J. Chem. Phys.
{\bf 100(8)}, 1 (1994).

\bibitem{A12}  
P. Piecuch and L. Adamowicz, Solving the Single-Reference 
Coupled-Cluster Equations Involving
Highly Excited Clusters in Quasidegenerate Situations, 
J. Chem. Phys. {\bf 100}, 5857 (1994).

\bibitem{A13}  
P. Piecuch and L. Adamowicz, State-Selective 
Multi-Reference Coupled-Cluster Theory Using
Multiconfiguration Self-Consistent Field Orbitals:  A 
Model Study on H8, Chem. Phys. Lett. {\bf 221}, 121
(1994).

\bibitem{A14}  Z. Zhang and L. Adamowicz, Explicitly--Correlated 
Gaussian Functions with $r_{ij}^{2n}$ Factors for Calculations
of the Ground State of the Helium Atom, J. Comp. Chem. 
{\bf 15}, 893 (1994).

\bibitem{A15}  
Z. Slanina, S.-L. Lee, J.-P. Francois, J. Kurtz 
and L. Adamowicz, Inversion of the $C_8$ Non--Planar Ring,
Chem. Phys. Lett. {\bf 223}, 397 (1994).

\bibitem{A16}  P.M. Kozlowski and L. Adamowicz, Effective 
Non--Adiabatic Calculations on $HD^{+}$ with 
Explicitly--Correlated Gaussian Functions, 
Int. J. Quant. Chem. {\bf 55}, 245 (1995).

\bibitem{A17}  
P. Piecuch and L. Adamowicz, Breaking 
Bonds with the State--Selective Multi--Reference Coupled-Cluster 
Method, J. Chem. Phys. {\bf 102}, 898 (1995).

\bibitem{A18}  
A. Sobolewski  and L. Adamowicz, {\it Ab-Initio} 
Characterization of Electronically Excited States in
Highly Unsaturated Hydrocarbons, J. Chem. Phys. 
{\bf 102}, 394 (1995) .

\bibitem{A19}  
V. Alexandrov. P. Piecuch and L. Adamowicz, 
State-Selective Multi-Reference Coupled-Cluster Theory
Employing the Single-Reference Formalism:  Application 
to Excited States of H8, J. Chem. Phys.
{\bf 102(8)}, 3301 (1995).

\bibitem{A20}  
P. M. Kozlowski and L. Adamowicz, Matrix Elements 
for  $J_z$ and $J^2$  Operators over Explicitely-Correlated
Cartesian Gaussian Functions, 
Int. J. Quant. Chem. {\bf 55}, 367 (1995).

\bibitem{A21}  
Z. Zhang and L. Adamowicz, Newton-Raphson Optimization 
of the Explicitly-Correlated Gaussian
Functions for the Ground State of the Be Atom, 
Int. J. Quantum Chem.  {\bf 54}, 281 (1995).

\bibitem{A22}  
K.B. Ghose and L. Adamowicz, Use of Recursively 
Generated Intermediates in State--Selective
Multi--reference Coupled--Cluster Method:  
A Numerical Example, J. Chem. Phys. {\bf 103}, 9324 (1995).

\bibitem{A23}  
Z. Slanina, S.-L. Lee, M. Smigel, J. Kurtz and 
L. Adamowicz, Smaller Carbon Clusters: Linear, Cyclic,
Polyhedral, in SCIENCE AND TECHNOLOGY OF 
FULLERENE MATERIALS, Eds. P. Bernier, D.S.
Bethune, L.Y. Chiang, T.W. Ebbesen, 
R.M. Metzger and J.W. Mintmire, Materials Research Society,
Pittsburgh, 1995, 163-168 (1995).

\bibitem{A24}  
W.J. McCarthy, L. Lapinski, M.J. Nowak and 
L. Adamowicz, Anharmonic Contributions to the
Inversion Vibration in 2-Aminopyrimidine, 
J. Chem. Phys. {\bf 103}, 656 (1995).

\bibitem{A25}  
K.B. Ghose, P. Piecuch and L. Adamowicz, 
Improved Computational Strategy for the State-Selective
Coupled-Cluster Theory with Semi-Internal Triexcited Clusters:  
Potential Energy Surface of HF
Molecule, J. Chem. Phys. {\bf 103}, 9331 (1995).

\bibitem{A26}  
K.B. Ghose, P. Piecuch, S. Pal and L. Adamowicz, 
State-Selective Multireference Coupled-Cluster
Theory:  In pursuit of Property Calculation, 
J. Chem. Phys. {\bf 104}, 6582 (1996).

\bibitem{A27}  
P. M. Kozlowski and L. Adamowicz, Variation 
Calculations of the Two-Photon Annihilation Rate of the
Positronium Molecule, J. Phys. Chem. {\bf 100}, 6266 (1996).

\bibitem{A28}  
G. Gutsev, A. Le\'{s} and L. Adamowicz, The Electronic 
and Geometrical Structure of Aluminum Flouride
Anions, $AlF_n, n = 1-4$, and Electron Affinities of 
Their Neutrals, J. Chem. Phys. {\bf 100}, 8925 (1994).

\bibitem{A29}  
G.L. Gutsev and L. Adamowicz, The Structure of 
the CF Anion and the Electron Affinity of the $CF_4$
Molecule, J. Chem. Phys. {\bf 102}, 9309 (1995).

\bibitem{A30}  
G. Gutsev, A. Sobolewski and L. Adamowicz, 
Theoretical Study on the Structure of Acetonitrile
$(CH_3CH)$ and its Anion $CH_3CH^-$, Chem. Phys. 
{\bf 196}, 1 (1995).

\bibitem{A31}  
G.L. Gutsev and L. Adamowicz, Relationship Between 
the Dipole Moments and the Electron Affinities
for Some Polar Organic Molecules, Chem. Phys. Lett. {\bf 235}, 
377 (1995).

\bibitem{A32}  
G.L. Gutsev and L. Adamowicz, The Electronic and 
Geometrical Structure of Dipole--Bound Anions
Formed by Polar Molecules, J. Phys. Chem. 
{\bf 99}, 13412 (1995).

\bibitem{A33}  
G.L. Gutsev and L. Adamowicz, The Valence and 
Dipole--Bound States of the Cyanomethide Ion,
$CH_2CN$ , Chem. Phys. Lett. 246, 245 (1995).

\bibitem{A34}  
L. Adamowicz and J.-P. Malrieu, Multi--Reference 
Self--Consistent Size--Extensive State--Selective
Configuration Interaction, J. Chem. Phys. 
{\bf 105}, 9240 (1996).

\bibitem{A35}  
L. Adamowicz, R. Caballol, J.-P. Malrieu and J. Meller, 
A General Bridge Between Configuration
Interaction and Coupled--Cluster Methods:  A Multistate 
Solution, Chem. Phys. Lett.
{\bf 259}, 619 (1996).

\bibitem{A36}  
L. Adamowicz and J.-P. Malrieu, Multi-Reference Self-Consistent 
Size--Extensive Configuration
Interaction (CI) - A Bridge Between the Coupled-Cluster 
Method and the CI Method, in MODERN
IDEAS IN COUPLED-CLUSTER METHODS, R.J. Bartlett, 
Ed., World Scientific Publishing, pp. 307-332 (1997).

\bibitem{A37}  
D.B. Kinghorn and L. Adamowicz, 
The Electron Affinity of Hydrogen, Deuterium and Tritium:  
A Non--Adiabatic Variational Calculation Using Explicitly-Correlated 
Gaussian Basis Set, J. Chem. Phys.,
{\bf 106}, 4589 (1997).

\bibitem{A38}  
D. Gilmore, P.M. Kozlowski, D.B. Kinghorn and 
L. Adamowicz, Analytic First Derivatives for
Explicitly-Correlated Multi-Center Gaussian Geminals, 
Int. J. Quantum Chem. {\bf 63}, 991 (1997).

\bibitem{A381}
W.J. McCarthy, L. Lapinski, M.J. Nowak and L. Adamowicz,
Out--of--Plane Vibrations of $NH_2$ in 2--Aminopyrimidine, 
J.Chem.Phys., {\bf 108}, 10116 (1998).  

\bibitem{A382}
D.B. Kinghorn and L. Adamowicz,
A New N--Body Potential and Basis Functions for
Variational Energy Calculations,
J.Chem.Phys., {\bf 106}, 8760 (1997).

\bibitem{A383}
D. B. Kinghorn and L. Adamowicz, in {\em Pauling's Chemical
Bonding}, edited by Z.B. Maksic and W.J. Orville--Thomas,
Elsevier Science, 1998, accepted for publication.

\bibitem{A384}
K.B. Ghose, L. Adamowicz and S. Pal,
The State Selective Coupled Cluster Method with
Restricted Sets of Triples and Quadruples:
Some Aspects of the Theory and its Recent Applications,
Int.J.Quantum Chemistry, submitted.

\bibitem{A385}
K.B. Ghose and L. Adamowicz,
The State--Selective Multi--Reference Coupled Cluster
Method with Restricted Sets of Triples and Quadruples:
Applications for the BH, CO and C$_2$ Molecules,
Chem.Phys.Lett., submitted.


\bibitem{A3851}
L. Adamowicz, P. Piecuch and K. Ghose, 
The State--Selective Multi--Reference
Coupled Cluster Method, Mol. Phys., 
{\bf 94}, 225 (1998).



\bibitem{A3860}
N.A. Oyler and L. Adamowicz, Theoretical Ab--Initio
Calculations of the Electron Affinity of 
Thymine, Chem. Phys. Lett. {\bf 219}, 223 (1994).


\bibitem{A3861}
G.H. Roehrig, N.A. Oyler and L. Adamowicz, 
Can Electron Attachment Change Tautomeric
Equilibrium of Guanine?, Chem. Phys. Lett. {\bf 225}, 265 (1994).


\bibitem{A3862}
G.H. Roehrig, N.A. Olyer
and L. Adamowicz, Electron Affinity of Adenine.  
Theoretical Study, J. Phys. Chem. {\bf 99},
14285 (1995). 


\bibitem{A3863}
J. Smets, W.J. McCarthy and 
L. Adamowicz, Dipole--Bound Electron
Attachment of Uracil--Water Complex.  Theoretical 
{\it Ab--Initio} Study, 
J. Phys. Chem., {\bf 100}, 14655 (1996).

\bibitem{A3864}
J. Smets, 
W.J. McCarthy and L. Adamowicz, Water Molecules
Enhances Dipole--Bound Electron Affinity of 1--Methyl--Cytosine, 
Chem. Phys. Lett. {\bf 256}, 360 (1996).

\bibitem{A3865}
Y. Elkadi and L. Adamowicz, 
Dipole--Bound Electron Attachment
to Ethylene  Glycol Dimer.  Theoretical {\it Ab--Initio} 
Study, Chem. Phys. Lett., {\bf 261}, 507 (1996).

\bibitem{A3866}
D.M.A. Smith, J. Smets, Y. Elkadi and 
L. Adamowicz, Methylation Reduces
Electron Affinity of Uracil.  {\it Ab--Initio} 
Theoretical Study, J. Phys. Chem.
{\bf 101}, 8123 (1997).

\bibitem{A3867}
J. Smets, D.M.A. Smith, Y. Elkadi 
and L. Adamowicz, {\it Ab--Initio} Theoretical Study
of Dipole--Bound Anions of Molecular 
Complexes.  Water Molecule Inhibits or
Enhances Electron Affinity of N,N-Dimethyl--Aminoadenine, 
Pol. J. Chem., {\bf 72}, xxx (1998). 

\bibitem{A3868}
J. Smets, D.M.A. Smith, Y. Elkadi and 
L. Adamowicz, The Search for Stable
Anions of Uracil Water Clusters.  {\it Ab--Initio} 
Theoretical Studies, J. Phys. Chem.
{\bf 101}, 9152 (1997).

\bibitem{A3869}
D.M.A. Smith, J. Smets, Y. Elkadi and 
L. Adamowicz, {\it Ab--Initio} Theoretical Study
of Dipole--Bound Anions of Molecular 
Complexes.  Water Trimer Anion, J. Chem.
Phys.
{\bf 107}, 5788 (1997).

\bibitem{A3870}
R. Ramaekers, D.M.A. Smith, Y. Elkadi and 
L. Adamowicz, Dipole--Bound Anion
of Hydrogen Fluoride Dimer.  Theoretical 
{\it Ab--Initio} Study, Chem. Phys. Lett.,
{\bf 277}, 269 (1997).

\bibitem{A3871}
R. Ramaekers, D.M.A. Smith, J. Smets 
and L. Adamowicz, Ab--Initio Theoretical
Study of Dipole--Bound Anions of Molecular 
Complexes. (HF)$_3^-$ and (HF)$_4^-$ Anions,
J. Chem. Phys.
{\bf 107}, 9475 (1997).

\bibitem{A3872}
V. Alexandrov, L. Adamowicz, and S. Stepanian,
Theoretical {\it Ab-initio} Study of OH Vibrational Band
in Gas-Phase Glycine Conformers, Chem.Phys.Lett.,
{\bf 291}, 110 (1998).

\bibitem{A3873}
D.M.A. Smith, J.Smets, Y. Elkadi, and L. Adamowicz,
{\it Ab-initio} Theoretical Study of Dipole-Bound Anions
of Molecular Complexes. [H$_2$O \dots HCN]$^-$ and
[HCN $\dots$ H$_2$O]$^-$ Anions,
Chem.Phys.Lett., {\bf 288}, 609 (1988).

\bibitem{A3874}
V. Alexandrov, D.M.A. Smith, H. Rostkowska, M.J. Nowak, 
L. Adamowicz and W. McCarthy, 
Theoretical Study of the OH Stretching Band in 3-Hydroxy-2-Methyl-4-Pyrone,
J.Chem.Phys., {\bf 108}, 9685 (1998).

\bibitem{A3875}
D.M.A. Smith, J. Smeths, Y. Elkadi, and L. Adamowicz,
{\it Ab-initio} Theoretical Study of Dipole-Bound Anions
of Molecular COmplexes. Water Tetramer Anions,
J.Chem.Phys., {\bf 109}, 1238 (1998).


\bibitem{A3876}
C. Desfran\c{o}ois, V. P\'{e}riquet, S. Carles, B.H. Bowen,
J.P. Schermann, and L. Adamowicz,
Neutral and Negatively-Charged Formamide, N-methyl-formamide
and N,N-ddimethyl-formamide Clusters, Chem.Phys.,
accepted for publication.

\bibitem{A3877}
C. Desfran\c{o}ois, V. P\'{e}riquet, Y. Bouteille, 
L. Adamowicz, B.H. Bowen,
nd J.P. Schermann,
Experimental Evidence for Electron Binding to
Quadrupolar Molecules and Clusters,
J.Phys. B., accepted for publication.


\bibitem{A3878}
D.M.A. Smith, J. Smets, Y. Elkadi, and L. Adamowicz,
{\it Ab-initio} Theoretical Study of  Dipole-Bound Anions
of Molecular COmplexes. Formaldehyde Dimer Anion,
Chem.Phys.Lett., submitted.


\bibitem{A3879}
D.M.A. Smith, J. Smets, and L. Adamowicz,
{\it Ab-initio} Theoretical Study of Dipole-Bound Anions of
Molecular Complexes. Water Pentamer Anion,
J.Chem.Phys., submitted.


\bibitem{A3880}
J. Smets, D.M.A. Smith, and L. Adamowicz,
{\it Ab-initio} Theoretical Study of Dipole-Bound
Anions of Molecular Complexes. Anions of H-Bonded Hydrogen Cyanide Polymers,
Chem.Phys.Lett., submitted.




\bibitem{A3881}
C. Desfran\c{o}ois, V. P\'{e}riquet, S. Carles, J.P. Schermann,
D.M.A. Smith, and L. Adamowicz,
Experimental and {\it Ab-initio} Theoretical Study of Electron
Binding to Formamide, N-Methylformamide and N,N-Dimethylformamide,
J.Chem.Phys., submitted.


\bibitem{ad1}
D.B. Kinghorn and L. Adamowicz,
Improved non--adiabatic ground state energy
upper bound for dihydrogen,
Phys.Rev.Lett., accepted for publication.


%\bibitem{ref:A39}  
%L. Lapinski, M.J. Nowak, J. Fulara, A. Le\'{s} and 
%L. Adamowicz, Experimental and Theoretical Studies
%on the IR Spectra of 5--Methylcytosine,
%J. Phys. Chem. {\bf 94}, 6555 (1990).  


%\bibitem{ref:A40}  
%M.J. Nowak, L. Lapinski, H. Rostkowska, 
%A. Le\'{s}  and L. Adamowicz, Theoretical and Matrix-Isolation
%Experimental Study on 2(1H)--Pyridinethione/2--Pyridinethiol, 
%J. Phys. Chem., {\bf 94}, 7406 (1990).

%\bibitem{ref:A41}  
%M.J. Nowak, L. Lapinski, J. Fulara, A. Le\'{s}  and 
%L. Adamowicz, Theoretical and Infrared Matrix--Isolation Study on 
%4(3H)--Pyrimidinethione and 3(2H)--Pyridazinethione--Tautomerism and
%Phototautomerism, J. Phys. Chem. {\bf 95}, 2404 (1991).

%\bibitem{ref:A42}  
%J. Fulara, M.J. Nowak, L. Lapinski, A. Le\'{s}  and 
%L. Adamowicz, Theoretical and Matrix--Isolation
%Experimental Study of the Infrared Spectra of 5--Azauracil 
%and 6--Azauracil, Spectrochimica Acta {\bf 47A},
%595 (1991).

%\bibitem{ref:A43}  
%M. Nowak, L. Lapinski, Jan Fulara, A. Le\'{s}  and 
%L. Adamowicz, Matrix--Isolation IR Spectroscopy of
%Tautomeric Systems and its Theoretical Interpretation.  
%2--Hydroxy--Pyridine/2(1H)--Pyridinone, J. Phys.
%Chem. {\bf 96}, 1562 (1992).

%\bibitem{ref:A44}  
%A. Le\'{s}, L. Adamowicz, M.J. Nowak and L. Lapinski, 
%Theoretical Interpretation of the Gas Phase
%Equilibrium of 2--Hydroxypyridine/2(1H)--Pyridinone, 
%J. Mol. Structure {\bf 277}, 313 (1992).
%
%\bibitem{ref:A45}  
%L. Lapinski, M.J. Nowak, J. Fulara, A. Le\'{s}  and 
%L. Adamowicz, The Relation Between Structure and
%Tautomerism in Diazinones and Diazinethiones. An 
%Experimental Matrix--Isolation and Theoretical {\it Ab--Initio} Study, 
%J. Phys. Chem. {\bf 96}, 6250 (1992).

%\bibitem{ref:A46}  
%A. Le\'{s}, L. Adamowicz, M.J. Nowak and L. Lapinski, 
%Assignment of the Matrix--Isolation IR Spectra
%of Uracil and Tymine Based on New {\it Ab-Initio} Theoretical 
%Calculations, Spectrochim. Acta {\bf 48A} (10),
%1385 (1992).

%\bibitem{ref:A47}  
%H. Rostkowska, M.J. Nowak, L. Lapinski, 
%M. Bretner, T. Kulikowski, A. Le\'{s}  and L. Adamowicz,
%Infrared Spectra of 2--Thiocytosine and 5--Fluoro--2--Thiocytosine; 
%Experimental and {\it Ab-Initio} Studies,
%Spectrochim. Acta Part A, {\bf 49A}, 551 (1993).

%\bibitem{ref:A48}  
%H. Rostkowska, M.J. Nowak, L. Lapinski, M. Bretner, 
%T. Kulikowski, A. Le\'{s} and L. Adamowicz,
%Theoretical and Matrix--Isolation Experimental Studies 
%on 2--Thiocytosine and 5--Flouro--2--Thiocytosine,
%Biochimica et Biophysica Acta {\bf 1172}, 239 (1993).

%\bibitem{ref:A49}  
%H. Vranken, J. Smets, L. Lapinski, M.J. Nowak, 
%L. Adamowicz and G. Maes, Infrared Spectra and
%Tautomerism of Isocytosine.  An {\it Ab-Initio} and Matrix--Isolation 
%Study, Spectrochim. Acta Part A, {\bf 50A},
%875 (1994).

%\bibitem{ref:A50} 
%L. Lapinski, M.J. Nowak, A. Le\'{s}  and L. Adamowicz, 
%Photochemistry of Matrix--Isolated 4(3H)--Pyrimidinones, 
%J. Am. Chem. Soc. {\bf 116}, 1461 (1994).

%\bibitem{ref:A51}  
%A. Le\'{s}, L. Adamowicz, M.J. Nowak and L. Lapinski, 
%Concerted Biprotonic Tautomerism of 2--Hydroxypyridine, 
%J. Mol. Structure {\bf 312}, 157 (1994).

%\bibitem{ref:A52}  
%M.J. Nowak, A. Le\'{s}  and L. Adamowicz, Application 
%of {\it Ab-Initio} Quantum Mechanical Calculations
%to Assign Matrix--Isolation IR Spectra of Oxopyrimidines, 
%in Trends in Physical Chemistry, published
%by the Council of Scientific Information, India, 137-168 (1994).

%\bibitem{ref:A53}  
%L. Lapinski, M.J. Nowak, A. Le\'{s} and L. Adamowicz, 
%Comparison of {\it Ab--Initio} HF/6-31$G^{**}$, HF/6-31++$G^{**}$ 
%and MP2/6-31$^{**}$ IR Spectra of 4-Pyrimidinone Tautomers with 
%Matrix--Isolation Spectra,
%Vibrational Spectroscopy  {\bf 8}, 331 (1995).

%\bibitem{ref:A54}  
%L. Lapinski, D. Prusinowska, M.J. Nowak, 
%M. Bretner, K. Felczak, G. Maes and L. Adamowicz,
%Infrared Spectra of 6--aza--Thiouraciles:  
%An Experimental Matrix--Isolation and Theoretical {\it Ab-Initio}
%Study, Spectrochimica Acta {\bf 52}, 645 (1996).

%\bibitem{ref:A55}  
%D. Prusinowska, L. Lapinski, M.J. Nowak and 
%L. Adamowicz, Tautomerism, Phototautomerism and
%Infrared Spectra of Matrix--Isolated 2--Quinolinethione, 
%Spectrochemica Acta Part {\bf A51}, 1809 (1995).
%

%\bibitem{ref:A56}  
%M. Van Bael, K. Schoone, L. Houben,  J. Smets, 
%W. McCarthy, L. Adamowicz, M. Nowak and G. Maes,
%Matrix--Isolation FT--IR studies and {\it Ab-Initio} 
%Calculations of Hydrogen--bonded Complexes of Imidazole
%Comparison Between Experimental Data and 
%Different Theoretical Methods, J. Phys. Chem., submitted. 

%\bibitem{ref:A561}
%M. Rostkowska, M.J. Nowak, L. Lapinski, D. Smith, and L. Adamowicz,
%Molecular Structure and Infrared Spectra of
%3,4--Dihydroxy--3--Cyclobutene--1,2--Dione:
%Experimental Matrix--Isolation and Theoretical Hartree--Fock
%and Post Hartree--Fock Study, Spectrochimica Acta, 
%accepted for publication.  


%\bibitem{ref:A57}  
%A. Sobolewski and L. Adamowicz, Theoretical 
%Investigations of Proton Transfer Reaction in Hydrogen
%Bonded Complexes of Cytosine and Water, 
%J. Chem. Phys. {\bf 102}, 5708 (1995).

%\bibitem{ref:A58}  
%A. Sobolewski and L. Adamowicz, 
%Theoretical Investigations of the Excited--State Intra--molecular
%Proton Transfer Reaction in N--substituted--3--hydroxypyridinones, 
%Chem. Phys. Lett. {\bf 93}, 67 (1995).

%\bibitem{ref:A59}  
%A. Sobolewski and L. Adamowicz, Theoretical 
%Investigations of the Proton Transfer Reaction in
%Hydrogen--Bonded Complexes of Cytosine with HNO, 
%Chem. Phys. Lett. {\bf 234}, 94 (1995).

%\bibitem{ref:A60}  
%A. Sobolewski and L. Adamowicz, Theoretical 
%Investigations of the Proton Transfer Reaction in the
%Hydrogen--Bonded Complexes of 2--Pyrimidinone with Water, 
%J. Phys. Chem. {\bf 99}, 14277 (1995).

%\bibitem{ref:A61}  
%A. Sobolewski and L. Adamowicz, 
%Photophysics of 2--hydroxypyridine:  An {\it Ab-Initio} Study, J. Phys.
%Chem. {\bf 100}, 3933 (1996).

%\bibitem{ref:A62}  
%A. Sobolewski and L. Adamowicz, 
%Double--Proton Transfer in [2,2--Bipyridine]--3,3'--Diol:  
%An {\it Ab-Initio} Study, Chem. Phys. Lett. {\bf 252}, 33 (1996)


%\bibitem{ref:A63}  
%H. Saint--Martin, I. Ortega--Blake, A. Le\'{s}  and 
%L. Adamowicz, The Role of Hydration in the Hydrolysis
%of Pyrophosphate.  A Monte--Carlo Simulation 
%with Polarizable--type Interaction Potentials, Biochimica
%et Biophysica Acta {\bf 1207}, 12 (1994).

%\bibitem{ref:A64}  
%A. Le\'{s} and L. Adamowicz, {\it Ab-Initio} 
%Calculations of Biomolecules.  Proceedings of the CAM '94
%Physics Meeting in Cancun, Mexico, AIP 
%Conference Proceedings {\bf 342}, 190 (1995).


\bibitem{BO1927} M. Born and J. P. Oppenheimer, 
Zur Quantentheorie der Molekeln,
Annalen der Physik 
{\bf 84}, 457 (1927).

\bibitem{CS1980} J.M. Cobes and R. Seiler, Quantum Dynamics of Molecules,
ed. R. G.  Woolley, Plenum Press, New York, 1980, p. 435.

\bibitem{KM1992} M. Klein, A. Martinez, R. Seiler and X. Wang, Commun.
Math.Phys. {\bf 143}  607 (1992).

\bibitem{C}
A. Carington and R.A. Kennedy, Gas Phase Ion Chemistry;
Ed. M.T. Bowers, Academic Press, New York, vol.3, p.393.


\bibitem{ref:handy}
SPECTRO, written by J.F. Gaw, A. Willetts, W.H. Green, and N.C. Handy.

\bibitem{ref:ch21}
S. Carter and N. C. Handy, 
On the Calculation of Vibration--Rotation Energy Levels of
Quasi--Linear Molecules,
J.Mol.Spectrosc. {\bf 95}, 9 (1982).

\bibitem{ref:bowman}
J.M. Bowman and B. Gazdy, 
A Truncation/Recoupling Method for Basis Set Calculations of
Eigen-values and Eigen-functions,
J.Chem.Phys.
{\bf 94}, 454 (1991);
D.A. Jelski, R.H. Halay, and J.M. Bowman,
New Vibrational Self-Consistent Field Program
for Large Molecules,
J.Comp.Chem. {\bf 17}, 1645 (1996).


%\bibitem{ref:schwenke}
%D.W. Schwenke, 
%On the Computation of Ro--vibrational Energy Levels of
%Triatomic Molecules,
%Comput.Phys.Commun. {\bf 70}, 1 (1992).



%\bibitem{ref:podolski}
%B. Podolski, Quantum--Mechanically Correct Form of
%Hamiltonian Function for Conservative Systems,
%Phys.Rev. {\bf 32}, 812 (1928).

%\bibitem{ref:malloy}
%T.B. Malloy, J.Mol.Struct. {\bf 44}, 504 (1972).

%\bibitem{ref:wilson}
%E.B. Wilson, J.C. Decius, and P.C. Cross, in {\em
%Molecular Vibrations}, McGraw-Hill, New York 1972.


%\bibitem{ref:eckard}
%H.M. Pickett, 
%Vibration--Rotation Interactions and the Choice of
%Rotating Axes for Polyatomic Molecules,
%J.Chem.Phys. {\bf 56}, 1715 (1971).



%\bibitem{Bartlett95}
%R.~J. Bartlett,
%\newblock Coupled-cluster theory: An overview of recent developments,
%\newblock in {\em Modern Electronic Structure Theory}, edited by D.~R. Yarkony,
%World Scientific, Singapore, 1995.

\bibitem{Eckart35}
C.~Eckart,
Some Studies Concerning Rotating Axes and Polyatomic
Molecules,
\newblock Phys.Rev. {\bf 47}, 552 (1935).

\bibitem{WilsonHoward36}
E.~B. Wilson and J.~B. Howard,
The Vibration--Rotation Energy Levels
of Polyatomic Molecules,
\newblock J.Chem.Phys. {\bf 4}, 260 (1936).

\bibitem{Watson68}
J.~K.~G. Watson,
Simlification of the Molecular Vibration-Rotation
Hamiltonian,
Mol.Phys. {\bf 15}, 479 (1968);
{\it ibid.} 
The Vibration--Rotation Hamiltonian of Linear Molecules,
{\bf 19}, 465 (1970).


%\bibitem{SutcliffeTennyson87}
%B.~T. Sutcliffe and J.~Tennyson,
%\newblock J.Chem.Soc.Faraday Trans. 2 {\bf 83}, 1663
%(1987).

\bibitem{Jensen88}
P.~Jensen,
A new Morse Oscillator--Rigid Bender Internal
Dynamics (MORBID) Hamiltonian for Triatomic Molecules,
\newblock J.Mol.Spectr. {\bf 128}, 478 (1988).

\bibitem{LeRoy95}
R.~J. Le{R}oy,
\newblock {LEVEL 6.0} a 
computer program for bound and quasibound levels, and
calculating various expectation values and matrix elements,
\newblock Technical report, University of Waterloo.

\bibitem{Hutson93}
J.~M. Hutson,
\newblock {BOUND}: A program for calculating bound-state energies for weakly
bound molecular complexes, version 1 (1984) to version 5 (1993),
\newblock Computational Project No. 6 of the Science and Engineering Research
 Council, on Heavy Particle Dynamics, 1993.
%\newblock e-mail: J.M.Hutson@durham.ac.uk.

\bibitem{ref:ten1}
J. Tennyson, J. R. Henderson, and N. G. Fulton,
DVR3D: For the Fully Pointwise Calculations of Ro-vibrational
Spectra of Triatomic Molecules,
Comp.Phys.Comm. {\bf 86}, 175 (1995).



\bibitem{ref:kcn1}
J. Tennyson and B. T. Sutcliffe, 
{\it Ab--initio} Vibrational--Rotational Spectrum of Potasium Cyanide,
KCN,
Mol.Phys. {\bf 46}, 97 (1982).


\bibitem{ref:kcn2}
J. Tennyson and A. van der Avoird, 
{\it Ab--initio} Vibrational--Rotational Spectrum
of Potasium Cyanide, KCN. II. Large Amplitude Motions
and Ro-vibrational Coupling,
J.Chem.Phys. 
{\bf 76}, 5710 (1982).


\bibitem{ref:kcn3}
J. Tennyson and B. T. Sutcliffe, 
The {\it Ab--initio} Calculation of the 
Vibrational--Rotational Spectrum of Triatomic Systems in the  
Close--Coupling Approach with KCN and H$_2$Ne as Examples,
J.Chem.Phys. {\bf 77}, 4061 (1982).


\bibitem{ref:ch22}
B. T. Sutcliffe and J. Tennyson, 
Variational Methods for the Calculation of Ro-vibrational Energy
Levels of Small Molecules,
J.Chem.Soc.Faraday Trans. 2, 
{\bf 83}, 1663 (1987).


\bibitem{ref:ch23}
J. S. Lee and D. Secrest, 
A Calculation of the Rotation--Vibration States of He$_2$H$^+$,
J.Chem.Phys. {\bf 85}, 6565 (1986).

\bibitem{ref:j22}
C.-L. Chen, B. Maessen, and M. Wolfsberg, 
Variational Calculations of Rotational--Vibrational Energy
Levels of Water,
J.Chem.Phys. {\bf 83}, 1795 
(1985).

\bibitem{ref:j26}
B. T. Sutcliffe, in {\em Current Aspects of Quantum Chemistry}
(R. Carbo, ed.), Studies in Theoretical Chemistry, Vol. 21,
pp 99-125, Elsevier, New York (1982).

\bibitem{ref:j27}
N. C. Handy, 
The Derivation of Vibration Kinetic Energy
operator in Internal Coordinates,
Mol.Phys. {\bf 61}, 207 (1987).

\bibitem{ref:j32}
B. T. Sutcliffe and J. Tennyson, 
A Generalized Approach to the Calculation of Ro--vibrational Spectra
of Triatomic Molecules,
Mol.Phys. {\bf 58}, 1053 (1986).

\bibitem{ref:j33}
D. Estes and D. Secrest, 
The Vibration--Rotation Hamiltonian; A Unified
Treatment of Linear and Non--Linear Molecules,
Mol.Phys. {\bf 59}, 569 (1986).

\bibitem{ref:j34}
J. Makarewicz, 
Ro-vibrational Hamiltonian of a Triatomic Molecule in 
Local and Collective Internal Coordinates,
J.Phys. B: At.Mol.Opt.Phys. {\bf 21}, 1803 (1988).

\bibitem{ref:j35}
J. Makarewicz and W. Lodyga, 
Self--Consistent Internal Axes for a Rotating-Vibrating
Triatomic Molecules,
Mol.Phys. {\bf 64}, 899 (1988).

\bibitem{ref:j36}
J. M. Bowman, J. Zuniga, and A. Wierzbicki, 
Investigations of Transformed Mass--Selected Jacobi Coordinates
for Vibrations of Polyatomic Molecules with Applications to H$_2$O,
J.Chem.Phys. {\bf 90}, 2708 (1989).

\bibitem{ref:j37}
B. T. Sutcliffe and J. Tennyson, 
A General Treatment of Vibrational--Rotational Coordinates
For Triatomic Molecules,
Int.J.Quantum Chem. {\bf 39},
183 (1991).

\bibitem{ref:j25}
R. M. Whitnell and J. C. Light, J.Chem.Phys. 
Efficient Pointwise Representations for
Vibrational Wave Functions: Eigen-functions of H$_3^+$,
{\bf 90}, 1774 (1989).

\bibitem{ref:j45}
Z. Ba\u{c}i\'{c} and J. C. Light, 
Theoretical Methods for Ro-vibrational States of Floppy 
Molecules,
Ann.Rev.Phys.Chem. {\bf 40}, 
469 (1989).

\bibitem{ref:j46}
S. E. Choi and J. C. Light, 
Determination of the Bound and Quasibound States of Ar--HCl
van--der--Waals Complex:
Discrete Variable Representation Method,
J.Chem.Phys. {\bf 92}, 2129 (1990).

\bibitem{ref:j50}
J. Tennyson and B. T. Sutcliffe, 
Highly Rotationally Excited
States of Floppy Molecules: H$_2$D$^+$ with $J \leq 20$,
Mol. Phys. {\bf 58}, 1067 (1986).


\bibitem{Dunham32a}
J.~L. Dunham,
The Wentzel--Brillouin--Kramers Method of Solving
the Wave Equation,
\newblock Phys.Rev. {\bf 41}, 713 (1932).

\bibitem{Dunham32b}
J.~L. Dunham,
The Energy Levels of a Rotating Vibrator,
\newblock Phys.Rev. {\bf 41}, 721 (1932).

\bibitem{Simons73}
G.~Simons, R.~G. Parr, and J.~M. Finlan,
New Alternative to the Dunham Potential for Diatomic Molecules,
\newblock J.Chem.Phys. {\bf 59}, 3229 (1973).

\bibitem{Ogilvie81}
J.~F. Ogilvie,
A General Potential Energy Function for Diatomic Molecules,
\newblock Proc.Royal Soc. London A {\bf 378}, 287 (1981).

\bibitem{Thakkar75}
A.~Thakkar,
A New Generalized Expansion for the Potential Energy 
Curves of Diatomic Molecules,
\newblock J.Chem.Phys. {\bf 62}, 1693 (1975).

\bibitem{Huffaker76}
J.~N. Huffaker,
Diatomic Molecules as Morse Oscillators. 1. Energy Levels,
\newblock J.Chem.Phys. {\bf 64}, 3175 (1976).

\bibitem{Morse29}
P.~M. Morse,
Diatomic Molecules According to the Wave Mechanics. II. 
Vibrational Levels.
\newblock Phys.Rev. {\bf 34}, 57 (1929).

\bibitem{Jordan74}
K.~D. Jordan, J.~L. Kinsey, and R.~Silbey,
Use of Pade' Approximants in the Construction of 
Diabatic Potential Energy Curves for Ionic Molecules,
\newblock J.Chem.Phys. {\bf 61}, 911 (1974).

%\bibitem{Jorish79}
%V.~S. Jorish and N.~B. Shcherbak,
%\newblock Chem.Phys.Lett. {\bf 67}, 160 (1979).

\bibitem{Sonnleitner81}
S.~A. Sonnleitner, C.~L. Beckel, A.~J. Colucci, and E.~R. Scaggs,
Rational Fraction Representation of Diatomic 
Vibrational Potentials. V. The $^{3} \Sigma_g^+ $ State of H$_2^+$,
\newblock J.Chem.Phys. {\bf 75}, 2018 (1981).

\bibitem{Pardo86}
A.~Pardo, J.~J. Camacho, and J.~M.~L. Poyato,
The Pade' Approximant Method and its Applications to the Construction of
Potential Energy Curves for the Lithium Hydride Molecule,
\newblock Chem.Phys.Lett. {\bf 131}, 490 (1986).


\bibitem{Jorish79}
J.M.L. Martin, T.J. Lee, and P.R. Taylor,
An Accurate {\it Ab-initio} Quadratic Force Field for
Formaldehyde and its Isotopomers,
J.Mol.Spectr. {\bf 160}, 105 (1993).

%\bibitem{ref:t2}
%J.M.L. Martin, P.R. Taylor, J.T. Yustein, T.R. Burkholder, and L. Andrews,
%Pulsed Laser Evaporation of Boron/Carbon pellets:
%Infrared Spectra and Quantum Chemical Structures and Frequencies for 
%$BC_2$,
%J.Chem.Phys. {\bf 99}, 12 (1993).
%
%\bibitem{ref:t3}
%T.J. Lee, C.E. Dateo, B. Gazdy and J.M. Bowman,
%Accurate Quartic Force Fields and Vibrational Frequencies for HCN and HNC,
%J.Phys.Chem. {\bf 97}, 8937 (1993).

\bibitem{Dateo94}
C.E. Dateo, T.J. Lee, and D.W. Schwenke,
An Accurate Quartic Force Field and Vibrational Frequencies
for $HNO$ and $DNO$,
J.Chem.Phys. {\bf 101}, 5853 (1994).

%\bibitem{ref:t5}
%J.M.L. Martin and P.R. Taylor,
%{\it Ab-initio} Study of the Isoelectronic Molecules
%$BCN$, $BNC$, and $C_3$ Including Anharmonicity,
%J.Phys.Chem. {\bf 98}, 6105 (1994).

%\bibitem{ref:t6}
%J.M.L. Martin and P.R. Taylor,
%{\it Ab-initio} Study of the Molecules $BC$ and $B_2C$,
%J.Chem.Phys, {\bf 100}, 9002 (1994).

%\bibitem{ref:t7}
%J.M.L. Martin, P.R. Taylor,
%J.P. Fran\c{c}ois and R. Gijbels,
%{\it Ab-initio} Study of the Spectroscopy and Thermochemistry
%of the $C_2N$ and $CN_2$ molecules.
%Chem.Phys.Lett. {\bf 226}, 475 (1994).

%\bibitem{ref:t8}
%J.M.L. Martin, P.R. Taylor,
%J.P. Fran\c{c}ois and R. Gijbels,
%{\it Ab-initio} Study of the Spectroscopy, Kinetics and 
%Thermochemistry
%of the $BN_2$ molecule,
%Chem.Phys.Lett. {\bf 222}, 517 (1994).

\bibitem{Lee95a}
T.J. Lee, J.M.L. Martin and P.R. Taylor,
An Accurate {\it Ab-initio} Quartic Force Field and Vibrational
Frequencies for $CH_4$ and Isotopomers,
J.Chem.Phys. {\bf 102}, 254 (1995).

\bibitem{Martin95}
J.M.L. Martin, T.J. Lee, P.R. Taylor,
and J.P. Fran\c{c}ois,
The Anharmonic Force Field of Ethylene, $C_2H_4$, by Means of
Accurate {\it Ab-initio} Calculations,
J.Chem.Phys. {\bf 103}, 2589 (1995).

\bibitem{Lee95b}
T.J. Lee,
J.M.L. Martin, C.E. Dateo, and P.R. Taylor,
Accurate {\it Ab-initio} Quartic Force Fields,
Vibrational Frequencies, and Heats of Formation
for $FCN$, $FNC$, $ClCN$, and $ClNC$,
J.Phys.Chem. {\bf 99}, 15858 (1995).

\bibitem{ref:t12}
J.M.L. Martin and P.R. Taylor,
The Geometry, Vibrational Frequencies, and Total Atomization
Energy of Ethylene. A Calibration Study,
Chem.Phys.Lett. {\bf 248}, 336 (1996).

\bibitem{ref:t13}
J.M.L. Martin, D.W. Schwenke, T.J. Lee, and P.R. Taylor,
Is There Evidence for Detection of Cyclic $C_4$ in IR spectra?
An Accurate {\it ab-initio} Computed Quartic Force Field,
J.Chem.Phys. {\bf 104}, 4657 (1996).

\bibitem{ref:t14}
J.M.L. Martin and P.R. Taylor,
Structure and Vibrations of Small Carbon Clusters from 
Coupled--Cluster Calculations,
J.Phys.Chem. {\bf 100}, 6047 (1996).

%\bibitem{Lee95a}
%T.~J. Lee, J.~M.~L. Martin, and P.~R. Taylor,
%\newblock J.Chem.Phys. {\bf 102}, 254 (1995).

%\bibitem{Dateo94}
%C.~E. Dateo, T.~J. Lee, and D.~W. Schwenke,
%\newblock J.Chem.Phys. {\bf 101}, 5853 (1994).

%\bibitem{Martin93}
%J.~M.~L. Martin and T.~J. Lee,
%\newblock J.Mol.Spectr. {\bf 160}, 105 (1993).

%\bibitem{Lee95b}
%T.~J. Lee, J.~M.~L. Martin, C.~E. Dateo, and P.~R. Taylor,
%\newblock J.Phys.Chem. {\bf 99}, 15858 (1995).

%\bibitem{Martin95}
%J.~M.~L. Martin, T.~J. Lee, P.~R. Taylor, and J.-P. Fran\c{c}ois,
%\newblock J.Chem.Phys. {\bf 103}, 2589 (1995).

\bibitem{Koizumi91}
H.~Koizumi and G.~C. Schatz,
A Coupled Channel Study of HN$_2$ Unimolecular Decay Based on a
Global {\it Ab-initio} Potential Surface,
\newblock J.Chem.Phys. {\bf 95}, 4130 (1991).

\bibitem{Bentley92}
J.~A. Bentley, J.~M. Bowman, and B.~Gazdy,
A Global {\it ab--initio} Potential for HCN/HNC,
exact vibrational energies, and comparison to experiment,
\newblock Chem.Phys.Lett. {\bf 198}, 563 (1992).

\bibitem{Bowman75}
J.~M. Bowman and A.~Kuppermann,
A Semi-Numerical Approach to the Construction and Fitting of Triatomic 
Potential Energy Surfaces,
\newblock Chem.Phys.Lett. {\bf 34}, 523 (1975).

\bibitem{Connor75}
J.~N.~L. Connor, W.~Jakubetz, and J.~Manz,
Exact Quantum Transition Probabilities by the State Path Sum Method:
Collinear F + H$_2$ Reaction,
\newblock Mol.Phys. {\bf 29}, 347 (1975).

\bibitem{Forsythe77}
G.~E. Forsythe,
\newblock {\em Computer methods for mathematical computations},
\newblock Prentice-Hall, Englewood Cliffs, N.J., 1977.

\bibitem{Malik80}
D.~J. Malik, J.~Eccles, and D.~Secrest,
On Quantal Bound State Solutions and Potential Energy Surface Fitting.
A Comparison of the Finite Element Numerov-Cooley and Finite Difference 
\newblock J.Comp.Phys. {\bf 38}, 157 (1980).

\bibitem{Sathyamurthy75}
N.~Sathyamurthy and L.~M. Raff,
Quasiclassical Trajectory 
Studies Using 3D Spline Interpolation of {\it Ab--initio} Surfaces,
\newblock J.Chem.Phys. {\bf 63}, 464 (1975).

\bibitem{Dunne87}
S.~J. Dunne, D.~J. Searles, and E.~I. {von Nagy-Felsobuki},
{\it Ab--initio} 
Model of the Raman Spectrum of Li$_3^+$ : Breathe Mode Frequencies,
\newblock Spectrochim. Acta {\bf 43A}, 699 (1987).

\bibitem{Bruehl88}
M.~Bruehl and G.~Schatz,
Theoretical Studies of Collisional Energy Transfer in Highly-excited 
Molecules: Temperature and Potential Surface Dependence of Relaxation
in H$_2$, Ne, Ar + CS$_2$,
\newblock J.Phys.Chem. {\bf 92}, 7223 (1988).

\bibitem{Murrell84}
J.~N. Murrell, S.~Carter, 
S.~C. Farantos, P.~Huxley, and A.~J.~C. Varandas,
\newblock {\em Molecular Potential Energy Functions},
\newblock John Wiley, Chichester, 1984.

\bibitem{Sorbie75}
K.~S. Sorbie and J.~N. Murrell,
Analytical Potentials for Triatomic Molecules from Spectroscopic Data,
\newblock Mol.Phys. {\bf 29}, 1387 (1975).

\bibitem{Whitehead76}
R.~J. Whitehead and N.~C. Handy,
Variational Calculation of Low--Lying and Excited Vibrational Levels of
the Water Molecule,
\newblock J.Mol.Spectr. {\bf 59}, 459 (1976).


\bibitem{Carter82}
S.~Carter, I.~M. Mills, J.~N. Murrell, and A.~J.~C. Varandas,
Analytical Potentials for Triatomic Molecules IX. The Prediction of 
Anharmonic Force Constants from Potential Energy Surfaces Based on
Harmonic Force Fields and Dissociation Energies for SO$_2$ and O$_3$,
\newblock Mol.Phys. {\bf 45}, 1053 (1982).

\bibitem{Searles93}
D.~Searles and E.~{von Nagy-Felsobuki},
\newblock {\em Ab Initio Variational Calculations of Molecular
Vibrational-Rotational Spectra},
\newblock Springer-Verlag, Berlin, 1993.

\bibitem{Mezey87}
P.~G. Mezey,
\newblock {\em Potential Energy Hypersurfaces},
\newblock Elsevier, Amsterdam, 1987.



\bibitem{kozlowski91}
P. M. Kozlowski and L. Adamowicz, 
An Effective Method for Generating Nonadiabatic
Many--Body Wave Functions Using Explicitly Correlated
Gaussian--Type Functions,
J. Chem. Phys., {\bf 95}, 6681 (1991).

\bibitem{kozlowski92a}
P. M. Kozlowski and L. Adamowicz, 
Multi-Center and Multi-particle Integrals for
Explicitly Correlated Gaussian-Type Functions,
J. Comput. Chem., {\bf 13}, 602 (1992).


\bibitem{kozlowski92b}
P. M. Kozlowski and L. Adamowicz, 
Implementation of Analytical First Derivatives for
Evaluation of the Many--Body Non-Adiabatic Wave Function
with Explicitly Correlated Gaussian Functions,
J. Chem. Phys., {\bf 96}, 9013 (1992).

\bibitem{kozlowski92c}
P. M. Kozlowski and L. Adamowicz, 
Newton--Raphson Optimization of the Many Body
Non-Adiabatic Wave Function Expressed in Terms of Explicitly
Correlated Gaussian Functions,
J. Chem. Phys., {\bf 97}, 5063 (1992).


%\bibitem{ref:A3}   
%P.M. Kozlowski and L. Adamowicz, 
%Phys. Rev. {\bf A 48}, 1903 (1993).
%
%
%\bibitem{ref:A6}   
%P.M. Kozlowski and L. Adamowicz, 
%Chem. Rev. {\bf 93}, 2007 (1993).
%
%\bibitem{ref:A8}   
%Z. Zhang, P.M. Kozlowski and L. Adamowicz,
%J. Comp. Chem.
%{\bf 15}, 54 (1994).
%
%\bibitem{ref:A14}  Z. Zhang and L. Adamowicz, 
%J. Comp. Chem. {\bf 15}, 893 (1994).
%
%
%\bibitem{ref:A16}  P.M. Kozlowski and L. Adamowicz, 
%Int. J. Quant. Chem. {\bf 55}, 245 (1995).
%
%\bibitem{ref:A20}  
%P. M. Kozlowski and L. Adamowicz, 
%Int. J. Quant. Chem. {\bf 55}, 367 (1995).
%
%\bibitem{ref:A21}  
%Z. Zhang and L. Adamowicz, 
%Int. J. Quantum Chem.  {\bf 54}, 281 (1995).
%
%
%\bibitem{ref:A27}  
%P. M. Kozlowski and L. Adamowicz, 
%J. Phys. Chem. {\bf 100}, 6266 (1996).
%
%
%\bibitem{ref:A37}  
%D.B. Kinghorn and L. Adamowicz, 
%J. Chem. Phys., accepted for publication.
%
%\bibitem{ref:A38}  
%D. W. Gilmore, P.M. Kozlowski, D.B. Kinghorn and 
%L. Adamowicz,
%Int. J. Quantum Chem., accepted for publication.
%
%\bibitem{ref:A382}
%D.B. Kinghorn and L. Adamowicz,
%J.Chem.Phys., submitted.
%

\bibitem{kpc}
D.~B. Kinghorn and L. Adamowicz, we have recently derived
compact integral algorithms allowing for negative $m_{kij}$ powers
of $r_{ij}$ in the $\phi_k$ functions; unpublished result.


\bibitem{Kinghorn95a}
D.~B. Kinghorn,
Implementation of Gradient Formulas for Correlated
Gaussians: He, $^{\infty}$He, Ps$_2$, $^9$Be, and $^{\infty}$Be
test results,
\newblock Int.J.Quantum Chem. {\bf 57}, 141 (1996).

\bibitem{Kinghorn93}
D.~B. Kinghorn and R.~D. Poshusta,
Nonadiabatic Variational Calculations on Dipositronium
Using Explicitly Correlated Gaussian Basis Functions,
\newblock Phys.Rev. A {\bf 47}, 3671 (1993).

\bibitem{Kinghorn95b}
D.~B. Kinghorn and R.~D. Poshusta,
Density Matrices for Correlated Gaussians: Helium and
Dipositronium,
\newblock Int.J.Quantum Chem. {\bf 60}, 213 (1996).

\bibitem{Poshusta83}
R.~D. Poshusta,
Nonadiabatic Singer Polynomial Wave Functions for 
Three--Particle Systems,
\newblock Int.J.Quantum Chem. 
{\bf 24}, 65 (1983).


\bibitem{k15}
H. A. Bethe and E.E. Salpeter, {\rm Quantum Mechanics of One- and
Two--Electron Systems}, Academic Press, New York, 1957.

\bibitem{kk}
W. Kolos, Hydrogen Molecule. Test of Quantum Chemistry,
Polish J.Chem., {\bf 67}, 553 (1993).



\bibitem{sz1}
R. Bukowski, B. Jeziorski, and K. Szalewicz,
New Effective Strategy of Generating Gaussian-type Geminal
Basis Sets for Correlation Energy Calculations,
J.Chem.Phys. {\bf 100}, 1366 (1994);
R. Bukowski, B. Jeziorski, S. Rybak, and K. Szalewicz,
Second-order Correlation energy for H$_2$O Using
Explicitly Correlated Gaussian Geminals,
J.Chem.Phys. {\bf 102}, 888 (1995);
R. Bukowski, B. Jeziorski, and K. Szalewicz,
Basis Set Superposition Problem in Interaction Energy
Calculations with Explicitly Correlated Bases:
Saturated Second- and Third-Order Energies for He$_2$,
J.Chem.Phys. {\bf 104}, 3306 (1996);
H.L. Williams, T. Korona, R. Bukowski, B. Jeziorski, and K. Szalewicz,
Helium Dimer Potential from Symmetry-Adapted
Perturbation Theory,
Chem.Phys.Lett. {\bf 262}, 431 (1996);
T. Korona, H.L. Williams, R. Bukowski,
B. Jeziorski, and K. Szalewicz,
Helium Dimer Potential from Symmetry Adapted Perturbation Theory
Calculations Using Large Gaussian Geminal and Orbital Basis Sets,
J.Chem.Phys. {\bf 106}, 5109 (1997);
and reference therein.


\bibitem{Biedenharn81}
L.~C. Biedenharn and J.~D. Louck,
\newblock {\em Angular Momentum in Quantum Physics. Theory and Application},
\newblock Encyclopedia of Mathematics and Its Applications, Addison-{W}esley,
Reading, {MA}, 1981.

\bibitem{Varga95}
K.~Varga and Y.~Suzuki,
Precise Solution of Few--Body Problems with the
Stochastic Variational Method on a Correlated Gaussian Basis,
\newblock Phys.Rev. C {\bf 52}, 2885 (1995).


\bibitem{Varga96}
K.~Varga and Y.~Suzuki,
Global Vector Representation of the Angular Motion of
Few--Particle Systems,
\newblock Phys.Rev. C  (1997),
\newblock in press.

\bibitem{Kolos65}
W.~Kolos and L.~Wolniewicz,
Potential--Energy Curves for the X$^{1} \Sigma _g^+$, 
b $^{3}\Sigma_u ^+,$
and C $^{1} \Pi_u$ States of the Hydrogen Molecule,
\newblock J.Chem.Phys. {\bf 43}, 2429 (1965).

\bibitem{Mathematica}
Wolfram Research, Inc., 
100 Trade Center Drive, Champaign, Illinois 61820-7237
USA,
\newblock {\em Mathematica}.

\bibitem{Wolniewicz66}
L.~Wolniewicz,
Vibrational--Rotational Study of the Electronic Ground State 
of the Hydrogen Molecule,
\newblock J.Chem.Phys. {\bf 45}, 515 (1966).

\bibitem{LeRoy68}
R.~J. Le{R}oy and R.~B. Bernstein,
Dissociation Energy and Vibrational Terms of Ground State 
(X$^{1} \Sigma _g ^+$) Hydrogen,
\newblock  J.Chem.Phys. {\bf 49}, 4312 (1968).



\bibitem{k29}
W.C. Stwalley, Chem.Phys.Lett. {\bf 6}, 241 (1970).

\bibitem{k42}
E.E. Eyler and N. Melikechi, Near-threshold continuum structure
and the dissociation energies of H$_2$, HD, and D$_2$,
Phys.Rev. A {\bf 48}, R18-R21, 1993.

\bibitem{k42p}
J.M. Gilligan and E.E. Eyler,
Precise determinations of ionization potentials and EF--state
energy levels of H$_2$, HD and D$_2$,
Phys.Rev. A {\bf 46}, 3676 (1992).

\bibitem{k52}
W.L. Glab and J.P. Hessler,
Multiphoton excitation of high singlet $np$ Rydberg states of
molecular hydrogen - spectroscopy and dynamics,
Phys.Rev. A {\bf 35} 2102 (1987).

\bibitem{k53}
E. McCormack, J.M. Gilligan, C. Cornaggie, and E.E. Eyler,
Measurment of high Rdberg states and the ionization potential
of H$_2$,
Phys.Rev. A {\bf 39}, 2260 (1989).

\bibitem{k57}
C. Jungen, I. Dabrowski, G. Herzberg, and M. Vervloet,
The ionization potential of D$_2$,
J.Mol.Spectr., {\bf 153}, 11 (1992).

\bibitem{k12}
W. Kolos and J. Rychlewski,
Improved theoretical dissociation energy and
ionization potential for the ground state of the
hydrogen molecule,
J.Chem.Phys. {\bf 98}, 3960 (1993).


\bibitem{k58}
C.A. Leach and R.E. Moss,
Spectroscopy and quantum--mechanics of the hydrogen 
molecular cation - a test of molecular quantum mechanics,
Annu.Rev.Phys.Chem. {\bf 46}, 55 (1995). 


\bibitem{k59}
R. Bukowski, B. Jeziorski, R. Moszynski, and W. Kolos,
Bethe logarithm and Lamb shift for the hydrogen molecular 
ion,
Intern.J.Quantum Chem. {\bf 42}, 287 (1992).

\bibitem{w1}
D.M. Bishop and L.M. Cheung,
Rigorous theoretical investigation of the ground state of H$_2$,
Phys.Rev. A {\bf 18}, 1846 (1978).

\bibitem{w2}
L. Wolniewicz,
Nonadiabatic energies of the ground state of the hydrogen
molecule,
J.Chem.Phys. {\bf 103}, 1792 (1995).



\bibitem{w9}
P. Quadrelli, D. Dressler, and L. Wolniewicz,
Weak predissociation of the EF, GK, and H$^1\Sigma^+_g$ states
of the H$_2$ molecule by non--adiabatic coupling
with the electronic ground state,
J.Chem.Phys. {\bf 93}, 4958 (1990).


\bibitem{w10}
S. Yu and K. Dressler, 
Improved theoretical $^1\Sigma_u^+$ and $^1\Pi_u$ wave functions,
energies, and transition moments fro the H$_2$, D$_2$, and T$_2$,
J.Chem.Phys. {\bf 101}, 7692 (1994).

\bibitem{w28}
E. Reinhold, W. Hogervorst, and W. Ubachs,
Complete g--u symmetry breaking in highly excited valence states
of the HD molecule,
Chem.Phys.Lett. {\bf 296}, 411 (1998).

\bibitem{ww}
E. Reinhold, W. Hogervorst, W. Ubachs, and L. Wolniewicz,
Experimental and theoretical investigation of the 
$H\overline{H}^1\Sigma_g^+$ state in H$_2$, D$_2$, and HD,
and the $B"\overline{B}^1\Sigma_u^+$ state in HD,
Phys.Rev. A {\bf 60}, 1258 (1999).


\bibitem{ref:k35}
T. Oka, 
Observation of the Infrared Spectrum of H$_3^+$, 
Phys.Rev.Lett. {\bf 45}, 531 (1980).

\bibitem{ref:k36}
G. D. Carney and R. N. Porter, 
{\it Ab--Initio} Prediction of the Rotation--Vibration 
Spectrum of H$_3^+$ and D$_3^+$,
Phys.Rev.Lett. {\bf 45}, 537 (1980);
H$_3^+$: Geometry Dependence of Electronic Properties, J.Chem.Phys. 
J.Chem.Phys. {\bf 60}, 4251 (1974);
H$_3^+$: {\it Ab--Initio} Calculation of the Vibration Spectrum,
J.Chem.Phys. {\bf 65}, 3547 (1976).

\bibitem{ref:k41}
P. Drossart, J.-P. Maillard, J. Caldwell, S. J. Kimm,
J. K. G. Watson, W. A. Majewski, J. Tennyson, S. Miller, S. K. Atreya,
J. T. Clarke, J. W. Waite, Jr., and R. Wagener, 
Detection of H$_3^+$ on Jupiter,
Nature {\bf 340},
539 (1989).

\bibitem{ref:k44}
T. R. Geballe, M.-F. Jagod and T. Oka, 
Detection of H$_3^+$ Infrared Emission Lines
in Saturn,
Astrophys. J. {\bf 408},
L108 (1993).

\bibitem{ref:k45}
L. Trafton, T. R. Geballe, S. Miller, J. Tennyson, and
G. E. Ballester, 
Detection of H$_3^+$ from Uranus,
Astrophys. J. {\bf 405}, 761 (1993).

\bibitem{ref:ten2}
L. Neale, S. Miller, and J. Tennyson, 
Spectroscopic Properties of the H$_3^+$ molecule:
a New Calculated Line List,
Astrophys. J. {\bf 464}, 516
(1996).

\bibitem{ref:ten3}
B. M. Dinelli, O. L. Polyansky and J. Tennyson, 
Spectroscopically Determined Born--Openheimer and
adiabatic surfaces for 
H$_3^+$, H$_2$D$^+$, D$_2$H$^+$, and D$_3^+$,
J.Chem.Phys. {\bf 103},
10433 (1995).

\bibitem{ref:ten4}
J. Tennyson, in {\em Physics World}, 
H$_3^+$: from Chaos to the Cosmos,
vol.{\bf 8}, p. 33 (1995).

\bibitem{ref:ten5}
O. L. Polyansky, B. M. Dinelli, C. R. Le Sueur, and J. Tennyson,
Asymmetric Adiabatic Correction to the Rotation--Vibration levels
of H$_2$D$^+$ and D$_2$H$^+$,
J.Chem.Phys. {\bf 102}, 9322 (1995).

\bibitem{ref:ten6}
J. Tennyson, 
Spectroscopy of H$_3^+$: Planets, Chaos and the Universe,
in {\em Reports on Progress in Physics}, {\bf 58},
421 (1995).

\bibitem{ref:ten7}
B. M. Dinelli, C. R. Le Sueur, J. Tennyson, and R. D. Amos,
{\it Ab--initio} Ro--vibrational Levels of H$_3^+$ beyond the
Born--Oppenheimer Approximation,
Chem.Phys.Lett. {\bf 232}, 295 (1995).

\bibitem{ref:ten8}
S. Miller, An Lam. Hoanh, and J. Tennyson,
What Astronomy Has Learn from Observation of H$_3^+$,
Can.J.Phys. {\bf 72}, 760 (1994).

\bibitem{ref:ten9}
J. Tennyson and S. Miller, 
H$_3^+$: from First Principles to Jupiter,
Contemp.Phys. {\bf 35}, 105 (1994).

\bibitem{mccall}
B.J. McCall, T.R. Geballe, K.H. Hinkley and T. Oka,
Observations of H$_3^+$ in dense molecular clouds,
Astophys.J. {\bf 522}, 338 (1999).

\bibitem{stark}
R. Stark, F.F.S. van der Tak, and E.F. van Dishoeck,
Astrophys.J. {\bf 521}, L67 (1999).

\bibitem{azinovic}
D. Azinovic, R. Bruckmeister, C. Wunderlich, H. Figger,
G. Theodorakopoulos, and I.D. Petsalakis,
Dynamics on the ground--state potential surfaces of H$_3$
and its isotopomers from their UV spectra,
Phys.Rev. A {\bf 58}, 1115 (1998).

\bibitem{muller}
U. Muller and P.C. Cosby,
Three--body decay of the 3s(2) A(1)' (N=1,K=0) and 3d E-2"
(N=1,G=0,R=1) Rydberg states of the triatomic hydrogen
molecule H$_3$,
Phys.Rev. A {\bf 59}, 3632 (1999).

\bibitem{rob}
F. Robicheaux,
Simple asymptotic potential model for finding
weakly bound negative ions,
Phys.Rev. {\bf 60}, 1706 (1999).

\bibitem{sarkas}
H.W. Sarkas, J.H. Hendricks, S.T. Arnold, and K.H. Bowen,
Photoelectron spectroscopy of lithium hydride anion,
J.Chem.Phys. {\bf 100}, 1884 (1994).

\bibitem{ref:bern1}
L. Wallace, P. F. Bernath, W. Livingston, K. Hinkle, J. Busler,
B. Guo, and K. Zhang, 
Water on the Sun,
Science {\bf 268}, 1155 (1995).

\bibitem{ref:ten10}
F. Allard, P. H. Hauschildt, S. Miller, and J. Tennyson,
The Influence of H$_2$O Line Blanketing on the Spectra of
Cool Dwarf Stars,
Astroph.J.,Lett. {\bf 426}, L39 (1994).

\bibitem{ref:ten11}
C. D. Paulse and J. Tennyson, 
An Empirical Potential Energy Surface for Water Accounting for
States with High Angular Momentum,
J.Mol.Spectr. {\bf 168}, 313 (1994).

\bibitem{ref:ten12}
A. E. Lynas--Gray, S. Miller, and J. Tennyson,
Infrared Transition Intensities for Water: A Comparison of
{\it Ab--initio} and Fitted Dipole Moment Surfaces,
J.Mol.Spectr. {\bf 169}, 458 (1995).

\bibitem{ref:ten13}
J. H. Schryber, S. Miller, and J. Tennyson,
Computed Infrared Absorption Properties if Hot Water Vapour,
J.Quantit.Spectr.Radiat.Transf. {\bf 53}, 373 (1995).


\bibitem{ref:ten14}
N. F. Zobov, O. L. Polyansky, C. R. Le Sueur, and J. Tennyson,
Vibration--Rotation Levels of Water beyond the 
Born--Oppenheimer Approximation,
Chem.Phys.Lett. {\bf 260}, 381 (1996).


%\bibitem{ref:A1}   
%N. Oliphant and L. Adamowicz, 
%Int. Rev. Phys. Chem. {\bf 12}, 339 (1993).

%\bibitem{ref:A4}   
%P. Piecuch, N. Oliphant and L. Adamowicz, 
%J. Chem. Phys. {\bf 99}, 1875 (1993).
%
%\bibitem{ref:A11}  
%P. Piecuch and L. Adamowicz, 
%J. Chem. Phys.
%{\bf 100(8)}, 1 (1994).
%
%\bibitem{ref:A17}  
%P. Piecuch and L. Adamowicz, 
%J. Chem. Phys. {\bf 102}, 898 (1995).
%
%\bibitem{ref:A22}  
%K.B. Ghose and L. Adamowicz, 
%J. Chem. Phys. {\bf 103}, 9324 (1995).
%
%\bibitem{ref:A25}  
%K.B. Ghose, P. Piecuch and L. Adamowicz, 
%J. Chem. Phys. {\bf 103}, 9331 (1995).

\bibitem{ref:sch1} 
C. D. Sherrill, G. Vacek, Y. Yamaguchi, H. F. Schaefer, III, 
J. F. Stanton, and J. Gauss,
The A$^1_u$ State and the T$_2$ potential surface
of Acetylene: Implications for Triplet Perturbations in the
Fluorescence Spectra of teh A State,
J.Chem.Phys. {\bf 104}, 8507 (1996).

\bibitem{ref:sch2} 
G. Vacek, J. R. Thomas, B.J. DeLeeuw, Y. Yamaguchi, H. F. Schaefer, III, 
J. F. Stanton, and J. Gauss,
Isomerization Reactions of the Lowest Potential
Energy Hypersurface of Triplet Vinylidene and Triplet Acetylene,
J.Chem.Phys. {\bf 98}, 4766 (1993).

\bibitem{ref:dup1}
P. Dupre, P.G. Green, R.W. Field, 
Quantum Beat Spectroscopic Studies of Zeeman Anticrossing
in the A$^1_u$ State of the Acetylene Molecule
(C$_2$H$_2$),
Chem.Phys. {\bf 198}, 211 (1995). 

\bibitem{klopper}
W. Klopper, M. Quack, and M.A. Suhm,
A New {\it Ab--initio} Based Six--Dimensional Semi--Empirical Pair
Interaction Potential for HF,
Chem.Phys.Lett. {\bf 261}, 35 (1996).

\bibitem{Chung92}
Kwong T. Chung and Paul Fullbright,
Electron Affinity of Lithium,
Physica Scripta. {\bf 45}, 445 (1992)

\bibitem{beowulfwww}
see for example; {\tt http://www.beowulf.org}

\bibitem{HPFwww}
see for example the documentation at the 
Maui High Performance Computing Center;
{\tt http://www.mhpcc.edu/doc/hpf/hpf.html},
and the Portland Group Inc. web site;
{\tt http://www.pgroup.com}

\bibitem{PVMwww}
see for example; 
{\tt http://www.epm.ornl.gov/pvm/pvm$\_$home.html}

\bibitem{Kinghorn99a}
D.~B. Kinghorn and L.~Adamowicz,
A Correlated Basis Set for Nonadiabatic Energy Calculations
on Diatomic Systems,
J. Chem. Phys. {\bf 110}, 7166 (1999).


\end{thebibliography}

\end{document}


