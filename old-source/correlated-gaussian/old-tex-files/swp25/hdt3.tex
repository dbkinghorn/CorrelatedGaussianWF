%% This document created by Scientific Word (R) Version 2.0
%% Starting shell: mathart1


\documentclass[12pt,thmsa]{article}
%%%%%%%%%%%%%%%%%%%%%%%%%%%%%%%%%%%%%%%%%%%%%%%%%%%%%%%%%%%%%%%%%%%%%%%%%%%%%%%%%%%%%%%%%%%%%%%%%%%%%%%%%%%%%%%%%%%%%%%%%%%%
\usepackage{sw20aip}
\usepackage{doublesp}

%TCIDATA{Created=Mon Apr 15 10:38:32 1996}
%TCIDATA{LastRevised=Fri Nov 15 14:59:43 1996}
%TCIDATA{Language=American English}

\input tcilatex
\begin{document}

\author{Donald B. Kinghorn and Ludwik Adamowicz \\
%EndAName
Department of Chemistry, \\
University of Arizona \\
Tucson, AZ 85721, U.S.A.}
\title{The Electron Affinity of Hydrogen, Deuterium, and Tritium: A Non-adiabatic
Variational Calculation Using Explicitly Correlated Gaussian Basis Functions}
\date{November 14, 1996}
\maketitle

\begin{abstract}
Non-adiabatic variational calculations for the anion ground state energies,
mass shifts, and electron affinities of Hydrogen, Deuterium and Tritium are
reported. Electron affinities values are 6083.0994$cm^{-1}$, 6086.7137$%
cm^{-1}$, and 6087.9168$cm^{-1}$ for Hydrogen, Deuterium and Tritium,
respectively. These results were obtained using a basis of explicitly
correlated Gaussians. Exact nonrelativistic energy bounds are carefully
predicted: $E\left( H^{-}\right) =-.5274458811$, $E\left( D^{-}\right)
=-.5275983247$ and $E\left( T^{-}\right) =-.5276490482$ in hartree energy
units. It is shown that these new bounds cannot be obtained using the
infinite nuclear mass approximation plus Rydberg scaling and the usual first
order mass polarization correction.
\end{abstract}

\section{Introduction}

The use of explicitly correlated Gaussian basis functions can provide a
means of obtaining high accuracy energy results for few particle systems,
especially when there use is combined with a non-adiabatic approach.
Calculations using this basis are hampered by the formidable task of
optimizing the many non-linear parameters need for a reasonable expansion of
the wave function. Past calculations were often left only partially
optimized suggesting much slower convergence than this basis is capable of.
Recent development and implementation of compact formulas for analytic
gradients with this basis has greatly reduced optimization costs\cite
{Kinghorn95a,Kinghorn95b}. The present highly accurate (errors less than $%
10^{-3}cm^{-1}$) non-adiabatic calculations for $H^{-},\,\,D^{-},\,\,$and $%
T^{-}$ reflect these recent improvements in the utilization of this basis.

Experimental values for the electron affinity of Hydrogen have been steadily
improving. In the mid 70's a value of 6081$\pm 16cm^{-1}$ was reported by
Feldmann\cite{Feldmann75} and independently by McCulloh and Walker\cite
{McCulloh74}. Greater accuracy was then achieved by Chupka, Dehmer and Jivery%
\cite{Chupka75} giving 6083$\left( +11,-3\right) cm^{-1}$. In 1979 Scherk%
\cite{Scherk79} reported a value of 6085.5$\pm 3.3cm^{-1}$. Then in 1991
Lykke, Murray and Lineberger\cite{Lykke91} achieved a 20 fold improvement in
accuracy obtaining 6082.99$\pm 0.15cm^{-1}$ and 6086.2$\pm 0.6cm^{-1}$ for
the electron affinity of Hydrogen and Deuterium respectively. Recent
conversation with Professor Lineberger suggested that improved error limits
especially for Deuterium may be possible. In anticipation of these improved
experimental results the present work gives highly accurate theoretical
predictions for the electron affinity of Hydrogen, Deuterium and Tritium.

There have been several highly accurate ground state energy calculations for
the negative Hydrogen ion\cite
{Pekeris62,Frankowski66,Drake88,Baker90,Thakkar94,Ackermann95}. The most
accurate of these calculations\cite{Drake88,Baker90,Thakkar94} appear to be
converged to the point where the uncertainties in the calculated energies is
less than $10^{-13}$ hartree ($10^{-8}cm^{-1}$). This uncertainty is several
orders of magnitude less than the uncertainty in even the most accurately
known fundamental physical constants. For example, the Rydberg constant is
uncertain in the 10th digit\cite{Petley92} and the atomic mass of Hydrogen
is uncertain in the 11th digit\cite{Audi93} (the electron mass is uncertain
in the 8th digit\cite{Audi93}). As remarkable as these calculations are they
suffer from one major shortcoming ---the computed wave functions are for a
fictitious system with an infinitely heavy nuclear mass. It is common to add
corrections to these computed energies. The corrections consist of scaling
using a reduced-mass-corrected Rydberg constant, the reduced Rydberg, and
adding a first order perturbation correction for what is usually called mass
polarization\cite{BetheSalpeter}. These corrections to the infinite nuclear
mass approximation are very good, however, they do not account for the
entire energy shift due to nuclear motion. For highly accurate energy
calculations on the systems $H^{-},\,\,D^{-},\,\,$and $T^{-}$ (and probably
for Helium) it is necessary to use a Hamiltonian that explicitly includes
the nuclear motion ---the non-adiabatic Hamiltonian. We demonstrate this in
the present work by providing highly accurate variational upper bounds for
the non-relativistic ground state energy of $H^{-}$, $D^{-}$ and $T^{-}.$
These bounds are computed directly using the finite mass non-adiabatic
Hamiltonian. Our $H^{-}$ energy bound is nearly 13 nano-hartree lower than
can be achieved using the infinite nuclear mass approximation plus Rydberg
scaling and first order mass polarization corrections, and we know from the
essentially exact $H^{-}$ result of Drake\cite{Drake88} that the exact
energy lowering should be closer to 18 nano-hartrees. It appears that this
18 nano-hartrees can only be recovered rigorously using the non-adiabatic
Hamiltonian. This is an energy correction approximately equal to that of the
Lamb shift (17 nano-hartrees\cite{Drake88}) and thus represents a
significant deficiency in the infinite nuclear mass approximation. We use
our non-relativistic energy bounds together with relativistic and other
small corrections computed by Drake\cite{Drake88} to obtain what should be
reliable theoretical values for the electron affinity of Hydrogen, Deuterium
and Tritium. We hope that this work will spark renewed interest in high
accuracy electron affinity determination among both experimentalists and
theoreticians.

\section{Theory}

\subsection{The Hamiltonian}

In general the total Hamiltonian for a three-particle system with masses $%
\left\{ M_1,M_2,M_3\right\} $ and charges $\left\{ Q_1,Q_2,Q_3\right\} $
interacting under a Coulomb potential is represented (in atomic units) by 
\begin{equation}
H_{tot}=\frac{P_1^2}{2M_1}+\frac{P_2^2}{2M_2}+\frac{P_3^2}{2M_3}+\frac{Q_1Q_2%
}{R_{12}}+\frac{Q_1Q_3}{R_{13}}+\frac{Q_2Q_3}{R_{23}}  \label{Htot}
\end{equation}
Where $R_i$, $P_i$ are position and momentum vectors for particle $i$ and $%
R_{ij}=\left\| R_i-R_j\right\| .$ This Hamiltonian is separable into a
Hamiltonian for the kinetic energy of the center of mass and a Hamiltonian
describing the internal interaction energy. A transformation leading to this
separation is given by\cite{Poshusta83} 
\begin{equation}
\begin{array}{llll}
r_0=\frac{M_1R_1+M_2R_2+M_3R_3}{m_0}, & p_0=P_1+P_2+P_3, & m_0=M_1+M_2+M_{3,}
& q_0=Q_1 \\ 
r_1=R_2-R_1, & p_1=P_2-\frac{M_2}{m_0}p_0, & m_1=\frac{M_1M_2}{M_1+M_2}, & 
q_1=Q_2 \\ 
r_2=R_3-R_1, & p_2=P_3-\frac{M_3}{m_0}p_0, & m_2=\frac{M_2M_3}{M_2+M_3}, & 
q_2=Q_3
\end{array}
\label{CMtran}
\end{equation}
Applying this transformation to the total Hamiltonian gives $%
H_{tot}=H_{cm}+H $ where 
\begin{equation}
H_{cm}=\frac{p_0^2}{2m_0}
\end{equation}
and 
\begin{equation}
H=\frac{p_1^2}{2m_1}+\frac{p_2^2}{2m_2}+\frac{p_1\cdot p_2}{M_1}+\frac{q_0q_1%
}{r_1}+\frac{q_0q_2}{r_2}+\frac{q_1q_2}{r_{12}}  \label{H1}
\end{equation}
The Hamiltonian in eqn.(\ref{H1}) can be viewed as representing the internal
motions of a three-particle system or as the total energy of a two-particle
system with fictitious masses $m_1$, $m_2$ and charges $q_1,\,q_2$
interacting with a charge $q_0$ at the origin together with their mutual
Coulomb interaction and a momentum dependent ``mass polarization potential''.

Now, for $H^{-}$, $D^{-}$, and $T^{-}$, we assign the nucleus to particle
one and the electrons to particles two and three giving the nonrelativistic,
nonadiabatic, Hamiltonian used in these calculations 
\begin{equation}
H=-\frac{\nabla _1^2}{2\mu _1}-\frac{\nabla _2^2}{2\mu _2}-\frac{\nabla
_1\cdot \nabla _2}{M_N}-\frac 1{r_1}-\frac 1{r_2}+\frac 1{r_{12}}
\label{ham1}
\end{equation}
Here $\mu _1=\mu _2$ is the reduced electron mass, $M_N$ is the nuclear
mass, $r_1$ and $r_2$ are relative coordinates described in the center of
mass transformation, eqn.(\ref{CMtran}), and we have written the momentum
operator $p$ as $-i\nabla $. For the values of the nuclear masses see table(%
\ref{masstab}).

It is possible to write eqn(\ref{ham1}) in a form that makes scaling and
perturbation corrections to the infinite nuclear mass approximation more
obvious\cite{Drake96}. Since for the cases being considered $\mu =\mu _1=\mu
_2$, coordinate scaling $\rho _i=\left( \mu /m_e\right) r_i$ gives a
Hamiltonian, 
\begin{equation}
H=\frac \mu {m_e}\left[ \left\{ -\frac{\nabla _{\rho _1}^2}2-\frac{\nabla
_{\rho _2}^2}2-\frac 1{\rho _1}-\frac 1{\rho _2}+\frac 1{\rho _{12}}\right\}
-\frac \mu {M_N}\nabla _{\rho _1}\cdot \nabla _{\rho _2}\right] .
\label{ham2}
\end{equation}
that has the same eigenvalues as eqn(\ref{ham1}). The terms in braces make
up the infinite nuclear mass Hamiltonian, thus suggesting the following
perturbation expansion, 
\begin{equation}
E=\frac \mu {m_e}\left[ \varepsilon _0+\varepsilon _1\left( \frac \mu
{M_N}\right) +\varepsilon _2\left( \frac \mu {M_N}\right) ^2+\cdots \right]
\label{Epert}
\end{equation}
where, $\varepsilon _0=\left\langle H\right\rangle _\infty $ is the energy
in the infinite nuclear mass approximation and $\varepsilon _1=-\left\langle
\nabla _1\cdot \nabla _2\right\rangle _\infty $ is a first order mass
polarization correction.

\subsection{Wave Function}

Calculations where performed using explicitly correlated Gaussian basis
functions of the form ( $^{\prime }$ represents vector/matrix transposition
and $\otimes $ is the Kronecker product symbol) 
\begin{eqnarray}
\phi _k &=&\exp \left[ -r^{\prime }\left( L_kL_k^{\prime }\otimes I_3\right)
r\right]   \label{ECG} \\
&=&\exp \left[ -r^{\prime }\left( A_k\otimes I_3\right) r\right]   \nonumber
\\
&=&\exp \left[ -\sum_{i,j}a_{ij}\left( r_i\cdot r_j\right) \right]  
\nonumber
\end{eqnarray}
Where, in general, for an $n$ particle (pseudo-particle) system $r$ is the $%
3n\times 1$ vector of Cartesian coordinates for the $n$ particles, $L_k$ is
an $n\times n$ rank $n$ lower triangular matrix of nonlinear variation
parameters, and $I_3$ is the $3\times 3$ identity matrix. The Kronecker
product with the identity insures rotational invariance of the basis
functions (the $\phi _k$ are angular momentum eigen-functions with J=0 and
are thus suitable for the ground state calculations we are considering). The
exponent parameters are written in Cholesky factored form, $L_kL_k^{\prime },
$ to insure positive definiteness of the quadratic form. Correlation is more
obvious when one notes the relation $r_{ij}^2=r_i\cdot r_i+r_j\cdot
r_j-2r_i\cdot r_j.$ This form of the basis is equivalent to the (single
center) form of the basis introduced by Boys\cite{Boys} and then, in matrix
form, by Singer\cite{Singer1} in 1960. For a more complete discussion of
this basis and derivation of the Hamiltonian matrix elements and derivatives
in matrix form see Kinghorn\cite{Kinghorn95a,Kinghorn95b}. For the
particular cases we are considering in this work $r=\left( r_1^{\prime
},\,r_2^{\prime }\right) ^{\prime }$ is the $6\times 1$ vector of relative
coordinates defined above. The ground state spatial wave function (symmetric
with respect to exchange of electrons) is then given as the symmetry
projected linear combination of the $\phi _k$, 
\begin{equation}
\Psi =\sum_kc_k\left( \exp \left[ -r^{\prime }\left( L_kL_k^{\prime }\otimes
I_3\right) r\right] +\exp \left[ -r^{\prime }\left( \tau ^{\prime
}L_kL_k^{\prime }\tau \otimes I_3\right) r\right] \right) 
\end{equation}
where $\tau $ is the permutation matrix $\left( 
\begin{array}{cc}
0 & 1 \\ 
1 & 0
\end{array}
\right) .$

\subsection{Optimization}

The wave functions are optimized for the ground state energy by minimizing
the Rayleigh quotient 
\begin{equation}
E\left( a;c\right) =\min_{\left\{ a,c\right\} }\frac{c^{\prime }H(a)c}{%
c^{\prime }S(a)c}  \label{energy}
\end{equation}
$H\left( a\right) $ and $S\left( a\right) $ are the Hamiltonian and overlap
matrices which are functions of the nonlinear parameters contained in the
basis set exponent matrices $L_k$. We write $a$ for the collection of these
nonlinear parameters. $c$ is the vector of linear coefficients in the basis
expansion of $\Psi .$ Thus, if we let $N$ be the number of basis functions,
the energy is, in general, a function of $Nn\left( n+1\right) /2+N$
variables which in our particular case is $4N$ variables. Obviously a
reduction of $N$ variables could be achieved by solving the eigen-problem
associated with eqn. (\ref{energy}) to obtain the linear coefficients and
then iteratively optimize the nonlinear parameters. However, we found that
much more thorough optimization could be achieved by letting the optimizer
simultaneously vary both the linear and nonlinear parameters.

The optimization software employed was the package TN by Stephen Nash\cite
{NashTN} --- available from netlib\cite{netlib}. TN is a truncated Newton
method utilizing a user supplied gradient. The analytic gradient of the
energy functional was derived using matrix differential calculus\cite
{Kinghorn95a,Kinghorn95b}.

\subsection{Mass values and other constants}

The values for nuclear masses and other constants used in these calculations
are presented in table(\ref{masstab}). The nuclear masses where computed
using the atomic masses given in \emph{The 1993 atomic mass evaluation} of
Audi and Wapstra\cite{Audi93}. The nuclear mass is derived from the atomic
mass using 
\begin{equation}
M_N=M_A-Z\times m_e+B_e
\end{equation}
Here $M_A$ is the atomic mass, $Z$ the nuclear charge, $m_e$ the electron
mass, and $B_e$ the total electron binding energy. $B_e$ is the negative of
the ground state energy and is converted to mass units using $%
E=mc^2=m/\alpha ^2.$ We use quantum units in this work except where
otherwise noted. Thus, $\hbar =1,$ $m_e=1,$ energy is in hartree$\left(
=2R_\infty \right) $, and distance in bohr. The value of the Rydberg
constant, $R_\infty $, was taken from CODATA86\cite{codata86} and the fine
structure constant, $\alpha $, from Kinoshita\cite{Kinoshita95}. The finite
nuclear mass Hydrogen like atom ground state energy differs from the
infinite nuclear mass result by the factor $\mu /m_e=1-\mu /M_N=M_N/\left(
M_N+m_e\right) .$ This factor is used later in this work when effects of
nuclear motion are discussed.

It should be noted that the electron mass contains error in its eighth
significant digit ($m_e=548579.903\pm 0.013$ nano-atomic units)\cite{Audi93}%
, this, together with the error in the nuclear masses, propagates to an
error of approximately $\pm 6\times 10^{-10}$ in the reduced masses $\mu _i$%
. Thus, even the ``exact'' ground state energy for Hydrogen is in error in
the 10th digit. Therefore, to reflect (at least) the uncertainties in the
fundamental physical constants, results in this work will be presented with
at most 10 digits.

\section{Results}

\subsection{Expectation values}

Tables (\ref{Hinfexpvals}, \ref{Hexpvals},\ref{Dexpvals},\ref{Texpvals})
contain expectation values for Hydrogen (infinite nuclear mass
approximation), Hydrogen, Deuterium , and Tritium respectively. Results
using wave functions with from 64 to 512 terms are included to exhibit
convergence which we feel is surprisingly good for this basis. Included in
the tables are expectation values for the Hamiltonian , the kinetic and
potential energy, the virial coefficient ($\eta $), components of the
kinetic and potential energy, integrals for $r,$ $r^2,$ and $r^{-2}$ and
finally contact integrals.

The ground state energy of $H^{-}$ in the infinite nuclear mass
approximation has been computed to incredible precision with the best
results appearing to be converged to at least 13 digits. However, the
infinite nuclear mass approximation introduces errors in expectation values
in the 4th digit as is easily seen by comparing results in table (\ref
{Hinfexpvals}) with table (\ref{Hexpvals}). The standard corrections to the
energy introduced by Bethe and Salpeter\cite{BetheSalpeter}, ``reduced
Rydberg scaling'' and mass polarization corrections, leave error in the 8th
digit as will be demonstrated in section(\ref{nucmotian}). Even the Rydberg
constant, $R_\infty $, which would be used for energy unit conversion is
uncertain in the 10th digit\cite{Petley92}. With this clearly in mind we
include table (\ref{Litdata}) listing expectation values from other infinite
nuclear mass calculations.

\subsection{Finite Nuclear Mass Effects\label{nucmotian}\label{NMeffect}}

The energy shift due to finite nuclear mass is 
\begin{equation}
\Delta E_{NM}=\left\langle H\right\rangle _f-\left\langle H\right\rangle
_\infty  \label{dENM}
\end{equation}
Using eqn(\ref{Epert}) $\Delta E_{NM}$ can be given to very good
approximation, for two electron systems, by 
\begin{eqnarray}
\widetilde{\Delta E_{NM}} &=&\frac \mu {m_e}\left[ \left\langle
H\right\rangle _\infty -\frac \mu {M_N}\left\langle \nabla _1\cdot \nabla
_2\right\rangle _\infty \right] -\left\langle H\right\rangle _\infty 
\nonumber \\
&=&\left[ \left( 1-\frac \mu {M_N}\right) \left\langle H\right\rangle
_\infty -\frac{\mu ^2}{m_eM_N}\left\langle \nabla _1\cdot \nabla
_2\right\rangle _\infty \right] -\left\langle H\right\rangle _\infty 
\nonumber \\
&=&-\frac \mu {M_N}\left( \left\langle H\right\rangle _\infty -\frac \mu
{m_e}\left\langle \nabla _1\cdot \nabla _2\right\rangle _\infty \right)
\label{dNMapprox}
\end{eqnarray}
The first term in square brackets in eqn.(\ref{dNMapprox}) is the ``reduced
Rydberg scaling'' which could be interpreted as the energy change that would
result from changing units from hartree$(=2R_\infty )$ to reduced Rydberg
units $2R_M=2(\mu /m_e)R_\infty $. The second term is a perturbation, the
``mass polarization energy'', again in reduced Rydberg units.

In table (\ref{NMtab}) values for $\Delta E_{NM}$, $\widetilde{\Delta E_{NM}}
$ and $\Delta =\widetilde{\Delta E_{NM}}-\Delta E_{NM}$ are given using our
infinite nuclear mass wavefunction and non-adiabatic wave functions for
Hydrogen, Deuterium and Tritium anions. The first order approximation to the
nuclear mass energy shift for $H^{-}$ leaves 17 nano-hartree unaccounted for
which is unacceptable for high accuracy calculations.

Drake\cite{Drake88} has computed a total mass polarization correction given
as 
\begin{equation}
\Delta E_{mp}=.03287978125\left( \mu /M\right) -.059779493\left( \mu
/M\right) ^2.  \label{drakeEM}
\end{equation}
The coefficient of the first term in eqn.(\ref{drakeEM}) is $\left\langle
\nabla _1\cdot \nabla _2\right\rangle _\infty $. The coefficient of the
second term is derived by subtracting the leading term in eqn.(\ref{drakeEM}%
) from the total energy shift due to mass polarization. This total mass
polarization energy shift is computed by subtracting the infinite nuclear
mass energy from the non-adiabatic energy of $H^{-}$ (derived from the
Hamiltonian in eqn(\ref{ham2})). Thus, using the terms in eqn(\ref{drakeEM})
for the second and third terms in the perturbation expansion, eqn(\ref{Epert}%
), will, of course, give the exact result for $H^{-}$ and should give
excellent approximation for $D^{-}$ and $T^{-}$. However, we feel that since
the non-adiabatic energy for $H^{-}$ is used to obtain this expansion, the
non-adiabatic energy for the other isotopes may as well be computed without
approximation.

This section is concluded with table(\ref{boundstab}) presenting the
variational energy upper bounds computed in this work. Table(\ref{boundstab}%
) includes our estimation of the exact energy upper bounds. Our infinite
nuclear mass and $H^{-}$ energy values lie 4.9 nano-hartrees above the
essentially exact values reported by Drake\cite{Drake88} and we believe that
our non-adiabatic calculations for $D^{-}$ and $T^{-}$contain this same
deficiency. Therefore, we estimate the exact energy upper bounds by
correcting our computed values for this deficiency.

\subsection{Electron Affinity}

Electron affinities for Hydrogen Deuterium and Tritium are presented in
table (\ref{EAtab}). In the calculation of the electron affinities the
energies for the neutral atoms are those given in table(\ref{masstab}). The
calculations include highly accurate small corrections computed by Drake\cite
{Drake88}. These include relativistic, relativistic recoil, Lamb shift and
finite nuclear size corrections giving a total correction of 0.307505$%
cm^{-1} $, labeled $\Delta E_{corr}$ in table (\ref{EAtab}). We use this
correction with Rydberg scaling for Deuterium and Tritium. The electron
affinity is then given by 
\begin{equation}
EA=E_{\text{atom}}-E_{\text{anion}}-\Delta E_{corr}
\end{equation}
where $E_{\text{atom}}$ is the energy of the neutral atom, $E_{\text{anion}}$
is the non-adiabatic energy of the negative ion and $\Delta E_{corr}$ is the
correction described above. Included in table(\ref{EAtab}) are the
computational results for Hydrogen of Pekeris\cite{Pekeris62} and of Drake%
\cite{Drake88}, and the experimental results for Hydrogen and Deuterium of
Lykke, Murray and Lineberger\cite{Lykke91}. Also included is our estimates
of the exact electron affinity computed from the energy estimates given in
table(\ref{boundstab}). The result from Pekeris in table(\ref{EAtab})
includes only first order mass polarization correction and Drakes
relativistic corrections $\Delta E_{corr}.$

\subsection{Conclusion}

This work has provided reliable values for the electron affinity of
Hydrogen, Deuterium and Tritium. Constructing wave functions using a
relatively simple explicitly correlated Gaussian basis and a non-adiabatic
Hamiltonian has provided results unattainable with the infinite nuclear mass
approximation plus Rydberg scaling and first order mass polarization
correction. It was shown that for $H^{-},$ 17 nano-hartrees of the energy
shift due to nuclear motion cannot be accounted for by these corrections.
Our energy calculations are still in error by approximately 4.9
nano-hartrees and we hope to correct this by including products of powers of 
$r_{ij}$ with our correlated Gaussians, 
\begin{equation}
\phi _k=\prod_{i<j}r_{ij}^{m_{kij}}\exp \left[ -\mathbf{r}^{\prime
}(A_k\otimes I_3)\mathbf{r}\right] .
\end{equation}
We expect this new correlated n-body basis to have excellent convergence
characteristics. These results may be important for researchers testing the
validity of QED calculations such as the Lamb shift and other small energy
corrections and may also be significant for researchers attempting to
compute values for fundamental physical constants. We hope that others will
confirm and improve on our results, and that experimentalists will push the
envelope of error bars closer to the accuracy of the theoretical
calculations.

\noindent \textbf{Acknowledgment}

This work was supported by the National Science Foundation with grant
CHE-9300497

\bibliographystyle{aip}
\bibliography{4-prefs,hmhe,mcalc}

\newpage

%TCIMACRO{\TeXButton{B}{\begin{table}[t] \centering}}
%BeginExpansion
\begin{table}[t] \centering%
%EndExpansion
\begin{tabular}{ll}
\hline\hline
$M_H=1836.152693$ & $\mu _H=.9994556794$ \\ 
$M_D=3670.483008$ & $\mu _D=.9997276305$ \\ 
$M_T=5496.921571$ & $\mu _T=.9998181131$ \\ \hline
$R_\infty =109737.31534cm^{-1}$ &  \\ 
$\alpha ^{-1}=137.03599944$ &  \\ \hline
$E_0=-\mu /2$ &  \\ 
$E_0\left( H\right) =-.4997278397$ &  \\ 
$E_0\left( D\right) =-.4998638152$ &  \\ 
$E_0\left( T\right) =-.4999090565$ &  \\ \hline\hline
\end{tabular}
\caption{Mass values and other constants \label{masstab}}%
%TCIMACRO{\TeXButton{E}{\end{table}}}
%BeginExpansion
\end{table}%
%EndExpansion

\newpage

%TCIMACRO{\TeXButton{B}{\begin{table}[t] \centering} }
%BeginExpansion
\begin{table}[t] \centering%
%EndExpansion
\begin{tabular}{lllll}
\hline\hline
$N$ & $64$ & $128$ & $256$ & $512$ \\ \hline
$\left\langle H\right\rangle _\infty $ & \multicolumn{1}{r}{-.5277504812} & 
\multicolumn{1}{r}{-.5277509852} & \multicolumn{1}{r}{-.5277510093} & 
\multicolumn{1}{r}{-.5277510116} \\ 
$\left\langle T\right\rangle _\infty $ & \multicolumn{1}{r}{.5277504436} & 
\multicolumn{1}{r}{.5277509709} & \multicolumn{1}{r}{.5277510703} & 
\multicolumn{1}{r}{.5277510096} \\ 
$\left\langle V\right\rangle _\infty $ & \multicolumn{1}{r}{-1.0555009248} & 
\multicolumn{1}{r}{-1.0555019562} & \multicolumn{1}{r}{-1.0555020796} & 
\multicolumn{1}{r}{-1.0555020212} \\ 
$\eta $ & \multicolumn{1}{r}{1.0000000355} & \multicolumn{1}{r}{1.0000000135}
& \multicolumn{1}{r}{.9999999422} & \multicolumn{1}{r}{1.0000000019} \\ 
$\left\langle -\nabla _1^2\right\rangle _\infty $ & \multicolumn{1}{r}{
.5277504436} & \multicolumn{1}{r}{.5277509709} & \multicolumn{1}{r}{
.5277510702} & \multicolumn{1}{r}{.5277510095} \\ 
$\left\langle -\nabla _1\cdot \nabla _2\right\rangle _\infty $ & 
\multicolumn{1}{r}{.0328798325} & \multicolumn{1}{r}{.0328797865} & 
\multicolumn{1}{r}{.0328797911} & \multicolumn{1}{r}{.0328797798} \\ 
$\left\langle r_1^{-1}\right\rangle _\infty $ & \multicolumn{1}{r}{
.6832622892} & \multicolumn{1}{r}{.6832617853} & \multicolumn{1}{r}{
.6832618093} & \multicolumn{1}{r}{.6832617605} \\ 
$\left\langle r_{12}^{-1}\right\rangle _\infty $ & \multicolumn{1}{r}{
.3110236535} & \multicolumn{1}{r}{.3110216145} & \multicolumn{1}{r}{
.3110215390} & \multicolumn{1}{r}{.3110214997} \\ 
$\left\langle r_1\right\rangle _\infty $ & \multicolumn{1}{r}{2.7101080515}
& \multicolumn{1}{r}{2.7101737546} & \multicolumn{1}{r}{2.7101774512} & 
\multicolumn{1}{r}{2.7101782272} \\ 
$\left\langle r_{12}\right\rangle _\infty $ & \multicolumn{1}{r}{4.4125559666
} & \multicolumn{1}{r}{4.4126857296} & \multicolumn{1}{r}{4.4126929713} & 
\multicolumn{1}{r}{4.4126944006} \\ 
$\left\langle r_1^2\right\rangle _\infty $ & \multicolumn{1}{r}{11.9117467188
} & \multicolumn{1}{r}{11.9135603852} & \multicolumn{1}{r}{11.9136812624} & 
\multicolumn{1}{r}{11.9136972111} \\ 
$\left\langle r_{12}^2\right\rangle _\infty $ & \multicolumn{1}{r}{
25.1981384317} & \multicolumn{1}{r}{25.2017491899} & \multicolumn{1}{r}{
25.2019887841} & \multicolumn{1}{r}{25.2020204446} \\ 
$\left\langle r_1^{-2}\right\rangle _\infty $ & \multicolumn{1}{r}{
1.1166299163} & \multicolumn{1}{r}{1.1166554041} & \multicolumn{1}{r}{
1.1166570728} & \multicolumn{1}{r}{1.1166583435} \\ 
$\left\langle r_{12}^{-2}\right\rangle _\infty $ & \multicolumn{1}{r}{
.1551216229} & \multicolumn{1}{r}{.1551059984} & \multicolumn{1}{r}{
.1551054860} & \multicolumn{1}{r}{.1551054766} \\ 
$\left\langle \delta \left( r_1\right) \right\rangle _\infty $ & 
\multicolumn{1}{r}{.1635839048} & \multicolumn{1}{r}{.1640737909} & 
\multicolumn{1}{r}{.1640744018} & \multicolumn{1}{r}{.1641346963} \\ 
$\left\langle \delta \left( r_{12}\right) \right\rangle _\infty $ & 
\multicolumn{1}{r}{.0028090290} & \multicolumn{1}{r}{.0027610047} & 
\multicolumn{1}{r}{.0027593265} & \multicolumn{1}{r}{.0027596593} \\ 
\hline\hline
\end{tabular}
\caption{$^{\infty}$H$^{-}$ Expectation Values \label{Hinfexpvals}}%
%TCIMACRO{\TeXButton{E}{\end{table}}}
%BeginExpansion
\end{table}%
%EndExpansion

%TCIMACRO{\TeXButton{B}{\begin{table}[t] \centering} }
%BeginExpansion
\begin{table}[t] \centering%
%EndExpansion
\begin{tabular}{lllll}
\hline\hline
$N$ & $64$ & $128$ & $256$ & $512$ \\ \hline
$\left\langle H\right\rangle _H$ & \multicolumn{1}{r}{-.5274453611} & 
\multicolumn{1}{r}{-.5274458498} & \multicolumn{1}{r}{-.5274458739} & 
\multicolumn{1}{r}{-.5274458762} \\ 
$\left\langle T\right\rangle _H$ & \multicolumn{1}{r}{.5274453148} & 
\multicolumn{1}{r}{.5274458206} & \multicolumn{1}{r}{.5274458565} & 
\multicolumn{1}{r}{.5274458636} \\ 
$\left\langle V\right\rangle _H$ & \multicolumn{1}{r}{-1.0548906759} & 
\multicolumn{1}{r}{-1.0548916704} & \multicolumn{1}{r}{-1.0548917304} & 
\multicolumn{1}{r}{-1.0548917398} \\ 
$\eta $ & \multicolumn{1}{r}{1.0000000438} & \multicolumn{1}{r}{1.0000000276}
& \multicolumn{1}{r}{1.0000000165} & \multicolumn{1}{r}{1.0000000119} \\ 
$\left\langle -\nabla _1^2\right\rangle _H$ & \multicolumn{1}{r}{.5271403731}
& \multicolumn{1}{r}{.5271408787} & \multicolumn{1}{r}{.5271409145} & 
\multicolumn{1}{r}{.5271409216} \\ 
$\left\langle -\nabla _1\cdot \nabla _2\right\rangle _H$ & 
\multicolumn{1}{r}{.0327790347} & \multicolumn{1}{r}{.0327789973} & 
\multicolumn{1}{r}{.0327790013} & \multicolumn{1}{r}{.0327790010} \\ 
$\left\langle r_1^{-1}\right\rangle _H$ & \multicolumn{1}{r}{.6828538727} & 
\multicolumn{1}{r}{.6828533936} & \multicolumn{1}{r}{.6828533723} & 
\multicolumn{1}{r}{.6828533730} \\ 
$\left\langle r_{12}^{-1}\right\rangle _H$ & \multicolumn{1}{r}{.3108170695}
& \multicolumn{1}{r}{.3108151168} & \multicolumn{1}{r}{.3108150142} & 
\multicolumn{1}{r}{.3108150062} \\ 
$\left\langle r_1\right\rangle _H$ & \multicolumn{1}{r}{2.7120310559} & 
\multicolumn{1}{r}{2.7120911055} & \multicolumn{1}{r}{2.7120952238} & 
\multicolumn{1}{r}{2.7120955275} \\ 
$\left\langle r_{12}\right\rangle _H$ & \multicolumn{1}{r}{4.4155651280} & 
\multicolumn{1}{r}{4.4156837795} & \multicolumn{1}{r}{4.4156918099} & 
\multicolumn{1}{r}{4.4156924114} \\ 
$\left\langle r_1^2\right\rangle _H$ & \multicolumn{1}{r}{11.9300411389} & 
\multicolumn{1}{r}{11.9316074492} & \multicolumn{1}{r}{11.9317345507} & 
\multicolumn{1}{r}{11.9317438071} \\ 
$\left\langle r_{12}^2\right\rangle _H$ & \multicolumn{1}{r}{25.2337827345}
& \multicolumn{1}{r}{25.2368966127} & \multicolumn{1}{r}{25.2371493544} & 
\multicolumn{1}{r}{25.2371678863} \\ 
$\left\langle r_1^{-2}\right\rangle _H$ & \multicolumn{1}{r}{1.1153654889} & 
\multicolumn{1}{r}{1.1153908649} & \multicolumn{1}{r}{1.1153923541} & 
\multicolumn{1}{r}{1.1153937757} \\ 
$\left\langle r_{12}^{-2}\right\rangle _H$ & \multicolumn{1}{r}{.1549104913}
& \multicolumn{1}{r}{.1548959719} & \multicolumn{1}{r}{.1548954385} & 
\multicolumn{1}{r}{.1548954516} \\ 
$\left\langle \delta \left( r_1\right) \right\rangle _H$ & 
\multicolumn{1}{r}{.1633128785} & \multicolumn{1}{r}{.1638020720} & 
\multicolumn{1}{r}{.1638024979} & \multicolumn{1}{r}{.1638623602} \\ 
$\left\langle \delta \left( r_{12}\right) \right\rangle _H$ & 
\multicolumn{1}{r}{.0027992322} & \multicolumn{1}{r}{.0027545364} & 
\multicolumn{1}{r}{.0027528569} & \multicolumn{1}{r}{.0027531889} \\ 
\hline\hline
\end{tabular}
\caption{H$^{-}$ Expectation Values \label{Hexpvals}}%
%TCIMACRO{\TeXButton{E}{\end{table}}}
%BeginExpansion
\end{table}%
%EndExpansion

%TCIMACRO{\TeXButton{B}{\begin{table}[t] \centering} }
%BeginExpansion
\begin{table}[t] \centering%
%EndExpansion
\begin{tabular}{lllll}
\hline\hline
$N$ & $64$ & $128$ & $256$ & $512$ \\ \hline
$\left\langle H\right\rangle _D$ & \multicolumn{1}{r}{-.5275977894} & 
\multicolumn{1}{r}{-.5275982934} & \multicolumn{1}{r}{-.5275983175} & 
\multicolumn{1}{r}{-.5275983198} \\ 
$\left\langle T\right\rangle _D$ & \multicolumn{1}{r}{.5275977498} & 
\multicolumn{1}{r}{.5275982638} & \multicolumn{1}{r}{.5275983000} & 
\multicolumn{1}{r}{.5275983071} \\ 
$\left\langle V\right\rangle _D$ & \multicolumn{1}{r}{-1.0551955392} & 
\multicolumn{1}{r}{-1.0551965573} & \multicolumn{1}{r}{-1.0551966175} & 
\multicolumn{1}{r}{-1.0551966269} \\ 
$\eta $ & \multicolumn{1}{r}{1.0000000375} & \multicolumn{1}{r}{1.0000000279}
& \multicolumn{1}{r}{1.0000000164} & \multicolumn{1}{r}{1.0000000119} \\ 
$\left\langle -\nabla _1^2\right\rangle _D$ & \multicolumn{1}{r}{.5274451065}
& \multicolumn{1}{r}{.5274456205} & \multicolumn{1}{r}{.5274456566} & 
\multicolumn{1}{r}{.5274456637} \\ 
$\left\langle -\nabla _1\cdot \nabla _2\right\rangle _D$ & 
\multicolumn{1}{r}{.0328293594} & \multicolumn{1}{r}{.0328293211} & 
\multicolumn{1}{r}{.0328293246} & \multicolumn{1}{r}{.0328293249} \\ 
$\left\langle r_1^{-1}\right\rangle _D$ & \multicolumn{1}{r}{.6830579287} & 
\multicolumn{1}{r}{.6830574141} & \multicolumn{1}{r}{.6830573931} & 
\multicolumn{1}{r}{.6830573937} \\ 
$\left\langle r_{12}^{-1}\right\rangle _D$ & \multicolumn{1}{r}{.3109203181}
& \multicolumn{1}{r}{.3109182709} & \multicolumn{1}{r}{.3109181688} & 
\multicolumn{1}{r}{.3109181606} \\ 
$\left\langle r_1\right\rangle _D$ & \multicolumn{1}{r}{2.7110671655} & 
\multicolumn{1}{r}{2.7111329939} & \multicolumn{1}{r}{2.7111370994} & 
\multicolumn{1}{r}{2.7111374073} \\ 
$\left\langle r_{12}\right\rangle _D$ & \multicolumn{1}{r}{4.4140556422} & 
\multicolumn{1}{r}{4.4141856237} & \multicolumn{1}{r}{4.4141936235} & 
\multicolumn{1}{r}{4.4141942386} \\ 
$\left\langle r_1^2\right\rangle _D$ & \multicolumn{1}{r}{11.9207707499} & 
\multicolumn{1}{r}{11.9225874044} & \multicolumn{1}{r}{11.9227141243} & 
\multicolumn{1}{r}{11.9227234416} \\ 
$\left\langle r_{12}^2\right\rangle _D$ & \multicolumn{1}{r}{25.2157135447}
& \multicolumn{1}{r}{25.2193293841} & \multicolumn{1}{r}{25.2195812302} & 
\multicolumn{1}{r}{25.2196000783} \\ 
$\left\langle r_1^{-2}\right\rangle _D$ & \multicolumn{1}{r}{1.1159970928} & 
\multicolumn{1}{r}{1.1160225252} & \multicolumn{1}{r}{1.1160240187} & 
\multicolumn{1}{r}{1.1160254418} \\ 
$\left\langle r_{12}^{-2}\right\rangle _D$ & \multicolumn{1}{r}{.1550164901}
& \multicolumn{1}{r}{.1550008728} & \multicolumn{1}{r}{.1550003393} & 
\multicolumn{1}{r}{.1550003524} \\ 
$\left\langle \delta \left( r_1\right) \right\rangle _D$ & 
\multicolumn{1}{r}{.1634484158} & \multicolumn{1}{r}{.1639378072} & 
\multicolumn{1}{r}{.1639383038} & \multicolumn{1}{r}{.1639982438} \\ 
$\left\langle \delta \left( r_{12}\right) \right\rangle _D$ & 
\multicolumn{1}{r}{.0028057413} & \multicolumn{1}{r}{.0027577672} & 
\multicolumn{1}{r}{.0027560874} & \multicolumn{1}{r}{.0027564205} \\ 
\hline\hline
\end{tabular}
\caption{D$^{-}$ Expectation Values \label{Dexpvals}}%
%TCIMACRO{\TeXButton{E}{\end{table}}}
%BeginExpansion
\end{table}%
%EndExpansion

%TCIMACRO{\TeXButton{B}{\begin{table}[t] \centering} }
%BeginExpansion
\begin{table}[t] \centering%
%EndExpansion
\begin{tabular}{lllll}
\hline\hline
$N$ & $64$ & $128$ & $256$ & $512$ \\ \hline
$\left\langle H\right\rangle _T$ & \multicolumn{1}{r}{-.5276485136} & 
\multicolumn{1}{r}{-.5276490169} & \multicolumn{1}{r}{-.5276490410} & 
\multicolumn{1}{r}{-.5276490433} \\ 
$\left\langle T\right\rangle _T$ & \multicolumn{1}{r}{.5276484962} & 
\multicolumn{1}{r}{.5276489885} & \multicolumn{1}{r}{.5276490236} & 
\multicolumn{1}{r}{.5276490306} \\ 
$\left\langle V\right\rangle _T$ & \multicolumn{1}{r}{-1.0552970098} & 
\multicolumn{1}{r}{-1.0552980055} & \multicolumn{1}{r}{-1.0552980646} & 
\multicolumn{1}{r}{-1.0552980739} \\ 
$\eta $ & \multicolumn{1}{r}{1.0000000165} & \multicolumn{1}{r}{1.0000000268}
& \multicolumn{1}{r}{1.0000000164} & \multicolumn{1}{r}{1.0000000120} \\ 
$\left\langle -\nabla _1^2\right\rangle _T$ & \multicolumn{1}{r}{.5275465495}
& \multicolumn{1}{r}{.5275470418} & \multicolumn{1}{r}{.5275470769} & 
\multicolumn{1}{r}{.5275470838} \\ 
$\left\langle -\nabla _1\cdot \nabla _2\right\rangle _T$ & 
\multicolumn{1}{r}{.0328461179} & \multicolumn{1}{r}{.0328460647} & 
\multicolumn{1}{r}{.0328460796} & \multicolumn{1}{r}{.0328460808} \\ 
$\left\langle r_1^{-1}\right\rangle _T$ & \multicolumn{1}{r}{.6831258386} & 
\multicolumn{1}{r}{.6831253023} & \multicolumn{1}{r}{.6831252803} & 
\multicolumn{1}{r}{.6831252808} \\ 
$\left\langle r_{12}^{-1}\right\rangle _T$ & \multicolumn{1}{r}{.3109546673}
& \multicolumn{1}{r}{.3109525990} & \multicolumn{1}{r}{.3109524960} & 
\multicolumn{1}{r}{.3109524877} \\ 
$\left\langle r_1\right\rangle _T$ & \multicolumn{1}{r}{2.7107482436} & 
\multicolumn{1}{r}{2.7108143219} & \multicolumn{1}{r}{2.7108183967} & 
\multicolumn{1}{r}{2.7108187065} \\ 
$\left\langle r_{12}\right\rangle _T$ & \multicolumn{1}{r}{4.4135568905} & 
\multicolumn{1}{r}{4.4136872973} & \multicolumn{1}{r}{4.4136952742} & 
\multicolumn{1}{r}{4.4136958928} \\ 
$\left\langle r_1^2\right\rangle _T$ & \multicolumn{1}{r}{11.9177685588} & 
\multicolumn{1}{r}{11.9195898850} & \multicolumn{1}{r}{11.9197143110} & 
\multicolumn{1}{r}{11.9197236656} \\ 
$\left\langle r_{12}^2\right\rangle _T$ & \multicolumn{1}{r}{25.2098664126}
& \multicolumn{1}{r}{25.2134904549} & \multicolumn{1}{r}{25.2137388014} & 
\multicolumn{1}{r}{25.2137576949} \\ 
$\left\langle r_1^{-2}\right\rangle _T$ & \multicolumn{1}{r}{1.1162073447} & 
\multicolumn{1}{r}{1.1162327408} & \multicolumn{1}{r}{1.1162342331} & 
\multicolumn{1}{r}{1.1162356567} \\ 
$\left\langle r_{12}^{-2}\right\rangle _T$ & \multicolumn{1}{r}{.1550507158}
& \multicolumn{1}{r}{.1550357895} & \multicolumn{1}{r}{.1550352550} & 
\multicolumn{1}{r}{.1550352681} \\ 
$\left\langle \delta \left( r_1\right) \right\rangle _T$ & 
\multicolumn{1}{r}{.1634933913} & \multicolumn{1}{r}{.1639829810} & 
\multicolumn{1}{r}{.1639835043} & \multicolumn{1}{r}{.1640434722} \\ 
$\left\langle \delta \left( r_{12}\right) \right\rangle _T$ & 
\multicolumn{1}{r}{.0028046832} & \multicolumn{1}{r}{.0027588443} & 
\multicolumn{1}{r}{.0027571630} & \multicolumn{1}{r}{.0027574964} \\ 
\hline\hline
\end{tabular}
\caption{T$^{-}$ Expectation Values \label{Texpvals}}%
%TCIMACRO{\TeXButton{E}{\end{table}}}
%BeginExpansion
\end{table}%
%EndExpansion

%TCIMACRO{\TeXButton{B}{\begin{table}[t] \centering}}
%BeginExpansion
\begin{table}[t] \centering%
%EndExpansion
\begin{tabular}{lll}
\hline\hline
$\left\langle H\right\rangle _\infty $ & -.5277510062954 & {\footnotesize %
Pekeris}\cite{Pekeris62}{\footnotesize , 444 term Hylleraas type} \\ 
& -.52775101635 & {\footnotesize Frankowski and Pekeris}\cite{Frankowski66}%
{\footnotesize , 246 term Hylleraas + }$\ln $ \\ 
& -.527751016544203 & {\footnotesize Drake}\cite{Drake88}{\footnotesize ,
616 term Hylleraas with double exponents} \\ 
& -.527751016544375 & {\footnotesize Baker et al }\cite{Baker90}%
{\footnotesize , 476 term Hylleraas + }$\ln ,${\footnotesize \ 1/Z expansion}
\\ 
& -.527751016544240 & {\footnotesize Thakkar and Koga}\cite{Thakkar94}%
{\footnotesize , 455 term modified Hylleraas type} \\ 
& -.527751016532 & {\footnotesize Ackermann}\cite{Ackermann95}{\footnotesize %
, 2005 point Finite Element Method} \\ 
& -.5277510116 & {\footnotesize This work, 512 term correlated Gaussian} \\ 
\hline
$\left\langle -\nabla _1\cdot \nabla _2\right\rangle _\infty $ & .0328797771
& {\footnotesize Pekeris}\cite{Pekeris62} \\ 
& .03287978125 & {\footnotesize Drake}\cite{Drake88} \\ 
& .0328797798 & {\footnotesize This work} \\ \hline
$\left\langle \delta \left( r_1\right) \right\rangle _\infty $ & .164547396
& {\footnotesize Pekeris}\cite{Pekeris62} \\ 
& .164552873 & {\footnotesize Drake}\cite{Drake88} \\ 
& .1641346963 & {\footnotesize This work} \\ \hline
$\left\langle \delta \left( r_{12}\right) \right\rangle _\infty $ & 
.002740206 & {\footnotesize Pekeris}\cite{Pekeris62} \\ 
& .002737995 & {\footnotesize Drake}\cite{Drake88} \\ 
& .002759659 & {\footnotesize This work} \\ \hline
$\left\langle r_1\right\rangle _\infty $ & 2.71017831 & {\footnotesize %
Pekeris}\cite{Pekeris62} \\ 
& 2.710178263 & {\footnotesize Ackermann}\cite{Ackermann95} \\ 
& 2.710178227 & {\footnotesize This work} \\ \hline
$\left\langle r_{12}\right\rangle _\infty $ & 4.41269452 & {\footnotesize %
Pekeris}\cite{Pekeris62} \\ 
& 4.412694401 & {\footnotesize This work} \\ \hline
$\left\langle r_1^2\right\rangle _\infty $ & 11.913692 & {\footnotesize %
Pekeris}\cite{Pekeris62} \\ 
& 11.913699235 & {\footnotesize Ackermann}\cite{Ackermann95} \\ 
& 11.9136972111 & {\footnotesize This work} \\ \hline
$\left\langle r_{12}^2\right\rangle _\infty $ & 25.202010 & {\footnotesize %
Pekeris}\cite{Pekeris62} \\ 
& 25.20202044 & {\footnotesize This work} \\ \hline\hline
\end{tabular}
\caption{$H^{-}$  infinite nuclear mass literature values \label{Litdata}}%
%TCIMACRO{\TeXButton{E}{\end{table}}}
%BeginExpansion
\end{table}%
%EndExpansion

%TCIMACRO{\TeXButton{B}{\begin{table}[t] \centering}}
%BeginExpansion
\begin{table}[t] \centering%
%EndExpansion
\begin{tabular}{lll}
\hline\hline
$\Delta E_{NM}\left( H^{-}\right) $ & .0003051354 & 66.96948$cm^{-1}$ \\ 
$\widetilde{\Delta E_{NM}}\left( H^{-}\right) $ & .0003051531 & 
\multicolumn{1}{r}{66.96947$cm^{-1}$} \\ 
$\Delta \left( H^{-}\right) $ & $1.77\times 10^{-8}$ & \multicolumn{1}{r}{
.00389$cm^{-1}$} \\ \hline
$\Delta E_{NM}\left( D^{-}\right) $ & .0001526918 & \multicolumn{1}{r}{
33.51198$cm^{-1}$} \\ 
$\widetilde{\Delta E_{NM}}\left( D^{-}\right) $ & .0001526963 & 
\multicolumn{1}{r}{33.51296$cm^{-1}$} \\ 
$\Delta \left( D^{-}\right) $ & $4.49\times 10^{-9}$ & \multicolumn{1}{r}{
.00099$cm^{-1}$} \\ \hline
$\Delta E_{NM}\left( T^{-}\right) $ & .0001019683 & \multicolumn{1}{r}{
22.37946$cm^{-1}$} \\ 
$\widetilde{\Delta E_{NM}}\left( T^{-}\right) $ & .0001019703 & 
\multicolumn{1}{r}{22.37990$cm^{-1}$} \\ 
$\Delta \left( T^{-}\right) $ & $2.01\times 10^{-9}$ & \multicolumn{1}{r}{
.00044$cm^{-1}$} \\ \hline\hline
\end{tabular}
\caption{Energy shift due to finite nuclear mass\label{NMtab}}%
%TCIMACRO{\TeXButton{E}{\end{table}}}
%BeginExpansion
\end{table}%
%EndExpansion

\newpage

%TCIMACRO{\TeXButton{B}{\begin{table}[t] \centering}}
%BeginExpansion
\begin{table}[t] \centering%
%EndExpansion
\begin{tabular}{llll}
\hline\hline
& $H^{-}$ & $D^{-}$ & $T^{-}$ \\ 
This work & -.5274458762 & -.5275983198 & -.5276490433 \\ 
``Exact'' estimate & -.5274458811 & -.5275983247 & -.5276490482 \\ 
\hline\hline
\end{tabular}
\caption{Energy upper bounds \label{boundstab}}%
%TCIMACRO{\TeXButton{E}{\end{table}}}
%BeginExpansion
\end{table}%
%EndExpansion
\pagebreak \pagebreak \bigskip \vspace*{3in}\bigskip \bigskip \bigskip
\vspace*{4in}%
%TCIMACRO{\TeXButton{B}{\begin{table}[t] \centering}}
%BeginExpansion
\begin{table}[t] \centering%
%EndExpansion
\begin{tabular}{llll}
\hline\hline
& Hydrogen & Deuterium & Tritium \\ \hline
$E_H-E_{H^{-}}$ & 6083.4058$cm^{-1}$ & 6087.0201$cm^{-1}$ & 6088.2233$%
cm^{-1} $ \\ 
$\Delta E_{corr}$ & .307505$cm^{-1}$ & .307589$cm^{-1}$ & .307616$cm^{-1}$
\\ 
$EA$ & 6083.0983$cm^{-1}$ & 6086.7126$cm^{-1}$ & 6087.9157$cm^{-1}$ \\ 
``Exact'' estimate & 6083.0994$cm^{-1}$ & 6086.7137$cm^{-1}$ & 6087.9168$%
cm^{-1}$ \\ \hline
Drake\cite{Drake88} & 6083.099414$cm^{-1}$ &  &  \\ 
Pekeris\cite{Pekeris62}(using $\Delta E_{corr}$) & 6083.0909$cm^{-1}$ &  & 
\\ 
Lykke\cite{Lykke91}(experiment) & 6082.99$\pm 0.15cm^{-1}$ & 6086.2$\pm
0.6cm^{-1}$ &  \\ \hline\hline
\end{tabular}
\caption{Electron affinity of Hydrogen, Deuterium and Tritium. The  term $ \Delta E_{corr} $
contains relativistic, relativistic recoil, Lamb shift and finite nuclear size corrections   \label{EAtab}}%
%TCIMACRO{\TeXButton{E}{\end{table}}}
%BeginExpansion
\end{table}%
%EndExpansion

\end{document}
