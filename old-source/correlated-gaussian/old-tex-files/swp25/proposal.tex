%% This document created by Scientific Word (R) Version 2.0


\documentclass[12pt,thmsa]{article}
%%%%%%%%%%%%%%%%%%%%%%%%%%%%
\usepackage{sw20jart}

%TCIDATA{TCIstyle=Article/art4.lat,jart,sw20jart}

\input tcilatex
\QQQ{Language}{
American English
}

\begin{document}

\hspace{1.0in}

\vspace{1.0in}

Spectroscopy is one of the most important tools in chemistry. It is
indispensable for structure determination and chemical analysis. Indeed much
of what is known in chemistry is the result of spectroscopic investigation
or at least motivated by such investigations. Quantum chemistry itself is
the curious child of spectroscopy. It was the puzzlement of spectral
observations which motivated the development of quantum mechanics. Since
Plank's investigation of black-body radiation and Bohr's explanation of the
hydrogen spectra as transitions between quantum states, spectroscopic
observations have been interpreted using quantum theory. Theoretical quantum
chemistry has matured enormously since the turn of the century, this,
coupled with the phenomenal increase in the computational power of computers
has opened the door to ``\emph{ab initio} spectroscopy''. It is work in this
relatively new and undeniably important field which is suggested in this
proposal.

Our focus here is on a new method for computing vibration-rotation energy
levels. A variety of methods for investigation of vibration-rotation have
been used ranging from, simple harmonic oscillator rigid rotor models with
perturbation extensions based on the classical rovibrational Hamiltonians of
Eckart\cite{Eckart35}, Willison and Howard\cite{WilsonHoward36}, and the
quantum mechanical version of the Eckart Hamiltonian put forth by Watson\cite
{Watson68}, to the variational method of Sutcliffe and Tennyson\cite
{SutcliffeTennyson87}, and the Morse oscillator based Hamiltonian, (MORBID),
of Jenson\cite{Jenson88}. Most of the recent work utilizes complicated
coordinate transformations which are specific for the systems investigated,
di- and tri-atomic molecules, and are not easily generalized, and not
practical for investigations of larger systems. The method we propose does
not require exotic transformations of the Hamiltonian and is easily extended
to larger systems.

The method proposed here consists of three steps: 1) compute an accurate
potential energy surface for the electronic state of interest, 2) Model the
potential energy surface using a series of elliptic correlated gaussians, 3)
Perform a variational calculation for the vibration-rotation energy levels
using a basis of correlated gaussians.

Generating an accurate potential energy surface for a system with many
degrees of freedom is a computationally intensive task, there are, however,
many methods available for carrying out such calculations, for example, the
various forms of configuration interaction methods. Also, for systems with
few electrons highly accurate potential energy surfaces can be generated
using variational methods, currently under development by us, utilizing
explicitly correlated gaussian basis sets together with analytic derivatives
of the energy functional with respect to the nonlinear variational
parameters.

The second step in our procedure finding an expression for the potential
energy function has been referred to as the search for the ``Holy Grail of
Spectroscopy''. Methods have included expansions in terms of normal
coordinates, Morse functions, interpolating functions, power series
expansions, rational polynomial approximations and many body expansions; see
Searles\cite{Searles93} for a review. Our method will utilize a
multi-parameter fit of the potential energy surface using an expansion in
elliptic correlated gaussians of the form $V=\sum_ic_i\exp \left[ r^{\prime
}\left( A\otimes B\right) r\right] $ where $r$ is a $3n$ vector of nuclear
coordinates in an internal coordinate system, $A$ is a $n\times n$ matrix
with ``orbital'' coefficients on the diagonal and ``correlation''
coefficients off the diagonal and $B$ is a $3\times 3$ diagonal matrix of
parameters which determine the elliptic distortions. We expect this form of
the potential to be flexible enough to model even the most complex potential
energy surface.

The finial step in our procedure is a variational calculation over the
nuclear Hamiltonian utilizing our potential function from step 2. The
variation calculation will be carried out using a symmetry projected basis
of explicitly correlated gaussians with harmonic polynomial pre multipliers
to generate vibration-rotation eigenstates $\phi _k=\mathcal{P}\exp
[-r\left( A\otimes I_n\right) r]\theta _{LM_L}\left( r\right) $. This basis
should produce excellent results since it correlates all n-body
interactions. Notoriously difficult problems such as computing energy levels
for the highly excited vibrational states of H$_2$O and the vibration
rotation levels of the very floppy molecule H$_4^{+}$ should be easily
handled. All of the steps involved in this procedure are tractable. Elegant
and easily implementable mathematical forms for all of the required
integrals and derivatives can be obtained using the powerful matrix
differential calculus described by Kinghorn\cite{Kinghorn95}.

This project will culminate in the development of efficient, easy to use
computer code for performing \emph{ab initio} vibration-rotation energy
level calculations for molecular systems much larger than anything that has
been done to date. This work will mark a significant advance in the
interpretation and prediction of molecular spectra and should be of great
interest to many researchers in a wide variety of fields.

\begin{thebibliography}{9}
\bibitem{Eckart35}  C. Eckart, Phys. Rev. 47, 552 (1953)

\bibitem{WilsonHoward36}  E. B. Wilson and J. B. Howard, J. Chem. Phys. 4,
260 (1936)

\bibitem{Watson68}  J. K. G. Watson, Mol. Phys. 15,479 (1968)

\bibitem{SutcliffeTennyson87}  B. T. Sutcliffe and J. Tennyson, J. Chem.
Soc. Faraday Tran. 83(9), 1663 (1987)

\bibitem{Jenson88}  P. Jenson, J. Mol. Spec. 128, 478 (1988)

\bibitem{Searles93}  D. J. Searles and E. I. von Nagy-Felsobuki, Ab Inito
Variational Calculations of Molecular Vibrational-Rotational Spectra,
Springer-Verlag, Berlin (1993)

\bibitem{Kinghorn95}  D. B. Kinghorn, Int. J. Quant. Chem. (in press)
\end{thebibliography}

\end{document}
