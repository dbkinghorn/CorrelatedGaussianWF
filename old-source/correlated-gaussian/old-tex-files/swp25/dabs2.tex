%% This document created by Scientific Word (R) Version 2.0


\documentclass[12pt,thmsa]{article}
%%%%%%%%%%%%%%%%%%%%%%%%%%%%
\usepackage{sw20jart}

%TCIDATA{TCIstyle=Article/art4.lat,jart,sw20jart}

\input tcilatex

\begin{document}


\begin{center}
{EXPLICITLY CORRELATED GAUSSIAN BASIS FUNCTIONS:\\ DERIVATION AND
IMPLEMENTATION OF MATRIX \\ ELEMENTS AND GRADIENT FORMULAS USING \\ MATRIX
DIFFERENTIAL CALCULUS}

\textbf{Abstract} \\\ \\by Donald B. Kinghorn, Ph.D. \\Washington
State University \\August 1995
\end{center}

\ \newline
{\noindent Chair: R.D. Poshusta} \ \newline

The matrix differential calculus is introduced to the quantum chemistry
community via new matrix derivations of integral formulas and gradients for
Hamiltonian matrix elements in a basis of correlated Gaussian functions.
Requisite mathematical background material on Kronecker products, Hadamard
products, the vec and vech operators, linear structures, and matrix
differential calculus is presented. New matrix forms for the kinetic and
potential energy operators are presented. Integrals for overlap, kinetic
energy and potential energy matrix elements are derived in matrix form using
matrix calculus. The gradient of the energy functional with respect to the
correlated Gaussian exponent matrices is derived. Burdensome summation
notation is entirely replaced with a compact matrix notation that is both
theoretically and computationally insightful.

These new formulas in the basis of explicitly correlated Gaussian basis
functions, are implemented and applied to find variational upper bounds for
non-relativistic ground states of $^4$He, $^\infty $He, Ps$_2$, $^9$Be, and $%
^\infty $Be. Analytic gradients of the energy are included to speed
optimization of the exponential variational parameters. Five different
nonlinear optimization subroutines (algorithms) are compared: TN, truncated
Newton; DUMING, quasi-Newton; DUMIDH, modified Newton; DUMCGG, conjugate
gradient; POWELL, direction set (non-gradient). The new analytic gradient
formulas are found to significantly accelerate optimizations that require
gradients. The truncated Newton algorithm is found to outperform the other
optimizers for the selected test cases. Computer timings and energy bounds
are reported. The new TN bounds surpass previously reported bounds with the
same basis size.

\end{document}
