%% This document created by Scientific Word (R) Version 2.0
%% Starting shell: mathart1


\documentclass[12pt,thmsa]{article}
\usepackage{amsfonts}
%%%%%
\usepackage{sw20aip}
\usepackage{doublesp}

\input tcilatex
\QQQ{Language}{
American English
}

\begin{document}

\author{Donald B. Kinghorn and Ludwik Adamowicz \\
%EndAName
Department of Chemistry, University of Arizona, Tucson, Arizona 85721}
\title{Multi-Center Correlated Gaussian Basis Functions: Integrals and Derivatives
uning Matrix Differential Calculus }
\date{December 14 1995}
\maketitle

\begin{abstract}
Replace this text with your own abstract.
\end{abstract}

\section{Introduction}

\section{Multi-Center Correlated Gaussian Basis Functions}

The milti-center correlated Gaussian functions considered in this work are
negative exponitials of positive definite quadratic forms,

\begin{equation}
\phi _k=\exp \left[ -\left( r-s_k\right) ^{\prime }\left( L_kL_k^{\prime
}\otimes I_3\right) \left( r-s_k\right) \right]  \label{basisfcn}
\end{equation}
Here, $r$ is a $3n\times 1$ vector of Cartesian coordinates for the $n$
particles, $s_k$ is a $3n\times 1$ vector of centers for the Gaussian
functions (not nescessarily nuclear positions), $L_k$ is an $n\times n$
lower triangular matrix of rank $n$ and $I_3$ is the $3\times 3$ identity
matrix. $k$ would range from $1$ to $N$ where $N$ is the number of basis
functions. Integrals involving the functions $\phi _k$ are well defined only
if the exponent matrix is positive definite symmetric, this is assured by
using the Cholesky factorization $L_kL_k^{\prime }$. The Kronecker product
with $I_3$ is used to insure rotational invariance of the basis
functions*****????.

This form is similar to that presented by Singer\cite{Singer1} *****

\section{Integral Formulas}

\subsection{A Fundimental Integral}

The following integral from Friedman\cite{Friedman90int} will be useful in
what follows. Let $A$ be a positive definite $m\times m$ matrix, $a$ an $%
m\times 1$column vector, $x$ an $m\times 1$column vector variable and $%
f\left( u\right) $ an arbitrary integrable function of the variable $u$ over
the interval $\left( -\infty ,\infty \right) $, then, 
\begin{equation}
\int_{x\in \Bbb{R}^n}f(a^{\prime }x)\exp \left[ -x^{\prime }Ax\right]
\,\,dx=\pi ^{\left( m-1\right) /2}\left| A\right| ^{-1/2}\int_{-\infty
}^\infty f\left( \gamma u\right) e^{-u^2}\,\,du  \label{fint}
\end{equation}
with, 
\[
\gamma ^2=a^{\prime }A^{-1}a 
\]

Now, let 
\[
f\left( a^{\prime }x\right) =\exp \left[ a^{\prime }x\right] 
\]
then, using equation (\ref{fint}), 
\begin{eqnarray}
\int_{x\in \Bbb{R}^n}\exp \left[ a^{\prime }x-x^{\prime }Ax\right] \,\,dx
&=&\pi ^{\left( m-1\right) /2}\left| A\right| ^{-1/2}\int_{-\infty }^\infty
e^{\gamma u-u^2}\,\,du  \nonumber \\
&=&\pi ^{m/2}\left| A\right| ^{-1/2}\exp \left[ \left( a^{\prime
}A^{-1}a\right) /4\right]  \label{basicint}
\end{eqnarray}
Use of formula (\ref{basicint}) will allow a straight forward derivation of
the overlap and potential energy integrals.

\subsection{Overlap Integral}

Some notational simplification will make the formulas that follow easier to
read. Let 
\begin{eqnarray*}
A_k &=&L_kL_k^{\prime } \\
A_{kl} &=&A_k+A_l
\end{eqnarray*}
The Kronecker product of a matrix with the $3\times 3$ identity matrix will
be denoted by an over-bar, for example 
\[
\bar{A}_k=A_k\otimes I_3
\]
Continuing, let, 
\begin{eqnarray*}
a_k &=&\bar{A}_ks_k \\
a_{kl} &=&a_k+a_l \\
\gamma _k &=&s_k^{\prime }\bar{A}_ks_k \\
\gamma _{kl} &=&\gamma _k+\gamma _l
\end{eqnarray*}
With this notation the basis function, equation (\ref{basisfcn}), can be
written as 
\begin{eqnarray}
\phi _k &=&\exp \left[ -\left( r-s_k\right) ^{\prime }\bar{A}_k\left(
r-s_k\right) \right]   \nonumber \\
&=&\exp \left[ 2a_k^{\prime }r-r^{\prime }\bar{A}_kr\right] \,e^{-\gamma _k}
\end{eqnarray}
The product of two basis functions is then, 
\begin{equation}
\phi _k\phi _l=\exp \left[ 2a_{kl}^{\prime }r-r^{\prime }\bar{A}_kr\right]
\,e^{-\gamma _{kl}}
\end{equation}

The overlap integral now follows immediately from equation (\ref{basicint}) 
\begin{eqnarray}
\left\langle \phi _k\right. |\left. \phi _l\right\rangle  &=&\pi
^{3n/2}\,e^{-\gamma _{kl}}\,\,\left| \bar{A}_{kl}\right| ^{-1/2}\exp \left[
a_{kl}^{\prime }\bar{A}_{kl}^{-1}a\right]   \nonumber \\
&=&\alpha \,\pi ^{3n/2}\left| A_{kl}\otimes I_3\right| ^{-1/2}  \nonumber \\
&=&\alpha \,\pi ^{3n/2}\left| A_{kl}\right| ^{-3/2}
\end{eqnarray}
where $\alpha $ is the constant, 
\begin{equation}
\alpha =e^{-\gamma _{kl}}\,\exp \left[ a_{kl}^{\prime }\bar{A}%
_{kl}^{-1}a\right] 
\end{equation}
Note that $\alpha =1$ when $s_k=s_l=0,$ the single center case.

\subsection{Kinetic Energy Integral}

\subsection{Potential Energy Integral}

\subsection{Other Integral Formulas}

\section{Gradient of the Energy Functional}

\section{First Derivatives}

\subsection{Overlap Derivative}

\subsection{Kinetic Energy Derivative}

\subsection{Potential Energy Derivative}

\section{Conclusions}

\appendix 

\section{Some Useful Matrix Results}

\end{document}
