%% This document created by Scientific Word (R) Version 2.0


\documentclass[12pt,thmsa]{article}
%%%%%%%%%%%%%%%%%%%%%%%%%%%%
\usepackage{sw20jart}

%TCIDATA{TCIstyle=Article/art4.lat,jart,sw20jart}

\input tcilatex
\QQQ{Language}{
American English
}

\begin{document}

Specific research objectives for the postdoctoral appointment include;

\begin{itemize}
\item  derivation of formulas for the Hamiltonian matrix elements and energy
derivatives involving the various forms of the correlated Gaussians
described in the proposal,

\item  development of computer code for generating Born-Oppenheimer
potential energy surfaces for few particle systems using correlated
Gaussians including test calculations on H$_2$, H$_3^{+}$, H$_4^{+}$, Li$_2$
and other test cases,

\item  implementation of code for a formal test of the methodology for doing
vibration-rotation energy level calculations described in the proposal. The
test calculation will be on the often used test case H$_3^{+}$. 
\end{itemize}

Other activities related directly and indirectly to the objectives stated
above include;

\begin{itemize}
\item  review of the literature on methodology for vibration-rotation
calculations,

\item  participation in scientific meetings,

\item  increase skill at using existing computer packages such as Gaussian92
for computing potential energy surfaces and learn new methods such as the
coupled cluster approach of Adamowicz
\end{itemize}

The work that I have proposed makes heavy use of correlated Gaussian
functions, this is one of the reasons for choosing Professor Adamowicz as my
postdoctoral advisor, he is regarded as on of the principle experts in the
use of these functions. In addition to his expertise with correlated
Gaussians his work covers a wide range of topics in quantum chemistry, I am
sure I will learn a great deal working with his research group. I have
already started working with professor Adamowicz and find the interaction
with this 8 member group to be very stimulating and enjoyable.   

\end{document}
